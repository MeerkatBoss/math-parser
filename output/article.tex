\documentclass{article}
\usepackage{preamble}
\begin{document}
\maketitle
\begin{abstract}
Wonderful article
\end{abstract}
\newpage
\section{Derivative}
Let us find the derivative of the following function:
\begin{equation}
\sin x ^{3 } + \left( \cos 15 \cdot x \right) ^{4 } 
\end{equation}

We shall ponder the following:
\begin{equation}
15 \cdot x 
\end{equation}
Clearly, the derivative of this is equal to
\begin{equation}
0 \cdot x + 15 \cdot 1 
\end{equation}

Consider the following:
\begin{equation}
x ^{3 } 
\end{equation}
Obviously, the derivative of this is equal to
\begin{equation}
3 \cdot x ^{3 - 1 } \cdot 1 
\end{equation}

Now the proof that the derivative of this function is equal to
\begin{equation}
\cos x ^{3 } \cdot 3 \cdot x ^{3 - 1 } \cdot 1 + 4 \cdot \left( \cos 15 \cdot x \right) ^{4 - 1 } \cdot -\sin 15 \cdot x \cdot \left( 0 \cdot x + 15 \cdot 1 \right) 
\end{equation}
shall be considered an amusing exercise for the reader.
Unsurprisingly, if we simplify this we wil get
\begin{equation}
\cos x ^{3 } \cdot 3 \cdot x ^{2 } - 4 \cdot \left( \cos 15 \cdot x \right) ^{3 } \cdot \sin 15 \cdot x \cdot 15 
\end{equation}
\newpage
\section{Taylor series}
Let us find the Taylor series at $x = 0$ of the following function:
\begin{equation}
\sin x ^{3 } + \left( \cos 15 \cdot x \right) ^{4 } 
\end{equation}

The following is worth a closer look:
\begin{equation}
15 \cdot x 
\end{equation}
It is now obvious, that the derivative of this is equal to
\begin{equation}
0 \cdot x + 15 \cdot 1 
\end{equation}

We are going to study the following:
\begin{equation}
x ^{3 } 
\end{equation}
Any self-respecting mathematician would find it obvious, that the derivative of this is equal to
\begin{equation}
3 \cdot x ^{3 - 1 } \cdot 1 
\end{equation}

The object of our ultimate interest is the following:
\begin{equation}
15 
\end{equation}
Trivially, the derivative of this is equal to
\begin{equation}
0 
\end{equation}

Consider the following:
\begin{equation}
15 \cdot x 
\end{equation}
Unsurprisingly, the derivative of this is equal to
\begin{equation}
0 \cdot x + 15 \cdot 1 
\end{equation}

We shall ponder the following:
\begin{equation}
15 \cdot x 
\end{equation}
Any self-respecting mathematician would find it obvious, that the derivative of this is equal to
\begin{equation}
0 \cdot x + 15 \cdot 1 
\end{equation}

We are going to study the following:
\begin{equation}
4 
\end{equation}
Unsurprisingly, the derivative of this is equal to
\begin{equation}
0 
\end{equation}

We will allow ourselves to divert the reader's attention to this gem of mathematical wonder:
\begin{equation}
x ^{2 } 
\end{equation}
Unsurprisingly, the derivative of this is equal to
\begin{equation}
2 \cdot x ^{2 - 1 } \cdot 1 
\end{equation}

One shall regard the object in question with utmost interest:
\begin{equation}
3 
\end{equation}
Clearly, the derivative of this is equal to
\begin{equation}
0 
\end{equation}

We shall ponder the following:
\begin{equation}
x ^{3 } 
\end{equation}
It can be easily proved, that the derivative of this is equal to
\begin{equation}
3 \cdot x ^{3 - 1 } \cdot 1 
\end{equation}

Let us take a look at this:
\begin{equation}
15 
\end{equation}
Clearly, the derivative of this is equal to
\begin{equation}
0 
\end{equation}

Let us take a look at this:
\begin{equation}
15 \cdot x 
\end{equation}
Any self-respecting mathematician would find it obvious, that the derivative of this is equal to
\begin{equation}
0 \cdot x + 15 \cdot 1 
\end{equation}

The object of our ultimate interest is the following:
\begin{equation}
15 
\end{equation}
It is now obvious, that the derivative of this is equal to
\begin{equation}
0 
\end{equation}

We will allow ourselves to divert the reader's attention to this gem of mathematical wonder:
\begin{equation}
15 \cdot x 
\end{equation}
Clearly, the derivative of this is equal to
\begin{equation}
0 \cdot x + 15 \cdot 1 
\end{equation}

One shall regard the object in question with utmost interest:
\begin{equation}
15 \cdot x 
\end{equation}
It can be easily proved, that the derivative of this is equal to
\begin{equation}
0 \cdot x + 15 \cdot 1 
\end{equation}

Let us take a look at this:
\begin{equation}
3 
\end{equation}
Unsurprisingly, the derivative of this is equal to
\begin{equation}
0 
\end{equation}

We shall ponder the following:
\begin{equation}
4 
\end{equation}
Clearly, the derivative of this is equal to
\begin{equation}
0 
\end{equation}

We are going to study the following:
\begin{equation}
15 
\end{equation}
Clearly, the derivative of this is equal to
\begin{equation}
0 
\end{equation}

We are going to study the following:
\begin{equation}
15 
\end{equation}
Trivially, the derivative of this is equal to
\begin{equation}
0 
\end{equation}

One shall regard the object in question with utmost interest:
\begin{equation}
15 \cdot x 
\end{equation}
Obviously, the derivative of this is equal to
\begin{equation}
0 \cdot x + 15 \cdot 1 
\end{equation}

The following is worth a closer look:
\begin{equation}
15 \cdot x 
\end{equation}
It is now obvious, that the derivative of this is equal to
\begin{equation}
0 \cdot x + 15 \cdot 1 
\end{equation}

We shall ponder the following:
\begin{equation}
4 
\end{equation}
Trivially, the derivative of this is equal to
\begin{equation}
0 
\end{equation}

Consider the following:
\begin{equation}
x ^{2 } 
\end{equation}
Trivially, the derivative of this is equal to
\begin{equation}
2 \cdot x ^{2 - 1 } \cdot 1 
\end{equation}

We are going to study the following:
\begin{equation}
3 
\end{equation}
It can be easily proved, that the derivative of this is equal to
\begin{equation}
0 
\end{equation}

Let us take a look at this:
\begin{equation}
x ^{2 } 
\end{equation}
Any self-respecting mathematician would find it obvious, that the derivative of this is equal to
\begin{equation}
2 \cdot x ^{2 - 1 } \cdot 1 
\end{equation}

We will take a closer look at this:
\begin{equation}
3 
\end{equation}
Clearly, the derivative of this is equal to
\begin{equation}
0 
\end{equation}

We are going to study the following:
\begin{equation}
x ^{3 } 
\end{equation}
Unsurprisingly, the derivative of this is equal to
\begin{equation}
3 \cdot x ^{3 - 1 } \cdot 1 
\end{equation}

We will take a closer look at this:
\begin{equation}
2 \cdot x 
\end{equation}
As you can see, the derivative of this is equal to
\begin{equation}
0 \cdot x + 2 \cdot 1 
\end{equation}

The object of our ultimate interest is the following:
\begin{equation}
3 
\end{equation}
Clearly, the derivative of this is equal to
\begin{equation}
0 
\end{equation}

We will take a closer look at this:
\begin{equation}
x ^{3 } 
\end{equation}
As you can see, the derivative of this is equal to
\begin{equation}
3 \cdot x ^{3 - 1 } \cdot 1 
\end{equation}

Consider the following:
\begin{equation}
15 
\end{equation}
It can be easily proved, that the derivative of this is equal to
\begin{equation}
0 
\end{equation}

The object of our ultimate interest is the following:
\begin{equation}
15 
\end{equation}
Unsurprisingly, the derivative of this is equal to
\begin{equation}
0 
\end{equation}

Consider the following:
\begin{equation}
15 \cdot x 
\end{equation}
Any self-respecting mathematician would find it obvious, that the derivative of this is equal to
\begin{equation}
0 \cdot x + 15 \cdot 1 
\end{equation}

The following is worth a closer look:
\begin{equation}
15 
\end{equation}
It is now obvious, that the derivative of this is equal to
\begin{equation}
0 
\end{equation}

Consider the following:
\begin{equation}
15 \cdot x 
\end{equation}
As you can see, the derivative of this is equal to
\begin{equation}
0 \cdot x + 15 \cdot 1 
\end{equation}

We will take a closer look at this:
\begin{equation}
15 \cdot x 
\end{equation}
Obviously, the derivative of this is equal to
\begin{equation}
0 \cdot x + 15 \cdot 1 
\end{equation}

We shall ponder the following:
\begin{equation}
3 
\end{equation}
Unsurprisingly, the derivative of this is equal to
\begin{equation}
0 
\end{equation}

Consider the following:
\begin{equation}
4 
\end{equation}
Clearly, the derivative of this is equal to
\begin{equation}
0 
\end{equation}

Let us take a look at this:
\begin{equation}
15 
\end{equation}
Clearly, the derivative of this is equal to
\begin{equation}
0 
\end{equation}

We are going to study the following:
\begin{equation}
15 \cdot x 
\end{equation}
Clearly, the derivative of this is equal to
\begin{equation}
0 \cdot x + 15 \cdot 1 
\end{equation}

Consider the following:
\begin{equation}
15 
\end{equation}
Trivially, the derivative of this is equal to
\begin{equation}
0 
\end{equation}

We are going to study the following:
\begin{equation}
15 \cdot x 
\end{equation}
Unsurprisingly, the derivative of this is equal to
\begin{equation}
0 \cdot x + 15 \cdot 1 
\end{equation}

Let us take a look at this:
\begin{equation}
15 
\end{equation}
Obviously, the derivative of this is equal to
\begin{equation}
0 
\end{equation}

Let us take a look at this:
\begin{equation}
15 \cdot x 
\end{equation}
Trivially, the derivative of this is equal to
\begin{equation}
0 \cdot x + 15 \cdot 1 
\end{equation}

We are going to study the following:
\begin{equation}
15 \cdot x 
\end{equation}
It can be easily proved, that the derivative of this is equal to
\begin{equation}
0 \cdot x + 15 \cdot 1 
\end{equation}

Let us take a look at this:
\begin{equation}
2 
\end{equation}
Clearly, the derivative of this is equal to
\begin{equation}
0 
\end{equation}

One shall regard the object in question with utmost interest:
\begin{equation}
3 
\end{equation}
Trivially, the derivative of this is equal to
\begin{equation}
0 
\end{equation}

We are going to study the following:
\begin{equation}
15 
\end{equation}
As you can see, the derivative of this is equal to
\begin{equation}
0 
\end{equation}

Let us take a look at this:
\begin{equation}
15 
\end{equation}
As you can see, the derivative of this is equal to
\begin{equation}
0 
\end{equation}

We will take a closer look at this:
\begin{equation}
15 \cdot x 
\end{equation}
Trivially, the derivative of this is equal to
\begin{equation}
0 \cdot x + 15 \cdot 1 
\end{equation}

We will allow ourselves to divert the reader's attention to this gem of mathematical wonder:
\begin{equation}
15 \cdot x 
\end{equation}
It can be easily proved, that the derivative of this is equal to
\begin{equation}
0 \cdot x + 15 \cdot 1 
\end{equation}

We shall ponder the following:
\begin{equation}
3 
\end{equation}
As you can see, the derivative of this is equal to
\begin{equation}
0 
\end{equation}

One shall regard the object in question with utmost interest:
\begin{equation}
4 
\end{equation}
It can be easily proved, that the derivative of this is equal to
\begin{equation}
0 
\end{equation}

We will take a closer look at this:
\begin{equation}
15 
\end{equation}
As you can see, the derivative of this is equal to
\begin{equation}
0 
\end{equation}

One shall regard the object in question with utmost interest:
\begin{equation}
15 
\end{equation}
Any self-respecting mathematician would find it obvious, that the derivative of this is equal to
\begin{equation}
0 
\end{equation}

We shall ponder the following:
\begin{equation}
15 
\end{equation}
Clearly, the derivative of this is equal to
\begin{equation}
0 
\end{equation}

We will allow ourselves to divert the reader's attention to this gem of mathematical wonder:
\begin{equation}
15 \cdot x 
\end{equation}
Unsurprisingly, the derivative of this is equal to
\begin{equation}
0 \cdot x + 15 \cdot 1 
\end{equation}

We will take a closer look at this:
\begin{equation}
15 \cdot x 
\end{equation}
Trivially, the derivative of this is equal to
\begin{equation}
0 \cdot x + 15 \cdot 1 
\end{equation}

The following is worth a closer look:
\begin{equation}
4 
\end{equation}
It is now obvious, that the derivative of this is equal to
\begin{equation}
0 
\end{equation}

We will allow ourselves to divert the reader's attention to this gem of mathematical wonder:
\begin{equation}
15 
\end{equation}
It can be easily proved, that the derivative of this is equal to
\begin{equation}
0 
\end{equation}

We will allow ourselves to divert the reader's attention to this gem of mathematical wonder:
\begin{equation}
15 
\end{equation}
It is now obvious, that the derivative of this is equal to
\begin{equation}
0 
\end{equation}

We are going to study the following:
\begin{equation}
15 \cdot x 
\end{equation}
It can be easily proved, that the derivative of this is equal to
\begin{equation}
0 \cdot x + 15 \cdot 1 
\end{equation}

The following is worth a closer look:
\begin{equation}
15 
\end{equation}
Obviously, the derivative of this is equal to
\begin{equation}
0 
\end{equation}

The object of our ultimate interest is the following:
\begin{equation}
15 \cdot x 
\end{equation}
Clearly, the derivative of this is equal to
\begin{equation}
0 \cdot x + 15 \cdot 1 
\end{equation}

We are going to study the following:
\begin{equation}
15 \cdot x 
\end{equation}
Trivially, the derivative of this is equal to
\begin{equation}
0 \cdot x + 15 \cdot 1 
\end{equation}

We will allow ourselves to divert the reader's attention to this gem of mathematical wonder:
\begin{equation}
3 
\end{equation}
It is now obvious, that the derivative of this is equal to
\begin{equation}
0 
\end{equation}

Let us take a look at this:
\begin{equation}
4 
\end{equation}
As you can see, the derivative of this is equal to
\begin{equation}
0 
\end{equation}

One shall regard the object in question with utmost interest:
\begin{equation}
2 \cdot x 
\end{equation}
Trivially, the derivative of this is equal to
\begin{equation}
0 \cdot x + 2 \cdot 1 
\end{equation}

The object of our ultimate interest is the following:
\begin{equation}
3 
\end{equation}
It is now obvious, that the derivative of this is equal to
\begin{equation}
0 
\end{equation}

The object of our ultimate interest is the following:
\begin{equation}
x ^{2 } 
\end{equation}
As you can see, the derivative of this is equal to
\begin{equation}
2 \cdot x ^{2 - 1 } \cdot 1 
\end{equation}

The object of our ultimate interest is the following:
\begin{equation}
3 
\end{equation}
It can be easily proved, that the derivative of this is equal to
\begin{equation}
0 
\end{equation}

The object of our ultimate interest is the following:
\begin{equation}
x ^{3 } 
\end{equation}
Clearly, the derivative of this is equal to
\begin{equation}
3 \cdot x ^{3 - 1 } \cdot 1 
\end{equation}

We will take a closer look at this:
\begin{equation}
x ^{2 } 
\end{equation}
Unsurprisingly, the derivative of this is equal to
\begin{equation}
2 \cdot x ^{2 - 1 } \cdot 1 
\end{equation}

One shall regard the object in question with utmost interest:
\begin{equation}
3 
\end{equation}
Trivially, the derivative of this is equal to
\begin{equation}
0 
\end{equation}

Consider the following:
\begin{equation}
2 \cdot x 
\end{equation}
Trivially, the derivative of this is equal to
\begin{equation}
0 \cdot x + 2 \cdot 1 
\end{equation}

The object of our ultimate interest is the following:
\begin{equation}
3 
\end{equation}
It can be easily proved, that the derivative of this is equal to
\begin{equation}
0 
\end{equation}

We shall ponder the following:
\begin{equation}
x ^{3 } 
\end{equation}
Any self-respecting mathematician would find it obvious, that the derivative of this is equal to
\begin{equation}
3 \cdot x ^{3 - 1 } \cdot 1 
\end{equation}

We shall ponder the following:
\begin{equation}
x ^{2 } 
\end{equation}
Clearly, the derivative of this is equal to
\begin{equation}
2 \cdot x ^{2 - 1 } \cdot 1 
\end{equation}

One shall regard the object in question with utmost interest:
\begin{equation}
3 
\end{equation}
Trivially, the derivative of this is equal to
\begin{equation}
0 
\end{equation}

We will allow ourselves to divert the reader's attention to this gem of mathematical wonder:
\begin{equation}
x ^{2 } 
\end{equation}
Clearly, the derivative of this is equal to
\begin{equation}
2 \cdot x ^{2 - 1 } \cdot 1 
\end{equation}

The following is worth a closer look:
\begin{equation}
3 
\end{equation}
Unsurprisingly, the derivative of this is equal to
\begin{equation}
0 
\end{equation}

The object of our ultimate interest is the following:
\begin{equation}
x ^{3 } 
\end{equation}
Clearly, the derivative of this is equal to
\begin{equation}
3 \cdot x ^{3 - 1 } \cdot 1 
\end{equation}

One shall regard the object in question with utmost interest:
\begin{equation}
2 \cdot x 
\end{equation}
It is now obvious, that the derivative of this is equal to
\begin{equation}
0 \cdot x + 2 \cdot 1 
\end{equation}

We shall ponder the following:
\begin{equation}
3 
\end{equation}
Unsurprisingly, the derivative of this is equal to
\begin{equation}
0 
\end{equation}

Let us take a look at this:
\begin{equation}
x ^{2 } 
\end{equation}
Trivially, the derivative of this is equal to
\begin{equation}
2 \cdot x ^{2 - 1 } \cdot 1 
\end{equation}

We will take a closer look at this:
\begin{equation}
3 
\end{equation}
Unsurprisingly, the derivative of this is equal to
\begin{equation}
0 
\end{equation}

The following is worth a closer look:
\begin{equation}
x ^{3 } 
\end{equation}
Clearly, the derivative of this is equal to
\begin{equation}
3 \cdot x ^{3 - 1 } \cdot 1 
\end{equation}

We will allow ourselves to divert the reader's attention to this gem of mathematical wonder:
\begin{equation}
6 
\end{equation}
It can be easily proved, that the derivative of this is equal to
\begin{equation}
0 
\end{equation}

We shall ponder the following:
\begin{equation}
x ^{3 } 
\end{equation}
Any self-respecting mathematician would find it obvious, that the derivative of this is equal to
\begin{equation}
3 \cdot x ^{3 - 1 } \cdot 1 
\end{equation}

Consider the following:
\begin{equation}
6 
\end{equation}
Any self-respecting mathematician would find it obvious, that the derivative of this is equal to
\begin{equation}
0 
\end{equation}

We are going to study the following:
\begin{equation}
x ^{2 } 
\end{equation}
Obviously, the derivative of this is equal to
\begin{equation}
2 \cdot x ^{2 - 1 } \cdot 1 
\end{equation}

We will allow ourselves to divert the reader's attention to this gem of mathematical wonder:
\begin{equation}
3 
\end{equation}
Any self-respecting mathematician would find it obvious, that the derivative of this is equal to
\begin{equation}
0 
\end{equation}

The following is worth a closer look:
\begin{equation}
x ^{3 } 
\end{equation}
Any self-respecting mathematician would find it obvious, that the derivative of this is equal to
\begin{equation}
3 \cdot x ^{3 - 1 } \cdot 1 
\end{equation}

Consider the following:
\begin{equation}
2 \cdot x 
\end{equation}
It can be easily proved, that the derivative of this is equal to
\begin{equation}
0 \cdot x + 2 \cdot 1 
\end{equation}

We will allow ourselves to divert the reader's attention to this gem of mathematical wonder:
\begin{equation}
3 
\end{equation}
Obviously, the derivative of this is equal to
\begin{equation}
0 
\end{equation}

We are going to study the following:
\begin{equation}
2 \cdot x 
\end{equation}
Any self-respecting mathematician would find it obvious, that the derivative of this is equal to
\begin{equation}
0 \cdot x + 2 \cdot 1 
\end{equation}

Let us take a look at this:
\begin{equation}
3 
\end{equation}
It can be easily proved, that the derivative of this is equal to
\begin{equation}
0 
\end{equation}

The following is worth a closer look:
\begin{equation}
x ^{3 } 
\end{equation}
Obviously, the derivative of this is equal to
\begin{equation}
3 \cdot x ^{3 - 1 } \cdot 1 
\end{equation}

Consider the following:
\begin{equation}
x ^{2 } 
\end{equation}
Trivially, the derivative of this is equal to
\begin{equation}
2 \cdot x ^{2 - 1 } \cdot 1 
\end{equation}

The following is worth a closer look:
\begin{equation}
3 
\end{equation}
Clearly, the derivative of this is equal to
\begin{equation}
0 
\end{equation}

Let us take a look at this:
\begin{equation}
x ^{2 } 
\end{equation}
Unsurprisingly, the derivative of this is equal to
\begin{equation}
2 \cdot x ^{2 - 1 } \cdot 1 
\end{equation}

We will take a closer look at this:
\begin{equation}
3 
\end{equation}
Unsurprisingly, the derivative of this is equal to
\begin{equation}
0 
\end{equation}

We will take a closer look at this:
\begin{equation}
x ^{3 } 
\end{equation}
As you can see, the derivative of this is equal to
\begin{equation}
3 \cdot x ^{3 - 1 } \cdot 1 
\end{equation}

One shall regard the object in question with utmost interest:
\begin{equation}
2 \cdot x 
\end{equation}
It can be easily proved, that the derivative of this is equal to
\begin{equation}
0 \cdot x + 2 \cdot 1 
\end{equation}

The object of our ultimate interest is the following:
\begin{equation}
3 
\end{equation}
It can be easily proved, that the derivative of this is equal to
\begin{equation}
0 
\end{equation}

We shall ponder the following:
\begin{equation}
2 \cdot x 
\end{equation}
Obviously, the derivative of this is equal to
\begin{equation}
0 \cdot x + 2 \cdot 1 
\end{equation}

The following is worth a closer look:
\begin{equation}
3 
\end{equation}
Trivially, the derivative of this is equal to
\begin{equation}
0 
\end{equation}

We will allow ourselves to divert the reader's attention to this gem of mathematical wonder:
\begin{equation}
x ^{3 } 
\end{equation}
Any self-respecting mathematician would find it obvious, that the derivative of this is equal to
\begin{equation}
3 \cdot x ^{3 - 1 } \cdot 1 
\end{equation}

The object of our ultimate interest is the following:
\begin{equation}
x ^{2 } 
\end{equation}
Trivially, the derivative of this is equal to
\begin{equation}
2 \cdot x ^{2 - 1 } \cdot 1 
\end{equation}

One shall regard the object in question with utmost interest:
\begin{equation}
3 
\end{equation}
Trivially, the derivative of this is equal to
\begin{equation}
0 
\end{equation}

The object of our ultimate interest is the following:
\begin{equation}
x ^{2 } 
\end{equation}
Trivially, the derivative of this is equal to
\begin{equation}
2 \cdot x ^{2 - 1 } \cdot 1 
\end{equation}

Consider the following:
\begin{equation}
3 
\end{equation}
It can be easily proved, that the derivative of this is equal to
\begin{equation}
0 
\end{equation}

Let us take a look at this:
\begin{equation}
x ^{3 } 
\end{equation}
Trivially, the derivative of this is equal to
\begin{equation}
3 \cdot x ^{3 - 1 } \cdot 1 
\end{equation}

The object of our ultimate interest is the following:
\begin{equation}
x ^{2 } 
\end{equation}
It can be easily proved, that the derivative of this is equal to
\begin{equation}
2 \cdot x ^{2 - 1 } \cdot 1 
\end{equation}

The object of our ultimate interest is the following:
\begin{equation}
3 
\end{equation}
It is now obvious, that the derivative of this is equal to
\begin{equation}
0 
\end{equation}

Consider the following:
\begin{equation}
6 
\end{equation}
As you can see, the derivative of this is equal to
\begin{equation}
0 
\end{equation}

The object of our ultimate interest is the following:
\begin{equation}
x ^{3 } 
\end{equation}
Clearly, the derivative of this is equal to
\begin{equation}
3 \cdot x ^{3 - 1 } \cdot 1 
\end{equation}

Let us take a look at this:
\begin{equation}
2 \cdot x 
\end{equation}
Unsurprisingly, the derivative of this is equal to
\begin{equation}
0 \cdot x + 2 \cdot 1 
\end{equation}

The following is worth a closer look:
\begin{equation}
3 
\end{equation}
As you can see, the derivative of this is equal to
\begin{equation}
0 
\end{equation}

We will take a closer look at this:
\begin{equation}
x ^{2 } 
\end{equation}
Clearly, the derivative of this is equal to
\begin{equation}
2 \cdot x ^{2 - 1 } \cdot 1 
\end{equation}

The object of our ultimate interest is the following:
\begin{equation}
3 
\end{equation}
Clearly, the derivative of this is equal to
\begin{equation}
0 
\end{equation}

The object of our ultimate interest is the following:
\begin{equation}
x ^{3 } 
\end{equation}
It can be easily proved, that the derivative of this is equal to
\begin{equation}
3 \cdot x ^{3 - 1 } \cdot 1 
\end{equation}

We will allow ourselves to divert the reader's attention to this gem of mathematical wonder:
\begin{equation}
2 \cdot x 
\end{equation}
Obviously, the derivative of this is equal to
\begin{equation}
0 \cdot x + 2 \cdot 1 
\end{equation}

We shall ponder the following:
\begin{equation}
3 
\end{equation}
Clearly, the derivative of this is equal to
\begin{equation}
0 
\end{equation}

Let us take a look at this:
\begin{equation}
x ^{2 } 
\end{equation}
Any self-respecting mathematician would find it obvious, that the derivative of this is equal to
\begin{equation}
2 \cdot x ^{2 - 1 } \cdot 1 
\end{equation}

We will take a closer look at this:
\begin{equation}
3 
\end{equation}
Obviously, the derivative of this is equal to
\begin{equation}
0 
\end{equation}

We will take a closer look at this:
\begin{equation}
x ^{3 } 
\end{equation}
Any self-respecting mathematician would find it obvious, that the derivative of this is equal to
\begin{equation}
3 \cdot x ^{3 - 1 } \cdot 1 
\end{equation}

One shall regard the object in question with utmost interest:
\begin{equation}
x ^{2 } 
\end{equation}
Obviously, the derivative of this is equal to
\begin{equation}
2 \cdot x ^{2 - 1 } \cdot 1 
\end{equation}

We are going to study the following:
\begin{equation}
3 
\end{equation}
Trivially, the derivative of this is equal to
\begin{equation}
0 
\end{equation}

One shall regard the object in question with utmost interest:
\begin{equation}
x ^{2 } 
\end{equation}
Trivially, the derivative of this is equal to
\begin{equation}
2 \cdot x ^{2 - 1 } \cdot 1 
\end{equation}

Consider the following:
\begin{equation}
3 
\end{equation}
Unsurprisingly, the derivative of this is equal to
\begin{equation}
0 
\end{equation}

Consider the following:
\begin{equation}
x ^{2 } 
\end{equation}
It can be easily proved, that the derivative of this is equal to
\begin{equation}
2 \cdot x ^{2 - 1 } \cdot 1 
\end{equation}

The object of our ultimate interest is the following:
\begin{equation}
3 
\end{equation}
Obviously, the derivative of this is equal to
\begin{equation}
0 
\end{equation}

Let us take a look at this:
\begin{equation}
x ^{3 } 
\end{equation}
Clearly, the derivative of this is equal to
\begin{equation}
3 \cdot x ^{3 - 1 } \cdot 1 
\end{equation}

We are going to study the following:
\begin{equation}
2 \cdot x 
\end{equation}
It can be easily proved, that the derivative of this is equal to
\begin{equation}
0 \cdot x + 2 \cdot 1 
\end{equation}

The object of our ultimate interest is the following:
\begin{equation}
3 
\end{equation}
Obviously, the derivative of this is equal to
\begin{equation}
0 
\end{equation}

Consider the following:
\begin{equation}
x ^{3 } 
\end{equation}
Trivially, the derivative of this is equal to
\begin{equation}
3 \cdot x ^{3 - 1 } \cdot 1 
\end{equation}

Consider the following:
\begin{equation}
6 
\end{equation}
Unsurprisingly, the derivative of this is equal to
\begin{equation}
0 
\end{equation}

We will take a closer look at this:
\begin{equation}
x ^{2 } 
\end{equation}
It is now obvious, that the derivative of this is equal to
\begin{equation}
2 \cdot x ^{2 - 1 } \cdot 1 
\end{equation}

Let us take a look at this:
\begin{equation}
3 
\end{equation}
As you can see, the derivative of this is equal to
\begin{equation}
0 
\end{equation}

Consider the following:
\begin{equation}
x ^{3 } 
\end{equation}
Unsurprisingly, the derivative of this is equal to
\begin{equation}
3 \cdot x ^{3 - 1 } \cdot 1 
\end{equation}

The following is worth a closer look:
\begin{equation}
2 \cdot x 
\end{equation}
Unsurprisingly, the derivative of this is equal to
\begin{equation}
0 \cdot x + 2 \cdot 1 
\end{equation}

The object of our ultimate interest is the following:
\begin{equation}
3 
\end{equation}
It can be easily proved, that the derivative of this is equal to
\begin{equation}
0 
\end{equation}

We shall ponder the following:
\begin{equation}
2 \cdot x 
\end{equation}
Unsurprisingly, the derivative of this is equal to
\begin{equation}
0 \cdot x + 2 \cdot 1 
\end{equation}

One shall regard the object in question with utmost interest:
\begin{equation}
3 
\end{equation}
Obviously, the derivative of this is equal to
\begin{equation}
0 
\end{equation}

We will allow ourselves to divert the reader's attention to this gem of mathematical wonder:
\begin{equation}
x ^{3 } 
\end{equation}
It is now obvious, that the derivative of this is equal to
\begin{equation}
3 \cdot x ^{3 - 1 } \cdot 1 
\end{equation}

We are going to study the following:
\begin{equation}
x ^{2 } 
\end{equation}
Unsurprisingly, the derivative of this is equal to
\begin{equation}
2 \cdot x ^{2 - 1 } \cdot 1 
\end{equation}

The object of our ultimate interest is the following:
\begin{equation}
3 
\end{equation}
Clearly, the derivative of this is equal to
\begin{equation}
0 
\end{equation}

The object of our ultimate interest is the following:
\begin{equation}
x ^{2 } 
\end{equation}
As you can see, the derivative of this is equal to
\begin{equation}
2 \cdot x ^{2 - 1 } \cdot 1 
\end{equation}

We will allow ourselves to divert the reader's attention to this gem of mathematical wonder:
\begin{equation}
3 
\end{equation}
Any self-respecting mathematician would find it obvious, that the derivative of this is equal to
\begin{equation}
0 
\end{equation}

Let us take a look at this:
\begin{equation}
x ^{3 } 
\end{equation}
Obviously, the derivative of this is equal to
\begin{equation}
3 \cdot x ^{3 - 1 } \cdot 1 
\end{equation}

We will allow ourselves to divert the reader's attention to this gem of mathematical wonder:
\begin{equation}
6 
\end{equation}
Any self-respecting mathematician would find it obvious, that the derivative of this is equal to
\begin{equation}
0 
\end{equation}

The object of our ultimate interest is the following:
\begin{equation}
x ^{2 } 
\end{equation}
It can be easily proved, that the derivative of this is equal to
\begin{equation}
2 \cdot x ^{2 - 1 } \cdot 1 
\end{equation}

We will take a closer look at this:
\begin{equation}
3 
\end{equation}
It is now obvious, that the derivative of this is equal to
\begin{equation}
0 
\end{equation}

We are going to study the following:
\begin{equation}
x ^{3 } 
\end{equation}
As you can see, the derivative of this is equal to
\begin{equation}
3 \cdot x ^{3 - 1 } \cdot 1 
\end{equation}

We will take a closer look at this:
\begin{equation}
15 
\end{equation}
Trivially, the derivative of this is equal to
\begin{equation}
0 
\end{equation}

We shall ponder the following:
\begin{equation}
15 
\end{equation}
Obviously, the derivative of this is equal to
\begin{equation}
0 
\end{equation}

We will take a closer look at this:
\begin{equation}
15 
\end{equation}
Any self-respecting mathematician would find it obvious, that the derivative of this is equal to
\begin{equation}
0 
\end{equation}

One shall regard the object in question with utmost interest:
\begin{equation}
15 \cdot x 
\end{equation}
Trivially, the derivative of this is equal to
\begin{equation}
0 \cdot x + 15 \cdot 1 
\end{equation}

Consider the following:
\begin{equation}
15 
\end{equation}
As you can see, the derivative of this is equal to
\begin{equation}
0 
\end{equation}

We are going to study the following:
\begin{equation}
15 \cdot x 
\end{equation}
It is now obvious, that the derivative of this is equal to
\begin{equation}
0 \cdot x + 15 \cdot 1 
\end{equation}

Let us take a look at this:
\begin{equation}
15 \cdot x 
\end{equation}
Any self-respecting mathematician would find it obvious, that the derivative of this is equal to
\begin{equation}
0 \cdot x + 15 \cdot 1 
\end{equation}

Consider the following:
\begin{equation}
3 
\end{equation}
Unsurprisingly, the derivative of this is equal to
\begin{equation}
0 
\end{equation}

The object of our ultimate interest is the following:
\begin{equation}
4 
\end{equation}
It can be easily proved, that the derivative of this is equal to
\begin{equation}
0 
\end{equation}

The object of our ultimate interest is the following:
\begin{equation}
15 
\end{equation}
It can be easily proved, that the derivative of this is equal to
\begin{equation}
0 
\end{equation}

Consider the following:
\begin{equation}
15 
\end{equation}
It can be easily proved, that the derivative of this is equal to
\begin{equation}
0 
\end{equation}

The following is worth a closer look:
\begin{equation}
15 \cdot x 
\end{equation}
Obviously, the derivative of this is equal to
\begin{equation}
0 \cdot x + 15 \cdot 1 
\end{equation}

One shall regard the object in question with utmost interest:
\begin{equation}
15 
\end{equation}
Unsurprisingly, the derivative of this is equal to
\begin{equation}
0 
\end{equation}

We will take a closer look at this:
\begin{equation}
15 \cdot x 
\end{equation}
It can be easily proved, that the derivative of this is equal to
\begin{equation}
0 \cdot x + 15 \cdot 1 
\end{equation}

The object of our ultimate interest is the following:
\begin{equation}
15 
\end{equation}
Clearly, the derivative of this is equal to
\begin{equation}
0 
\end{equation}

Let us take a look at this:
\begin{equation}
15 \cdot x 
\end{equation}
Any self-respecting mathematician would find it obvious, that the derivative of this is equal to
\begin{equation}
0 \cdot x + 15 \cdot 1 
\end{equation}

We are going to study the following:
\begin{equation}
15 \cdot x 
\end{equation}
It is now obvious, that the derivative of this is equal to
\begin{equation}
0 \cdot x + 15 \cdot 1 
\end{equation}

We are going to study the following:
\begin{equation}
2 
\end{equation}
Unsurprisingly, the derivative of this is equal to
\begin{equation}
0 
\end{equation}

Let us take a look at this:
\begin{equation}
3 
\end{equation}
It can be easily proved, that the derivative of this is equal to
\begin{equation}
0 
\end{equation}

Let us take a look at this:
\begin{equation}
15 
\end{equation}
It is now obvious, that the derivative of this is equal to
\begin{equation}
0 
\end{equation}

Consider the following:
\begin{equation}
15 
\end{equation}
Unsurprisingly, the derivative of this is equal to
\begin{equation}
0 
\end{equation}

We will allow ourselves to divert the reader's attention to this gem of mathematical wonder:
\begin{equation}
15 \cdot x 
\end{equation}
Obviously, the derivative of this is equal to
\begin{equation}
0 \cdot x + 15 \cdot 1 
\end{equation}

The object of our ultimate interest is the following:
\begin{equation}
15 \cdot x 
\end{equation}
Obviously, the derivative of this is equal to
\begin{equation}
0 \cdot x + 15 \cdot 1 
\end{equation}

Let us take a look at this:
\begin{equation}
3 
\end{equation}
Clearly, the derivative of this is equal to
\begin{equation}
0 
\end{equation}

Let us take a look at this:
\begin{equation}
4 
\end{equation}
Unsurprisingly, the derivative of this is equal to
\begin{equation}
0 
\end{equation}

Consider the following:
\begin{equation}
15 
\end{equation}
It is now obvious, that the derivative of this is equal to
\begin{equation}
0 
\end{equation}

We are going to study the following:
\begin{equation}
15 \cdot x 
\end{equation}
Clearly, the derivative of this is equal to
\begin{equation}
0 \cdot x + 15 \cdot 1 
\end{equation}

The object of our ultimate interest is the following:
\begin{equation}
15 
\end{equation}
It can be easily proved, that the derivative of this is equal to
\begin{equation}
0 
\end{equation}

We will take a closer look at this:
\begin{equation}
15 
\end{equation}
Unsurprisingly, the derivative of this is equal to
\begin{equation}
0 
\end{equation}

We will allow ourselves to divert the reader's attention to this gem of mathematical wonder:
\begin{equation}
15 \cdot x 
\end{equation}
It is now obvious, that the derivative of this is equal to
\begin{equation}
0 \cdot x + 15 \cdot 1 
\end{equation}

Let us take a look at this:
\begin{equation}
15 
\end{equation}
Unsurprisingly, the derivative of this is equal to
\begin{equation}
0 
\end{equation}

Consider the following:
\begin{equation}
15 \cdot x 
\end{equation}
Obviously, the derivative of this is equal to
\begin{equation}
0 \cdot x + 15 \cdot 1 
\end{equation}

Consider the following:
\begin{equation}
15 \cdot x 
\end{equation}
Unsurprisingly, the derivative of this is equal to
\begin{equation}
0 \cdot x + 15 \cdot 1 
\end{equation}

One shall regard the object in question with utmost interest:
\begin{equation}
2 
\end{equation}
Trivially, the derivative of this is equal to
\begin{equation}
0 
\end{equation}

We will take a closer look at this:
\begin{equation}
3 
\end{equation}
It can be easily proved, that the derivative of this is equal to
\begin{equation}
0 
\end{equation}

We shall ponder the following:
\begin{equation}
15 
\end{equation}
Trivially, the derivative of this is equal to
\begin{equation}
0 
\end{equation}

We will allow ourselves to divert the reader's attention to this gem of mathematical wonder:
\begin{equation}
15 \cdot x 
\end{equation}
Clearly, the derivative of this is equal to
\begin{equation}
0 \cdot x + 15 \cdot 1 
\end{equation}

We will allow ourselves to divert the reader's attention to this gem of mathematical wonder:
\begin{equation}
15 
\end{equation}
As you can see, the derivative of this is equal to
\begin{equation}
0 
\end{equation}

We will take a closer look at this:
\begin{equation}
15 \cdot x 
\end{equation}
It can be easily proved, that the derivative of this is equal to
\begin{equation}
0 \cdot x + 15 \cdot 1 
\end{equation}

The following is worth a closer look:
\begin{equation}
15 
\end{equation}
As you can see, the derivative of this is equal to
\begin{equation}
0 
\end{equation}

We are going to study the following:
\begin{equation}
15 \cdot x 
\end{equation}
Any self-respecting mathematician would find it obvious, that the derivative of this is equal to
\begin{equation}
0 \cdot x + 15 \cdot 1 
\end{equation}

We shall ponder the following:
\begin{equation}
2 
\end{equation}
Unsurprisingly, the derivative of this is equal to
\begin{equation}
0 
\end{equation}

Let us take a look at this:
\begin{equation}
15 
\end{equation}
It is now obvious, that the derivative of this is equal to
\begin{equation}
0 
\end{equation}

One shall regard the object in question with utmost interest:
\begin{equation}
15 
\end{equation}
Obviously, the derivative of this is equal to
\begin{equation}
0 
\end{equation}

We will take a closer look at this:
\begin{equation}
15 \cdot x 
\end{equation}
Trivially, the derivative of this is equal to
\begin{equation}
0 \cdot x + 15 \cdot 1 
\end{equation}

The following is worth a closer look:
\begin{equation}
15 \cdot x 
\end{equation}
Any self-respecting mathematician would find it obvious, that the derivative of this is equal to
\begin{equation}
0 \cdot x + 15 \cdot 1 
\end{equation}

We will take a closer look at this:
\begin{equation}
2 
\end{equation}
It is now obvious, that the derivative of this is equal to
\begin{equation}
0 
\end{equation}

The object of our ultimate interest is the following:
\begin{equation}
3 
\end{equation}
Obviously, the derivative of this is equal to
\begin{equation}
0 
\end{equation}

Consider the following:
\begin{equation}
15 
\end{equation}
Trivially, the derivative of this is equal to
\begin{equation}
0 
\end{equation}

Consider the following:
\begin{equation}
15 
\end{equation}
Any self-respecting mathematician would find it obvious, that the derivative of this is equal to
\begin{equation}
0 
\end{equation}

We shall ponder the following:
\begin{equation}
15 
\end{equation}
It is now obvious, that the derivative of this is equal to
\begin{equation}
0 
\end{equation}

We are going to study the following:
\begin{equation}
15 \cdot x 
\end{equation}
Trivially, the derivative of this is equal to
\begin{equation}
0 \cdot x + 15 \cdot 1 
\end{equation}

Consider the following:
\begin{equation}
15 \cdot x 
\end{equation}
Clearly, the derivative of this is equal to
\begin{equation}
0 \cdot x + 15 \cdot 1 
\end{equation}

We shall ponder the following:
\begin{equation}
3 
\end{equation}
Clearly, the derivative of this is equal to
\begin{equation}
0 
\end{equation}

The following is worth a closer look:
\begin{equation}
15 
\end{equation}
Unsurprisingly, the derivative of this is equal to
\begin{equation}
0 
\end{equation}

The object of our ultimate interest is the following:
\begin{equation}
15 
\end{equation}
It can be easily proved, that the derivative of this is equal to
\begin{equation}
0 
\end{equation}

One shall regard the object in question with utmost interest:
\begin{equation}
15 \cdot x 
\end{equation}
Any self-respecting mathematician would find it obvious, that the derivative of this is equal to
\begin{equation}
0 \cdot x + 15 \cdot 1 
\end{equation}

We shall ponder the following:
\begin{equation}
15 
\end{equation}
It is now obvious, that the derivative of this is equal to
\begin{equation}
0 
\end{equation}

We are going to study the following:
\begin{equation}
15 \cdot x 
\end{equation}
Unsurprisingly, the derivative of this is equal to
\begin{equation}
0 \cdot x + 15 \cdot 1 
\end{equation}

The object of our ultimate interest is the following:
\begin{equation}
15 \cdot x 
\end{equation}
Obviously, the derivative of this is equal to
\begin{equation}
0 \cdot x + 15 \cdot 1 
\end{equation}

We are going to study the following:
\begin{equation}
2 
\end{equation}
Clearly, the derivative of this is equal to
\begin{equation}
0 
\end{equation}

Let us take a look at this:
\begin{equation}
3 
\end{equation}
Clearly, the derivative of this is equal to
\begin{equation}
0 
\end{equation}

The following is worth a closer look:
\begin{equation}
4 
\end{equation}
Obviously, the derivative of this is equal to
\begin{equation}
0 
\end{equation}

We will allow ourselves to divert the reader's attention to this gem of mathematical wonder:
\begin{equation}
15 
\end{equation}
Clearly, the derivative of this is equal to
\begin{equation}
0 
\end{equation}

Let us take a look at this:
\begin{equation}
15 
\end{equation}
It can be easily proved, that the derivative of this is equal to
\begin{equation}
0 
\end{equation}

We will take a closer look at this:
\begin{equation}
15 \cdot x 
\end{equation}
Trivially, the derivative of this is equal to
\begin{equation}
0 \cdot x + 15 \cdot 1 
\end{equation}

We are going to study the following:
\begin{equation}
15 
\end{equation}
Any self-respecting mathematician would find it obvious, that the derivative of this is equal to
\begin{equation}
0 
\end{equation}

Consider the following:
\begin{equation}
15 \cdot x 
\end{equation}
Any self-respecting mathematician would find it obvious, that the derivative of this is equal to
\begin{equation}
0 \cdot x + 15 \cdot 1 
\end{equation}

Let us take a look at this:
\begin{equation}
15 
\end{equation}
It can be easily proved, that the derivative of this is equal to
\begin{equation}
0 
\end{equation}

The object of our ultimate interest is the following:
\begin{equation}
15 \cdot x 
\end{equation}
As you can see, the derivative of this is equal to
\begin{equation}
0 \cdot x + 15 \cdot 1 
\end{equation}

We shall ponder the following:
\begin{equation}
15 \cdot x 
\end{equation}
Trivially, the derivative of this is equal to
\begin{equation}
0 \cdot x + 15 \cdot 1 
\end{equation}

The object of our ultimate interest is the following:
\begin{equation}
2 
\end{equation}
It can be easily proved, that the derivative of this is equal to
\begin{equation}
0 
\end{equation}

We will allow ourselves to divert the reader's attention to this gem of mathematical wonder:
\begin{equation}
3 
\end{equation}
Any self-respecting mathematician would find it obvious, that the derivative of this is equal to
\begin{equation}
0 
\end{equation}

The object of our ultimate interest is the following:
\begin{equation}
15 
\end{equation}
Obviously, the derivative of this is equal to
\begin{equation}
0 
\end{equation}

Consider the following:
\begin{equation}
15 
\end{equation}
Unsurprisingly, the derivative of this is equal to
\begin{equation}
0 
\end{equation}

The object of our ultimate interest is the following:
\begin{equation}
15 \cdot x 
\end{equation}
Any self-respecting mathematician would find it obvious, that the derivative of this is equal to
\begin{equation}
0 \cdot x + 15 \cdot 1 
\end{equation}

The following is worth a closer look:
\begin{equation}
15 \cdot x 
\end{equation}
Trivially, the derivative of this is equal to
\begin{equation}
0 \cdot x + 15 \cdot 1 
\end{equation}

One shall regard the object in question with utmost interest:
\begin{equation}
3 
\end{equation}
Obviously, the derivative of this is equal to
\begin{equation}
0 
\end{equation}

We will allow ourselves to divert the reader's attention to this gem of mathematical wonder:
\begin{equation}
4 
\end{equation}
Unsurprisingly, the derivative of this is equal to
\begin{equation}
0 
\end{equation}

Consider the following:
\begin{equation}
15 
\end{equation}
It is now obvious, that the derivative of this is equal to
\begin{equation}
0 
\end{equation}

Let us take a look at this:
\begin{equation}
15 
\end{equation}
It can be easily proved, that the derivative of this is equal to
\begin{equation}
0 
\end{equation}

The object of our ultimate interest is the following:
\begin{equation}
15 
\end{equation}
Unsurprisingly, the derivative of this is equal to
\begin{equation}
0 
\end{equation}

We are going to study the following:
\begin{equation}
15 \cdot x 
\end{equation}
Clearly, the derivative of this is equal to
\begin{equation}
0 \cdot x + 15 \cdot 1 
\end{equation}

Consider the following:
\begin{equation}
15 
\end{equation}
Unsurprisingly, the derivative of this is equal to
\begin{equation}
0 
\end{equation}

We will take a closer look at this:
\begin{equation}
15 \cdot x 
\end{equation}
Trivially, the derivative of this is equal to
\begin{equation}
0 \cdot x + 15 \cdot 1 
\end{equation}

The following is worth a closer look:
\begin{equation}
15 \cdot x 
\end{equation}
Clearly, the derivative of this is equal to
\begin{equation}
0 \cdot x + 15 \cdot 1 
\end{equation}

We will take a closer look at this:
\begin{equation}
3 
\end{equation}
Obviously, the derivative of this is equal to
\begin{equation}
0 
\end{equation}

The object of our ultimate interest is the following:
\begin{equation}
4 
\end{equation}
Unsurprisingly, the derivative of this is equal to
\begin{equation}
0 
\end{equation}

Consider the following:
\begin{equation}
15 
\end{equation}
Clearly, the derivative of this is equal to
\begin{equation}
0 
\end{equation}

The following is worth a closer look:
\begin{equation}
15 
\end{equation}
As you can see, the derivative of this is equal to
\begin{equation}
0 
\end{equation}

The object of our ultimate interest is the following:
\begin{equation}
15 
\end{equation}
Trivially, the derivative of this is equal to
\begin{equation}
0 
\end{equation}

Consider the following:
\begin{equation}
15 
\end{equation}
It is now obvious, that the derivative of this is equal to
\begin{equation}
0 
\end{equation}

Consider the following:
\begin{equation}
15 \cdot x 
\end{equation}
As you can see, the derivative of this is equal to
\begin{equation}
0 \cdot x + 15 \cdot 1 
\end{equation}

The object of our ultimate interest is the following:
\begin{equation}
15 \cdot x 
\end{equation}
Trivially, the derivative of this is equal to
\begin{equation}
0 \cdot x + 15 \cdot 1 
\end{equation}

One shall regard the object in question with utmost interest:
\begin{equation}
4 
\end{equation}
Trivially, the derivative of this is equal to
\begin{equation}
0 
\end{equation}

The following is worth a closer look:
\begin{equation}
15 
\end{equation}
Trivially, the derivative of this is equal to
\begin{equation}
0 
\end{equation}

We shall ponder the following:
\begin{equation}
15 
\end{equation}
It can be easily proved, that the derivative of this is equal to
\begin{equation}
0 
\end{equation}

Let us take a look at this:
\begin{equation}
15 
\end{equation}
It is now obvious, that the derivative of this is equal to
\begin{equation}
0 
\end{equation}

One shall regard the object in question with utmost interest:
\begin{equation}
15 \cdot x 
\end{equation}
Any self-respecting mathematician would find it obvious, that the derivative of this is equal to
\begin{equation}
0 \cdot x + 15 \cdot 1 
\end{equation}

We will allow ourselves to divert the reader's attention to this gem of mathematical wonder:
\begin{equation}
15 
\end{equation}
Trivially, the derivative of this is equal to
\begin{equation}
0 
\end{equation}

The following is worth a closer look:
\begin{equation}
15 \cdot x 
\end{equation}
Trivially, the derivative of this is equal to
\begin{equation}
0 \cdot x + 15 \cdot 1 
\end{equation}

Let us take a look at this:
\begin{equation}
15 \cdot x 
\end{equation}
Trivially, the derivative of this is equal to
\begin{equation}
0 \cdot x + 15 \cdot 1 
\end{equation}

The object of our ultimate interest is the following:
\begin{equation}
3 
\end{equation}
As you can see, the derivative of this is equal to
\begin{equation}
0 
\end{equation}

Consider the following:
\begin{equation}
4 
\end{equation}
Trivially, the derivative of this is equal to
\begin{equation}
0 
\end{equation}

We are going to study the following:
\begin{equation}
15 
\end{equation}
Any self-respecting mathematician would find it obvious, that the derivative of this is equal to
\begin{equation}
0 
\end{equation}

We shall ponder the following:
\begin{equation}
15 
\end{equation}
Clearly, the derivative of this is equal to
\begin{equation}
0 
\end{equation}

The following is worth a closer look:
\begin{equation}
15 \cdot x 
\end{equation}
Obviously, the derivative of this is equal to
\begin{equation}
0 \cdot x + 15 \cdot 1 
\end{equation}

We will take a closer look at this:
\begin{equation}
15 
\end{equation}
Unsurprisingly, the derivative of this is equal to
\begin{equation}
0 
\end{equation}

We shall ponder the following:
\begin{equation}
15 \cdot x 
\end{equation}
Trivially, the derivative of this is equal to
\begin{equation}
0 \cdot x + 15 \cdot 1 
\end{equation}

We will allow ourselves to divert the reader's attention to this gem of mathematical wonder:
\begin{equation}
15 
\end{equation}
Unsurprisingly, the derivative of this is equal to
\begin{equation}
0 
\end{equation}

Consider the following:
\begin{equation}
15 \cdot x 
\end{equation}
It can be easily proved, that the derivative of this is equal to
\begin{equation}
0 \cdot x + 15 \cdot 1 
\end{equation}

We are going to study the following:
\begin{equation}
15 \cdot x 
\end{equation}
Clearly, the derivative of this is equal to
\begin{equation}
0 \cdot x + 15 \cdot 1 
\end{equation}

The object of our ultimate interest is the following:
\begin{equation}
2 
\end{equation}
Any self-respecting mathematician would find it obvious, that the derivative of this is equal to
\begin{equation}
0 
\end{equation}

Let us take a look at this:
\begin{equation}
3 
\end{equation}
It is now obvious, that the derivative of this is equal to
\begin{equation}
0 
\end{equation}

We shall ponder the following:
\begin{equation}
15 
\end{equation}
Obviously, the derivative of this is equal to
\begin{equation}
0 
\end{equation}

We will take a closer look at this:
\begin{equation}
15 
\end{equation}
Any self-respecting mathematician would find it obvious, that the derivative of this is equal to
\begin{equation}
0 
\end{equation}

We will allow ourselves to divert the reader's attention to this gem of mathematical wonder:
\begin{equation}
15 \cdot x 
\end{equation}
Clearly, the derivative of this is equal to
\begin{equation}
0 \cdot x + 15 \cdot 1 
\end{equation}

We shall ponder the following:
\begin{equation}
15 \cdot x 
\end{equation}
Trivially, the derivative of this is equal to
\begin{equation}
0 \cdot x + 15 \cdot 1 
\end{equation}

We shall ponder the following:
\begin{equation}
3 
\end{equation}
It is now obvious, that the derivative of this is equal to
\begin{equation}
0 
\end{equation}

Consider the following:
\begin{equation}
4 
\end{equation}
As you can see, the derivative of this is equal to
\begin{equation}
0 
\end{equation}

Now the proof that the Taylor series of this function at $x = 0$ is equal to
\begin{equation}
A + 6 \cdot \frac{\left( x - 0 \right) ^{3 } }{6 } + 2.025e+06 \cdot \frac{\left( x - 0 \right) ^{4 } }{24 } 
\end{equation}
Where:
\begin{itemize}
	\item $A = 0 + 1 \cdot \frac{\left( x - 0 \right) ^{0 } }{1 } + 0 \cdot \frac{\left( x - 0 \right) ^{1 } }{1 } + -900 \cdot \frac{\left( x - 0 \right) ^{2 } }{2 } $
\end{itemize}

shall be considered an amusing exercise for the reader.
It is now obvious, that if we simplify this we wil get
\begin{equation}
1 - 900 \cdot \frac{x ^{2 } }{2 } + 6 \cdot \frac{x ^{3 } }{6 } + 2.025e+06 \cdot \frac{x ^{4 } }{24 } 
\end{equation}
\newpage
\section{Tangent}
Let us find the Taylor series at $x = 0$ of the following function:
\begin{equation}
\sin x ^{3 } + \left( \cos 15 \cdot x \right) ^{4 } 
\end{equation}

We are going to study the following:
\begin{equation}
15 \cdot x 
\end{equation}
Any self-respecting mathematician would find it obvious, that the derivative of this is equal to
\begin{equation}
0 \cdot x + 15 \cdot 1 
\end{equation}

We are going to study the following:
\begin{equation}
x ^{3 } 
\end{equation}
It is now obvious, that the derivative of this is equal to
\begin{equation}
3 \cdot x ^{3 - 1 } \cdot 1 
\end{equation}

We will take a closer look at this:
\begin{equation}
15 
\end{equation}
Obviously, the derivative of this is equal to
\begin{equation}
0 
\end{equation}

We will allow ourselves to divert the reader's attention to this gem of mathematical wonder:
\begin{equation}
15 \cdot x 
\end{equation}
It can be easily proved, that the derivative of this is equal to
\begin{equation}
0 \cdot x + 15 \cdot 1 
\end{equation}

We are going to study the following:
\begin{equation}
15 \cdot x 
\end{equation}
Unsurprisingly, the derivative of this is equal to
\begin{equation}
0 \cdot x + 15 \cdot 1 
\end{equation}

The object of our ultimate interest is the following:
\begin{equation}
4 
\end{equation}
Trivially, the derivative of this is equal to
\begin{equation}
0 
\end{equation}

We will take a closer look at this:
\begin{equation}
x ^{2 } 
\end{equation}
Unsurprisingly, the derivative of this is equal to
\begin{equation}
2 \cdot x ^{2 - 1 } \cdot 1 
\end{equation}

We are going to study the following:
\begin{equation}
3 
\end{equation}
It is now obvious, that the derivative of this is equal to
\begin{equation}
0 
\end{equation}

We shall ponder the following:
\begin{equation}
x ^{3 } 
\end{equation}
Any self-respecting mathematician would find it obvious, that the derivative of this is equal to
\begin{equation}
3 \cdot x ^{3 - 1 } \cdot 1 
\end{equation}

Now the proof that the Taylor series of this function at $x = 0$ is equal to
\begin{equation}
0 + 1 \cdot \frac{\left( x - 0 \right) ^{0 } }{1 } + 0 \cdot \frac{\left( x - 0 \right) ^{1 } }{1 } 
\end{equation}
is left out for the reader to solve themselves.
Any self-respecting mathematician would find it obvious, that if we simplify this we wil get
\begin{equation}
1 
\end{equation}
\includegraphics{"plot.png"}
\end{document}
