\documentclass{article}
\usepackage{preamble}
\begin{document}
\maketitle
\begin{abstract}
Wonderful article
\end{abstract}
\newpage
\section{Derivative}
Let us find the derivative of the following function:
\begin{equation}
\left( x + 1 \right) ^{\frac{\sin x }{2 } } \cdot \left( \arctan \sqrt {x ^{2 } + 1 } \right) ^{x - 2 } 
\end{equation}

The object of our ultimate interest is the following:
\begin{equation}
1 
\end{equation}
Clearly, the derivative of this is equal to
\begin{equation}
0 
\end{equation}

We are going to study the following:
\begin{equation}
x ^{2 } 
\end{equation}
It is now obvious, that the derivative of this is equal to
\begin{equation}
2 \cdot x ^{2 - 1 } \cdot 1 
\end{equation}

One shall regard the object in question with utmost interest:
\begin{equation}
x - 2 
\end{equation}
Unsurprisingly, the derivative of this is equal to
\begin{equation}
1 - 0 
\end{equation}

One shall regard the object in question with utmost interest:
\begin{equation}
x + 1 
\end{equation}
It can be easily proved, that the derivative of this is equal to
\begin{equation}
1 + 0 
\end{equation}

The object of our ultimate interest is the following:
\begin{equation}
2 
\end{equation}
Any self-respecting mathematician would find it obvious, that the derivative of this is equal to
\begin{equation}
0 
\end{equation}

The following is worth a closer look:
\begin{equation}
\sin x 
\end{equation}
It is now obvious, that the derivative of this is equal to
\begin{equation}
\cos x \cdot 1 
\end{equation}

Now the proof that the derivative of this function is equal to
\begin{equation}
\left( x + 1 \right) ^{\frac{\sin x }{2 } } \cdot \left( A \right) \cdot \left( \arctan \sqrt {x ^{2 } + 1 } \right) ^{x - 2 } + \left( x + 1 \right) ^{\frac{\sin x }{2 } } \cdot \left( \arctan \sqrt {x ^{2 } + 1 } \right) ^{x - 2 } \cdot \left( C \right) 
\end{equation}
Where:
\begin{itemize}
	\item $A = \frac{\cos x \cdot 1 \cdot 2 - \sin x \cdot 0 }{2 ^{2 } } \cdot \ln \left( x + 1 \right) + \frac{\sin x }{2 } \cdot \frac{1 + 0 }{x + 1 } $
	\item $B = \frac{1 }{1 + \left( \sqrt {x ^{2 } + 1 } \right) ^{2 } } \cdot \frac{1 }{2 \cdot \sqrt {x ^{2 } + 1 } } \cdot \left( 2 \cdot x ^{2 - 1 } \cdot 1 + 0 \right) $
	\item $C = \left( 1 - 0 \right) \cdot \ln \arctan \sqrt {x ^{2 } + 1 } + \left( x - 2 \right) \cdot \frac{B }{\arctan \sqrt {x ^{2 } + 1 } } $
\end{itemize}

is left out for the reader to solve themselves.
Unsurprisingly, if we simplify this we wil get
\begin{equation}
A \cdot \left( \arctan \sqrt {x ^{2 } + 1 } \right) ^{x - 2 } + \left( x + 1 \right) ^{\frac{\sin x }{2 } } \cdot C 
\end{equation}
Where:
\begin{itemize}
	\item $A = \left( x + 1 \right) ^{\frac{\sin x }{2 } } \cdot \left( \frac{\cos x \cdot 2 }{4 } \cdot \ln \left( x + 1 \right) + \frac{\sin x }{2 } \cdot \frac{1 }{x + 1 } \right) $
	\item $B = \frac{1 }{1 + \left( \sqrt {x ^{2 } + 1 } \right) ^{2 } } \cdot \frac{1 }{2 \cdot \sqrt {x ^{2 } + 1 } } \cdot 2 \cdot x $
	\item $C = \left( \arctan \sqrt {x ^{2 } + 1 } \right) ^{x - 2 } \cdot \left( \ln \arctan \sqrt {x ^{2 } + 1 } + \left( x - 2 \right) \cdot \frac{B }{\arctan \sqrt {x ^{2 } + 1 } } \right) $
\end{itemize}

\newpage
\section{Taylor series}
Let us find the Taylor series at $x = 0$ of the following function:
\begin{equation}
\left( x + 1 \right) ^{\frac{\sin x }{2 } } \cdot \left( \arctan \sqrt {x ^{2 } + 1 } \right) ^{x - 2 } 
\end{equation}

We will allow ourselves to divert the reader's attentions to this gem of mathematical wonder:
\begin{equation}
1 
\end{equation}
It can be easily proved, that the derivative of this is equal to
\begin{equation}
0 
\end{equation}

The object of our ultimate interest is the following:
\begin{equation}
x ^{2 } 
\end{equation}
It is now obvious, that the derivative of this is equal to
\begin{equation}
2 \cdot x ^{2 - 1 } \cdot 1 
\end{equation}

We are going to study the following:
\begin{equation}
x - 2 
\end{equation}
As you can see, the derivative of this is equal to
\begin{equation}
1 - 0 
\end{equation}

The following is worth a closer look:
\begin{equation}
x + 1 
\end{equation}
Unsurprisingly, the derivative of this is equal to
\begin{equation}
1 + 0 
\end{equation}

We will take a closer look at this:
\begin{equation}
2 
\end{equation}
It is now obvious, that the derivative of this is equal to
\begin{equation}
0 
\end{equation}

Let us take a look at this:
\begin{equation}
\sin x 
\end{equation}
Trivially, the derivative of this is equal to
\begin{equation}
\cos x \cdot 1 
\end{equation}

Let us take a look at this:
\begin{equation}
1 
\end{equation}
Any self-respecting mathematician would find it obvious, that the derivative of this is equal to
\begin{equation}
0 
\end{equation}

One shall regard the object in question with utmost interest:
\begin{equation}
x ^{2 } 
\end{equation}
Trivially, the derivative of this is equal to
\begin{equation}
2 \cdot x ^{2 - 1 } \cdot 1 
\end{equation}

We shall ponder the following:
\begin{equation}
2 \cdot x 
\end{equation}
Any self-respecting mathematician would find it obvious, that the derivative of this is equal to
\begin{equation}
0 \cdot x + 2 \cdot 1 
\end{equation}

The object of our ultimate interest is the following:
\begin{equation}
1 
\end{equation}
It is now obvious, that the derivative of this is equal to
\begin{equation}
0 
\end{equation}

The object of our ultimate interest is the following:
\begin{equation}
x ^{2 } 
\end{equation}
It is now obvious, that the derivative of this is equal to
\begin{equation}
2 \cdot x ^{2 - 1 } \cdot 1 
\end{equation}

We shall ponder the following:
\begin{equation}
2 
\end{equation}
Any self-respecting mathematician would find it obvious, that the derivative of this is equal to
\begin{equation}
0 
\end{equation}

We shall ponder the following:
\begin{equation}
1 
\end{equation}
Any self-respecting mathematician would find it obvious, that the derivative of this is equal to
\begin{equation}
0 
\end{equation}

We will allow ourselves to divert the reader's attentions to this gem of mathematical wonder:
\begin{equation}
1 
\end{equation}
Obviously, the derivative of this is equal to
\begin{equation}
0 
\end{equation}

We shall ponder the following:
\begin{equation}
x ^{2 } 
\end{equation}
It is now obvious, that the derivative of this is equal to
\begin{equation}
2 \cdot x ^{2 - 1 } \cdot 1 
\end{equation}

Let us take a look at this:
\begin{equation}
1 
\end{equation}
Obviously, the derivative of this is equal to
\begin{equation}
0 
\end{equation}

We are going to study the following:
\begin{equation}
1 
\end{equation}
It is now obvious, that the derivative of this is equal to
\begin{equation}
0 
\end{equation}

The object of our ultimate interest is the following:
\begin{equation}
x - 2 
\end{equation}
Obviously, the derivative of this is equal to
\begin{equation}
1 - 0 
\end{equation}

Let us take a look at this:
\begin{equation}
1 
\end{equation}
Obviously, the derivative of this is equal to
\begin{equation}
0 
\end{equation}

Consider the following:
\begin{equation}
x ^{2 } 
\end{equation}
As you can see, the derivative of this is equal to
\begin{equation}
2 \cdot x ^{2 - 1 } \cdot 1 
\end{equation}

Let us take a look at this:
\begin{equation}
1 
\end{equation}
As you can see, the derivative of this is equal to
\begin{equation}
0 
\end{equation}

We are going to study the following:
\begin{equation}
x ^{2 } 
\end{equation}
As you can see, the derivative of this is equal to
\begin{equation}
2 \cdot x ^{2 - 1 } \cdot 1 
\end{equation}

We will allow ourselves to divert the reader's attentions to this gem of mathematical wonder:
\begin{equation}
x - 2 
\end{equation}
As you can see, the derivative of this is equal to
\begin{equation}
1 - 0 
\end{equation}

The object of our ultimate interest is the following:
\begin{equation}
x + 1 
\end{equation}
Clearly, the derivative of this is equal to
\begin{equation}
1 + 0 
\end{equation}

Let us take a look at this:
\begin{equation}
2 
\end{equation}
Clearly, the derivative of this is equal to
\begin{equation}
0 
\end{equation}

Let us take a look at this:
\begin{equation}
\sin x 
\end{equation}
As you can see, the derivative of this is equal to
\begin{equation}
\cos x \cdot 1 
\end{equation}

We will take a closer look at this:
\begin{equation}
1 
\end{equation}
Unsurprisingly, the derivative of this is equal to
\begin{equation}
0 
\end{equation}

The following is worth a closer look:
\begin{equation}
x ^{2 } 
\end{equation}
Any self-respecting mathematician would find it obvious, that the derivative of this is equal to
\begin{equation}
2 \cdot x ^{2 - 1 } \cdot 1 
\end{equation}

We shall ponder the following:
\begin{equation}
x - 2 
\end{equation}
Clearly, the derivative of this is equal to
\begin{equation}
1 - 0 
\end{equation}

One shall regard the object in question with utmost interest:
\begin{equation}
x + 1 
\end{equation}
It can be easily proved, that the derivative of this is equal to
\begin{equation}
1 + 0 
\end{equation}

The following is worth a closer look:
\begin{equation}
1 
\end{equation}
Unsurprisingly, the derivative of this is equal to
\begin{equation}
0 
\end{equation}

We shall ponder the following:
\begin{equation}
2 
\end{equation}
It is now obvious, that the derivative of this is equal to
\begin{equation}
0 
\end{equation}

We will take a closer look at this:
\begin{equation}
\sin x 
\end{equation}
Trivially, the derivative of this is equal to
\begin{equation}
\cos x \cdot 1 
\end{equation}

We shall ponder the following:
\begin{equation}
x + 1 
\end{equation}
Unsurprisingly, the derivative of this is equal to
\begin{equation}
1 + 0 
\end{equation}

We will allow ourselves to divert the reader's attentions to this gem of mathematical wonder:
\begin{equation}
4 
\end{equation}
Unsurprisingly, the derivative of this is equal to
\begin{equation}
0 
\end{equation}

Consider the following:
\begin{equation}
2 
\end{equation}
It is now obvious, that the derivative of this is equal to
\begin{equation}
0 
\end{equation}

We shall ponder the following:
\begin{equation}
\cos x 
\end{equation}
Unsurprisingly, the derivative of this is equal to
\begin{equation}
-\sin x \cdot 1 
\end{equation}

We will allow ourselves to divert the reader's attentions to this gem of mathematical wonder:
\begin{equation}
x + 1 
\end{equation}
Obviously, the derivative of this is equal to
\begin{equation}
1 + 0 
\end{equation}

We are going to study the following:
\begin{equation}
2 
\end{equation}
It can be easily proved, that the derivative of this is equal to
\begin{equation}
0 
\end{equation}

Consider the following:
\begin{equation}
\sin x 
\end{equation}
It is now obvious, that the derivative of this is equal to
\begin{equation}
\cos x \cdot 1 
\end{equation}

We shall ponder the following:
\begin{equation}
1 
\end{equation}
Clearly, the derivative of this is equal to
\begin{equation}
0 
\end{equation}

We will allow ourselves to divert the reader's attentions to this gem of mathematical wonder:
\begin{equation}
x ^{2 } 
\end{equation}
Obviously, the derivative of this is equal to
\begin{equation}
2 \cdot x ^{2 - 1 } \cdot 1 
\end{equation}

We are going to study the following:
\begin{equation}
2 \cdot x 
\end{equation}
Any self-respecting mathematician would find it obvious, that the derivative of this is equal to
\begin{equation}
0 \cdot x + 2 \cdot 1 
\end{equation}

One shall regard the object in question with utmost interest:
\begin{equation}
1 
\end{equation}
Any self-respecting mathematician would find it obvious, that the derivative of this is equal to
\begin{equation}
0 
\end{equation}

Consider the following:
\begin{equation}
x ^{2 } 
\end{equation}
Clearly, the derivative of this is equal to
\begin{equation}
2 \cdot x ^{2 - 1 } \cdot 1 
\end{equation}

The object of our ultimate interest is the following:
\begin{equation}
2 
\end{equation}
Trivially, the derivative of this is equal to
\begin{equation}
0 
\end{equation}

One shall regard the object in question with utmost interest:
\begin{equation}
1 
\end{equation}
Any self-respecting mathematician would find it obvious, that the derivative of this is equal to
\begin{equation}
0 
\end{equation}

Consider the following:
\begin{equation}
1 
\end{equation}
Clearly, the derivative of this is equal to
\begin{equation}
0 
\end{equation}

One shall regard the object in question with utmost interest:
\begin{equation}
x ^{2 } 
\end{equation}
Trivially, the derivative of this is equal to
\begin{equation}
2 \cdot x ^{2 - 1 } \cdot 1 
\end{equation}

The following is worth a closer look:
\begin{equation}
1 
\end{equation}
Unsurprisingly, the derivative of this is equal to
\begin{equation}
0 
\end{equation}

Consider the following:
\begin{equation}
1 
\end{equation}
Clearly, the derivative of this is equal to
\begin{equation}
0 
\end{equation}

We shall ponder the following:
\begin{equation}
2 \cdot x 
\end{equation}
Trivially, the derivative of this is equal to
\begin{equation}
0 \cdot x + 2 \cdot 1 
\end{equation}

We shall ponder the following:
\begin{equation}
1 
\end{equation}
Any self-respecting mathematician would find it obvious, that the derivative of this is equal to
\begin{equation}
0 
\end{equation}

The object of our ultimate interest is the following:
\begin{equation}
x ^{2 } 
\end{equation}
It is now obvious, that the derivative of this is equal to
\begin{equation}
2 \cdot x ^{2 - 1 } \cdot 1 
\end{equation}

We are going to study the following:
\begin{equation}
2 
\end{equation}
Obviously, the derivative of this is equal to
\begin{equation}
0 
\end{equation}

Let us take a look at this:
\begin{equation}
1 
\end{equation}
Clearly, the derivative of this is equal to
\begin{equation}
0 
\end{equation}

We are going to study the following:
\begin{equation}
1 
\end{equation}
It can be easily proved, that the derivative of this is equal to
\begin{equation}
0 
\end{equation}

We will allow ourselves to divert the reader's attentions to this gem of mathematical wonder:
\begin{equation}
x ^{2 } 
\end{equation}
Any self-respecting mathematician would find it obvious, that the derivative of this is equal to
\begin{equation}
2 \cdot x ^{2 - 1 } \cdot 1 
\end{equation}

Consider the following:
\begin{equation}
1 
\end{equation}
Trivially, the derivative of this is equal to
\begin{equation}
0 
\end{equation}

Consider the following:
\begin{equation}
1 
\end{equation}
It can be easily proved, that the derivative of this is equal to
\begin{equation}
0 
\end{equation}

One shall regard the object in question with utmost interest:
\begin{equation}
1 
\end{equation}
It is now obvious, that the derivative of this is equal to
\begin{equation}
0 
\end{equation}

We shall ponder the following:
\begin{equation}
x ^{2 } 
\end{equation}
Unsurprisingly, the derivative of this is equal to
\begin{equation}
2 \cdot x ^{2 - 1 } \cdot 1 
\end{equation}

Consider the following:
\begin{equation}
2 \cdot x 
\end{equation}
Any self-respecting mathematician would find it obvious, that the derivative of this is equal to
\begin{equation}
0 \cdot x + 2 \cdot 1 
\end{equation}

The following is worth a closer look:
\begin{equation}
1 
\end{equation}
Unsurprisingly, the derivative of this is equal to
\begin{equation}
0 
\end{equation}

We are going to study the following:
\begin{equation}
x ^{2 } 
\end{equation}
It can be easily proved, that the derivative of this is equal to
\begin{equation}
2 \cdot x ^{2 - 1 } \cdot 1 
\end{equation}

Let us take a look at this:
\begin{equation}
2 
\end{equation}
As you can see, the derivative of this is equal to
\begin{equation}
0 
\end{equation}

Let us take a look at this:
\begin{equation}
1 
\end{equation}
Any self-respecting mathematician would find it obvious, that the derivative of this is equal to
\begin{equation}
0 
\end{equation}

We shall ponder the following:
\begin{equation}
1 
\end{equation}
Any self-respecting mathematician would find it obvious, that the derivative of this is equal to
\begin{equation}
0 
\end{equation}

One shall regard the object in question with utmost interest:
\begin{equation}
x ^{2 } 
\end{equation}
Obviously, the derivative of this is equal to
\begin{equation}
2 \cdot x ^{2 - 1 } \cdot 1 
\end{equation}

The following is worth a closer look:
\begin{equation}
1 
\end{equation}
As you can see, the derivative of this is equal to
\begin{equation}
0 
\end{equation}

We will allow ourselves to divert the reader's attentions to this gem of mathematical wonder:
\begin{equation}
2 \cdot x 
\end{equation}
As you can see, the derivative of this is equal to
\begin{equation}
0 \cdot x + 2 \cdot 1 
\end{equation}

Let us take a look at this:
\begin{equation}
1 
\end{equation}
Trivially, the derivative of this is equal to
\begin{equation}
0 
\end{equation}

We shall ponder the following:
\begin{equation}
x ^{2 } 
\end{equation}
It can be easily proved, that the derivative of this is equal to
\begin{equation}
2 \cdot x ^{2 - 1 } \cdot 1 
\end{equation}

Consider the following:
\begin{equation}
2 
\end{equation}
Unsurprisingly, the derivative of this is equal to
\begin{equation}
0 
\end{equation}

We will allow ourselves to divert the reader's attentions to this gem of mathematical wonder:
\begin{equation}
1 
\end{equation}
Obviously, the derivative of this is equal to
\begin{equation}
0 
\end{equation}

The following is worth a closer look:
\begin{equation}
1 
\end{equation}
Any self-respecting mathematician would find it obvious, that the derivative of this is equal to
\begin{equation}
0 
\end{equation}

We will allow ourselves to divert the reader's attentions to this gem of mathematical wonder:
\begin{equation}
x ^{2 } 
\end{equation}
Obviously, the derivative of this is equal to
\begin{equation}
2 \cdot x ^{2 - 1 } \cdot 1 
\end{equation}

We will allow ourselves to divert the reader's attentions to this gem of mathematical wonder:
\begin{equation}
2 
\end{equation}
It is now obvious, that the derivative of this is equal to
\begin{equation}
0 
\end{equation}

We shall ponder the following:
\begin{equation}
2 \cdot x 
\end{equation}
Trivially, the derivative of this is equal to
\begin{equation}
0 \cdot x + 2 \cdot 1 
\end{equation}

One shall regard the object in question with utmost interest:
\begin{equation}
1 
\end{equation}
Unsurprisingly, the derivative of this is equal to
\begin{equation}
0 
\end{equation}

We shall ponder the following:
\begin{equation}
x ^{2 } 
\end{equation}
Trivially, the derivative of this is equal to
\begin{equation}
2 \cdot x ^{2 - 1 } \cdot 1 
\end{equation}

Consider the following:
\begin{equation}
2 
\end{equation}
It can be easily proved, that the derivative of this is equal to
\begin{equation}
0 
\end{equation}

Consider the following:
\begin{equation}
2 \cdot x 
\end{equation}
Any self-respecting mathematician would find it obvious, that the derivative of this is equal to
\begin{equation}
0 \cdot x + 2 \cdot 1 
\end{equation}

The object of our ultimate interest is the following:
\begin{equation}
1 
\end{equation}
Clearly, the derivative of this is equal to
\begin{equation}
0 
\end{equation}

One shall regard the object in question with utmost interest:
\begin{equation}
x ^{2 } 
\end{equation}
Trivially, the derivative of this is equal to
\begin{equation}
2 \cdot x ^{2 - 1 } \cdot 1 
\end{equation}

We shall ponder the following:
\begin{equation}
2 
\end{equation}
Any self-respecting mathematician would find it obvious, that the derivative of this is equal to
\begin{equation}
0 
\end{equation}

The following is worth a closer look:
\begin{equation}
1 
\end{equation}
As you can see, the derivative of this is equal to
\begin{equation}
0 
\end{equation}

We are going to study the following:
\begin{equation}
2 
\end{equation}
It can be easily proved, that the derivative of this is equal to
\begin{equation}
0 
\end{equation}

We shall ponder the following:
\begin{equation}
2 
\end{equation}
Clearly, the derivative of this is equal to
\begin{equation}
0 
\end{equation}

Consider the following:
\begin{equation}
1 
\end{equation}
It can be easily proved, that the derivative of this is equal to
\begin{equation}
0 
\end{equation}

We will take a closer look at this:
\begin{equation}
x ^{2 } 
\end{equation}
Any self-respecting mathematician would find it obvious, that the derivative of this is equal to
\begin{equation}
2 \cdot x ^{2 - 1 } \cdot 1 
\end{equation}

We shall ponder the following:
\begin{equation}
2 
\end{equation}
Obviously, the derivative of this is equal to
\begin{equation}
0 
\end{equation}

The object of our ultimate interest is the following:
\begin{equation}
1 
\end{equation}
It is now obvious, that the derivative of this is equal to
\begin{equation}
0 
\end{equation}

The object of our ultimate interest is the following:
\begin{equation}
1 
\end{equation}
Unsurprisingly, the derivative of this is equal to
\begin{equation}
0 
\end{equation}

We are going to study the following:
\begin{equation}
x ^{2 } 
\end{equation}
Clearly, the derivative of this is equal to
\begin{equation}
2 \cdot x ^{2 - 1 } \cdot 1 
\end{equation}

We will take a closer look at this:
\begin{equation}
1 
\end{equation}
It is now obvious, that the derivative of this is equal to
\begin{equation}
0 
\end{equation}

Consider the following:
\begin{equation}
1 
\end{equation}
Obviously, the derivative of this is equal to
\begin{equation}
0 
\end{equation}

Let us take a look at this:
\begin{equation}
x - 2 
\end{equation}
Any self-respecting mathematician would find it obvious, that the derivative of this is equal to
\begin{equation}
1 - 0 
\end{equation}

Let us take a look at this:
\begin{equation}
1 
\end{equation}
Any self-respecting mathematician would find it obvious, that the derivative of this is equal to
\begin{equation}
0 
\end{equation}

The object of our ultimate interest is the following:
\begin{equation}
x ^{2 } 
\end{equation}
Unsurprisingly, the derivative of this is equal to
\begin{equation}
2 \cdot x ^{2 - 1 } \cdot 1 
\end{equation}

The following is worth a closer look:
\begin{equation}
2 \cdot x 
\end{equation}
As you can see, the derivative of this is equal to
\begin{equation}
0 \cdot x + 2 \cdot 1 
\end{equation}

The object of our ultimate interest is the following:
\begin{equation}
1 
\end{equation}
As you can see, the derivative of this is equal to
\begin{equation}
0 
\end{equation}

We are going to study the following:
\begin{equation}
x ^{2 } 
\end{equation}
It is now obvious, that the derivative of this is equal to
\begin{equation}
2 \cdot x ^{2 - 1 } \cdot 1 
\end{equation}

The following is worth a closer look:
\begin{equation}
2 
\end{equation}
Any self-respecting mathematician would find it obvious, that the derivative of this is equal to
\begin{equation}
0 
\end{equation}

We will allow ourselves to divert the reader's attentions to this gem of mathematical wonder:
\begin{equation}
1 
\end{equation}
Trivially, the derivative of this is equal to
\begin{equation}
0 
\end{equation}

The following is worth a closer look:
\begin{equation}
1 
\end{equation}
Obviously, the derivative of this is equal to
\begin{equation}
0 
\end{equation}

We will allow ourselves to divert the reader's attentions to this gem of mathematical wonder:
\begin{equation}
x ^{2 } 
\end{equation}
Clearly, the derivative of this is equal to
\begin{equation}
2 \cdot x ^{2 - 1 } \cdot 1 
\end{equation}

We will allow ourselves to divert the reader's attentions to this gem of mathematical wonder:
\begin{equation}
1 
\end{equation}
As you can see, the derivative of this is equal to
\begin{equation}
0 
\end{equation}

We will take a closer look at this:
\begin{equation}
1 
\end{equation}
Unsurprisingly, the derivative of this is equal to
\begin{equation}
0 
\end{equation}

Consider the following:
\begin{equation}
2 \cdot x 
\end{equation}
Trivially, the derivative of this is equal to
\begin{equation}
0 \cdot x + 2 \cdot 1 
\end{equation}

The object of our ultimate interest is the following:
\begin{equation}
1 
\end{equation}
Obviously, the derivative of this is equal to
\begin{equation}
0 
\end{equation}

Let us take a look at this:
\begin{equation}
x ^{2 } 
\end{equation}
Any self-respecting mathematician would find it obvious, that the derivative of this is equal to
\begin{equation}
2 \cdot x ^{2 - 1 } \cdot 1 
\end{equation}

Consider the following:
\begin{equation}
2 
\end{equation}
Clearly, the derivative of this is equal to
\begin{equation}
0 
\end{equation}

The object of our ultimate interest is the following:
\begin{equation}
1 
\end{equation}
It can be easily proved, that the derivative of this is equal to
\begin{equation}
0 
\end{equation}

We will take a closer look at this:
\begin{equation}
1 
\end{equation}
Any self-respecting mathematician would find it obvious, that the derivative of this is equal to
\begin{equation}
0 
\end{equation}

We will take a closer look at this:
\begin{equation}
x ^{2 } 
\end{equation}
Any self-respecting mathematician would find it obvious, that the derivative of this is equal to
\begin{equation}
2 \cdot x ^{2 - 1 } \cdot 1 
\end{equation}

The following is worth a closer look:
\begin{equation}
1 
\end{equation}
It is now obvious, that the derivative of this is equal to
\begin{equation}
0 
\end{equation}

Let us take a look at this:
\begin{equation}
1 
\end{equation}
Unsurprisingly, the derivative of this is equal to
\begin{equation}
0 
\end{equation}

We will take a closer look at this:
\begin{equation}
1 
\end{equation}
It can be easily proved, that the derivative of this is equal to
\begin{equation}
0 
\end{equation}

The following is worth a closer look:
\begin{equation}
x ^{2 } 
\end{equation}
Obviously, the derivative of this is equal to
\begin{equation}
2 \cdot x ^{2 - 1 } \cdot 1 
\end{equation}

We shall ponder the following:
\begin{equation}
1 
\end{equation}
As you can see, the derivative of this is equal to
\begin{equation}
0 
\end{equation}

We will take a closer look at this:
\begin{equation}
1 
\end{equation}
Trivially, the derivative of this is equal to
\begin{equation}
0 
\end{equation}

We will take a closer look at this:
\begin{equation}
x ^{2 } 
\end{equation}
It is now obvious, that the derivative of this is equal to
\begin{equation}
2 \cdot x ^{2 - 1 } \cdot 1 
\end{equation}

Consider the following:
\begin{equation}
x - 2 
\end{equation}
Trivially, the derivative of this is equal to
\begin{equation}
1 - 0 
\end{equation}

Let us take a look at this:
\begin{equation}
1 
\end{equation}
Clearly, the derivative of this is equal to
\begin{equation}
0 
\end{equation}

We will allow ourselves to divert the reader's attentions to this gem of mathematical wonder:
\begin{equation}
x ^{2 } 
\end{equation}
Unsurprisingly, the derivative of this is equal to
\begin{equation}
2 \cdot x ^{2 - 1 } \cdot 1 
\end{equation}

We are going to study the following:
\begin{equation}
2 \cdot x 
\end{equation}
Obviously, the derivative of this is equal to
\begin{equation}
0 \cdot x + 2 \cdot 1 
\end{equation}

Consider the following:
\begin{equation}
1 
\end{equation}
It is now obvious, that the derivative of this is equal to
\begin{equation}
0 
\end{equation}

The object of our ultimate interest is the following:
\begin{equation}
x ^{2 } 
\end{equation}
Trivially, the derivative of this is equal to
\begin{equation}
2 \cdot x ^{2 - 1 } \cdot 1 
\end{equation}

The following is worth a closer look:
\begin{equation}
2 
\end{equation}
It can be easily proved, that the derivative of this is equal to
\begin{equation}
0 
\end{equation}

We shall ponder the following:
\begin{equation}
1 
\end{equation}
It can be easily proved, that the derivative of this is equal to
\begin{equation}
0 
\end{equation}

Consider the following:
\begin{equation}
1 
\end{equation}
It is now obvious, that the derivative of this is equal to
\begin{equation}
0 
\end{equation}

We will allow ourselves to divert the reader's attentions to this gem of mathematical wonder:
\begin{equation}
x ^{2 } 
\end{equation}
It is now obvious, that the derivative of this is equal to
\begin{equation}
2 \cdot x ^{2 - 1 } \cdot 1 
\end{equation}

We shall ponder the following:
\begin{equation}
1 
\end{equation}
Clearly, the derivative of this is equal to
\begin{equation}
0 
\end{equation}

Consider the following:
\begin{equation}
1 
\end{equation}
As you can see, the derivative of this is equal to
\begin{equation}
0 
\end{equation}

We shall ponder the following:
\begin{equation}
x - 2 
\end{equation}
It is now obvious, that the derivative of this is equal to
\begin{equation}
1 - 0 
\end{equation}

Let us take a look at this:
\begin{equation}
1 
\end{equation}
Any self-respecting mathematician would find it obvious, that the derivative of this is equal to
\begin{equation}
0 
\end{equation}

Let us take a look at this:
\begin{equation}
x ^{2 } 
\end{equation}
Unsurprisingly, the derivative of this is equal to
\begin{equation}
2 \cdot x ^{2 - 1 } \cdot 1 
\end{equation}

We shall ponder the following:
\begin{equation}
1 
\end{equation}
Obviously, the derivative of this is equal to
\begin{equation}
0 
\end{equation}

We are going to study the following:
\begin{equation}
x ^{2 } 
\end{equation}
Trivially, the derivative of this is equal to
\begin{equation}
2 \cdot x ^{2 - 1 } \cdot 1 
\end{equation}

We will allow ourselves to divert the reader's attentions to this gem of mathematical wonder:
\begin{equation}
2 \cdot x 
\end{equation}
Obviously, the derivative of this is equal to
\begin{equation}
0 \cdot x + 2 \cdot 1 
\end{equation}

The following is worth a closer look:
\begin{equation}
1 
\end{equation}
Any self-respecting mathematician would find it obvious, that the derivative of this is equal to
\begin{equation}
0 
\end{equation}

We are going to study the following:
\begin{equation}
x ^{2 } 
\end{equation}
Unsurprisingly, the derivative of this is equal to
\begin{equation}
2 \cdot x ^{2 - 1 } \cdot 1 
\end{equation}

Consider the following:
\begin{equation}
2 
\end{equation}
Unsurprisingly, the derivative of this is equal to
\begin{equation}
0 
\end{equation}

Consider the following:
\begin{equation}
1 
\end{equation}
It is now obvious, that the derivative of this is equal to
\begin{equation}
0 
\end{equation}

The following is worth a closer look:
\begin{equation}
1 
\end{equation}
Clearly, the derivative of this is equal to
\begin{equation}
0 
\end{equation}

Consider the following:
\begin{equation}
x ^{2 } 
\end{equation}
It can be easily proved, that the derivative of this is equal to
\begin{equation}
2 \cdot x ^{2 - 1 } \cdot 1 
\end{equation}

One shall regard the object in question with utmost interest:
\begin{equation}
1 
\end{equation}
As you can see, the derivative of this is equal to
\begin{equation}
0 
\end{equation}

The object of our ultimate interest is the following:
\begin{equation}
1 
\end{equation}
Any self-respecting mathematician would find it obvious, that the derivative of this is equal to
\begin{equation}
0 
\end{equation}

We are going to study the following:
\begin{equation}
x - 2 
\end{equation}
As you can see, the derivative of this is equal to
\begin{equation}
1 - 0 
\end{equation}

We will allow ourselves to divert the reader's attentions to this gem of mathematical wonder:
\begin{equation}
1 
\end{equation}
It is now obvious, that the derivative of this is equal to
\begin{equation}
0 
\end{equation}

We are going to study the following:
\begin{equation}
x ^{2 } 
\end{equation}
Obviously, the derivative of this is equal to
\begin{equation}
2 \cdot x ^{2 - 1 } \cdot 1 
\end{equation}

We shall ponder the following:
\begin{equation}
1 
\end{equation}
Clearly, the derivative of this is equal to
\begin{equation}
0 
\end{equation}

The object of our ultimate interest is the following:
\begin{equation}
x ^{2 } 
\end{equation}
Trivially, the derivative of this is equal to
\begin{equation}
2 \cdot x ^{2 - 1 } \cdot 1 
\end{equation}

We are going to study the following:
\begin{equation}
x - 2 
\end{equation}
It is now obvious, that the derivative of this is equal to
\begin{equation}
1 - 0 
\end{equation}

The following is worth a closer look:
\begin{equation}
x + 1 
\end{equation}
It can be easily proved, that the derivative of this is equal to
\begin{equation}
1 + 0 
\end{equation}

The object of our ultimate interest is the following:
\begin{equation}
2 
\end{equation}
Any self-respecting mathematician would find it obvious, that the derivative of this is equal to
\begin{equation}
0 
\end{equation}

We will allow ourselves to divert the reader's attentions to this gem of mathematical wonder:
\begin{equation}
\sin x 
\end{equation}
It can be easily proved, that the derivative of this is equal to
\begin{equation}
\cos x \cdot 1 
\end{equation}

We will allow ourselves to divert the reader's attentions to this gem of mathematical wonder:
\begin{equation}
1 
\end{equation}
Any self-respecting mathematician would find it obvious, that the derivative of this is equal to
\begin{equation}
0 
\end{equation}

One shall regard the object in question with utmost interest:
\begin{equation}
x ^{2 } 
\end{equation}
As you can see, the derivative of this is equal to
\begin{equation}
2 \cdot x ^{2 - 1 } \cdot 1 
\end{equation}

We will take a closer look at this:
\begin{equation}
2 \cdot x 
\end{equation}
Any self-respecting mathematician would find it obvious, that the derivative of this is equal to
\begin{equation}
0 \cdot x + 2 \cdot 1 
\end{equation}

We are going to study the following:
\begin{equation}
1 
\end{equation}
Obviously, the derivative of this is equal to
\begin{equation}
0 
\end{equation}

The following is worth a closer look:
\begin{equation}
x ^{2 } 
\end{equation}
Clearly, the derivative of this is equal to
\begin{equation}
2 \cdot x ^{2 - 1 } \cdot 1 
\end{equation}

We shall ponder the following:
\begin{equation}
2 
\end{equation}
It can be easily proved, that the derivative of this is equal to
\begin{equation}
0 
\end{equation}

We are going to study the following:
\begin{equation}
1 
\end{equation}
Unsurprisingly, the derivative of this is equal to
\begin{equation}
0 
\end{equation}

We will take a closer look at this:
\begin{equation}
1 
\end{equation}
It can be easily proved, that the derivative of this is equal to
\begin{equation}
0 
\end{equation}

The following is worth a closer look:
\begin{equation}
x ^{2 } 
\end{equation}
It is now obvious, that the derivative of this is equal to
\begin{equation}
2 \cdot x ^{2 - 1 } \cdot 1 
\end{equation}

We shall ponder the following:
\begin{equation}
1 
\end{equation}
Unsurprisingly, the derivative of this is equal to
\begin{equation}
0 
\end{equation}

We are going to study the following:
\begin{equation}
1 
\end{equation}
Clearly, the derivative of this is equal to
\begin{equation}
0 
\end{equation}

The object of our ultimate interest is the following:
\begin{equation}
x - 2 
\end{equation}
It is now obvious, that the derivative of this is equal to
\begin{equation}
1 - 0 
\end{equation}

Consider the following:
\begin{equation}
1 
\end{equation}
Obviously, the derivative of this is equal to
\begin{equation}
0 
\end{equation}

We shall ponder the following:
\begin{equation}
x ^{2 } 
\end{equation}
Any self-respecting mathematician would find it obvious, that the derivative of this is equal to
\begin{equation}
2 \cdot x ^{2 - 1 } \cdot 1 
\end{equation}

We shall ponder the following:
\begin{equation}
1 
\end{equation}
It can be easily proved, that the derivative of this is equal to
\begin{equation}
0 
\end{equation}

Let us take a look at this:
\begin{equation}
x ^{2 } 
\end{equation}
Unsurprisingly, the derivative of this is equal to
\begin{equation}
2 \cdot x ^{2 - 1 } \cdot 1 
\end{equation}

The following is worth a closer look:
\begin{equation}
x - 2 
\end{equation}
Obviously, the derivative of this is equal to
\begin{equation}
1 - 0 
\end{equation}

We are going to study the following:
\begin{equation}
x + 1 
\end{equation}
It is now obvious, that the derivative of this is equal to
\begin{equation}
1 + 0 
\end{equation}

We shall ponder the following:
\begin{equation}
1 
\end{equation}
Unsurprisingly, the derivative of this is equal to
\begin{equation}
0 
\end{equation}

We will take a closer look at this:
\begin{equation}
2 
\end{equation}
Obviously, the derivative of this is equal to
\begin{equation}
0 
\end{equation}

Let us take a look at this:
\begin{equation}
\sin x 
\end{equation}
Clearly, the derivative of this is equal to
\begin{equation}
\cos x \cdot 1 
\end{equation}

The object of our ultimate interest is the following:
\begin{equation}
x + 1 
\end{equation}
It can be easily proved, that the derivative of this is equal to
\begin{equation}
1 + 0 
\end{equation}

We are going to study the following:
\begin{equation}
4 
\end{equation}
Any self-respecting mathematician would find it obvious, that the derivative of this is equal to
\begin{equation}
0 
\end{equation}

Let us take a look at this:
\begin{equation}
2 
\end{equation}
It can be easily proved, that the derivative of this is equal to
\begin{equation}
0 
\end{equation}

We shall ponder the following:
\begin{equation}
\cos x 
\end{equation}
Obviously, the derivative of this is equal to
\begin{equation}
-\sin x \cdot 1 
\end{equation}

One shall regard the object in question with utmost interest:
\begin{equation}
x + 1 
\end{equation}
Clearly, the derivative of this is equal to
\begin{equation}
1 + 0 
\end{equation}

Let us take a look at this:
\begin{equation}
2 
\end{equation}
It can be easily proved, that the derivative of this is equal to
\begin{equation}
0 
\end{equation}

We will take a closer look at this:
\begin{equation}
\sin x 
\end{equation}
Unsurprisingly, the derivative of this is equal to
\begin{equation}
\cos x \cdot 1 
\end{equation}

We are going to study the following:
\begin{equation}
1 
\end{equation}
Unsurprisingly, the derivative of this is equal to
\begin{equation}
0 
\end{equation}

Let us take a look at this:
\begin{equation}
x ^{2 } 
\end{equation}
Clearly, the derivative of this is equal to
\begin{equation}
2 \cdot x ^{2 - 1 } \cdot 1 
\end{equation}

Let us take a look at this:
\begin{equation}
2 \cdot x 
\end{equation}
Obviously, the derivative of this is equal to
\begin{equation}
0 \cdot x + 2 \cdot 1 
\end{equation}

The following is worth a closer look:
\begin{equation}
1 
\end{equation}
It is now obvious, that the derivative of this is equal to
\begin{equation}
0 
\end{equation}

One shall regard the object in question with utmost interest:
\begin{equation}
x ^{2 } 
\end{equation}
Unsurprisingly, the derivative of this is equal to
\begin{equation}
2 \cdot x ^{2 - 1 } \cdot 1 
\end{equation}

We are going to study the following:
\begin{equation}
2 
\end{equation}
Unsurprisingly, the derivative of this is equal to
\begin{equation}
0 
\end{equation}

We shall ponder the following:
\begin{equation}
1 
\end{equation}
As you can see, the derivative of this is equal to
\begin{equation}
0 
\end{equation}

Let us take a look at this:
\begin{equation}
1 
\end{equation}
Trivially, the derivative of this is equal to
\begin{equation}
0 
\end{equation}

One shall regard the object in question with utmost interest:
\begin{equation}
x ^{2 } 
\end{equation}
Any self-respecting mathematician would find it obvious, that the derivative of this is equal to
\begin{equation}
2 \cdot x ^{2 - 1 } \cdot 1 
\end{equation}

The object of our ultimate interest is the following:
\begin{equation}
1 
\end{equation}
As you can see, the derivative of this is equal to
\begin{equation}
0 
\end{equation}

Consider the following:
\begin{equation}
1 
\end{equation}
Unsurprisingly, the derivative of this is equal to
\begin{equation}
0 
\end{equation}

One shall regard the object in question with utmost interest:
\begin{equation}
x - 2 
\end{equation}
It is now obvious, that the derivative of this is equal to
\begin{equation}
1 - 0 
\end{equation}

The following is worth a closer look:
\begin{equation}
1 
\end{equation}
It can be easily proved, that the derivative of this is equal to
\begin{equation}
0 
\end{equation}

The object of our ultimate interest is the following:
\begin{equation}
x ^{2 } 
\end{equation}
Any self-respecting mathematician would find it obvious, that the derivative of this is equal to
\begin{equation}
2 \cdot x ^{2 - 1 } \cdot 1 
\end{equation}

One shall regard the object in question with utmost interest:
\begin{equation}
1 
\end{equation}
As you can see, the derivative of this is equal to
\begin{equation}
0 
\end{equation}

Consider the following:
\begin{equation}
x ^{2 } 
\end{equation}
It is now obvious, that the derivative of this is equal to
\begin{equation}
2 \cdot x ^{2 - 1 } \cdot 1 
\end{equation}

We will take a closer look at this:
\begin{equation}
x - 2 
\end{equation}
As you can see, the derivative of this is equal to
\begin{equation}
1 - 0 
\end{equation}

The following is worth a closer look:
\begin{equation}
x + 1 
\end{equation}
It is now obvious, that the derivative of this is equal to
\begin{equation}
1 + 0 
\end{equation}

Let us take a look at this:
\begin{equation}
1 
\end{equation}
Obviously, the derivative of this is equal to
\begin{equation}
0 
\end{equation}

We are going to study the following:
\begin{equation}
2 
\end{equation}
Trivially, the derivative of this is equal to
\begin{equation}
0 
\end{equation}

The following is worth a closer look:
\begin{equation}
\sin x 
\end{equation}
As you can see, the derivative of this is equal to
\begin{equation}
\cos x \cdot 1 
\end{equation}

We are going to study the following:
\begin{equation}
x + 1 
\end{equation}
It is now obvious, that the derivative of this is equal to
\begin{equation}
1 + 0 
\end{equation}

Let us take a look at this:
\begin{equation}
4 
\end{equation}
Trivially, the derivative of this is equal to
\begin{equation}
0 
\end{equation}

We shall ponder the following:
\begin{equation}
2 
\end{equation}
As you can see, the derivative of this is equal to
\begin{equation}
0 
\end{equation}

Let us take a look at this:
\begin{equation}
\cos x 
\end{equation}
Obviously, the derivative of this is equal to
\begin{equation}
-\sin x \cdot 1 
\end{equation}

One shall regard the object in question with utmost interest:
\begin{equation}
x + 1 
\end{equation}
Unsurprisingly, the derivative of this is equal to
\begin{equation}
1 + 0 
\end{equation}

We will allow ourselves to divert the reader's attentions to this gem of mathematical wonder:
\begin{equation}
2 
\end{equation}
It is now obvious, that the derivative of this is equal to
\begin{equation}
0 
\end{equation}

Let us take a look at this:
\begin{equation}
\sin x 
\end{equation}
Unsurprisingly, the derivative of this is equal to
\begin{equation}
\cos x \cdot 1 
\end{equation}

We will allow ourselves to divert the reader's attentions to this gem of mathematical wonder:
\begin{equation}
1 
\end{equation}
It can be easily proved, that the derivative of this is equal to
\begin{equation}
0 
\end{equation}

The object of our ultimate interest is the following:
\begin{equation}
x ^{2 } 
\end{equation}
Obviously, the derivative of this is equal to
\begin{equation}
2 \cdot x ^{2 - 1 } \cdot 1 
\end{equation}

We will take a closer look at this:
\begin{equation}
x - 2 
\end{equation}
It is now obvious, that the derivative of this is equal to
\begin{equation}
1 - 0 
\end{equation}

We will allow ourselves to divert the reader's attentions to this gem of mathematical wonder:
\begin{equation}
x + 1 
\end{equation}
Trivially, the derivative of this is equal to
\begin{equation}
1 + 0 
\end{equation}

The object of our ultimate interest is the following:
\begin{equation}
1 
\end{equation}
Unsurprisingly, the derivative of this is equal to
\begin{equation}
0 
\end{equation}

We will take a closer look at this:
\begin{equation}
2 
\end{equation}
Obviously, the derivative of this is equal to
\begin{equation}
0 
\end{equation}

We will take a closer look at this:
\begin{equation}
\sin x 
\end{equation}
Trivially, the derivative of this is equal to
\begin{equation}
\cos x \cdot 1 
\end{equation}

We are going to study the following:
\begin{equation}
x + 1 
\end{equation}
As you can see, the derivative of this is equal to
\begin{equation}
1 + 0 
\end{equation}

We are going to study the following:
\begin{equation}
1 
\end{equation}
Obviously, the derivative of this is equal to
\begin{equation}
0 
\end{equation}

We will allow ourselves to divert the reader's attentions to this gem of mathematical wonder:
\begin{equation}
4 
\end{equation}
Unsurprisingly, the derivative of this is equal to
\begin{equation}
0 
\end{equation}

One shall regard the object in question with utmost interest:
\begin{equation}
2 
\end{equation}
It is now obvious, that the derivative of this is equal to
\begin{equation}
0 
\end{equation}

We will take a closer look at this:
\begin{equation}
\cos x 
\end{equation}
As you can see, the derivative of this is equal to
\begin{equation}
-\sin x \cdot 1 
\end{equation}

Consider the following:
\begin{equation}
x + 1 
\end{equation}
Clearly, the derivative of this is equal to
\begin{equation}
1 + 0 
\end{equation}

One shall regard the object in question with utmost interest:
\begin{equation}
16 
\end{equation}
Unsurprisingly, the derivative of this is equal to
\begin{equation}
0 
\end{equation}

One shall regard the object in question with utmost interest:
\begin{equation}
0 
\end{equation}
As you can see, the derivative of this is equal to
\begin{equation}
0 
\end{equation}

We will allow ourselves to divert the reader's attentions to this gem of mathematical wonder:
\begin{equation}
4 
\end{equation}
It can be easily proved, that the derivative of this is equal to
\begin{equation}
0 
\end{equation}

We shall ponder the following:
\begin{equation}
2 
\end{equation}
Trivially, the derivative of this is equal to
\begin{equation}
0 
\end{equation}

We will allow ourselves to divert the reader's attentions to this gem of mathematical wonder:
\begin{equation}
1 
\end{equation}
Obviously, the derivative of this is equal to
\begin{equation}
0 
\end{equation}

The object of our ultimate interest is the following:
\begin{equation}
\sin x 
\end{equation}
Unsurprisingly, the derivative of this is equal to
\begin{equation}
\cos x \cdot 1 
\end{equation}

Let us take a look at this:
\begin{equation}
x + 1 
\end{equation}
It can be easily proved, that the derivative of this is equal to
\begin{equation}
1 + 0 
\end{equation}

We shall ponder the following:
\begin{equation}
1 
\end{equation}
Trivially, the derivative of this is equal to
\begin{equation}
0 
\end{equation}

We will allow ourselves to divert the reader's attentions to this gem of mathematical wonder:
\begin{equation}
4 
\end{equation}
Obviously, the derivative of this is equal to
\begin{equation}
0 
\end{equation}

The following is worth a closer look:
\begin{equation}
2 
\end{equation}
Clearly, the derivative of this is equal to
\begin{equation}
0 
\end{equation}

Consider the following:
\begin{equation}
\cos x 
\end{equation}
Clearly, the derivative of this is equal to
\begin{equation}
-\sin x \cdot 1 
\end{equation}

Let us take a look at this:
\begin{equation}
x + 1 
\end{equation}
Clearly, the derivative of this is equal to
\begin{equation}
1 + 0 
\end{equation}

We are going to study the following:
\begin{equation}
2 
\end{equation}
It can be easily proved, that the derivative of this is equal to
\begin{equation}
0 
\end{equation}

We shall ponder the following:
\begin{equation}
\sin x 
\end{equation}
Trivially, the derivative of this is equal to
\begin{equation}
\cos x \cdot 1 
\end{equation}

Let us take a look at this:
\begin{equation}
x + 1 
\end{equation}
As you can see, the derivative of this is equal to
\begin{equation}
1 + 0 
\end{equation}

Let us take a look at this:
\begin{equation}
1 
\end{equation}
Clearly, the derivative of this is equal to
\begin{equation}
0 
\end{equation}

We will allow ourselves to divert the reader's attentions to this gem of mathematical wonder:
\begin{equation}
2 
\end{equation}
Unsurprisingly, the derivative of this is equal to
\begin{equation}
0 
\end{equation}

Consider the following:
\begin{equation}
\sin x 
\end{equation}
It can be easily proved, that the derivative of this is equal to
\begin{equation}
\cos x \cdot 1 
\end{equation}

We shall ponder the following:
\begin{equation}
x + 1 
\end{equation}
It is now obvious, that the derivative of this is equal to
\begin{equation}
1 + 0 
\end{equation}

The following is worth a closer look:
\begin{equation}
4 
\end{equation}
It can be easily proved, that the derivative of this is equal to
\begin{equation}
0 
\end{equation}

One shall regard the object in question with utmost interest:
\begin{equation}
2 
\end{equation}
Clearly, the derivative of this is equal to
\begin{equation}
0 
\end{equation}

Let us take a look at this:
\begin{equation}
\cos x 
\end{equation}
Any self-respecting mathematician would find it obvious, that the derivative of this is equal to
\begin{equation}
-\sin x \cdot 1 
\end{equation}

We shall ponder the following:
\begin{equation}
x + 1 
\end{equation}
Clearly, the derivative of this is equal to
\begin{equation}
1 + 0 
\end{equation}

Consider the following:
\begin{equation}
1 
\end{equation}
As you can see, the derivative of this is equal to
\begin{equation}
0 
\end{equation}

Consider the following:
\begin{equation}
2 
\end{equation}
Obviously, the derivative of this is equal to
\begin{equation}
0 
\end{equation}

The object of our ultimate interest is the following:
\begin{equation}
\sin x 
\end{equation}
It is now obvious, that the derivative of this is equal to
\begin{equation}
\cos x \cdot 1 
\end{equation}

One shall regard the object in question with utmost interest:
\begin{equation}
x + 1 
\end{equation}
Unsurprisingly, the derivative of this is equal to
\begin{equation}
1 + 0 
\end{equation}

Let us take a look at this:
\begin{equation}
4 
\end{equation}
Clearly, the derivative of this is equal to
\begin{equation}
0 
\end{equation}

The object of our ultimate interest is the following:
\begin{equation}
2 
\end{equation}
Clearly, the derivative of this is equal to
\begin{equation}
0 
\end{equation}

We are going to study the following:
\begin{equation}
\cos x 
\end{equation}
Obviously, the derivative of this is equal to
\begin{equation}
-\sin x \cdot 1 
\end{equation}

We are going to study the following:
\begin{equation}
x + 1 
\end{equation}
It is now obvious, that the derivative of this is equal to
\begin{equation}
1 + 0 
\end{equation}

We will allow ourselves to divert the reader's attentions to this gem of mathematical wonder:
\begin{equation}
2 
\end{equation}
It can be easily proved, that the derivative of this is equal to
\begin{equation}
0 
\end{equation}

We will allow ourselves to divert the reader's attentions to this gem of mathematical wonder:
\begin{equation}
\sin x 
\end{equation}
As you can see, the derivative of this is equal to
\begin{equation}
\cos x \cdot 1 
\end{equation}

Let us take a look at this:
\begin{equation}
1 
\end{equation}
Trivially, the derivative of this is equal to
\begin{equation}
0 
\end{equation}

One shall regard the object in question with utmost interest:
\begin{equation}
x ^{2 } 
\end{equation}
As you can see, the derivative of this is equal to
\begin{equation}
2 \cdot x ^{2 - 1 } \cdot 1 
\end{equation}

The object of our ultimate interest is the following:
\begin{equation}
2 \cdot x 
\end{equation}
As you can see, the derivative of this is equal to
\begin{equation}
0 \cdot x + 2 \cdot 1 
\end{equation}

Consider the following:
\begin{equation}
1 
\end{equation}
Clearly, the derivative of this is equal to
\begin{equation}
0 
\end{equation}

We will allow ourselves to divert the reader's attentions to this gem of mathematical wonder:
\begin{equation}
x ^{2 } 
\end{equation}
As you can see, the derivative of this is equal to
\begin{equation}
2 \cdot x ^{2 - 1 } \cdot 1 
\end{equation}

The object of our ultimate interest is the following:
\begin{equation}
2 
\end{equation}
Trivially, the derivative of this is equal to
\begin{equation}
0 
\end{equation}

We will allow ourselves to divert the reader's attentions to this gem of mathematical wonder:
\begin{equation}
1 
\end{equation}
Obviously, the derivative of this is equal to
\begin{equation}
0 
\end{equation}

The following is worth a closer look:
\begin{equation}
1 
\end{equation}
Obviously, the derivative of this is equal to
\begin{equation}
0 
\end{equation}

We will allow ourselves to divert the reader's attentions to this gem of mathematical wonder:
\begin{equation}
x ^{2 } 
\end{equation}
Trivially, the derivative of this is equal to
\begin{equation}
2 \cdot x ^{2 - 1 } \cdot 1 
\end{equation}

We are going to study the following:
\begin{equation}
1 
\end{equation}
Unsurprisingly, the derivative of this is equal to
\begin{equation}
0 
\end{equation}

Consider the following:
\begin{equation}
1 
\end{equation}
Clearly, the derivative of this is equal to
\begin{equation}
0 
\end{equation}

We shall ponder the following:
\begin{equation}
1 
\end{equation}
As you can see, the derivative of this is equal to
\begin{equation}
0 
\end{equation}

Consider the following:
\begin{equation}
x ^{2 } 
\end{equation}
It is now obvious, that the derivative of this is equal to
\begin{equation}
2 \cdot x ^{2 - 1 } \cdot 1 
\end{equation}

Consider the following:
\begin{equation}
2 
\end{equation}
As you can see, the derivative of this is equal to
\begin{equation}
0 
\end{equation}

The following is worth a closer look:
\begin{equation}
2 \cdot x 
\end{equation}
Any self-respecting mathematician would find it obvious, that the derivative of this is equal to
\begin{equation}
0 \cdot x + 2 \cdot 1 
\end{equation}

The following is worth a closer look:
\begin{equation}
1 
\end{equation}
Trivially, the derivative of this is equal to
\begin{equation}
0 
\end{equation}

We will allow ourselves to divert the reader's attentions to this gem of mathematical wonder:
\begin{equation}
x ^{2 } 
\end{equation}
Obviously, the derivative of this is equal to
\begin{equation}
2 \cdot x ^{2 - 1 } \cdot 1 
\end{equation}

The object of our ultimate interest is the following:
\begin{equation}
2 
\end{equation}
As you can see, the derivative of this is equal to
\begin{equation}
0 
\end{equation}

We are going to study the following:
\begin{equation}
1 
\end{equation}
Any self-respecting mathematician would find it obvious, that the derivative of this is equal to
\begin{equation}
0 
\end{equation}

We will take a closer look at this:
\begin{equation}
1 
\end{equation}
Clearly, the derivative of this is equal to
\begin{equation}
0 
\end{equation}

We will take a closer look at this:
\begin{equation}
x ^{2 } 
\end{equation}
It can be easily proved, that the derivative of this is equal to
\begin{equation}
2 \cdot x ^{2 - 1 } \cdot 1 
\end{equation}

The object of our ultimate interest is the following:
\begin{equation}
1 
\end{equation}
It is now obvious, that the derivative of this is equal to
\begin{equation}
0 
\end{equation}

The following is worth a closer look:
\begin{equation}
1 
\end{equation}
As you can see, the derivative of this is equal to
\begin{equation}
0 
\end{equation}

The object of our ultimate interest is the following:
\begin{equation}
2 \cdot x 
\end{equation}
Clearly, the derivative of this is equal to
\begin{equation}
0 \cdot x + 2 \cdot 1 
\end{equation}

One shall regard the object in question with utmost interest:
\begin{equation}
1 
\end{equation}
It is now obvious, that the derivative of this is equal to
\begin{equation}
0 
\end{equation}

We are going to study the following:
\begin{equation}
x ^{2 } 
\end{equation}
As you can see, the derivative of this is equal to
\begin{equation}
2 \cdot x ^{2 - 1 } \cdot 1 
\end{equation}

We shall ponder the following:
\begin{equation}
2 
\end{equation}
As you can see, the derivative of this is equal to
\begin{equation}
0 
\end{equation}

We shall ponder the following:
\begin{equation}
1 
\end{equation}
Clearly, the derivative of this is equal to
\begin{equation}
0 
\end{equation}

Consider the following:
\begin{equation}
1 
\end{equation}
It can be easily proved, that the derivative of this is equal to
\begin{equation}
0 
\end{equation}

We will take a closer look at this:
\begin{equation}
x ^{2 } 
\end{equation}
Trivially, the derivative of this is equal to
\begin{equation}
2 \cdot x ^{2 - 1 } \cdot 1 
\end{equation}

The object of our ultimate interest is the following:
\begin{equation}
1 
\end{equation}
Obviously, the derivative of this is equal to
\begin{equation}
0 
\end{equation}

We will allow ourselves to divert the reader's attentions to this gem of mathematical wonder:
\begin{equation}
1 
\end{equation}
It can be easily proved, that the derivative of this is equal to
\begin{equation}
0 
\end{equation}

The following is worth a closer look:
\begin{equation}
1 
\end{equation}
Unsurprisingly, the derivative of this is equal to
\begin{equation}
0 
\end{equation}

Consider the following:
\begin{equation}
x ^{2 } 
\end{equation}
Unsurprisingly, the derivative of this is equal to
\begin{equation}
2 \cdot x ^{2 - 1 } \cdot 1 
\end{equation}

One shall regard the object in question with utmost interest:
\begin{equation}
2 \cdot x 
\end{equation}
Any self-respecting mathematician would find it obvious, that the derivative of this is equal to
\begin{equation}
0 \cdot x + 2 \cdot 1 
\end{equation}

The following is worth a closer look:
\begin{equation}
1 
\end{equation}
Clearly, the derivative of this is equal to
\begin{equation}
0 
\end{equation}

The following is worth a closer look:
\begin{equation}
x ^{2 } 
\end{equation}
Obviously, the derivative of this is equal to
\begin{equation}
2 \cdot x ^{2 - 1 } \cdot 1 
\end{equation}

Consider the following:
\begin{equation}
2 
\end{equation}
Trivially, the derivative of this is equal to
\begin{equation}
0 
\end{equation}

The object of our ultimate interest is the following:
\begin{equation}
1 
\end{equation}
As you can see, the derivative of this is equal to
\begin{equation}
0 
\end{equation}

We are going to study the following:
\begin{equation}
1 
\end{equation}
Unsurprisingly, the derivative of this is equal to
\begin{equation}
0 
\end{equation}

One shall regard the object in question with utmost interest:
\begin{equation}
x ^{2 } 
\end{equation}
Unsurprisingly, the derivative of this is equal to
\begin{equation}
2 \cdot x ^{2 - 1 } \cdot 1 
\end{equation}

We are going to study the following:
\begin{equation}
1 
\end{equation}
Unsurprisingly, the derivative of this is equal to
\begin{equation}
0 
\end{equation}

The following is worth a closer look:
\begin{equation}
2 \cdot x 
\end{equation}
Obviously, the derivative of this is equal to
\begin{equation}
0 \cdot x + 2 \cdot 1 
\end{equation}

We shall ponder the following:
\begin{equation}
1 
\end{equation}
Unsurprisingly, the derivative of this is equal to
\begin{equation}
0 
\end{equation}

We will take a closer look at this:
\begin{equation}
x ^{2 } 
\end{equation}
Any self-respecting mathematician would find it obvious, that the derivative of this is equal to
\begin{equation}
2 \cdot x ^{2 - 1 } \cdot 1 
\end{equation}

The object of our ultimate interest is the following:
\begin{equation}
2 
\end{equation}
As you can see, the derivative of this is equal to
\begin{equation}
0 
\end{equation}

The object of our ultimate interest is the following:
\begin{equation}
1 
\end{equation}
As you can see, the derivative of this is equal to
\begin{equation}
0 
\end{equation}

The following is worth a closer look:
\begin{equation}
1 
\end{equation}
Obviously, the derivative of this is equal to
\begin{equation}
0 
\end{equation}

We will take a closer look at this:
\begin{equation}
x ^{2 } 
\end{equation}
It can be easily proved, that the derivative of this is equal to
\begin{equation}
2 \cdot x ^{2 - 1 } \cdot 1 
\end{equation}

We will allow ourselves to divert the reader's attentions to this gem of mathematical wonder:
\begin{equation}
2 
\end{equation}
Obviously, the derivative of this is equal to
\begin{equation}
0 
\end{equation}

The following is worth a closer look:
\begin{equation}
2 \cdot x 
\end{equation}
Obviously, the derivative of this is equal to
\begin{equation}
0 \cdot x + 2 \cdot 1 
\end{equation}

The object of our ultimate interest is the following:
\begin{equation}
1 
\end{equation}
It can be easily proved, that the derivative of this is equal to
\begin{equation}
0 
\end{equation}

One shall regard the object in question with utmost interest:
\begin{equation}
x ^{2 } 
\end{equation}
Any self-respecting mathematician would find it obvious, that the derivative of this is equal to
\begin{equation}
2 \cdot x ^{2 - 1 } \cdot 1 
\end{equation}

The object of our ultimate interest is the following:
\begin{equation}
2 
\end{equation}
Any self-respecting mathematician would find it obvious, that the derivative of this is equal to
\begin{equation}
0 
\end{equation}

Consider the following:
\begin{equation}
2 \cdot x 
\end{equation}
Clearly, the derivative of this is equal to
\begin{equation}
0 \cdot x + 2 \cdot 1 
\end{equation}

We will allow ourselves to divert the reader's attentions to this gem of mathematical wonder:
\begin{equation}
1 
\end{equation}
Trivially, the derivative of this is equal to
\begin{equation}
0 
\end{equation}

One shall regard the object in question with utmost interest:
\begin{equation}
x ^{2 } 
\end{equation}
Clearly, the derivative of this is equal to
\begin{equation}
2 \cdot x ^{2 - 1 } \cdot 1 
\end{equation}

We will take a closer look at this:
\begin{equation}
2 
\end{equation}
As you can see, the derivative of this is equal to
\begin{equation}
0 
\end{equation}

We are going to study the following:
\begin{equation}
1 
\end{equation}
As you can see, the derivative of this is equal to
\begin{equation}
0 
\end{equation}

We will allow ourselves to divert the reader's attentions to this gem of mathematical wonder:
\begin{equation}
2 
\end{equation}
Obviously, the derivative of this is equal to
\begin{equation}
0 
\end{equation}

We are going to study the following:
\begin{equation}
2 
\end{equation}
Unsurprisingly, the derivative of this is equal to
\begin{equation}
0 
\end{equation}

We are going to study the following:
\begin{equation}
1 
\end{equation}
It can be easily proved, that the derivative of this is equal to
\begin{equation}
0 
\end{equation}

The following is worth a closer look:
\begin{equation}
x ^{2 } 
\end{equation}
Clearly, the derivative of this is equal to
\begin{equation}
2 \cdot x ^{2 - 1 } \cdot 1 
\end{equation}

The object of our ultimate interest is the following:
\begin{equation}
2 
\end{equation}
Trivially, the derivative of this is equal to
\begin{equation}
0 
\end{equation}

Consider the following:
\begin{equation}
1 
\end{equation}
Any self-respecting mathematician would find it obvious, that the derivative of this is equal to
\begin{equation}
0 
\end{equation}

The object of our ultimate interest is the following:
\begin{equation}
1 
\end{equation}
It can be easily proved, that the derivative of this is equal to
\begin{equation}
0 
\end{equation}

We will allow ourselves to divert the reader's attentions to this gem of mathematical wonder:
\begin{equation}
x ^{2 } 
\end{equation}
Any self-respecting mathematician would find it obvious, that the derivative of this is equal to
\begin{equation}
2 \cdot x ^{2 - 1 } \cdot 1 
\end{equation}

We will take a closer look at this:
\begin{equation}
1 
\end{equation}
Trivially, the derivative of this is equal to
\begin{equation}
0 
\end{equation}

Let us take a look at this:
\begin{equation}
1 
\end{equation}
Clearly, the derivative of this is equal to
\begin{equation}
0 
\end{equation}

We will take a closer look at this:
\begin{equation}
1 
\end{equation}
Clearly, the derivative of this is equal to
\begin{equation}
0 
\end{equation}

The object of our ultimate interest is the following:
\begin{equation}
x ^{2 } 
\end{equation}
Clearly, the derivative of this is equal to
\begin{equation}
2 \cdot x ^{2 - 1 } \cdot 1 
\end{equation}

The following is worth a closer look:
\begin{equation}
2 \cdot x 
\end{equation}
Unsurprisingly, the derivative of this is equal to
\begin{equation}
0 \cdot x + 2 \cdot 1 
\end{equation}

We are going to study the following:
\begin{equation}
1 
\end{equation}
As you can see, the derivative of this is equal to
\begin{equation}
0 
\end{equation}

Let us take a look at this:
\begin{equation}
x ^{2 } 
\end{equation}
Clearly, the derivative of this is equal to
\begin{equation}
2 \cdot x ^{2 - 1 } \cdot 1 
\end{equation}

The object of our ultimate interest is the following:
\begin{equation}
2 
\end{equation}
It is now obvious, that the derivative of this is equal to
\begin{equation}
0 
\end{equation}

Let us take a look at this:
\begin{equation}
1 
\end{equation}
Trivially, the derivative of this is equal to
\begin{equation}
0 
\end{equation}

The following is worth a closer look:
\begin{equation}
1 
\end{equation}
As you can see, the derivative of this is equal to
\begin{equation}
0 
\end{equation}

The following is worth a closer look:
\begin{equation}
x ^{2 } 
\end{equation}
It is now obvious, that the derivative of this is equal to
\begin{equation}
2 \cdot x ^{2 - 1 } \cdot 1 
\end{equation}

We will take a closer look at this:
\begin{equation}
1 
\end{equation}
It is now obvious, that the derivative of this is equal to
\begin{equation}
0 
\end{equation}

We will allow ourselves to divert the reader's attentions to this gem of mathematical wonder:
\begin{equation}
2 \cdot x 
\end{equation}
Trivially, the derivative of this is equal to
\begin{equation}
0 \cdot x + 2 \cdot 1 
\end{equation}

One shall regard the object in question with utmost interest:
\begin{equation}
1 
\end{equation}
Obviously, the derivative of this is equal to
\begin{equation}
0 
\end{equation}

We will allow ourselves to divert the reader's attentions to this gem of mathematical wonder:
\begin{equation}
x ^{2 } 
\end{equation}
Obviously, the derivative of this is equal to
\begin{equation}
2 \cdot x ^{2 - 1 } \cdot 1 
\end{equation}

We are going to study the following:
\begin{equation}
2 
\end{equation}
Any self-respecting mathematician would find it obvious, that the derivative of this is equal to
\begin{equation}
0 
\end{equation}

We are going to study the following:
\begin{equation}
1 
\end{equation}
Trivially, the derivative of this is equal to
\begin{equation}
0 
\end{equation}

We are going to study the following:
\begin{equation}
1 
\end{equation}
Obviously, the derivative of this is equal to
\begin{equation}
0 
\end{equation}

Let us take a look at this:
\begin{equation}
x ^{2 } 
\end{equation}
Trivially, the derivative of this is equal to
\begin{equation}
2 \cdot x ^{2 - 1 } \cdot 1 
\end{equation}

We will allow ourselves to divert the reader's attentions to this gem of mathematical wonder:
\begin{equation}
2 
\end{equation}
It can be easily proved, that the derivative of this is equal to
\begin{equation}
0 
\end{equation}

We shall ponder the following:
\begin{equation}
2 \cdot x 
\end{equation}
Trivially, the derivative of this is equal to
\begin{equation}
0 \cdot x + 2 \cdot 1 
\end{equation}

We will allow ourselves to divert the reader's attentions to this gem of mathematical wonder:
\begin{equation}
1 
\end{equation}
Obviously, the derivative of this is equal to
\begin{equation}
0 
\end{equation}

We shall ponder the following:
\begin{equation}
x ^{2 } 
\end{equation}
Clearly, the derivative of this is equal to
\begin{equation}
2 \cdot x ^{2 - 1 } \cdot 1 
\end{equation}

One shall regard the object in question with utmost interest:
\begin{equation}
2 
\end{equation}
Obviously, the derivative of this is equal to
\begin{equation}
0 
\end{equation}

One shall regard the object in question with utmost interest:
\begin{equation}
2 \cdot x 
\end{equation}
Obviously, the derivative of this is equal to
\begin{equation}
0 \cdot x + 2 \cdot 1 
\end{equation}

We will allow ourselves to divert the reader's attentions to this gem of mathematical wonder:
\begin{equation}
1 
\end{equation}
It can be easily proved, that the derivative of this is equal to
\begin{equation}
0 
\end{equation}

We will take a closer look at this:
\begin{equation}
x ^{2 } 
\end{equation}
Any self-respecting mathematician would find it obvious, that the derivative of this is equal to
\begin{equation}
2 \cdot x ^{2 - 1 } \cdot 1 
\end{equation}

We shall ponder the following:
\begin{equation}
2 
\end{equation}
As you can see, the derivative of this is equal to
\begin{equation}
0 
\end{equation}

The following is worth a closer look:
\begin{equation}
1 
\end{equation}
Clearly, the derivative of this is equal to
\begin{equation}
0 
\end{equation}

We are going to study the following:
\begin{equation}
2 
\end{equation}
Clearly, the derivative of this is equal to
\begin{equation}
0 
\end{equation}

The object of our ultimate interest is the following:
\begin{equation}
2 
\end{equation}
Clearly, the derivative of this is equal to
\begin{equation}
0 
\end{equation}

Consider the following:
\begin{equation}
1 
\end{equation}
Any self-respecting mathematician would find it obvious, that the derivative of this is equal to
\begin{equation}
0 
\end{equation}

Let us take a look at this:
\begin{equation}
x ^{2 } 
\end{equation}
Clearly, the derivative of this is equal to
\begin{equation}
2 \cdot x ^{2 - 1 } \cdot 1 
\end{equation}

Consider the following:
\begin{equation}
2 
\end{equation}
Any self-respecting mathematician would find it obvious, that the derivative of this is equal to
\begin{equation}
0 
\end{equation}

The following is worth a closer look:
\begin{equation}
1 
\end{equation}
It can be easily proved, that the derivative of this is equal to
\begin{equation}
0 
\end{equation}

One shall regard the object in question with utmost interest:
\begin{equation}
1 
\end{equation}
Unsurprisingly, the derivative of this is equal to
\begin{equation}
0 
\end{equation}

Let us take a look at this:
\begin{equation}
x ^{2 } 
\end{equation}
Unsurprisingly, the derivative of this is equal to
\begin{equation}
2 \cdot x ^{2 - 1 } \cdot 1 
\end{equation}

We are going to study the following:
\begin{equation}
1 
\end{equation}
Trivially, the derivative of this is equal to
\begin{equation}
0 
\end{equation}

One shall regard the object in question with utmost interest:
\begin{equation}
1 
\end{equation}
Unsurprisingly, the derivative of this is equal to
\begin{equation}
0 
\end{equation}

We will take a closer look at this:
\begin{equation}
2 \cdot x 
\end{equation}
As you can see, the derivative of this is equal to
\begin{equation}
0 \cdot x + 2 \cdot 1 
\end{equation}

We will allow ourselves to divert the reader's attentions to this gem of mathematical wonder:
\begin{equation}
1 
\end{equation}
It can be easily proved, that the derivative of this is equal to
\begin{equation}
0 
\end{equation}

The following is worth a closer look:
\begin{equation}
x ^{2 } 
\end{equation}
Any self-respecting mathematician would find it obvious, that the derivative of this is equal to
\begin{equation}
2 \cdot x ^{2 - 1 } \cdot 1 
\end{equation}

Consider the following:
\begin{equation}
2 
\end{equation}
Trivially, the derivative of this is equal to
\begin{equation}
0 
\end{equation}

We will allow ourselves to divert the reader's attentions to this gem of mathematical wonder:
\begin{equation}
1 
\end{equation}
Unsurprisingly, the derivative of this is equal to
\begin{equation}
0 
\end{equation}

We will take a closer look at this:
\begin{equation}
1 
\end{equation}
Any self-respecting mathematician would find it obvious, that the derivative of this is equal to
\begin{equation}
0 
\end{equation}

Let us take a look at this:
\begin{equation}
x ^{2 } 
\end{equation}
Clearly, the derivative of this is equal to
\begin{equation}
2 \cdot x ^{2 - 1 } \cdot 1 
\end{equation}

One shall regard the object in question with utmost interest:
\begin{equation}
1 
\end{equation}
As you can see, the derivative of this is equal to
\begin{equation}
0 
\end{equation}

We are going to study the following:
\begin{equation}
1 
\end{equation}
Any self-respecting mathematician would find it obvious, that the derivative of this is equal to
\begin{equation}
0 
\end{equation}

Consider the following:
\begin{equation}
2 \cdot x 
\end{equation}
Unsurprisingly, the derivative of this is equal to
\begin{equation}
0 \cdot x + 2 \cdot 1 
\end{equation}

One shall regard the object in question with utmost interest:
\begin{equation}
1 
\end{equation}
Obviously, the derivative of this is equal to
\begin{equation}
0 
\end{equation}

Consider the following:
\begin{equation}
x ^{2 } 
\end{equation}
Clearly, the derivative of this is equal to
\begin{equation}
2 \cdot x ^{2 - 1 } \cdot 1 
\end{equation}

Consider the following:
\begin{equation}
2 
\end{equation}
Clearly, the derivative of this is equal to
\begin{equation}
0 
\end{equation}

We are going to study the following:
\begin{equation}
1 
\end{equation}
As you can see, the derivative of this is equal to
\begin{equation}
0 
\end{equation}

We are going to study the following:
\begin{equation}
1 
\end{equation}
It is now obvious, that the derivative of this is equal to
\begin{equation}
0 
\end{equation}

The object of our ultimate interest is the following:
\begin{equation}
x ^{2 } 
\end{equation}
Clearly, the derivative of this is equal to
\begin{equation}
2 \cdot x ^{2 - 1 } \cdot 1 
\end{equation}

We will allow ourselves to divert the reader's attentions to this gem of mathematical wonder:
\begin{equation}
1 
\end{equation}
It can be easily proved, that the derivative of this is equal to
\begin{equation}
0 
\end{equation}

Consider the following:
\begin{equation}
1 
\end{equation}
As you can see, the derivative of this is equal to
\begin{equation}
0 
\end{equation}

One shall regard the object in question with utmost interest:
\begin{equation}
2 \cdot x 
\end{equation}
It is now obvious, that the derivative of this is equal to
\begin{equation}
0 \cdot x + 2 \cdot 1 
\end{equation}

The following is worth a closer look:
\begin{equation}
1 
\end{equation}
As you can see, the derivative of this is equal to
\begin{equation}
0 
\end{equation}

Let us take a look at this:
\begin{equation}
x ^{2 } 
\end{equation}
Trivially, the derivative of this is equal to
\begin{equation}
2 \cdot x ^{2 - 1 } \cdot 1 
\end{equation}

Let us take a look at this:
\begin{equation}
2 
\end{equation}
Any self-respecting mathematician would find it obvious, that the derivative of this is equal to
\begin{equation}
0 
\end{equation}

One shall regard the object in question with utmost interest:
\begin{equation}
1 
\end{equation}
As you can see, the derivative of this is equal to
\begin{equation}
0 
\end{equation}

The following is worth a closer look:
\begin{equation}
1 
\end{equation}
As you can see, the derivative of this is equal to
\begin{equation}
0 
\end{equation}

We will allow ourselves to divert the reader's attentions to this gem of mathematical wonder:
\begin{equation}
x ^{2 } 
\end{equation}
Unsurprisingly, the derivative of this is equal to
\begin{equation}
2 \cdot x ^{2 - 1 } \cdot 1 
\end{equation}

We shall ponder the following:
\begin{equation}
1 
\end{equation}
It is now obvious, that the derivative of this is equal to
\begin{equation}
0 
\end{equation}

We shall ponder the following:
\begin{equation}
2 \cdot x 
\end{equation}
Clearly, the derivative of this is equal to
\begin{equation}
0 \cdot x + 2 \cdot 1 
\end{equation}

We will take a closer look at this:
\begin{equation}
1 
\end{equation}
Obviously, the derivative of this is equal to
\begin{equation}
0 
\end{equation}

The object of our ultimate interest is the following:
\begin{equation}
x ^{2 } 
\end{equation}
It is now obvious, that the derivative of this is equal to
\begin{equation}
2 \cdot x ^{2 - 1 } \cdot 1 
\end{equation}

Let us take a look at this:
\begin{equation}
2 
\end{equation}
It can be easily proved, that the derivative of this is equal to
\begin{equation}
0 
\end{equation}

The following is worth a closer look:
\begin{equation}
1 
\end{equation}
Trivially, the derivative of this is equal to
\begin{equation}
0 
\end{equation}

We will take a closer look at this:
\begin{equation}
1 
\end{equation}
Unsurprisingly, the derivative of this is equal to
\begin{equation}
0 
\end{equation}

Let us take a look at this:
\begin{equation}
x ^{2 } 
\end{equation}
Obviously, the derivative of this is equal to
\begin{equation}
2 \cdot x ^{2 - 1 } \cdot 1 
\end{equation}

We will take a closer look at this:
\begin{equation}
2 
\end{equation}
Trivially, the derivative of this is equal to
\begin{equation}
0 
\end{equation}

We are going to study the following:
\begin{equation}
2 \cdot x 
\end{equation}
Any self-respecting mathematician would find it obvious, that the derivative of this is equal to
\begin{equation}
0 \cdot x + 2 \cdot 1 
\end{equation}

The following is worth a closer look:
\begin{equation}
1 
\end{equation}
Any self-respecting mathematician would find it obvious, that the derivative of this is equal to
\begin{equation}
0 
\end{equation}

The following is worth a closer look:
\begin{equation}
x ^{2 } 
\end{equation}
Obviously, the derivative of this is equal to
\begin{equation}
2 \cdot x ^{2 - 1 } \cdot 1 
\end{equation}

Let us take a look at this:
\begin{equation}
2 
\end{equation}
Trivially, the derivative of this is equal to
\begin{equation}
0 
\end{equation}

Consider the following:
\begin{equation}
2 \cdot x 
\end{equation}
Clearly, the derivative of this is equal to
\begin{equation}
0 \cdot x + 2 \cdot 1 
\end{equation}

Consider the following:
\begin{equation}
1 
\end{equation}
Any self-respecting mathematician would find it obvious, that the derivative of this is equal to
\begin{equation}
0 
\end{equation}

We shall ponder the following:
\begin{equation}
x ^{2 } 
\end{equation}
Trivially, the derivative of this is equal to
\begin{equation}
2 \cdot x ^{2 - 1 } \cdot 1 
\end{equation}

We shall ponder the following:
\begin{equation}
2 
\end{equation}
Obviously, the derivative of this is equal to
\begin{equation}
0 
\end{equation}

The object of our ultimate interest is the following:
\begin{equation}
1 
\end{equation}
As you can see, the derivative of this is equal to
\begin{equation}
0 
\end{equation}

The object of our ultimate interest is the following:
\begin{equation}
2 
\end{equation}
As you can see, the derivative of this is equal to
\begin{equation}
0 
\end{equation}

We shall ponder the following:
\begin{equation}
2 
\end{equation}
Trivially, the derivative of this is equal to
\begin{equation}
0 
\end{equation}

Let us take a look at this:
\begin{equation}
1 
\end{equation}
Any self-respecting mathematician would find it obvious, that the derivative of this is equal to
\begin{equation}
0 
\end{equation}

We will take a closer look at this:
\begin{equation}
x ^{2 } 
\end{equation}
Any self-respecting mathematician would find it obvious, that the derivative of this is equal to
\begin{equation}
2 \cdot x ^{2 - 1 } \cdot 1 
\end{equation}

Let us take a look at this:
\begin{equation}
2 
\end{equation}
Clearly, the derivative of this is equal to
\begin{equation}
0 
\end{equation}

We shall ponder the following:
\begin{equation}
1 
\end{equation}
It can be easily proved, that the derivative of this is equal to
\begin{equation}
0 
\end{equation}

Let us take a look at this:
\begin{equation}
1 
\end{equation}
Obviously, the derivative of this is equal to
\begin{equation}
0 
\end{equation}

We will take a closer look at this:
\begin{equation}
x ^{2 } 
\end{equation}
Trivially, the derivative of this is equal to
\begin{equation}
2 \cdot x ^{2 - 1 } \cdot 1 
\end{equation}

We will allow ourselves to divert the reader's attentions to this gem of mathematical wonder:
\begin{equation}
1 
\end{equation}
It can be easily proved, that the derivative of this is equal to
\begin{equation}
0 
\end{equation}

The following is worth a closer look:
\begin{equation}
1 
\end{equation}
Any self-respecting mathematician would find it obvious, that the derivative of this is equal to
\begin{equation}
0 
\end{equation}

Consider the following:
\begin{equation}
1 
\end{equation}
Obviously, the derivative of this is equal to
\begin{equation}
0 
\end{equation}

We will allow ourselves to divert the reader's attentions to this gem of mathematical wonder:
\begin{equation}
x ^{2 } 
\end{equation}
Clearly, the derivative of this is equal to
\begin{equation}
2 \cdot x ^{2 - 1 } \cdot 1 
\end{equation}

We are going to study the following:
\begin{equation}
2 \cdot x 
\end{equation}
Obviously, the derivative of this is equal to
\begin{equation}
0 \cdot x + 2 \cdot 1 
\end{equation}

One shall regard the object in question with utmost interest:
\begin{equation}
1 
\end{equation}
Obviously, the derivative of this is equal to
\begin{equation}
0 
\end{equation}

Let us take a look at this:
\begin{equation}
x ^{2 } 
\end{equation}
Any self-respecting mathematician would find it obvious, that the derivative of this is equal to
\begin{equation}
2 \cdot x ^{2 - 1 } \cdot 1 
\end{equation}

The object of our ultimate interest is the following:
\begin{equation}
2 
\end{equation}
As you can see, the derivative of this is equal to
\begin{equation}
0 
\end{equation}

We will take a closer look at this:
\begin{equation}
2 \cdot x 
\end{equation}
Clearly, the derivative of this is equal to
\begin{equation}
0 \cdot x + 2 \cdot 1 
\end{equation}

We will allow ourselves to divert the reader's attentions to this gem of mathematical wonder:
\begin{equation}
1 
\end{equation}
As you can see, the derivative of this is equal to
\begin{equation}
0 
\end{equation}

Consider the following:
\begin{equation}
x ^{2 } 
\end{equation}
Any self-respecting mathematician would find it obvious, that the derivative of this is equal to
\begin{equation}
2 \cdot x ^{2 - 1 } \cdot 1 
\end{equation}

The following is worth a closer look:
\begin{equation}
2 
\end{equation}
It can be easily proved, that the derivative of this is equal to
\begin{equation}
0 
\end{equation}

We will allow ourselves to divert the reader's attentions to this gem of mathematical wonder:
\begin{equation}
1 
\end{equation}
Any self-respecting mathematician would find it obvious, that the derivative of this is equal to
\begin{equation}
0 
\end{equation}

One shall regard the object in question with utmost interest:
\begin{equation}
2 
\end{equation}
As you can see, the derivative of this is equal to
\begin{equation}
0 
\end{equation}

The object of our ultimate interest is the following:
\begin{equation}
2 
\end{equation}
It can be easily proved, that the derivative of this is equal to
\begin{equation}
0 
\end{equation}

The object of our ultimate interest is the following:
\begin{equation}
1 
\end{equation}
As you can see, the derivative of this is equal to
\begin{equation}
0 
\end{equation}

Consider the following:
\begin{equation}
x ^{2 } 
\end{equation}
Clearly, the derivative of this is equal to
\begin{equation}
2 \cdot x ^{2 - 1 } \cdot 1 
\end{equation}

Let us take a look at this:
\begin{equation}
2 
\end{equation}
It can be easily proved, that the derivative of this is equal to
\begin{equation}
0 
\end{equation}

We are going to study the following:
\begin{equation}
1 
\end{equation}
Obviously, the derivative of this is equal to
\begin{equation}
0 
\end{equation}

We are going to study the following:
\begin{equation}
1 
\end{equation}
Unsurprisingly, the derivative of this is equal to
\begin{equation}
0 
\end{equation}

We will allow ourselves to divert the reader's attentions to this gem of mathematical wonder:
\begin{equation}
x ^{2 } 
\end{equation}
Clearly, the derivative of this is equal to
\begin{equation}
2 \cdot x ^{2 - 1 } \cdot 1 
\end{equation}

We are going to study the following:
\begin{equation}
1 
\end{equation}
Clearly, the derivative of this is equal to
\begin{equation}
0 
\end{equation}

Let us take a look at this:
\begin{equation}
2 \cdot x 
\end{equation}
It can be easily proved, that the derivative of this is equal to
\begin{equation}
0 \cdot x + 2 \cdot 1 
\end{equation}

We will allow ourselves to divert the reader's attentions to this gem of mathematical wonder:
\begin{equation}
1 
\end{equation}
It is now obvious, that the derivative of this is equal to
\begin{equation}
0 
\end{equation}

One shall regard the object in question with utmost interest:
\begin{equation}
x ^{2 } 
\end{equation}
Unsurprisingly, the derivative of this is equal to
\begin{equation}
2 \cdot x ^{2 - 1 } \cdot 1 
\end{equation}

We will take a closer look at this:
\begin{equation}
2 
\end{equation}
Any self-respecting mathematician would find it obvious, that the derivative of this is equal to
\begin{equation}
0 
\end{equation}

Consider the following:
\begin{equation}
1 
\end{equation}
Clearly, the derivative of this is equal to
\begin{equation}
0 
\end{equation}

We are going to study the following:
\begin{equation}
1 
\end{equation}
As you can see, the derivative of this is equal to
\begin{equation}
0 
\end{equation}

We are going to study the following:
\begin{equation}
x ^{2 } 
\end{equation}
Trivially, the derivative of this is equal to
\begin{equation}
2 \cdot x ^{2 - 1 } \cdot 1 
\end{equation}

Let us take a look at this:
\begin{equation}
2 
\end{equation}
Clearly, the derivative of this is equal to
\begin{equation}
0 
\end{equation}

The object of our ultimate interest is the following:
\begin{equation}
2 \cdot x 
\end{equation}
As you can see, the derivative of this is equal to
\begin{equation}
0 \cdot x + 2 \cdot 1 
\end{equation}

We shall ponder the following:
\begin{equation}
1 
\end{equation}
It can be easily proved, that the derivative of this is equal to
\begin{equation}
0 
\end{equation}

Consider the following:
\begin{equation}
x ^{2 } 
\end{equation}
Clearly, the derivative of this is equal to
\begin{equation}
2 \cdot x ^{2 - 1 } \cdot 1 
\end{equation}

Let us take a look at this:
\begin{equation}
2 
\end{equation}
It is now obvious, that the derivative of this is equal to
\begin{equation}
0 
\end{equation}

Let us take a look at this:
\begin{equation}
1 
\end{equation}
As you can see, the derivative of this is equal to
\begin{equation}
0 
\end{equation}

The object of our ultimate interest is the following:
\begin{equation}
1 
\end{equation}
Obviously, the derivative of this is equal to
\begin{equation}
0 
\end{equation}

We shall ponder the following:
\begin{equation}
x ^{2 } 
\end{equation}
Clearly, the derivative of this is equal to
\begin{equation}
2 \cdot x ^{2 - 1 } \cdot 1 
\end{equation}

The object of our ultimate interest is the following:
\begin{equation}
1 
\end{equation}
It is now obvious, that the derivative of this is equal to
\begin{equation}
0 
\end{equation}

Let us take a look at this:
\begin{equation}
2 \cdot x 
\end{equation}
Obviously, the derivative of this is equal to
\begin{equation}
0 \cdot x + 2 \cdot 1 
\end{equation}

The following is worth a closer look:
\begin{equation}
1 
\end{equation}
It can be easily proved, that the derivative of this is equal to
\begin{equation}
0 
\end{equation}

Let us take a look at this:
\begin{equation}
x ^{2 } 
\end{equation}
Trivially, the derivative of this is equal to
\begin{equation}
2 \cdot x ^{2 - 1 } \cdot 1 
\end{equation}

We will take a closer look at this:
\begin{equation}
2 
\end{equation}
Obviously, the derivative of this is equal to
\begin{equation}
0 
\end{equation}

The object of our ultimate interest is the following:
\begin{equation}
1 
\end{equation}
As you can see, the derivative of this is equal to
\begin{equation}
0 
\end{equation}

We are going to study the following:
\begin{equation}
1 
\end{equation}
Obviously, the derivative of this is equal to
\begin{equation}
0 
\end{equation}

The following is worth a closer look:
\begin{equation}
x ^{2 } 
\end{equation}
Obviously, the derivative of this is equal to
\begin{equation}
2 \cdot x ^{2 - 1 } \cdot 1 
\end{equation}

We will allow ourselves to divert the reader's attentions to this gem of mathematical wonder:
\begin{equation}
2 
\end{equation}
It can be easily proved, that the derivative of this is equal to
\begin{equation}
0 
\end{equation}

Let us take a look at this:
\begin{equation}
1 
\end{equation}
It can be easily proved, that the derivative of this is equal to
\begin{equation}
0 
\end{equation}

We will take a closer look at this:
\begin{equation}
x ^{2 } 
\end{equation}
Any self-respecting mathematician would find it obvious, that the derivative of this is equal to
\begin{equation}
2 \cdot x ^{2 - 1 } \cdot 1 
\end{equation}

We shall ponder the following:
\begin{equation}
1 
\end{equation}
As you can see, the derivative of this is equal to
\begin{equation}
0 
\end{equation}

Let us take a look at this:
\begin{equation}
2 
\end{equation}
It is now obvious, that the derivative of this is equal to
\begin{equation}
0 
\end{equation}

Consider the following:
\begin{equation}
2 \cdot x 
\end{equation}
It is now obvious, that the derivative of this is equal to
\begin{equation}
0 \cdot x + 2 \cdot 1 
\end{equation}

We shall ponder the following:
\begin{equation}
1 
\end{equation}
As you can see, the derivative of this is equal to
\begin{equation}
0 
\end{equation}

We will take a closer look at this:
\begin{equation}
x ^{2 } 
\end{equation}
As you can see, the derivative of this is equal to
\begin{equation}
2 \cdot x ^{2 - 1 } \cdot 1 
\end{equation}

We shall ponder the following:
\begin{equation}
2 
\end{equation}
It is now obvious, that the derivative of this is equal to
\begin{equation}
0 
\end{equation}

One shall regard the object in question with utmost interest:
\begin{equation}
1 
\end{equation}
Any self-respecting mathematician would find it obvious, that the derivative of this is equal to
\begin{equation}
0 
\end{equation}

We shall ponder the following:
\begin{equation}
1 
\end{equation}
Clearly, the derivative of this is equal to
\begin{equation}
0 
\end{equation}

Consider the following:
\begin{equation}
x ^{2 } 
\end{equation}
Unsurprisingly, the derivative of this is equal to
\begin{equation}
2 \cdot x ^{2 - 1 } \cdot 1 
\end{equation}

The object of our ultimate interest is the following:
\begin{equation}
2 
\end{equation}
Obviously, the derivative of this is equal to
\begin{equation}
0 
\end{equation}

Let us take a look at this:
\begin{equation}
1 
\end{equation}
Any self-respecting mathematician would find it obvious, that the derivative of this is equal to
\begin{equation}
0 
\end{equation}

The following is worth a closer look:
\begin{equation}
x ^{2 } 
\end{equation}
Trivially, the derivative of this is equal to
\begin{equation}
2 \cdot x ^{2 - 1 } \cdot 1 
\end{equation}

Consider the following:
\begin{equation}
1 
\end{equation}
As you can see, the derivative of this is equal to
\begin{equation}
0 
\end{equation}

The object of our ultimate interest is the following:
\begin{equation}
2 \cdot x 
\end{equation}
It is now obvious, that the derivative of this is equal to
\begin{equation}
0 \cdot x + 2 \cdot 1 
\end{equation}

We shall ponder the following:
\begin{equation}
1 
\end{equation}
Clearly, the derivative of this is equal to
\begin{equation}
0 
\end{equation}

Consider the following:
\begin{equation}
x ^{2 } 
\end{equation}
It is now obvious, that the derivative of this is equal to
\begin{equation}
2 \cdot x ^{2 - 1 } \cdot 1 
\end{equation}

The following is worth a closer look:
\begin{equation}
2 
\end{equation}
It can be easily proved, that the derivative of this is equal to
\begin{equation}
0 
\end{equation}

We are going to study the following:
\begin{equation}
2 \cdot x 
\end{equation}
It is now obvious, that the derivative of this is equal to
\begin{equation}
0 \cdot x + 2 \cdot 1 
\end{equation}

Consider the following:
\begin{equation}
1 
\end{equation}
It can be easily proved, that the derivative of this is equal to
\begin{equation}
0 
\end{equation}

Consider the following:
\begin{equation}
x ^{2 } 
\end{equation}
Any self-respecting mathematician would find it obvious, that the derivative of this is equal to
\begin{equation}
2 \cdot x ^{2 - 1 } \cdot 1 
\end{equation}

We shall ponder the following:
\begin{equation}
2 
\end{equation}
It can be easily proved, that the derivative of this is equal to
\begin{equation}
0 
\end{equation}

The following is worth a closer look:
\begin{equation}
1 
\end{equation}
Trivially, the derivative of this is equal to
\begin{equation}
0 
\end{equation}

Consider the following:
\begin{equation}
2 
\end{equation}
Clearly, the derivative of this is equal to
\begin{equation}
0 
\end{equation}

We shall ponder the following:
\begin{equation}
2 
\end{equation}
Clearly, the derivative of this is equal to
\begin{equation}
0 
\end{equation}

We shall ponder the following:
\begin{equation}
1 
\end{equation}
Any self-respecting mathematician would find it obvious, that the derivative of this is equal to
\begin{equation}
0 
\end{equation}

The object of our ultimate interest is the following:
\begin{equation}
x ^{2 } 
\end{equation}
It can be easily proved, that the derivative of this is equal to
\begin{equation}
2 \cdot x ^{2 - 1 } \cdot 1 
\end{equation}

Let us take a look at this:
\begin{equation}
2 
\end{equation}
Unsurprisingly, the derivative of this is equal to
\begin{equation}
0 
\end{equation}

The following is worth a closer look:
\begin{equation}
1 
\end{equation}
As you can see, the derivative of this is equal to
\begin{equation}
0 
\end{equation}

We are going to study the following:
\begin{equation}
1 
\end{equation}
Trivially, the derivative of this is equal to
\begin{equation}
0 
\end{equation}

One shall regard the object in question with utmost interest:
\begin{equation}
x ^{2 } 
\end{equation}
Obviously, the derivative of this is equal to
\begin{equation}
2 \cdot x ^{2 - 1 } \cdot 1 
\end{equation}

We will take a closer look at this:
\begin{equation}
2 
\end{equation}
Any self-respecting mathematician would find it obvious, that the derivative of this is equal to
\begin{equation}
0 
\end{equation}

We will take a closer look at this:
\begin{equation}
2 \cdot x 
\end{equation}
As you can see, the derivative of this is equal to
\begin{equation}
0 \cdot x + 2 \cdot 1 
\end{equation}

We shall ponder the following:
\begin{equation}
1 
\end{equation}
Trivially, the derivative of this is equal to
\begin{equation}
0 
\end{equation}

Let us take a look at this:
\begin{equation}
x ^{2 } 
\end{equation}
Any self-respecting mathematician would find it obvious, that the derivative of this is equal to
\begin{equation}
2 \cdot x ^{2 - 1 } \cdot 1 
\end{equation}

We are going to study the following:
\begin{equation}
2 
\end{equation}
Clearly, the derivative of this is equal to
\begin{equation}
0 
\end{equation}

Consider the following:
\begin{equation}
1 
\end{equation}
Clearly, the derivative of this is equal to
\begin{equation}
0 
\end{equation}

One shall regard the object in question with utmost interest:
\begin{equation}
2 \cdot x 
\end{equation}
Any self-respecting mathematician would find it obvious, that the derivative of this is equal to
\begin{equation}
0 \cdot x + 2 \cdot 1 
\end{equation}

Let us take a look at this:
\begin{equation}
1 
\end{equation}
Clearly, the derivative of this is equal to
\begin{equation}
0 
\end{equation}

The following is worth a closer look:
\begin{equation}
x ^{2 } 
\end{equation}
Clearly, the derivative of this is equal to
\begin{equation}
2 \cdot x ^{2 - 1 } \cdot 1 
\end{equation}

The following is worth a closer look:
\begin{equation}
2 
\end{equation}
Clearly, the derivative of this is equal to
\begin{equation}
0 
\end{equation}

We will take a closer look at this:
\begin{equation}
1 
\end{equation}
Any self-respecting mathematician would find it obvious, that the derivative of this is equal to
\begin{equation}
0 
\end{equation}

One shall regard the object in question with utmost interest:
\begin{equation}
2 
\end{equation}
Trivially, the derivative of this is equal to
\begin{equation}
0 
\end{equation}

One shall regard the object in question with utmost interest:
\begin{equation}
2 
\end{equation}
It is now obvious, that the derivative of this is equal to
\begin{equation}
0 
\end{equation}

Consider the following:
\begin{equation}
1 
\end{equation}
It can be easily proved, that the derivative of this is equal to
\begin{equation}
0 
\end{equation}

We shall ponder the following:
\begin{equation}
x ^{2 } 
\end{equation}
It can be easily proved, that the derivative of this is equal to
\begin{equation}
2 \cdot x ^{2 - 1 } \cdot 1 
\end{equation}

We shall ponder the following:
\begin{equation}
2 
\end{equation}
It is now obvious, that the derivative of this is equal to
\begin{equation}
0 
\end{equation}

We are going to study the following:
\begin{equation}
2 \cdot x 
\end{equation}
Any self-respecting mathematician would find it obvious, that the derivative of this is equal to
\begin{equation}
0 \cdot x + 2 \cdot 1 
\end{equation}

One shall regard the object in question with utmost interest:
\begin{equation}
1 
\end{equation}
Any self-respecting mathematician would find it obvious, that the derivative of this is equal to
\begin{equation}
0 
\end{equation}

The object of our ultimate interest is the following:
\begin{equation}
x ^{2 } 
\end{equation}
Trivially, the derivative of this is equal to
\begin{equation}
2 \cdot x ^{2 - 1 } \cdot 1 
\end{equation}

Consider the following:
\begin{equation}
2 
\end{equation}
Unsurprisingly, the derivative of this is equal to
\begin{equation}
0 
\end{equation}

The object of our ultimate interest is the following:
\begin{equation}
1 
\end{equation}
It is now obvious, that the derivative of this is equal to
\begin{equation}
0 
\end{equation}

We will allow ourselves to divert the reader's attentions to this gem of mathematical wonder:
\begin{equation}
2 
\end{equation}
It can be easily proved, that the derivative of this is equal to
\begin{equation}
0 
\end{equation}

We will take a closer look at this:
\begin{equation}
2 \cdot x 
\end{equation}
It is now obvious, that the derivative of this is equal to
\begin{equation}
0 \cdot x + 2 \cdot 1 
\end{equation}

The following is worth a closer look:
\begin{equation}
1 
\end{equation}
Obviously, the derivative of this is equal to
\begin{equation}
0 
\end{equation}

We will take a closer look at this:
\begin{equation}
x ^{2 } 
\end{equation}
It can be easily proved, that the derivative of this is equal to
\begin{equation}
2 \cdot x ^{2 - 1 } \cdot 1 
\end{equation}

We shall ponder the following:
\begin{equation}
2 
\end{equation}
Clearly, the derivative of this is equal to
\begin{equation}
0 
\end{equation}

We are going to study the following:
\begin{equation}
2 \cdot x 
\end{equation}
As you can see, the derivative of this is equal to
\begin{equation}
0 \cdot x + 2 \cdot 1 
\end{equation}

We will take a closer look at this:
\begin{equation}
1 
\end{equation}
Clearly, the derivative of this is equal to
\begin{equation}
0 
\end{equation}

Consider the following:
\begin{equation}
x ^{2 } 
\end{equation}
Trivially, the derivative of this is equal to
\begin{equation}
2 \cdot x ^{2 - 1 } \cdot 1 
\end{equation}

One shall regard the object in question with utmost interest:
\begin{equation}
2 
\end{equation}
Any self-respecting mathematician would find it obvious, that the derivative of this is equal to
\begin{equation}
0 
\end{equation}

One shall regard the object in question with utmost interest:
\begin{equation}
1 
\end{equation}
Unsurprisingly, the derivative of this is equal to
\begin{equation}
0 
\end{equation}

Consider the following:
\begin{equation}
2 
\end{equation}
Clearly, the derivative of this is equal to
\begin{equation}
0 
\end{equation}

One shall regard the object in question with utmost interest:
\begin{equation}
1 
\end{equation}
Obviously, the derivative of this is equal to
\begin{equation}
0 
\end{equation}

We will allow ourselves to divert the reader's attentions to this gem of mathematical wonder:
\begin{equation}
x ^{2 } 
\end{equation}
It can be easily proved, that the derivative of this is equal to
\begin{equation}
2 \cdot x ^{2 - 1 } \cdot 1 
\end{equation}

The object of our ultimate interest is the following:
\begin{equation}
2 
\end{equation}
It is now obvious, that the derivative of this is equal to
\begin{equation}
0 
\end{equation}

We shall ponder the following:
\begin{equation}
2 
\end{equation}
It is now obvious, that the derivative of this is equal to
\begin{equation}
0 
\end{equation}

Let us take a look at this:
\begin{equation}
2 \cdot x 
\end{equation}
It can be easily proved, that the derivative of this is equal to
\begin{equation}
0 \cdot x + 2 \cdot 1 
\end{equation}

We shall ponder the following:
\begin{equation}
1 
\end{equation}
It is now obvious, that the derivative of this is equal to
\begin{equation}
0 
\end{equation}

The object of our ultimate interest is the following:
\begin{equation}
x ^{2 } 
\end{equation}
It is now obvious, that the derivative of this is equal to
\begin{equation}
2 \cdot x ^{2 - 1 } \cdot 1 
\end{equation}

We are going to study the following:
\begin{equation}
2 
\end{equation}
It can be easily proved, that the derivative of this is equal to
\begin{equation}
0 
\end{equation}

We shall ponder the following:
\begin{equation}
1 
\end{equation}
Any self-respecting mathematician would find it obvious, that the derivative of this is equal to
\begin{equation}
0 
\end{equation}

Consider the following:
\begin{equation}
2 
\end{equation}
Trivially, the derivative of this is equal to
\begin{equation}
0 
\end{equation}

We will allow ourselves to divert the reader's attentions to this gem of mathematical wonder:
\begin{equation}
1 
\end{equation}
Unsurprisingly, the derivative of this is equal to
\begin{equation}
0 
\end{equation}

The object of our ultimate interest is the following:
\begin{equation}
x ^{2 } 
\end{equation}
It can be easily proved, that the derivative of this is equal to
\begin{equation}
2 \cdot x ^{2 - 1 } \cdot 1 
\end{equation}

We will allow ourselves to divert the reader's attentions to this gem of mathematical wonder:
\begin{equation}
2 
\end{equation}
Any self-respecting mathematician would find it obvious, that the derivative of this is equal to
\begin{equation}
0 
\end{equation}

The following is worth a closer look:
\begin{equation}
2 \cdot x 
\end{equation}
As you can see, the derivative of this is equal to
\begin{equation}
0 \cdot x + 2 \cdot 1 
\end{equation}

We will take a closer look at this:
\begin{equation}
1 
\end{equation}
Any self-respecting mathematician would find it obvious, that the derivative of this is equal to
\begin{equation}
0 
\end{equation}

The object of our ultimate interest is the following:
\begin{equation}
x ^{2 } 
\end{equation}
It is now obvious, that the derivative of this is equal to
\begin{equation}
2 \cdot x ^{2 - 1 } \cdot 1 
\end{equation}

We will allow ourselves to divert the reader's attentions to this gem of mathematical wonder:
\begin{equation}
2 
\end{equation}
Obviously, the derivative of this is equal to
\begin{equation}
0 
\end{equation}

Consider the following:
\begin{equation}
2 \cdot x 
\end{equation}
As you can see, the derivative of this is equal to
\begin{equation}
0 \cdot x + 2 \cdot 1 
\end{equation}

We shall ponder the following:
\begin{equation}
1 
\end{equation}
Trivially, the derivative of this is equal to
\begin{equation}
0 
\end{equation}

We are going to study the following:
\begin{equation}
x ^{2 } 
\end{equation}
It can be easily proved, that the derivative of this is equal to
\begin{equation}
2 \cdot x ^{2 - 1 } \cdot 1 
\end{equation}

We are going to study the following:
\begin{equation}
2 
\end{equation}
Clearly, the derivative of this is equal to
\begin{equation}
0 
\end{equation}

One shall regard the object in question with utmost interest:
\begin{equation}
1 
\end{equation}
As you can see, the derivative of this is equal to
\begin{equation}
0 
\end{equation}

Consider the following:
\begin{equation}
2 
\end{equation}
Clearly, the derivative of this is equal to
\begin{equation}
0 
\end{equation}

We shall ponder the following:
\begin{equation}
2 
\end{equation}
Unsurprisingly, the derivative of this is equal to
\begin{equation}
0 
\end{equation}

We shall ponder the following:
\begin{equation}
1 
\end{equation}
It can be easily proved, that the derivative of this is equal to
\begin{equation}
0 
\end{equation}

We will take a closer look at this:
\begin{equation}
x ^{2 } 
\end{equation}
Clearly, the derivative of this is equal to
\begin{equation}
2 \cdot x ^{2 - 1 } \cdot 1 
\end{equation}

We shall ponder the following:
\begin{equation}
2 
\end{equation}
Any self-respecting mathematician would find it obvious, that the derivative of this is equal to
\begin{equation}
0 
\end{equation}

The following is worth a closer look:
\begin{equation}
1 
\end{equation}
Any self-respecting mathematician would find it obvious, that the derivative of this is equal to
\begin{equation}
0 
\end{equation}

We shall ponder the following:
\begin{equation}
2 
\end{equation}
Clearly, the derivative of this is equal to
\begin{equation}
0 
\end{equation}

We will take a closer look at this:
\begin{equation}
2 
\end{equation}
Obviously, the derivative of this is equal to
\begin{equation}
0 
\end{equation}

We will take a closer look at this:
\begin{equation}
1 
\end{equation}
Any self-respecting mathematician would find it obvious, that the derivative of this is equal to
\begin{equation}
0 
\end{equation}

The following is worth a closer look:
\begin{equation}
x ^{2 } 
\end{equation}
Any self-respecting mathematician would find it obvious, that the derivative of this is equal to
\begin{equation}
2 \cdot x ^{2 - 1 } \cdot 1 
\end{equation}

The object of our ultimate interest is the following:
\begin{equation}
2 
\end{equation}
It can be easily proved, that the derivative of this is equal to
\begin{equation}
0 
\end{equation}

The object of our ultimate interest is the following:
\begin{equation}
2 \cdot x 
\end{equation}
Any self-respecting mathematician would find it obvious, that the derivative of this is equal to
\begin{equation}
0 \cdot x + 2 \cdot 1 
\end{equation}

We will take a closer look at this:
\begin{equation}
1 
\end{equation}
Unsurprisingly, the derivative of this is equal to
\begin{equation}
0 
\end{equation}

We will allow ourselves to divert the reader's attentions to this gem of mathematical wonder:
\begin{equation}
x ^{2 } 
\end{equation}
It is now obvious, that the derivative of this is equal to
\begin{equation}
2 \cdot x ^{2 - 1 } \cdot 1 
\end{equation}

We shall ponder the following:
\begin{equation}
2 
\end{equation}
It can be easily proved, that the derivative of this is equal to
\begin{equation}
0 
\end{equation}

We shall ponder the following:
\begin{equation}
1 
\end{equation}
As you can see, the derivative of this is equal to
\begin{equation}
0 
\end{equation}

The following is worth a closer look:
\begin{equation}
2 
\end{equation}
Trivially, the derivative of this is equal to
\begin{equation}
0 
\end{equation}

Let us take a look at this:
\begin{equation}
1 
\end{equation}
Unsurprisingly, the derivative of this is equal to
\begin{equation}
0 
\end{equation}

We will take a closer look at this:
\begin{equation}
x ^{2 } 
\end{equation}
Any self-respecting mathematician would find it obvious, that the derivative of this is equal to
\begin{equation}
2 \cdot x ^{2 - 1 } \cdot 1 
\end{equation}

Let us take a look at this:
\begin{equation}
1 
\end{equation}
Any self-respecting mathematician would find it obvious, that the derivative of this is equal to
\begin{equation}
0 
\end{equation}

The object of our ultimate interest is the following:
\begin{equation}
1 
\end{equation}
As you can see, the derivative of this is equal to
\begin{equation}
0 
\end{equation}

The object of our ultimate interest is the following:
\begin{equation}
2 \cdot x 
\end{equation}
Clearly, the derivative of this is equal to
\begin{equation}
0 \cdot x + 2 \cdot 1 
\end{equation}

One shall regard the object in question with utmost interest:
\begin{equation}
1 
\end{equation}
Any self-respecting mathematician would find it obvious, that the derivative of this is equal to
\begin{equation}
0 
\end{equation}

One shall regard the object in question with utmost interest:
\begin{equation}
x ^{2 } 
\end{equation}
Any self-respecting mathematician would find it obvious, that the derivative of this is equal to
\begin{equation}
2 \cdot x ^{2 - 1 } \cdot 1 
\end{equation}

We will take a closer look at this:
\begin{equation}
2 
\end{equation}
It is now obvious, that the derivative of this is equal to
\begin{equation}
0 
\end{equation}

We shall ponder the following:
\begin{equation}
2 \cdot x 
\end{equation}
Obviously, the derivative of this is equal to
\begin{equation}
0 \cdot x + 2 \cdot 1 
\end{equation}

We will take a closer look at this:
\begin{equation}
1 
\end{equation}
Obviously, the derivative of this is equal to
\begin{equation}
0 
\end{equation}

The object of our ultimate interest is the following:
\begin{equation}
x ^{2 } 
\end{equation}
Any self-respecting mathematician would find it obvious, that the derivative of this is equal to
\begin{equation}
2 \cdot x ^{2 - 1 } \cdot 1 
\end{equation}

Let us take a look at this:
\begin{equation}
2 
\end{equation}
Clearly, the derivative of this is equal to
\begin{equation}
0 
\end{equation}

The following is worth a closer look:
\begin{equation}
1 
\end{equation}
Any self-respecting mathematician would find it obvious, that the derivative of this is equal to
\begin{equation}
0 
\end{equation}

The following is worth a closer look:
\begin{equation}
2 
\end{equation}
Obviously, the derivative of this is equal to
\begin{equation}
0 
\end{equation}

The object of our ultimate interest is the following:
\begin{equation}
2 
\end{equation}
It is now obvious, that the derivative of this is equal to
\begin{equation}
0 
\end{equation}

Consider the following:
\begin{equation}
1 
\end{equation}
Obviously, the derivative of this is equal to
\begin{equation}
0 
\end{equation}

We are going to study the following:
\begin{equation}
x ^{2 } 
\end{equation}
Unsurprisingly, the derivative of this is equal to
\begin{equation}
2 \cdot x ^{2 - 1 } \cdot 1 
\end{equation}

We will allow ourselves to divert the reader's attentions to this gem of mathematical wonder:
\begin{equation}
2 
\end{equation}
As you can see, the derivative of this is equal to
\begin{equation}
0 
\end{equation}

The following is worth a closer look:
\begin{equation}
1 
\end{equation}
Unsurprisingly, the derivative of this is equal to
\begin{equation}
0 
\end{equation}

We shall ponder the following:
\begin{equation}
1 
\end{equation}
It can be easily proved, that the derivative of this is equal to
\begin{equation}
0 
\end{equation}

We will allow ourselves to divert the reader's attentions to this gem of mathematical wonder:
\begin{equation}
x ^{2 } 
\end{equation}
Trivially, the derivative of this is equal to
\begin{equation}
2 \cdot x ^{2 - 1 } \cdot 1 
\end{equation}

We are going to study the following:
\begin{equation}
1 
\end{equation}
Trivially, the derivative of this is equal to
\begin{equation}
0 
\end{equation}

The following is worth a closer look:
\begin{equation}
2 \cdot x 
\end{equation}
Any self-respecting mathematician would find it obvious, that the derivative of this is equal to
\begin{equation}
0 \cdot x + 2 \cdot 1 
\end{equation}

The object of our ultimate interest is the following:
\begin{equation}
1 
\end{equation}
It can be easily proved, that the derivative of this is equal to
\begin{equation}
0 
\end{equation}

We will allow ourselves to divert the reader's attentions to this gem of mathematical wonder:
\begin{equation}
x ^{2 } 
\end{equation}
Unsurprisingly, the derivative of this is equal to
\begin{equation}
2 \cdot x ^{2 - 1 } \cdot 1 
\end{equation}

We will take a closer look at this:
\begin{equation}
2 
\end{equation}
Trivially, the derivative of this is equal to
\begin{equation}
0 
\end{equation}

We shall ponder the following:
\begin{equation}
1 
\end{equation}
Unsurprisingly, the derivative of this is equal to
\begin{equation}
0 
\end{equation}

One shall regard the object in question with utmost interest:
\begin{equation}
1 
\end{equation}
Trivially, the derivative of this is equal to
\begin{equation}
0 
\end{equation}

The object of our ultimate interest is the following:
\begin{equation}
x ^{2 } 
\end{equation}
Any self-respecting mathematician would find it obvious, that the derivative of this is equal to
\begin{equation}
2 \cdot x ^{2 - 1 } \cdot 1 
\end{equation}

We are going to study the following:
\begin{equation}
2 
\end{equation}
Clearly, the derivative of this is equal to
\begin{equation}
0 
\end{equation}

One shall regard the object in question with utmost interest:
\begin{equation}
2 \cdot x 
\end{equation}
Unsurprisingly, the derivative of this is equal to
\begin{equation}
0 \cdot x + 2 \cdot 1 
\end{equation}

The following is worth a closer look:
\begin{equation}
1 
\end{equation}
Any self-respecting mathematician would find it obvious, that the derivative of this is equal to
\begin{equation}
0 
\end{equation}

The following is worth a closer look:
\begin{equation}
x ^{2 } 
\end{equation}
It can be easily proved, that the derivative of this is equal to
\begin{equation}
2 \cdot x ^{2 - 1 } \cdot 1 
\end{equation}

One shall regard the object in question with utmost interest:
\begin{equation}
2 
\end{equation}
Any self-respecting mathematician would find it obvious, that the derivative of this is equal to
\begin{equation}
0 
\end{equation}

The object of our ultimate interest is the following:
\begin{equation}
1 
\end{equation}
Trivially, the derivative of this is equal to
\begin{equation}
0 
\end{equation}

Let us take a look at this:
\begin{equation}
1 
\end{equation}
Any self-respecting mathematician would find it obvious, that the derivative of this is equal to
\begin{equation}
0 
\end{equation}

We are going to study the following:
\begin{equation}
x ^{2 } 
\end{equation}
Unsurprisingly, the derivative of this is equal to
\begin{equation}
2 \cdot x ^{2 - 1 } \cdot 1 
\end{equation}

Let us take a look at this:
\begin{equation}
1 
\end{equation}
Trivially, the derivative of this is equal to
\begin{equation}
0 
\end{equation}

We are going to study the following:
\begin{equation}
1 
\end{equation}
It is now obvious, that the derivative of this is equal to
\begin{equation}
0 
\end{equation}

Consider the following:
\begin{equation}
2 \cdot x 
\end{equation}
Trivially, the derivative of this is equal to
\begin{equation}
0 \cdot x + 2 \cdot 1 
\end{equation}

The object of our ultimate interest is the following:
\begin{equation}
1 
\end{equation}
Unsurprisingly, the derivative of this is equal to
\begin{equation}
0 
\end{equation}

We shall ponder the following:
\begin{equation}
x ^{2 } 
\end{equation}
Obviously, the derivative of this is equal to
\begin{equation}
2 \cdot x ^{2 - 1 } \cdot 1 
\end{equation}

We will allow ourselves to divert the reader's attentions to this gem of mathematical wonder:
\begin{equation}
2 
\end{equation}
Any self-respecting mathematician would find it obvious, that the derivative of this is equal to
\begin{equation}
0 
\end{equation}

Let us take a look at this:
\begin{equation}
1 
\end{equation}
Any self-respecting mathematician would find it obvious, that the derivative of this is equal to
\begin{equation}
0 
\end{equation}

One shall regard the object in question with utmost interest:
\begin{equation}
1 
\end{equation}
It is now obvious, that the derivative of this is equal to
\begin{equation}
0 
\end{equation}

We will allow ourselves to divert the reader's attentions to this gem of mathematical wonder:
\begin{equation}
x ^{2 } 
\end{equation}
Obviously, the derivative of this is equal to
\begin{equation}
2 \cdot x ^{2 - 1 } \cdot 1 
\end{equation}

We will take a closer look at this:
\begin{equation}
1 
\end{equation}
Clearly, the derivative of this is equal to
\begin{equation}
0 
\end{equation}

We will take a closer look at this:
\begin{equation}
2 \cdot x 
\end{equation}
Unsurprisingly, the derivative of this is equal to
\begin{equation}
0 \cdot x + 2 \cdot 1 
\end{equation}

One shall regard the object in question with utmost interest:
\begin{equation}
1 
\end{equation}
As you can see, the derivative of this is equal to
\begin{equation}
0 
\end{equation}

The following is worth a closer look:
\begin{equation}
x ^{2 } 
\end{equation}
Obviously, the derivative of this is equal to
\begin{equation}
2 \cdot x ^{2 - 1 } \cdot 1 
\end{equation}

One shall regard the object in question with utmost interest:
\begin{equation}
2 
\end{equation}
It is now obvious, that the derivative of this is equal to
\begin{equation}
0 
\end{equation}

We are going to study the following:
\begin{equation}
1 
\end{equation}
It is now obvious, that the derivative of this is equal to
\begin{equation}
0 
\end{equation}

Consider the following:
\begin{equation}
1 
\end{equation}
It is now obvious, that the derivative of this is equal to
\begin{equation}
0 
\end{equation}

We will allow ourselves to divert the reader's attentions to this gem of mathematical wonder:
\begin{equation}
x ^{2 } 
\end{equation}
Clearly, the derivative of this is equal to
\begin{equation}
2 \cdot x ^{2 - 1 } \cdot 1 
\end{equation}

Let us take a look at this:
\begin{equation}
2 
\end{equation}
Obviously, the derivative of this is equal to
\begin{equation}
0 
\end{equation}

We are going to study the following:
\begin{equation}
2 \cdot x 
\end{equation}
It is now obvious, that the derivative of this is equal to
\begin{equation}
0 \cdot x + 2 \cdot 1 
\end{equation}

We are going to study the following:
\begin{equation}
1 
\end{equation}
It can be easily proved, that the derivative of this is equal to
\begin{equation}
0 
\end{equation}

The following is worth a closer look:
\begin{equation}
x ^{2 } 
\end{equation}
Clearly, the derivative of this is equal to
\begin{equation}
2 \cdot x ^{2 - 1 } \cdot 1 
\end{equation}

The following is worth a closer look:
\begin{equation}
2 
\end{equation}
Any self-respecting mathematician would find it obvious, that the derivative of this is equal to
\begin{equation}
0 
\end{equation}

We are going to study the following:
\begin{equation}
2 \cdot x 
\end{equation}
It can be easily proved, that the derivative of this is equal to
\begin{equation}
0 \cdot x + 2 \cdot 1 
\end{equation}

Let us take a look at this:
\begin{equation}
1 
\end{equation}
Obviously, the derivative of this is equal to
\begin{equation}
0 
\end{equation}

One shall regard the object in question with utmost interest:
\begin{equation}
x ^{2 } 
\end{equation}
Trivially, the derivative of this is equal to
\begin{equation}
2 \cdot x ^{2 - 1 } \cdot 1 
\end{equation}

We will allow ourselves to divert the reader's attentions to this gem of mathematical wonder:
\begin{equation}
2 
\end{equation}
It is now obvious, that the derivative of this is equal to
\begin{equation}
0 
\end{equation}

Consider the following:
\begin{equation}
1 
\end{equation}
It can be easily proved, that the derivative of this is equal to
\begin{equation}
0 
\end{equation}

We will take a closer look at this:
\begin{equation}
2 
\end{equation}
Any self-respecting mathematician would find it obvious, that the derivative of this is equal to
\begin{equation}
0 
\end{equation}

Consider the following:
\begin{equation}
2 
\end{equation}
It is now obvious, that the derivative of this is equal to
\begin{equation}
0 
\end{equation}

We shall ponder the following:
\begin{equation}
1 
\end{equation}
As you can see, the derivative of this is equal to
\begin{equation}
0 
\end{equation}

The following is worth a closer look:
\begin{equation}
x ^{2 } 
\end{equation}
Trivially, the derivative of this is equal to
\begin{equation}
2 \cdot x ^{2 - 1 } \cdot 1 
\end{equation}

The following is worth a closer look:
\begin{equation}
2 
\end{equation}
Obviously, the derivative of this is equal to
\begin{equation}
0 
\end{equation}

Consider the following:
\begin{equation}
1 
\end{equation}
Unsurprisingly, the derivative of this is equal to
\begin{equation}
0 
\end{equation}

The following is worth a closer look:
\begin{equation}
1 
\end{equation}
Any self-respecting mathematician would find it obvious, that the derivative of this is equal to
\begin{equation}
0 
\end{equation}

We are going to study the following:
\begin{equation}
x ^{2 } 
\end{equation}
Clearly, the derivative of this is equal to
\begin{equation}
2 \cdot x ^{2 - 1 } \cdot 1 
\end{equation}

One shall regard the object in question with utmost interest:
\begin{equation}
1 
\end{equation}
It can be easily proved, that the derivative of this is equal to
\begin{equation}
0 
\end{equation}

The object of our ultimate interest is the following:
\begin{equation}
1 
\end{equation}
Trivially, the derivative of this is equal to
\begin{equation}
0 
\end{equation}

Let us take a look at this:
\begin{equation}
x - 2 
\end{equation}
It is now obvious, that the derivative of this is equal to
\begin{equation}
1 - 0 
\end{equation}

Let us take a look at this:
\begin{equation}
1 
\end{equation}
Trivially, the derivative of this is equal to
\begin{equation}
0 
\end{equation}

We will take a closer look at this:
\begin{equation}
x ^{2 } 
\end{equation}
Unsurprisingly, the derivative of this is equal to
\begin{equation}
2 \cdot x ^{2 - 1 } \cdot 1 
\end{equation}

Consider the following:
\begin{equation}
2 \cdot x 
\end{equation}
Obviously, the derivative of this is equal to
\begin{equation}
0 \cdot x + 2 \cdot 1 
\end{equation}

Let us take a look at this:
\begin{equation}
1 
\end{equation}
As you can see, the derivative of this is equal to
\begin{equation}
0 
\end{equation}

The object of our ultimate interest is the following:
\begin{equation}
x ^{2 } 
\end{equation}
Any self-respecting mathematician would find it obvious, that the derivative of this is equal to
\begin{equation}
2 \cdot x ^{2 - 1 } \cdot 1 
\end{equation}

Consider the following:
\begin{equation}
2 
\end{equation}
As you can see, the derivative of this is equal to
\begin{equation}
0 
\end{equation}

We will take a closer look at this:
\begin{equation}
1 
\end{equation}
Clearly, the derivative of this is equal to
\begin{equation}
0 
\end{equation}

We shall ponder the following:
\begin{equation}
1 
\end{equation}
Obviously, the derivative of this is equal to
\begin{equation}
0 
\end{equation}

We are going to study the following:
\begin{equation}
x ^{2 } 
\end{equation}
It can be easily proved, that the derivative of this is equal to
\begin{equation}
2 \cdot x ^{2 - 1 } \cdot 1 
\end{equation}

Let us take a look at this:
\begin{equation}
1 
\end{equation}
Clearly, the derivative of this is equal to
\begin{equation}
0 
\end{equation}

One shall regard the object in question with utmost interest:
\begin{equation}
1 
\end{equation}
Clearly, the derivative of this is equal to
\begin{equation}
0 
\end{equation}

We will take a closer look at this:
\begin{equation}
2 \cdot x 
\end{equation}
It is now obvious, that the derivative of this is equal to
\begin{equation}
0 \cdot x + 2 \cdot 1 
\end{equation}

The object of our ultimate interest is the following:
\begin{equation}
1 
\end{equation}
Any self-respecting mathematician would find it obvious, that the derivative of this is equal to
\begin{equation}
0 
\end{equation}

We will take a closer look at this:
\begin{equation}
x ^{2 } 
\end{equation}
Unsurprisingly, the derivative of this is equal to
\begin{equation}
2 \cdot x ^{2 - 1 } \cdot 1 
\end{equation}

Let us take a look at this:
\begin{equation}
2 
\end{equation}
As you can see, the derivative of this is equal to
\begin{equation}
0 
\end{equation}

One shall regard the object in question with utmost interest:
\begin{equation}
1 
\end{equation}
Any self-respecting mathematician would find it obvious, that the derivative of this is equal to
\begin{equation}
0 
\end{equation}

Consider the following:
\begin{equation}
1 
\end{equation}
It can be easily proved, that the derivative of this is equal to
\begin{equation}
0 
\end{equation}

We shall ponder the following:
\begin{equation}
x ^{2 } 
\end{equation}
As you can see, the derivative of this is equal to
\begin{equation}
2 \cdot x ^{2 - 1 } \cdot 1 
\end{equation}

The following is worth a closer look:
\begin{equation}
1 
\end{equation}
Any self-respecting mathematician would find it obvious, that the derivative of this is equal to
\begin{equation}
0 
\end{equation}

We will allow ourselves to divert the reader's attentions to this gem of mathematical wonder:
\begin{equation}
1 
\end{equation}
Obviously, the derivative of this is equal to
\begin{equation}
0 
\end{equation}

We will take a closer look at this:
\begin{equation}
1 
\end{equation}
Unsurprisingly, the derivative of this is equal to
\begin{equation}
0 
\end{equation}

Let us take a look at this:
\begin{equation}
x ^{2 } 
\end{equation}
Clearly, the derivative of this is equal to
\begin{equation}
2 \cdot x ^{2 - 1 } \cdot 1 
\end{equation}

Let us take a look at this:
\begin{equation}
2 \cdot x 
\end{equation}
Any self-respecting mathematician would find it obvious, that the derivative of this is equal to
\begin{equation}
0 \cdot x + 2 \cdot 1 
\end{equation}

Consider the following:
\begin{equation}
1 
\end{equation}
As you can see, the derivative of this is equal to
\begin{equation}
0 
\end{equation}

Let us take a look at this:
\begin{equation}
x ^{2 } 
\end{equation}
Clearly, the derivative of this is equal to
\begin{equation}
2 \cdot x ^{2 - 1 } \cdot 1 
\end{equation}

We will take a closer look at this:
\begin{equation}
2 
\end{equation}
It can be easily proved, that the derivative of this is equal to
\begin{equation}
0 
\end{equation}

Let us take a look at this:
\begin{equation}
1 
\end{equation}
It is now obvious, that the derivative of this is equal to
\begin{equation}
0 
\end{equation}

We are going to study the following:
\begin{equation}
1 
\end{equation}
Unsurprisingly, the derivative of this is equal to
\begin{equation}
0 
\end{equation}

One shall regard the object in question with utmost interest:
\begin{equation}
x ^{2 } 
\end{equation}
It can be easily proved, that the derivative of this is equal to
\begin{equation}
2 \cdot x ^{2 - 1 } \cdot 1 
\end{equation}

We will take a closer look at this:
\begin{equation}
1 
\end{equation}
Unsurprisingly, the derivative of this is equal to
\begin{equation}
0 
\end{equation}

Let us take a look at this:
\begin{equation}
2 \cdot x 
\end{equation}
Trivially, the derivative of this is equal to
\begin{equation}
0 \cdot x + 2 \cdot 1 
\end{equation}

The following is worth a closer look:
\begin{equation}
1 
\end{equation}
Obviously, the derivative of this is equal to
\begin{equation}
0 
\end{equation}

Consider the following:
\begin{equation}
x ^{2 } 
\end{equation}
Clearly, the derivative of this is equal to
\begin{equation}
2 \cdot x ^{2 - 1 } \cdot 1 
\end{equation}

We are going to study the following:
\begin{equation}
2 
\end{equation}
It is now obvious, that the derivative of this is equal to
\begin{equation}
0 
\end{equation}

We will take a closer look at this:
\begin{equation}
1 
\end{equation}
As you can see, the derivative of this is equal to
\begin{equation}
0 
\end{equation}

Let us take a look at this:
\begin{equation}
1 
\end{equation}
Obviously, the derivative of this is equal to
\begin{equation}
0 
\end{equation}

We are going to study the following:
\begin{equation}
x ^{2 } 
\end{equation}
It can be easily proved, that the derivative of this is equal to
\begin{equation}
2 \cdot x ^{2 - 1 } \cdot 1 
\end{equation}

One shall regard the object in question with utmost interest:
\begin{equation}
2 
\end{equation}
Unsurprisingly, the derivative of this is equal to
\begin{equation}
0 
\end{equation}

We will take a closer look at this:
\begin{equation}
2 \cdot x 
\end{equation}
Trivially, the derivative of this is equal to
\begin{equation}
0 \cdot x + 2 \cdot 1 
\end{equation}

We shall ponder the following:
\begin{equation}
1 
\end{equation}
Obviously, the derivative of this is equal to
\begin{equation}
0 
\end{equation}

One shall regard the object in question with utmost interest:
\begin{equation}
x ^{2 } 
\end{equation}
It is now obvious, that the derivative of this is equal to
\begin{equation}
2 \cdot x ^{2 - 1 } \cdot 1 
\end{equation}

One shall regard the object in question with utmost interest:
\begin{equation}
2 
\end{equation}
It is now obvious, that the derivative of this is equal to
\begin{equation}
0 
\end{equation}

One shall regard the object in question with utmost interest:
\begin{equation}
2 \cdot x 
\end{equation}
Unsurprisingly, the derivative of this is equal to
\begin{equation}
0 \cdot x + 2 \cdot 1 
\end{equation}

Let us take a look at this:
\begin{equation}
1 
\end{equation}
Obviously, the derivative of this is equal to
\begin{equation}
0 
\end{equation}

Consider the following:
\begin{equation}
x ^{2 } 
\end{equation}
Any self-respecting mathematician would find it obvious, that the derivative of this is equal to
\begin{equation}
2 \cdot x ^{2 - 1 } \cdot 1 
\end{equation}

One shall regard the object in question with utmost interest:
\begin{equation}
2 
\end{equation}
It is now obvious, that the derivative of this is equal to
\begin{equation}
0 
\end{equation}

We are going to study the following:
\begin{equation}
1 
\end{equation}
It is now obvious, that the derivative of this is equal to
\begin{equation}
0 
\end{equation}

One shall regard the object in question with utmost interest:
\begin{equation}
2 
\end{equation}
Any self-respecting mathematician would find it obvious, that the derivative of this is equal to
\begin{equation}
0 
\end{equation}

We are going to study the following:
\begin{equation}
2 
\end{equation}
Any self-respecting mathematician would find it obvious, that the derivative of this is equal to
\begin{equation}
0 
\end{equation}

We are going to study the following:
\begin{equation}
1 
\end{equation}
Any self-respecting mathematician would find it obvious, that the derivative of this is equal to
\begin{equation}
0 
\end{equation}

Consider the following:
\begin{equation}
x ^{2 } 
\end{equation}
It can be easily proved, that the derivative of this is equal to
\begin{equation}
2 \cdot x ^{2 - 1 } \cdot 1 
\end{equation}

The object of our ultimate interest is the following:
\begin{equation}
2 
\end{equation}
Clearly, the derivative of this is equal to
\begin{equation}
0 
\end{equation}

One shall regard the object in question with utmost interest:
\begin{equation}
1 
\end{equation}
Trivially, the derivative of this is equal to
\begin{equation}
0 
\end{equation}

The following is worth a closer look:
\begin{equation}
1 
\end{equation}
Unsurprisingly, the derivative of this is equal to
\begin{equation}
0 
\end{equation}

Consider the following:
\begin{equation}
x ^{2 } 
\end{equation}
Obviously, the derivative of this is equal to
\begin{equation}
2 \cdot x ^{2 - 1 } \cdot 1 
\end{equation}

The following is worth a closer look:
\begin{equation}
1 
\end{equation}
Unsurprisingly, the derivative of this is equal to
\begin{equation}
0 
\end{equation}

One shall regard the object in question with utmost interest:
\begin{equation}
1 
\end{equation}
Unsurprisingly, the derivative of this is equal to
\begin{equation}
0 
\end{equation}

The object of our ultimate interest is the following:
\begin{equation}
1 
\end{equation}
Clearly, the derivative of this is equal to
\begin{equation}
0 
\end{equation}

We shall ponder the following:
\begin{equation}
x ^{2 } 
\end{equation}
Trivially, the derivative of this is equal to
\begin{equation}
2 \cdot x ^{2 - 1 } \cdot 1 
\end{equation}

One shall regard the object in question with utmost interest:
\begin{equation}
2 \cdot x 
\end{equation}
It can be easily proved, that the derivative of this is equal to
\begin{equation}
0 \cdot x + 2 \cdot 1 
\end{equation}

We will allow ourselves to divert the reader's attentions to this gem of mathematical wonder:
\begin{equation}
1 
\end{equation}
Unsurprisingly, the derivative of this is equal to
\begin{equation}
0 
\end{equation}

One shall regard the object in question with utmost interest:
\begin{equation}
x ^{2 } 
\end{equation}
Any self-respecting mathematician would find it obvious, that the derivative of this is equal to
\begin{equation}
2 \cdot x ^{2 - 1 } \cdot 1 
\end{equation}

Let us take a look at this:
\begin{equation}
2 
\end{equation}
It can be easily proved, that the derivative of this is equal to
\begin{equation}
0 
\end{equation}

Let us take a look at this:
\begin{equation}
1 
\end{equation}
Clearly, the derivative of this is equal to
\begin{equation}
0 
\end{equation}

Let us take a look at this:
\begin{equation}
1 
\end{equation}
As you can see, the derivative of this is equal to
\begin{equation}
0 
\end{equation}

The following is worth a closer look:
\begin{equation}
x ^{2 } 
\end{equation}
It can be easily proved, that the derivative of this is equal to
\begin{equation}
2 \cdot x ^{2 - 1 } \cdot 1 
\end{equation}

One shall regard the object in question with utmost interest:
\begin{equation}
1 
\end{equation}
It is now obvious, that the derivative of this is equal to
\begin{equation}
0 
\end{equation}

The following is worth a closer look:
\begin{equation}
1 
\end{equation}
Any self-respecting mathematician would find it obvious, that the derivative of this is equal to
\begin{equation}
0 
\end{equation}

Let us take a look at this:
\begin{equation}
2 \cdot x 
\end{equation}
It can be easily proved, that the derivative of this is equal to
\begin{equation}
0 \cdot x + 2 \cdot 1 
\end{equation}

Consider the following:
\begin{equation}
1 
\end{equation}
As you can see, the derivative of this is equal to
\begin{equation}
0 
\end{equation}

The following is worth a closer look:
\begin{equation}
x ^{2 } 
\end{equation}
It is now obvious, that the derivative of this is equal to
\begin{equation}
2 \cdot x ^{2 - 1 } \cdot 1 
\end{equation}

One shall regard the object in question with utmost interest:
\begin{equation}
2 
\end{equation}
It can be easily proved, that the derivative of this is equal to
\begin{equation}
0 
\end{equation}

We are going to study the following:
\begin{equation}
1 
\end{equation}
Trivially, the derivative of this is equal to
\begin{equation}
0 
\end{equation}

Consider the following:
\begin{equation}
1 
\end{equation}
Unsurprisingly, the derivative of this is equal to
\begin{equation}
0 
\end{equation}

We will allow ourselves to divert the reader's attentions to this gem of mathematical wonder:
\begin{equation}
x ^{2 } 
\end{equation}
Any self-respecting mathematician would find it obvious, that the derivative of this is equal to
\begin{equation}
2 \cdot x ^{2 - 1 } \cdot 1 
\end{equation}

The following is worth a closer look:
\begin{equation}
1 
\end{equation}
Any self-respecting mathematician would find it obvious, that the derivative of this is equal to
\begin{equation}
0 
\end{equation}

The following is worth a closer look:
\begin{equation}
1 
\end{equation}
Clearly, the derivative of this is equal to
\begin{equation}
0 
\end{equation}

We are going to study the following:
\begin{equation}
1 
\end{equation}
Trivially, the derivative of this is equal to
\begin{equation}
0 
\end{equation}

We will allow ourselves to divert the reader's attentions to this gem of mathematical wonder:
\begin{equation}
x ^{2 } 
\end{equation}
Any self-respecting mathematician would find it obvious, that the derivative of this is equal to
\begin{equation}
2 \cdot x ^{2 - 1 } \cdot 1 
\end{equation}

The object of our ultimate interest is the following:
\begin{equation}
2 \cdot x 
\end{equation}
As you can see, the derivative of this is equal to
\begin{equation}
0 \cdot x + 2 \cdot 1 
\end{equation}

We will allow ourselves to divert the reader's attentions to this gem of mathematical wonder:
\begin{equation}
1 
\end{equation}
Trivially, the derivative of this is equal to
\begin{equation}
0 
\end{equation}

We are going to study the following:
\begin{equation}
x ^{2 } 
\end{equation}
Obviously, the derivative of this is equal to
\begin{equation}
2 \cdot x ^{2 - 1 } \cdot 1 
\end{equation}

The following is worth a closer look:
\begin{equation}
2 
\end{equation}
Any self-respecting mathematician would find it obvious, that the derivative of this is equal to
\begin{equation}
0 
\end{equation}

The object of our ultimate interest is the following:
\begin{equation}
1 
\end{equation}
Trivially, the derivative of this is equal to
\begin{equation}
0 
\end{equation}

Consider the following:
\begin{equation}
1 
\end{equation}
Clearly, the derivative of this is equal to
\begin{equation}
0 
\end{equation}

The object of our ultimate interest is the following:
\begin{equation}
x ^{2 } 
\end{equation}
Obviously, the derivative of this is equal to
\begin{equation}
2 \cdot x ^{2 - 1 } \cdot 1 
\end{equation}

We shall ponder the following:
\begin{equation}
1 
\end{equation}
Clearly, the derivative of this is equal to
\begin{equation}
0 
\end{equation}

We will allow ourselves to divert the reader's attentions to this gem of mathematical wonder:
\begin{equation}
2 \cdot x 
\end{equation}
Clearly, the derivative of this is equal to
\begin{equation}
0 \cdot x + 2 \cdot 1 
\end{equation}

We shall ponder the following:
\begin{equation}
1 
\end{equation}
Trivially, the derivative of this is equal to
\begin{equation}
0 
\end{equation}

We will take a closer look at this:
\begin{equation}
x ^{2 } 
\end{equation}
It is now obvious, that the derivative of this is equal to
\begin{equation}
2 \cdot x ^{2 - 1 } \cdot 1 
\end{equation}

We will take a closer look at this:
\begin{equation}
2 
\end{equation}
Any self-respecting mathematician would find it obvious, that the derivative of this is equal to
\begin{equation}
0 
\end{equation}

The object of our ultimate interest is the following:
\begin{equation}
1 
\end{equation}
Obviously, the derivative of this is equal to
\begin{equation}
0 
\end{equation}

One shall regard the object in question with utmost interest:
\begin{equation}
1 
\end{equation}
It can be easily proved, that the derivative of this is equal to
\begin{equation}
0 
\end{equation}

One shall regard the object in question with utmost interest:
\begin{equation}
x ^{2 } 
\end{equation}
As you can see, the derivative of this is equal to
\begin{equation}
2 \cdot x ^{2 - 1 } \cdot 1 
\end{equation}

One shall regard the object in question with utmost interest:
\begin{equation}
2 
\end{equation}
As you can see, the derivative of this is equal to
\begin{equation}
0 
\end{equation}

One shall regard the object in question with utmost interest:
\begin{equation}
2 \cdot x 
\end{equation}
It can be easily proved, that the derivative of this is equal to
\begin{equation}
0 \cdot x + 2 \cdot 1 
\end{equation}

The following is worth a closer look:
\begin{equation}
1 
\end{equation}
Any self-respecting mathematician would find it obvious, that the derivative of this is equal to
\begin{equation}
0 
\end{equation}

One shall regard the object in question with utmost interest:
\begin{equation}
x ^{2 } 
\end{equation}
Trivially, the derivative of this is equal to
\begin{equation}
2 \cdot x ^{2 - 1 } \cdot 1 
\end{equation}

We shall ponder the following:
\begin{equation}
2 
\end{equation}
As you can see, the derivative of this is equal to
\begin{equation}
0 
\end{equation}

We are going to study the following:
\begin{equation}
2 \cdot x 
\end{equation}
It can be easily proved, that the derivative of this is equal to
\begin{equation}
0 \cdot x + 2 \cdot 1 
\end{equation}

The object of our ultimate interest is the following:
\begin{equation}
1 
\end{equation}
It can be easily proved, that the derivative of this is equal to
\begin{equation}
0 
\end{equation}

We are going to study the following:
\begin{equation}
x ^{2 } 
\end{equation}
As you can see, the derivative of this is equal to
\begin{equation}
2 \cdot x ^{2 - 1 } \cdot 1 
\end{equation}

We will take a closer look at this:
\begin{equation}
2 
\end{equation}
As you can see, the derivative of this is equal to
\begin{equation}
0 
\end{equation}

We will take a closer look at this:
\begin{equation}
1 
\end{equation}
Trivially, the derivative of this is equal to
\begin{equation}
0 
\end{equation}

The following is worth a closer look:
\begin{equation}
2 
\end{equation}
Any self-respecting mathematician would find it obvious, that the derivative of this is equal to
\begin{equation}
0 
\end{equation}

Let us take a look at this:
\begin{equation}
2 
\end{equation}
Unsurprisingly, the derivative of this is equal to
\begin{equation}
0 
\end{equation}

We will take a closer look at this:
\begin{equation}
1 
\end{equation}
Unsurprisingly, the derivative of this is equal to
\begin{equation}
0 
\end{equation}

One shall regard the object in question with utmost interest:
\begin{equation}
x ^{2 } 
\end{equation}
Unsurprisingly, the derivative of this is equal to
\begin{equation}
2 \cdot x ^{2 - 1 } \cdot 1 
\end{equation}

We will allow ourselves to divert the reader's attentions to this gem of mathematical wonder:
\begin{equation}
2 
\end{equation}
Clearly, the derivative of this is equal to
\begin{equation}
0 
\end{equation}

We are going to study the following:
\begin{equation}
1 
\end{equation}
As you can see, the derivative of this is equal to
\begin{equation}
0 
\end{equation}

Consider the following:
\begin{equation}
1 
\end{equation}
Unsurprisingly, the derivative of this is equal to
\begin{equation}
0 
\end{equation}

We will take a closer look at this:
\begin{equation}
x ^{2 } 
\end{equation}
Unsurprisingly, the derivative of this is equal to
\begin{equation}
2 \cdot x ^{2 - 1 } \cdot 1 
\end{equation}

One shall regard the object in question with utmost interest:
\begin{equation}
1 
\end{equation}
Any self-respecting mathematician would find it obvious, that the derivative of this is equal to
\begin{equation}
0 
\end{equation}

We are going to study the following:
\begin{equation}
1 
\end{equation}
Clearly, the derivative of this is equal to
\begin{equation}
0 
\end{equation}

We will allow ourselves to divert the reader's attentions to this gem of mathematical wonder:
\begin{equation}
2 \cdot x 
\end{equation}
Clearly, the derivative of this is equal to
\begin{equation}
0 \cdot x + 2 \cdot 1 
\end{equation}

The object of our ultimate interest is the following:
\begin{equation}
1 
\end{equation}
It is now obvious, that the derivative of this is equal to
\begin{equation}
0 
\end{equation}

Consider the following:
\begin{equation}
x ^{2 } 
\end{equation}
Obviously, the derivative of this is equal to
\begin{equation}
2 \cdot x ^{2 - 1 } \cdot 1 
\end{equation}

One shall regard the object in question with utmost interest:
\begin{equation}
2 
\end{equation}
Trivially, the derivative of this is equal to
\begin{equation}
0 
\end{equation}

The following is worth a closer look:
\begin{equation}
1 
\end{equation}
It can be easily proved, that the derivative of this is equal to
\begin{equation}
0 
\end{equation}

We are going to study the following:
\begin{equation}
1 
\end{equation}
Clearly, the derivative of this is equal to
\begin{equation}
0 
\end{equation}

Consider the following:
\begin{equation}
x ^{2 } 
\end{equation}
Clearly, the derivative of this is equal to
\begin{equation}
2 \cdot x ^{2 - 1 } \cdot 1 
\end{equation}

One shall regard the object in question with utmost interest:
\begin{equation}
1 
\end{equation}
Obviously, the derivative of this is equal to
\begin{equation}
0 
\end{equation}

One shall regard the object in question with utmost interest:
\begin{equation}
1 
\end{equation}
Clearly, the derivative of this is equal to
\begin{equation}
0 
\end{equation}

One shall regard the object in question with utmost interest:
\begin{equation}
1 
\end{equation}
Any self-respecting mathematician would find it obvious, that the derivative of this is equal to
\begin{equation}
0 
\end{equation}

We are going to study the following:
\begin{equation}
x ^{2 } 
\end{equation}
Unsurprisingly, the derivative of this is equal to
\begin{equation}
2 \cdot x ^{2 - 1 } \cdot 1 
\end{equation}

We will allow ourselves to divert the reader's attentions to this gem of mathematical wonder:
\begin{equation}
2 \cdot x 
\end{equation}
As you can see, the derivative of this is equal to
\begin{equation}
0 \cdot x + 2 \cdot 1 
\end{equation}

We are going to study the following:
\begin{equation}
1 
\end{equation}
Trivially, the derivative of this is equal to
\begin{equation}
0 
\end{equation}

The object of our ultimate interest is the following:
\begin{equation}
x ^{2 } 
\end{equation}
Clearly, the derivative of this is equal to
\begin{equation}
2 \cdot x ^{2 - 1 } \cdot 1 
\end{equation}

The following is worth a closer look:
\begin{equation}
2 
\end{equation}
Trivially, the derivative of this is equal to
\begin{equation}
0 
\end{equation}

We will take a closer look at this:
\begin{equation}
1 
\end{equation}
It is now obvious, that the derivative of this is equal to
\begin{equation}
0 
\end{equation}

One shall regard the object in question with utmost interest:
\begin{equation}
1 
\end{equation}
Any self-respecting mathematician would find it obvious, that the derivative of this is equal to
\begin{equation}
0 
\end{equation}

One shall regard the object in question with utmost interest:
\begin{equation}
x ^{2 } 
\end{equation}
As you can see, the derivative of this is equal to
\begin{equation}
2 \cdot x ^{2 - 1 } \cdot 1 
\end{equation}

One shall regard the object in question with utmost interest:
\begin{equation}
1 
\end{equation}
It can be easily proved, that the derivative of this is equal to
\begin{equation}
0 
\end{equation}

Let us take a look at this:
\begin{equation}
1 
\end{equation}
It is now obvious, that the derivative of this is equal to
\begin{equation}
0 
\end{equation}

One shall regard the object in question with utmost interest:
\begin{equation}
2 \cdot x 
\end{equation}
It is now obvious, that the derivative of this is equal to
\begin{equation}
0 \cdot x + 2 \cdot 1 
\end{equation}

Let us take a look at this:
\begin{equation}
1 
\end{equation}
It is now obvious, that the derivative of this is equal to
\begin{equation}
0 
\end{equation}

We are going to study the following:
\begin{equation}
x ^{2 } 
\end{equation}
Unsurprisingly, the derivative of this is equal to
\begin{equation}
2 \cdot x ^{2 - 1 } \cdot 1 
\end{equation}

We will take a closer look at this:
\begin{equation}
2 
\end{equation}
As you can see, the derivative of this is equal to
\begin{equation}
0 
\end{equation}

We will allow ourselves to divert the reader's attentions to this gem of mathematical wonder:
\begin{equation}
1 
\end{equation}
Obviously, the derivative of this is equal to
\begin{equation}
0 
\end{equation}

Let us take a look at this:
\begin{equation}
1 
\end{equation}
It can be easily proved, that the derivative of this is equal to
\begin{equation}
0 
\end{equation}

Let us take a look at this:
\begin{equation}
x ^{2 } 
\end{equation}
Obviously, the derivative of this is equal to
\begin{equation}
2 \cdot x ^{2 - 1 } \cdot 1 
\end{equation}

The object of our ultimate interest is the following:
\begin{equation}
1 
\end{equation}
Unsurprisingly, the derivative of this is equal to
\begin{equation}
0 
\end{equation}

One shall regard the object in question with utmost interest:
\begin{equation}
2 \cdot x 
\end{equation}
Any self-respecting mathematician would find it obvious, that the derivative of this is equal to
\begin{equation}
0 \cdot x + 2 \cdot 1 
\end{equation}

We will allow ourselves to divert the reader's attentions to this gem of mathematical wonder:
\begin{equation}
1 
\end{equation}
Obviously, the derivative of this is equal to
\begin{equation}
0 
\end{equation}

The following is worth a closer look:
\begin{equation}
x ^{2 } 
\end{equation}
Any self-respecting mathematician would find it obvious, that the derivative of this is equal to
\begin{equation}
2 \cdot x ^{2 - 1 } \cdot 1 
\end{equation}

We will take a closer look at this:
\begin{equation}
2 
\end{equation}
Obviously, the derivative of this is equal to
\begin{equation}
0 
\end{equation}

The following is worth a closer look:
\begin{equation}
1 
\end{equation}
It can be easily proved, that the derivative of this is equal to
\begin{equation}
0 
\end{equation}

The following is worth a closer look:
\begin{equation}
1 
\end{equation}
Obviously, the derivative of this is equal to
\begin{equation}
0 
\end{equation}

We shall ponder the following:
\begin{equation}
x ^{2 } 
\end{equation}
As you can see, the derivative of this is equal to
\begin{equation}
2 \cdot x ^{2 - 1 } \cdot 1 
\end{equation}

Consider the following:
\begin{equation}
2 
\end{equation}
Any self-respecting mathematician would find it obvious, that the derivative of this is equal to
\begin{equation}
0 
\end{equation}

The object of our ultimate interest is the following:
\begin{equation}
2 \cdot x 
\end{equation}
It can be easily proved, that the derivative of this is equal to
\begin{equation}
0 \cdot x + 2 \cdot 1 
\end{equation}

The object of our ultimate interest is the following:
\begin{equation}
1 
\end{equation}
As you can see, the derivative of this is equal to
\begin{equation}
0 
\end{equation}

We shall ponder the following:
\begin{equation}
x ^{2 } 
\end{equation}
It is now obvious, that the derivative of this is equal to
\begin{equation}
2 \cdot x ^{2 - 1 } \cdot 1 
\end{equation}

The following is worth a closer look:
\begin{equation}
2 
\end{equation}
It is now obvious, that the derivative of this is equal to
\begin{equation}
0 
\end{equation}

The following is worth a closer look:
\begin{equation}
2 \cdot x 
\end{equation}
Any self-respecting mathematician would find it obvious, that the derivative of this is equal to
\begin{equation}
0 \cdot x + 2 \cdot 1 
\end{equation}

We are going to study the following:
\begin{equation}
1 
\end{equation}
Obviously, the derivative of this is equal to
\begin{equation}
0 
\end{equation}

Consider the following:
\begin{equation}
x ^{2 } 
\end{equation}
Unsurprisingly, the derivative of this is equal to
\begin{equation}
2 \cdot x ^{2 - 1 } \cdot 1 
\end{equation}

The object of our ultimate interest is the following:
\begin{equation}
2 
\end{equation}
Unsurprisingly, the derivative of this is equal to
\begin{equation}
0 
\end{equation}

We are going to study the following:
\begin{equation}
1 
\end{equation}
It is now obvious, that the derivative of this is equal to
\begin{equation}
0 
\end{equation}

Let us take a look at this:
\begin{equation}
2 
\end{equation}
As you can see, the derivative of this is equal to
\begin{equation}
0 
\end{equation}

Consider the following:
\begin{equation}
2 
\end{equation}
As you can see, the derivative of this is equal to
\begin{equation}
0 
\end{equation}

Let us take a look at this:
\begin{equation}
1 
\end{equation}
Any self-respecting mathematician would find it obvious, that the derivative of this is equal to
\begin{equation}
0 
\end{equation}

We shall ponder the following:
\begin{equation}
x ^{2 } 
\end{equation}
Trivially, the derivative of this is equal to
\begin{equation}
2 \cdot x ^{2 - 1 } \cdot 1 
\end{equation}

We will allow ourselves to divert the reader's attentions to this gem of mathematical wonder:
\begin{equation}
2 
\end{equation}
Unsurprisingly, the derivative of this is equal to
\begin{equation}
0 
\end{equation}

We shall ponder the following:
\begin{equation}
1 
\end{equation}
Unsurprisingly, the derivative of this is equal to
\begin{equation}
0 
\end{equation}

We shall ponder the following:
\begin{equation}
1 
\end{equation}
As you can see, the derivative of this is equal to
\begin{equation}
0 
\end{equation}

Let us take a look at this:
\begin{equation}
x ^{2 } 
\end{equation}
Clearly, the derivative of this is equal to
\begin{equation}
2 \cdot x ^{2 - 1 } \cdot 1 
\end{equation}

Let us take a look at this:
\begin{equation}
1 
\end{equation}
It is now obvious, that the derivative of this is equal to
\begin{equation}
0 
\end{equation}

The following is worth a closer look:
\begin{equation}
1 
\end{equation}
As you can see, the derivative of this is equal to
\begin{equation}
0 
\end{equation}

The object of our ultimate interest is the following:
\begin{equation}
1 
\end{equation}
It is now obvious, that the derivative of this is equal to
\begin{equation}
0 
\end{equation}

We will take a closer look at this:
\begin{equation}
x ^{2 } 
\end{equation}
As you can see, the derivative of this is equal to
\begin{equation}
2 \cdot x ^{2 - 1 } \cdot 1 
\end{equation}

Let us take a look at this:
\begin{equation}
1 
\end{equation}
Clearly, the derivative of this is equal to
\begin{equation}
0 
\end{equation}

The object of our ultimate interest is the following:
\begin{equation}
1 
\end{equation}
Any self-respecting mathematician would find it obvious, that the derivative of this is equal to
\begin{equation}
0 
\end{equation}

Let us take a look at this:
\begin{equation}
x ^{2 } 
\end{equation}
It is now obvious, that the derivative of this is equal to
\begin{equation}
2 \cdot x ^{2 - 1 } \cdot 1 
\end{equation}

One shall regard the object in question with utmost interest:
\begin{equation}
x - 2 
\end{equation}
Any self-respecting mathematician would find it obvious, that the derivative of this is equal to
\begin{equation}
1 - 0 
\end{equation}

Let us take a look at this:
\begin{equation}
1 
\end{equation}
It can be easily proved, that the derivative of this is equal to
\begin{equation}
0 
\end{equation}

One shall regard the object in question with utmost interest:
\begin{equation}
x ^{2 } 
\end{equation}
Trivially, the derivative of this is equal to
\begin{equation}
2 \cdot x ^{2 - 1 } \cdot 1 
\end{equation}

One shall regard the object in question with utmost interest:
\begin{equation}
2 \cdot x 
\end{equation}
It is now obvious, that the derivative of this is equal to
\begin{equation}
0 \cdot x + 2 \cdot 1 
\end{equation}

One shall regard the object in question with utmost interest:
\begin{equation}
1 
\end{equation}
Clearly, the derivative of this is equal to
\begin{equation}
0 
\end{equation}

We will allow ourselves to divert the reader's attentions to this gem of mathematical wonder:
\begin{equation}
x ^{2 } 
\end{equation}
As you can see, the derivative of this is equal to
\begin{equation}
2 \cdot x ^{2 - 1 } \cdot 1 
\end{equation}

One shall regard the object in question with utmost interest:
\begin{equation}
2 
\end{equation}
Clearly, the derivative of this is equal to
\begin{equation}
0 
\end{equation}

Consider the following:
\begin{equation}
1 
\end{equation}
Obviously, the derivative of this is equal to
\begin{equation}
0 
\end{equation}

The object of our ultimate interest is the following:
\begin{equation}
1 
\end{equation}
Clearly, the derivative of this is equal to
\begin{equation}
0 
\end{equation}

One shall regard the object in question with utmost interest:
\begin{equation}
x ^{2 } 
\end{equation}
As you can see, the derivative of this is equal to
\begin{equation}
2 \cdot x ^{2 - 1 } \cdot 1 
\end{equation}

Consider the following:
\begin{equation}
1 
\end{equation}
Clearly, the derivative of this is equal to
\begin{equation}
0 
\end{equation}

We are going to study the following:
\begin{equation}
1 
\end{equation}
Trivially, the derivative of this is equal to
\begin{equation}
0 
\end{equation}

Let us take a look at this:
\begin{equation}
2 \cdot x 
\end{equation}
Obviously, the derivative of this is equal to
\begin{equation}
0 \cdot x + 2 \cdot 1 
\end{equation}

Consider the following:
\begin{equation}
1 
\end{equation}
It can be easily proved, that the derivative of this is equal to
\begin{equation}
0 
\end{equation}

We will take a closer look at this:
\begin{equation}
x ^{2 } 
\end{equation}
Clearly, the derivative of this is equal to
\begin{equation}
2 \cdot x ^{2 - 1 } \cdot 1 
\end{equation}

One shall regard the object in question with utmost interest:
\begin{equation}
2 
\end{equation}
Clearly, the derivative of this is equal to
\begin{equation}
0 
\end{equation}

The object of our ultimate interest is the following:
\begin{equation}
1 
\end{equation}
Any self-respecting mathematician would find it obvious, that the derivative of this is equal to
\begin{equation}
0 
\end{equation}

We shall ponder the following:
\begin{equation}
1 
\end{equation}
Clearly, the derivative of this is equal to
\begin{equation}
0 
\end{equation}

We will allow ourselves to divert the reader's attentions to this gem of mathematical wonder:
\begin{equation}
x ^{2 } 
\end{equation}
It can be easily proved, that the derivative of this is equal to
\begin{equation}
2 \cdot x ^{2 - 1 } \cdot 1 
\end{equation}

We will allow ourselves to divert the reader's attentions to this gem of mathematical wonder:
\begin{equation}
1 
\end{equation}
Clearly, the derivative of this is equal to
\begin{equation}
0 
\end{equation}

We will take a closer look at this:
\begin{equation}
1 
\end{equation}
Unsurprisingly, the derivative of this is equal to
\begin{equation}
0 
\end{equation}

Let us take a look at this:
\begin{equation}
1 
\end{equation}
Clearly, the derivative of this is equal to
\begin{equation}
0 
\end{equation}

We are going to study the following:
\begin{equation}
x ^{2 } 
\end{equation}
Unsurprisingly, the derivative of this is equal to
\begin{equation}
2 \cdot x ^{2 - 1 } \cdot 1 
\end{equation}

We are going to study the following:
\begin{equation}
2 \cdot x 
\end{equation}
Any self-respecting mathematician would find it obvious, that the derivative of this is equal to
\begin{equation}
0 \cdot x + 2 \cdot 1 
\end{equation}

We will allow ourselves to divert the reader's attentions to this gem of mathematical wonder:
\begin{equation}
1 
\end{equation}
It is now obvious, that the derivative of this is equal to
\begin{equation}
0 
\end{equation}

The object of our ultimate interest is the following:
\begin{equation}
x ^{2 } 
\end{equation}
As you can see, the derivative of this is equal to
\begin{equation}
2 \cdot x ^{2 - 1 } \cdot 1 
\end{equation}

The object of our ultimate interest is the following:
\begin{equation}
2 
\end{equation}
Unsurprisingly, the derivative of this is equal to
\begin{equation}
0 
\end{equation}

The following is worth a closer look:
\begin{equation}
1 
\end{equation}
It is now obvious, that the derivative of this is equal to
\begin{equation}
0 
\end{equation}

We shall ponder the following:
\begin{equation}
1 
\end{equation}
It is now obvious, that the derivative of this is equal to
\begin{equation}
0 
\end{equation}

We shall ponder the following:
\begin{equation}
x ^{2 } 
\end{equation}
Trivially, the derivative of this is equal to
\begin{equation}
2 \cdot x ^{2 - 1 } \cdot 1 
\end{equation}

The following is worth a closer look:
\begin{equation}
1 
\end{equation}
Any self-respecting mathematician would find it obvious, that the derivative of this is equal to
\begin{equation}
0 
\end{equation}

The object of our ultimate interest is the following:
\begin{equation}
2 \cdot x 
\end{equation}
Any self-respecting mathematician would find it obvious, that the derivative of this is equal to
\begin{equation}
0 \cdot x + 2 \cdot 1 
\end{equation}

The following is worth a closer look:
\begin{equation}
1 
\end{equation}
Any self-respecting mathematician would find it obvious, that the derivative of this is equal to
\begin{equation}
0 
\end{equation}

We will allow ourselves to divert the reader's attentions to this gem of mathematical wonder:
\begin{equation}
x ^{2 } 
\end{equation}
It is now obvious, that the derivative of this is equal to
\begin{equation}
2 \cdot x ^{2 - 1 } \cdot 1 
\end{equation}

We will allow ourselves to divert the reader's attentions to this gem of mathematical wonder:
\begin{equation}
2 
\end{equation}
As you can see, the derivative of this is equal to
\begin{equation}
0 
\end{equation}

We shall ponder the following:
\begin{equation}
1 
\end{equation}
Trivially, the derivative of this is equal to
\begin{equation}
0 
\end{equation}

One shall regard the object in question with utmost interest:
\begin{equation}
1 
\end{equation}
It is now obvious, that the derivative of this is equal to
\begin{equation}
0 
\end{equation}

We will take a closer look at this:
\begin{equation}
x ^{2 } 
\end{equation}
Clearly, the derivative of this is equal to
\begin{equation}
2 \cdot x ^{2 - 1 } \cdot 1 
\end{equation}

Let us take a look at this:
\begin{equation}
2 
\end{equation}
It can be easily proved, that the derivative of this is equal to
\begin{equation}
0 
\end{equation}

The object of our ultimate interest is the following:
\begin{equation}
2 \cdot x 
\end{equation}
As you can see, the derivative of this is equal to
\begin{equation}
0 \cdot x + 2 \cdot 1 
\end{equation}

One shall regard the object in question with utmost interest:
\begin{equation}
1 
\end{equation}
It is now obvious, that the derivative of this is equal to
\begin{equation}
0 
\end{equation}

We are going to study the following:
\begin{equation}
x ^{2 } 
\end{equation}
Obviously, the derivative of this is equal to
\begin{equation}
2 \cdot x ^{2 - 1 } \cdot 1 
\end{equation}

We will take a closer look at this:
\begin{equation}
2 
\end{equation}
Any self-respecting mathematician would find it obvious, that the derivative of this is equal to
\begin{equation}
0 
\end{equation}

Consider the following:
\begin{equation}
2 \cdot x 
\end{equation}
Any self-respecting mathematician would find it obvious, that the derivative of this is equal to
\begin{equation}
0 \cdot x + 2 \cdot 1 
\end{equation}

Consider the following:
\begin{equation}
1 
\end{equation}
Obviously, the derivative of this is equal to
\begin{equation}
0 
\end{equation}

Consider the following:
\begin{equation}
x ^{2 } 
\end{equation}
As you can see, the derivative of this is equal to
\begin{equation}
2 \cdot x ^{2 - 1 } \cdot 1 
\end{equation}

One shall regard the object in question with utmost interest:
\begin{equation}
2 
\end{equation}
Obviously, the derivative of this is equal to
\begin{equation}
0 
\end{equation}

We will allow ourselves to divert the reader's attentions to this gem of mathematical wonder:
\begin{equation}
1 
\end{equation}
Trivially, the derivative of this is equal to
\begin{equation}
0 
\end{equation}

Consider the following:
\begin{equation}
2 
\end{equation}
Unsurprisingly, the derivative of this is equal to
\begin{equation}
0 
\end{equation}

The object of our ultimate interest is the following:
\begin{equation}
2 
\end{equation}
It is now obvious, that the derivative of this is equal to
\begin{equation}
0 
\end{equation}

One shall regard the object in question with utmost interest:
\begin{equation}
1 
\end{equation}
Clearly, the derivative of this is equal to
\begin{equation}
0 
\end{equation}

We shall ponder the following:
\begin{equation}
x ^{2 } 
\end{equation}
It can be easily proved, that the derivative of this is equal to
\begin{equation}
2 \cdot x ^{2 - 1 } \cdot 1 
\end{equation}

We will allow ourselves to divert the reader's attentions to this gem of mathematical wonder:
\begin{equation}
2 
\end{equation}
Obviously, the derivative of this is equal to
\begin{equation}
0 
\end{equation}

We shall ponder the following:
\begin{equation}
1 
\end{equation}
Any self-respecting mathematician would find it obvious, that the derivative of this is equal to
\begin{equation}
0 
\end{equation}

Let us take a look at this:
\begin{equation}
1 
\end{equation}
It can be easily proved, that the derivative of this is equal to
\begin{equation}
0 
\end{equation}

The object of our ultimate interest is the following:
\begin{equation}
x ^{2 } 
\end{equation}
Unsurprisingly, the derivative of this is equal to
\begin{equation}
2 \cdot x ^{2 - 1 } \cdot 1 
\end{equation}

Consider the following:
\begin{equation}
1 
\end{equation}
It can be easily proved, that the derivative of this is equal to
\begin{equation}
0 
\end{equation}

We shall ponder the following:
\begin{equation}
1 
\end{equation}
As you can see, the derivative of this is equal to
\begin{equation}
0 
\end{equation}

One shall regard the object in question with utmost interest:
\begin{equation}
x - 2 
\end{equation}
Trivially, the derivative of this is equal to
\begin{equation}
1 - 0 
\end{equation}

One shall regard the object in question with utmost interest:
\begin{equation}
1 
\end{equation}
Any self-respecting mathematician would find it obvious, that the derivative of this is equal to
\begin{equation}
0 
\end{equation}

We are going to study the following:
\begin{equation}
x ^{2 } 
\end{equation}
Trivially, the derivative of this is equal to
\begin{equation}
2 \cdot x ^{2 - 1 } \cdot 1 
\end{equation}

We will take a closer look at this:
\begin{equation}
2 \cdot x 
\end{equation}
Trivially, the derivative of this is equal to
\begin{equation}
0 \cdot x + 2 \cdot 1 
\end{equation}

We will take a closer look at this:
\begin{equation}
1 
\end{equation}
It can be easily proved, that the derivative of this is equal to
\begin{equation}
0 
\end{equation}

Consider the following:
\begin{equation}
x ^{2 } 
\end{equation}
Clearly, the derivative of this is equal to
\begin{equation}
2 \cdot x ^{2 - 1 } \cdot 1 
\end{equation}

We will take a closer look at this:
\begin{equation}
2 
\end{equation}
Unsurprisingly, the derivative of this is equal to
\begin{equation}
0 
\end{equation}

One shall regard the object in question with utmost interest:
\begin{equation}
1 
\end{equation}
Unsurprisingly, the derivative of this is equal to
\begin{equation}
0 
\end{equation}

We are going to study the following:
\begin{equation}
1 
\end{equation}
Any self-respecting mathematician would find it obvious, that the derivative of this is equal to
\begin{equation}
0 
\end{equation}

We shall ponder the following:
\begin{equation}
x ^{2 } 
\end{equation}
Trivially, the derivative of this is equal to
\begin{equation}
2 \cdot x ^{2 - 1 } \cdot 1 
\end{equation}

We will take a closer look at this:
\begin{equation}
1 
\end{equation}
It can be easily proved, that the derivative of this is equal to
\begin{equation}
0 
\end{equation}

We will take a closer look at this:
\begin{equation}
1 
\end{equation}
It is now obvious, that the derivative of this is equal to
\begin{equation}
0 
\end{equation}

We are going to study the following:
\begin{equation}
2 \cdot x 
\end{equation}
It can be easily proved, that the derivative of this is equal to
\begin{equation}
0 \cdot x + 2 \cdot 1 
\end{equation}

We will allow ourselves to divert the reader's attentions to this gem of mathematical wonder:
\begin{equation}
1 
\end{equation}
It is now obvious, that the derivative of this is equal to
\begin{equation}
0 
\end{equation}

We will allow ourselves to divert the reader's attentions to this gem of mathematical wonder:
\begin{equation}
x ^{2 } 
\end{equation}
Trivially, the derivative of this is equal to
\begin{equation}
2 \cdot x ^{2 - 1 } \cdot 1 
\end{equation}

We shall ponder the following:
\begin{equation}
2 
\end{equation}
It is now obvious, that the derivative of this is equal to
\begin{equation}
0 
\end{equation}

The object of our ultimate interest is the following:
\begin{equation}
1 
\end{equation}
It can be easily proved, that the derivative of this is equal to
\begin{equation}
0 
\end{equation}

We will allow ourselves to divert the reader's attentions to this gem of mathematical wonder:
\begin{equation}
1 
\end{equation}
Unsurprisingly, the derivative of this is equal to
\begin{equation}
0 
\end{equation}

The following is worth a closer look:
\begin{equation}
x ^{2 } 
\end{equation}
It can be easily proved, that the derivative of this is equal to
\begin{equation}
2 \cdot x ^{2 - 1 } \cdot 1 
\end{equation}

We will allow ourselves to divert the reader's attentions to this gem of mathematical wonder:
\begin{equation}
1 
\end{equation}
Clearly, the derivative of this is equal to
\begin{equation}
0 
\end{equation}

Consider the following:
\begin{equation}
1 
\end{equation}
Clearly, the derivative of this is equal to
\begin{equation}
0 
\end{equation}

We will take a closer look at this:
\begin{equation}
1 
\end{equation}
Obviously, the derivative of this is equal to
\begin{equation}
0 
\end{equation}

We are going to study the following:
\begin{equation}
x ^{2 } 
\end{equation}
Obviously, the derivative of this is equal to
\begin{equation}
2 \cdot x ^{2 - 1 } \cdot 1 
\end{equation}

One shall regard the object in question with utmost interest:
\begin{equation}
1 
\end{equation}
Clearly, the derivative of this is equal to
\begin{equation}
0 
\end{equation}

The object of our ultimate interest is the following:
\begin{equation}
1 
\end{equation}
As you can see, the derivative of this is equal to
\begin{equation}
0 
\end{equation}

The following is worth a closer look:
\begin{equation}
x ^{2 } 
\end{equation}
As you can see, the derivative of this is equal to
\begin{equation}
2 \cdot x ^{2 - 1 } \cdot 1 
\end{equation}

The object of our ultimate interest is the following:
\begin{equation}
2 \cdot x 
\end{equation}
Clearly, the derivative of this is equal to
\begin{equation}
0 \cdot x + 2 \cdot 1 
\end{equation}

We are going to study the following:
\begin{equation}
1 
\end{equation}
Clearly, the derivative of this is equal to
\begin{equation}
0 
\end{equation}

The object of our ultimate interest is the following:
\begin{equation}
x ^{2 } 
\end{equation}
Trivially, the derivative of this is equal to
\begin{equation}
2 \cdot x ^{2 - 1 } \cdot 1 
\end{equation}

We will take a closer look at this:
\begin{equation}
2 
\end{equation}
Obviously, the derivative of this is equal to
\begin{equation}
0 
\end{equation}

We are going to study the following:
\begin{equation}
1 
\end{equation}
Any self-respecting mathematician would find it obvious, that the derivative of this is equal to
\begin{equation}
0 
\end{equation}

The following is worth a closer look:
\begin{equation}
1 
\end{equation}
Clearly, the derivative of this is equal to
\begin{equation}
0 
\end{equation}

We will allow ourselves to divert the reader's attentions to this gem of mathematical wonder:
\begin{equation}
x ^{2 } 
\end{equation}
It is now obvious, that the derivative of this is equal to
\begin{equation}
2 \cdot x ^{2 - 1 } \cdot 1 
\end{equation}

We will take a closer look at this:
\begin{equation}
1 
\end{equation}
Unsurprisingly, the derivative of this is equal to
\begin{equation}
0 
\end{equation}

We will take a closer look at this:
\begin{equation}
1 
\end{equation}
Clearly, the derivative of this is equal to
\begin{equation}
0 
\end{equation}

We will take a closer look at this:
\begin{equation}
x - 2 
\end{equation}
As you can see, the derivative of this is equal to
\begin{equation}
1 - 0 
\end{equation}

The object of our ultimate interest is the following:
\begin{equation}
1 
\end{equation}
Trivially, the derivative of this is equal to
\begin{equation}
0 
\end{equation}

The following is worth a closer look:
\begin{equation}
x ^{2 } 
\end{equation}
Clearly, the derivative of this is equal to
\begin{equation}
2 \cdot x ^{2 - 1 } \cdot 1 
\end{equation}

One shall regard the object in question with utmost interest:
\begin{equation}
1 
\end{equation}
Any self-respecting mathematician would find it obvious, that the derivative of this is equal to
\begin{equation}
0 
\end{equation}

Consider the following:
\begin{equation}
x ^{2 } 
\end{equation}
Obviously, the derivative of this is equal to
\begin{equation}
2 \cdot x ^{2 - 1 } \cdot 1 
\end{equation}

We will allow ourselves to divert the reader's attentions to this gem of mathematical wonder:
\begin{equation}
x - 2 
\end{equation}
Clearly, the derivative of this is equal to
\begin{equation}
1 - 0 
\end{equation}

The following is worth a closer look:
\begin{equation}
1 
\end{equation}
As you can see, the derivative of this is equal to
\begin{equation}
0 
\end{equation}

Let us take a look at this:
\begin{equation}
x ^{2 } 
\end{equation}
Any self-respecting mathematician would find it obvious, that the derivative of this is equal to
\begin{equation}
2 \cdot x ^{2 - 1 } \cdot 1 
\end{equation}

One shall regard the object in question with utmost interest:
\begin{equation}
2 \cdot x 
\end{equation}
Trivially, the derivative of this is equal to
\begin{equation}
0 \cdot x + 2 \cdot 1 
\end{equation}

The following is worth a closer look:
\begin{equation}
1 
\end{equation}
Trivially, the derivative of this is equal to
\begin{equation}
0 
\end{equation}

The following is worth a closer look:
\begin{equation}
x ^{2 } 
\end{equation}
It can be easily proved, that the derivative of this is equal to
\begin{equation}
2 \cdot x ^{2 - 1 } \cdot 1 
\end{equation}

The following is worth a closer look:
\begin{equation}
2 
\end{equation}
Unsurprisingly, the derivative of this is equal to
\begin{equation}
0 
\end{equation}

We will allow ourselves to divert the reader's attentions to this gem of mathematical wonder:
\begin{equation}
1 
\end{equation}
Obviously, the derivative of this is equal to
\begin{equation}
0 
\end{equation}

We shall ponder the following:
\begin{equation}
1 
\end{equation}
It is now obvious, that the derivative of this is equal to
\begin{equation}
0 
\end{equation}

The object of our ultimate interest is the following:
\begin{equation}
x ^{2 } 
\end{equation}
As you can see, the derivative of this is equal to
\begin{equation}
2 \cdot x ^{2 - 1 } \cdot 1 
\end{equation}

The object of our ultimate interest is the following:
\begin{equation}
1 
\end{equation}
It is now obvious, that the derivative of this is equal to
\begin{equation}
0 
\end{equation}

We will allow ourselves to divert the reader's attentions to this gem of mathematical wonder:
\begin{equation}
1 
\end{equation}
Trivially, the derivative of this is equal to
\begin{equation}
0 
\end{equation}

The following is worth a closer look:
\begin{equation}
2 \cdot x 
\end{equation}
Any self-respecting mathematician would find it obvious, that the derivative of this is equal to
\begin{equation}
0 \cdot x + 2 \cdot 1 
\end{equation}

Consider the following:
\begin{equation}
1 
\end{equation}
As you can see, the derivative of this is equal to
\begin{equation}
0 
\end{equation}

We will take a closer look at this:
\begin{equation}
x ^{2 } 
\end{equation}
Clearly, the derivative of this is equal to
\begin{equation}
2 \cdot x ^{2 - 1 } \cdot 1 
\end{equation}

One shall regard the object in question with utmost interest:
\begin{equation}
2 
\end{equation}
Any self-respecting mathematician would find it obvious, that the derivative of this is equal to
\begin{equation}
0 
\end{equation}

We are going to study the following:
\begin{equation}
1 
\end{equation}
Any self-respecting mathematician would find it obvious, that the derivative of this is equal to
\begin{equation}
0 
\end{equation}

The object of our ultimate interest is the following:
\begin{equation}
1 
\end{equation}
Clearly, the derivative of this is equal to
\begin{equation}
0 
\end{equation}

We shall ponder the following:
\begin{equation}
x ^{2 } 
\end{equation}
Obviously, the derivative of this is equal to
\begin{equation}
2 \cdot x ^{2 - 1 } \cdot 1 
\end{equation}

One shall regard the object in question with utmost interest:
\begin{equation}
1 
\end{equation}
Any self-respecting mathematician would find it obvious, that the derivative of this is equal to
\begin{equation}
0 
\end{equation}

We shall ponder the following:
\begin{equation}
1 
\end{equation}
Any self-respecting mathematician would find it obvious, that the derivative of this is equal to
\begin{equation}
0 
\end{equation}

The object of our ultimate interest is the following:
\begin{equation}
1 
\end{equation}
Clearly, the derivative of this is equal to
\begin{equation}
0 
\end{equation}

The object of our ultimate interest is the following:
\begin{equation}
x ^{2 } 
\end{equation}
Clearly, the derivative of this is equal to
\begin{equation}
2 \cdot x ^{2 - 1 } \cdot 1 
\end{equation}

Let us take a look at this:
\begin{equation}
2 \cdot x 
\end{equation}
Any self-respecting mathematician would find it obvious, that the derivative of this is equal to
\begin{equation}
0 \cdot x + 2 \cdot 1 
\end{equation}

We shall ponder the following:
\begin{equation}
1 
\end{equation}
As you can see, the derivative of this is equal to
\begin{equation}
0 
\end{equation}

The following is worth a closer look:
\begin{equation}
x ^{2 } 
\end{equation}
Clearly, the derivative of this is equal to
\begin{equation}
2 \cdot x ^{2 - 1 } \cdot 1 
\end{equation}

We shall ponder the following:
\begin{equation}
2 
\end{equation}
Any self-respecting mathematician would find it obvious, that the derivative of this is equal to
\begin{equation}
0 
\end{equation}

The object of our ultimate interest is the following:
\begin{equation}
1 
\end{equation}
It can be easily proved, that the derivative of this is equal to
\begin{equation}
0 
\end{equation}

The object of our ultimate interest is the following:
\begin{equation}
1 
\end{equation}
Any self-respecting mathematician would find it obvious, that the derivative of this is equal to
\begin{equation}
0 
\end{equation}

We are going to study the following:
\begin{equation}
x ^{2 } 
\end{equation}
As you can see, the derivative of this is equal to
\begin{equation}
2 \cdot x ^{2 - 1 } \cdot 1 
\end{equation}

We are going to study the following:
\begin{equation}
1 
\end{equation}
Unsurprisingly, the derivative of this is equal to
\begin{equation}
0 
\end{equation}

We shall ponder the following:
\begin{equation}
2 \cdot x 
\end{equation}
Trivially, the derivative of this is equal to
\begin{equation}
0 \cdot x + 2 \cdot 1 
\end{equation}

We will take a closer look at this:
\begin{equation}
1 
\end{equation}
Any self-respecting mathematician would find it obvious, that the derivative of this is equal to
\begin{equation}
0 
\end{equation}

We will take a closer look at this:
\begin{equation}
x ^{2 } 
\end{equation}
Unsurprisingly, the derivative of this is equal to
\begin{equation}
2 \cdot x ^{2 - 1 } \cdot 1 
\end{equation}

Let us take a look at this:
\begin{equation}
2 
\end{equation}
Any self-respecting mathematician would find it obvious, that the derivative of this is equal to
\begin{equation}
0 
\end{equation}

Consider the following:
\begin{equation}
1 
\end{equation}
Unsurprisingly, the derivative of this is equal to
\begin{equation}
0 
\end{equation}

The object of our ultimate interest is the following:
\begin{equation}
1 
\end{equation}
It is now obvious, that the derivative of this is equal to
\begin{equation}
0 
\end{equation}

The following is worth a closer look:
\begin{equation}
x ^{2 } 
\end{equation}
It can be easily proved, that the derivative of this is equal to
\begin{equation}
2 \cdot x ^{2 - 1 } \cdot 1 
\end{equation}

Consider the following:
\begin{equation}
2 
\end{equation}
Any self-respecting mathematician would find it obvious, that the derivative of this is equal to
\begin{equation}
0 
\end{equation}

One shall regard the object in question with utmost interest:
\begin{equation}
2 \cdot x 
\end{equation}
Trivially, the derivative of this is equal to
\begin{equation}
0 \cdot x + 2 \cdot 1 
\end{equation}

We shall ponder the following:
\begin{equation}
1 
\end{equation}
It is now obvious, that the derivative of this is equal to
\begin{equation}
0 
\end{equation}

The object of our ultimate interest is the following:
\begin{equation}
x ^{2 } 
\end{equation}
Clearly, the derivative of this is equal to
\begin{equation}
2 \cdot x ^{2 - 1 } \cdot 1 
\end{equation}

We will take a closer look at this:
\begin{equation}
2 
\end{equation}
It is now obvious, that the derivative of this is equal to
\begin{equation}
0 
\end{equation}

We will allow ourselves to divert the reader's attentions to this gem of mathematical wonder:
\begin{equation}
2 \cdot x 
\end{equation}
Any self-respecting mathematician would find it obvious, that the derivative of this is equal to
\begin{equation}
0 \cdot x + 2 \cdot 1 
\end{equation}

Consider the following:
\begin{equation}
1 
\end{equation}
Unsurprisingly, the derivative of this is equal to
\begin{equation}
0 
\end{equation}

We will take a closer look at this:
\begin{equation}
x ^{2 } 
\end{equation}
Obviously, the derivative of this is equal to
\begin{equation}
2 \cdot x ^{2 - 1 } \cdot 1 
\end{equation}

The following is worth a closer look:
\begin{equation}
2 
\end{equation}
Trivially, the derivative of this is equal to
\begin{equation}
0 
\end{equation}

The following is worth a closer look:
\begin{equation}
1 
\end{equation}
Obviously, the derivative of this is equal to
\begin{equation}
0 
\end{equation}

Let us take a look at this:
\begin{equation}
2 
\end{equation}
As you can see, the derivative of this is equal to
\begin{equation}
0 
\end{equation}

Consider the following:
\begin{equation}
2 
\end{equation}
It can be easily proved, that the derivative of this is equal to
\begin{equation}
0 
\end{equation}

We shall ponder the following:
\begin{equation}
1 
\end{equation}
As you can see, the derivative of this is equal to
\begin{equation}
0 
\end{equation}

One shall regard the object in question with utmost interest:
\begin{equation}
x ^{2 } 
\end{equation}
Clearly, the derivative of this is equal to
\begin{equation}
2 \cdot x ^{2 - 1 } \cdot 1 
\end{equation}

We are going to study the following:
\begin{equation}
2 
\end{equation}
As you can see, the derivative of this is equal to
\begin{equation}
0 
\end{equation}

The object of our ultimate interest is the following:
\begin{equation}
1 
\end{equation}
Obviously, the derivative of this is equal to
\begin{equation}
0 
\end{equation}

One shall regard the object in question with utmost interest:
\begin{equation}
1 
\end{equation}
Any self-respecting mathematician would find it obvious, that the derivative of this is equal to
\begin{equation}
0 
\end{equation}

The following is worth a closer look:
\begin{equation}
x ^{2 } 
\end{equation}
As you can see, the derivative of this is equal to
\begin{equation}
2 \cdot x ^{2 - 1 } \cdot 1 
\end{equation}

We shall ponder the following:
\begin{equation}
1 
\end{equation}
It is now obvious, that the derivative of this is equal to
\begin{equation}
0 
\end{equation}

Let us take a look at this:
\begin{equation}
1 
\end{equation}
Obviously, the derivative of this is equal to
\begin{equation}
0 
\end{equation}

Let us take a look at this:
\begin{equation}
x - 2 
\end{equation}
It is now obvious, that the derivative of this is equal to
\begin{equation}
1 - 0 
\end{equation}

The object of our ultimate interest is the following:
\begin{equation}
1 
\end{equation}
Trivially, the derivative of this is equal to
\begin{equation}
0 
\end{equation}

We will take a closer look at this:
\begin{equation}
x ^{2 } 
\end{equation}
Clearly, the derivative of this is equal to
\begin{equation}
2 \cdot x ^{2 - 1 } \cdot 1 
\end{equation}

The following is worth a closer look:
\begin{equation}
2 \cdot x 
\end{equation}
It can be easily proved, that the derivative of this is equal to
\begin{equation}
0 \cdot x + 2 \cdot 1 
\end{equation}

The object of our ultimate interest is the following:
\begin{equation}
1 
\end{equation}
Unsurprisingly, the derivative of this is equal to
\begin{equation}
0 
\end{equation}

We shall ponder the following:
\begin{equation}
x ^{2 } 
\end{equation}
It is now obvious, that the derivative of this is equal to
\begin{equation}
2 \cdot x ^{2 - 1 } \cdot 1 
\end{equation}

Consider the following:
\begin{equation}
2 
\end{equation}
It is now obvious, that the derivative of this is equal to
\begin{equation}
0 
\end{equation}

Let us take a look at this:
\begin{equation}
1 
\end{equation}
It is now obvious, that the derivative of this is equal to
\begin{equation}
0 
\end{equation}

We will allow ourselves to divert the reader's attentions to this gem of mathematical wonder:
\begin{equation}
1 
\end{equation}
As you can see, the derivative of this is equal to
\begin{equation}
0 
\end{equation}

One shall regard the object in question with utmost interest:
\begin{equation}
x ^{2 } 
\end{equation}
Any self-respecting mathematician would find it obvious, that the derivative of this is equal to
\begin{equation}
2 \cdot x ^{2 - 1 } \cdot 1 
\end{equation}

We will allow ourselves to divert the reader's attentions to this gem of mathematical wonder:
\begin{equation}
1 
\end{equation}
Clearly, the derivative of this is equal to
\begin{equation}
0 
\end{equation}

We are going to study the following:
\begin{equation}
1 
\end{equation}
It is now obvious, that the derivative of this is equal to
\begin{equation}
0 
\end{equation}

We will allow ourselves to divert the reader's attentions to this gem of mathematical wonder:
\begin{equation}
2 \cdot x 
\end{equation}
Clearly, the derivative of this is equal to
\begin{equation}
0 \cdot x + 2 \cdot 1 
\end{equation}

We will allow ourselves to divert the reader's attentions to this gem of mathematical wonder:
\begin{equation}
1 
\end{equation}
It can be easily proved, that the derivative of this is equal to
\begin{equation}
0 
\end{equation}

We shall ponder the following:
\begin{equation}
x ^{2 } 
\end{equation}
Clearly, the derivative of this is equal to
\begin{equation}
2 \cdot x ^{2 - 1 } \cdot 1 
\end{equation}

We shall ponder the following:
\begin{equation}
2 
\end{equation}
As you can see, the derivative of this is equal to
\begin{equation}
0 
\end{equation}

One shall regard the object in question with utmost interest:
\begin{equation}
1 
\end{equation}
Unsurprisingly, the derivative of this is equal to
\begin{equation}
0 
\end{equation}

Let us take a look at this:
\begin{equation}
1 
\end{equation}
Clearly, the derivative of this is equal to
\begin{equation}
0 
\end{equation}

The following is worth a closer look:
\begin{equation}
x ^{2 } 
\end{equation}
It is now obvious, that the derivative of this is equal to
\begin{equation}
2 \cdot x ^{2 - 1 } \cdot 1 
\end{equation}

Consider the following:
\begin{equation}
1 
\end{equation}
Clearly, the derivative of this is equal to
\begin{equation}
0 
\end{equation}

We shall ponder the following:
\begin{equation}
1 
\end{equation}
Any self-respecting mathematician would find it obvious, that the derivative of this is equal to
\begin{equation}
0 
\end{equation}

Let us take a look at this:
\begin{equation}
1 
\end{equation}
Any self-respecting mathematician would find it obvious, that the derivative of this is equal to
\begin{equation}
0 
\end{equation}

Consider the following:
\begin{equation}
x ^{2 } 
\end{equation}
It is now obvious, that the derivative of this is equal to
\begin{equation}
2 \cdot x ^{2 - 1 } \cdot 1 
\end{equation}

We are going to study the following:
\begin{equation}
1 
\end{equation}
Any self-respecting mathematician would find it obvious, that the derivative of this is equal to
\begin{equation}
0 
\end{equation}

We will take a closer look at this:
\begin{equation}
1 
\end{equation}
Any self-respecting mathematician would find it obvious, that the derivative of this is equal to
\begin{equation}
0 
\end{equation}

We shall ponder the following:
\begin{equation}
x ^{2 } 
\end{equation}
Unsurprisingly, the derivative of this is equal to
\begin{equation}
2 \cdot x ^{2 - 1 } \cdot 1 
\end{equation}

Let us take a look at this:
\begin{equation}
2 \cdot x 
\end{equation}
Any self-respecting mathematician would find it obvious, that the derivative of this is equal to
\begin{equation}
0 \cdot x + 2 \cdot 1 
\end{equation}

Let us take a look at this:
\begin{equation}
1 
\end{equation}
It is now obvious, that the derivative of this is equal to
\begin{equation}
0 
\end{equation}

We are going to study the following:
\begin{equation}
x ^{2 } 
\end{equation}
It is now obvious, that the derivative of this is equal to
\begin{equation}
2 \cdot x ^{2 - 1 } \cdot 1 
\end{equation}

The object of our ultimate interest is the following:
\begin{equation}
2 
\end{equation}
Any self-respecting mathematician would find it obvious, that the derivative of this is equal to
\begin{equation}
0 
\end{equation}

We will take a closer look at this:
\begin{equation}
1 
\end{equation}
Obviously, the derivative of this is equal to
\begin{equation}
0 
\end{equation}

We are going to study the following:
\begin{equation}
1 
\end{equation}
Trivially, the derivative of this is equal to
\begin{equation}
0 
\end{equation}

The object of our ultimate interest is the following:
\begin{equation}
x ^{2 } 
\end{equation}
Unsurprisingly, the derivative of this is equal to
\begin{equation}
2 \cdot x ^{2 - 1 } \cdot 1 
\end{equation}

One shall regard the object in question with utmost interest:
\begin{equation}
1 
\end{equation}
Unsurprisingly, the derivative of this is equal to
\begin{equation}
0 
\end{equation}

The following is worth a closer look:
\begin{equation}
1 
\end{equation}
Obviously, the derivative of this is equal to
\begin{equation}
0 
\end{equation}

We are going to study the following:
\begin{equation}
x - 2 
\end{equation}
Obviously, the derivative of this is equal to
\begin{equation}
1 - 0 
\end{equation}

Let us take a look at this:
\begin{equation}
1 
\end{equation}
Obviously, the derivative of this is equal to
\begin{equation}
0 
\end{equation}

We are going to study the following:
\begin{equation}
x ^{2 } 
\end{equation}
Trivially, the derivative of this is equal to
\begin{equation}
2 \cdot x ^{2 - 1 } \cdot 1 
\end{equation}

The following is worth a closer look:
\begin{equation}
1 
\end{equation}
Clearly, the derivative of this is equal to
\begin{equation}
0 
\end{equation}

We shall ponder the following:
\begin{equation}
x ^{2 } 
\end{equation}
It can be easily proved, that the derivative of this is equal to
\begin{equation}
2 \cdot x ^{2 - 1 } \cdot 1 
\end{equation}

Consider the following:
\begin{equation}
x - 2 
\end{equation}
Obviously, the derivative of this is equal to
\begin{equation}
1 - 0 
\end{equation}

We will allow ourselves to divert the reader's attentions to this gem of mathematical wonder:
\begin{equation}
1 
\end{equation}
Obviously, the derivative of this is equal to
\begin{equation}
0 
\end{equation}

We shall ponder the following:
\begin{equation}
x ^{2 } 
\end{equation}
Any self-respecting mathematician would find it obvious, that the derivative of this is equal to
\begin{equation}
2 \cdot x ^{2 - 1 } \cdot 1 
\end{equation}

One shall regard the object in question with utmost interest:
\begin{equation}
2 \cdot x 
\end{equation}
As you can see, the derivative of this is equal to
\begin{equation}
0 \cdot x + 2 \cdot 1 
\end{equation}

One shall regard the object in question with utmost interest:
\begin{equation}
1 
\end{equation}
It can be easily proved, that the derivative of this is equal to
\begin{equation}
0 
\end{equation}

Let us take a look at this:
\begin{equation}
x ^{2 } 
\end{equation}
Any self-respecting mathematician would find it obvious, that the derivative of this is equal to
\begin{equation}
2 \cdot x ^{2 - 1 } \cdot 1 
\end{equation}

We will allow ourselves to divert the reader's attentions to this gem of mathematical wonder:
\begin{equation}
2 
\end{equation}
Unsurprisingly, the derivative of this is equal to
\begin{equation}
0 
\end{equation}

We will allow ourselves to divert the reader's attentions to this gem of mathematical wonder:
\begin{equation}
1 
\end{equation}
It can be easily proved, that the derivative of this is equal to
\begin{equation}
0 
\end{equation}

We shall ponder the following:
\begin{equation}
1 
\end{equation}
Obviously, the derivative of this is equal to
\begin{equation}
0 
\end{equation}

The object of our ultimate interest is the following:
\begin{equation}
x ^{2 } 
\end{equation}
As you can see, the derivative of this is equal to
\begin{equation}
2 \cdot x ^{2 - 1 } \cdot 1 
\end{equation}

Consider the following:
\begin{equation}
1 
\end{equation}
It is now obvious, that the derivative of this is equal to
\begin{equation}
0 
\end{equation}

We will take a closer look at this:
\begin{equation}
1 
\end{equation}
It is now obvious, that the derivative of this is equal to
\begin{equation}
0 
\end{equation}

We are going to study the following:
\begin{equation}
x - 2 
\end{equation}
Clearly, the derivative of this is equal to
\begin{equation}
1 - 0 
\end{equation}

We are going to study the following:
\begin{equation}
1 
\end{equation}
It can be easily proved, that the derivative of this is equal to
\begin{equation}
0 
\end{equation}

We are going to study the following:
\begin{equation}
x ^{2 } 
\end{equation}
Trivially, the derivative of this is equal to
\begin{equation}
2 \cdot x ^{2 - 1 } \cdot 1 
\end{equation}

We shall ponder the following:
\begin{equation}
1 
\end{equation}
Unsurprisingly, the derivative of this is equal to
\begin{equation}
0 
\end{equation}

We will allow ourselves to divert the reader's attentions to this gem of mathematical wonder:
\begin{equation}
x ^{2 } 
\end{equation}
Any self-respecting mathematician would find it obvious, that the derivative of this is equal to
\begin{equation}
2 \cdot x ^{2 - 1 } \cdot 1 
\end{equation}

We are going to study the following:
\begin{equation}
2 \cdot x 
\end{equation}
It can be easily proved, that the derivative of this is equal to
\begin{equation}
0 \cdot x + 2 \cdot 1 
\end{equation}

We will take a closer look at this:
\begin{equation}
1 
\end{equation}
It is now obvious, that the derivative of this is equal to
\begin{equation}
0 
\end{equation}

The object of our ultimate interest is the following:
\begin{equation}
x ^{2 } 
\end{equation}
Any self-respecting mathematician would find it obvious, that the derivative of this is equal to
\begin{equation}
2 \cdot x ^{2 - 1 } \cdot 1 
\end{equation}

We will allow ourselves to divert the reader's attentions to this gem of mathematical wonder:
\begin{equation}
2 
\end{equation}
Clearly, the derivative of this is equal to
\begin{equation}
0 
\end{equation}

Let us take a look at this:
\begin{equation}
1 
\end{equation}
Obviously, the derivative of this is equal to
\begin{equation}
0 
\end{equation}

The following is worth a closer look:
\begin{equation}
1 
\end{equation}
Trivially, the derivative of this is equal to
\begin{equation}
0 
\end{equation}

We will allow ourselves to divert the reader's attentions to this gem of mathematical wonder:
\begin{equation}
x ^{2 } 
\end{equation}
Clearly, the derivative of this is equal to
\begin{equation}
2 \cdot x ^{2 - 1 } \cdot 1 
\end{equation}

We shall ponder the following:
\begin{equation}
1 
\end{equation}
It is now obvious, that the derivative of this is equal to
\begin{equation}
0 
\end{equation}

We will allow ourselves to divert the reader's attentions to this gem of mathematical wonder:
\begin{equation}
1 
\end{equation}
Clearly, the derivative of this is equal to
\begin{equation}
0 
\end{equation}

The following is worth a closer look:
\begin{equation}
2 \cdot x 
\end{equation}
It is now obvious, that the derivative of this is equal to
\begin{equation}
0 \cdot x + 2 \cdot 1 
\end{equation}

We shall ponder the following:
\begin{equation}
1 
\end{equation}
Trivially, the derivative of this is equal to
\begin{equation}
0 
\end{equation}

We will take a closer look at this:
\begin{equation}
x ^{2 } 
\end{equation}
Unsurprisingly, the derivative of this is equal to
\begin{equation}
2 \cdot x ^{2 - 1 } \cdot 1 
\end{equation}

Consider the following:
\begin{equation}
2 
\end{equation}
It can be easily proved, that the derivative of this is equal to
\begin{equation}
0 
\end{equation}

Let us take a look at this:
\begin{equation}
1 
\end{equation}
It can be easily proved, that the derivative of this is equal to
\begin{equation}
0 
\end{equation}

The following is worth a closer look:
\begin{equation}
1 
\end{equation}
As you can see, the derivative of this is equal to
\begin{equation}
0 
\end{equation}

The object of our ultimate interest is the following:
\begin{equation}
x ^{2 } 
\end{equation}
It can be easily proved, that the derivative of this is equal to
\begin{equation}
2 \cdot x ^{2 - 1 } \cdot 1 
\end{equation}

We shall ponder the following:
\begin{equation}
1 
\end{equation}
Any self-respecting mathematician would find it obvious, that the derivative of this is equal to
\begin{equation}
0 
\end{equation}

Let us take a look at this:
\begin{equation}
1 
\end{equation}
Trivially, the derivative of this is equal to
\begin{equation}
0 
\end{equation}

We shall ponder the following:
\begin{equation}
1 
\end{equation}
As you can see, the derivative of this is equal to
\begin{equation}
0 
\end{equation}

The object of our ultimate interest is the following:
\begin{equation}
x ^{2 } 
\end{equation}
Obviously, the derivative of this is equal to
\begin{equation}
2 \cdot x ^{2 - 1 } \cdot 1 
\end{equation}

We will take a closer look at this:
\begin{equation}
2 \cdot x 
\end{equation}
It is now obvious, that the derivative of this is equal to
\begin{equation}
0 \cdot x + 2 \cdot 1 
\end{equation}

The following is worth a closer look:
\begin{equation}
1 
\end{equation}
Any self-respecting mathematician would find it obvious, that the derivative of this is equal to
\begin{equation}
0 
\end{equation}

One shall regard the object in question with utmost interest:
\begin{equation}
x ^{2 } 
\end{equation}
Unsurprisingly, the derivative of this is equal to
\begin{equation}
2 \cdot x ^{2 - 1 } \cdot 1 
\end{equation}

Let us take a look at this:
\begin{equation}
2 
\end{equation}
Any self-respecting mathematician would find it obvious, that the derivative of this is equal to
\begin{equation}
0 
\end{equation}

We shall ponder the following:
\begin{equation}
1 
\end{equation}
It is now obvious, that the derivative of this is equal to
\begin{equation}
0 
\end{equation}

We will allow ourselves to divert the reader's attentions to this gem of mathematical wonder:
\begin{equation}
1 
\end{equation}
It can be easily proved, that the derivative of this is equal to
\begin{equation}
0 
\end{equation}

Let us take a look at this:
\begin{equation}
x ^{2 } 
\end{equation}
Unsurprisingly, the derivative of this is equal to
\begin{equation}
2 \cdot x ^{2 - 1 } \cdot 1 
\end{equation}

The object of our ultimate interest is the following:
\begin{equation}
1 
\end{equation}
It is now obvious, that the derivative of this is equal to
\begin{equation}
0 
\end{equation}

The following is worth a closer look:
\begin{equation}
2 \cdot x 
\end{equation}
Trivially, the derivative of this is equal to
\begin{equation}
0 \cdot x + 2 \cdot 1 
\end{equation}

One shall regard the object in question with utmost interest:
\begin{equation}
1 
\end{equation}
Obviously, the derivative of this is equal to
\begin{equation}
0 
\end{equation}

The following is worth a closer look:
\begin{equation}
x ^{2 } 
\end{equation}
Any self-respecting mathematician would find it obvious, that the derivative of this is equal to
\begin{equation}
2 \cdot x ^{2 - 1 } \cdot 1 
\end{equation}

One shall regard the object in question with utmost interest:
\begin{equation}
2 
\end{equation}
It is now obvious, that the derivative of this is equal to
\begin{equation}
0 
\end{equation}

The object of our ultimate interest is the following:
\begin{equation}
1 
\end{equation}
It is now obvious, that the derivative of this is equal to
\begin{equation}
0 
\end{equation}

One shall regard the object in question with utmost interest:
\begin{equation}
1 
\end{equation}
Obviously, the derivative of this is equal to
\begin{equation}
0 
\end{equation}

We will take a closer look at this:
\begin{equation}
x ^{2 } 
\end{equation}
It is now obvious, that the derivative of this is equal to
\begin{equation}
2 \cdot x ^{2 - 1 } \cdot 1 
\end{equation}

We will take a closer look at this:
\begin{equation}
2 
\end{equation}
Unsurprisingly, the derivative of this is equal to
\begin{equation}
0 
\end{equation}

The object of our ultimate interest is the following:
\begin{equation}
2 \cdot x 
\end{equation}
Trivially, the derivative of this is equal to
\begin{equation}
0 \cdot x + 2 \cdot 1 
\end{equation}

Consider the following:
\begin{equation}
1 
\end{equation}
Trivially, the derivative of this is equal to
\begin{equation}
0 
\end{equation}

We will take a closer look at this:
\begin{equation}
x ^{2 } 
\end{equation}
Any self-respecting mathematician would find it obvious, that the derivative of this is equal to
\begin{equation}
2 \cdot x ^{2 - 1 } \cdot 1 
\end{equation}

We are going to study the following:
\begin{equation}
2 
\end{equation}
Any self-respecting mathematician would find it obvious, that the derivative of this is equal to
\begin{equation}
0 
\end{equation}

Consider the following:
\begin{equation}
2 \cdot x 
\end{equation}
As you can see, the derivative of this is equal to
\begin{equation}
0 \cdot x + 2 \cdot 1 
\end{equation}

The object of our ultimate interest is the following:
\begin{equation}
1 
\end{equation}
Clearly, the derivative of this is equal to
\begin{equation}
0 
\end{equation}

We are going to study the following:
\begin{equation}
x ^{2 } 
\end{equation}
Trivially, the derivative of this is equal to
\begin{equation}
2 \cdot x ^{2 - 1 } \cdot 1 
\end{equation}

We will take a closer look at this:
\begin{equation}
2 
\end{equation}
Trivially, the derivative of this is equal to
\begin{equation}
0 
\end{equation}

One shall regard the object in question with utmost interest:
\begin{equation}
1 
\end{equation}
As you can see, the derivative of this is equal to
\begin{equation}
0 
\end{equation}

We will allow ourselves to divert the reader's attentions to this gem of mathematical wonder:
\begin{equation}
2 
\end{equation}
Clearly, the derivative of this is equal to
\begin{equation}
0 
\end{equation}

We will allow ourselves to divert the reader's attentions to this gem of mathematical wonder:
\begin{equation}
2 
\end{equation}
Clearly, the derivative of this is equal to
\begin{equation}
0 
\end{equation}

We will take a closer look at this:
\begin{equation}
1 
\end{equation}
It can be easily proved, that the derivative of this is equal to
\begin{equation}
0 
\end{equation}

We are going to study the following:
\begin{equation}
x ^{2 } 
\end{equation}
Unsurprisingly, the derivative of this is equal to
\begin{equation}
2 \cdot x ^{2 - 1 } \cdot 1 
\end{equation}

We will take a closer look at this:
\begin{equation}
2 
\end{equation}
As you can see, the derivative of this is equal to
\begin{equation}
0 
\end{equation}

The object of our ultimate interest is the following:
\begin{equation}
1 
\end{equation}
Unsurprisingly, the derivative of this is equal to
\begin{equation}
0 
\end{equation}

One shall regard the object in question with utmost interest:
\begin{equation}
1 
\end{equation}
It is now obvious, that the derivative of this is equal to
\begin{equation}
0 
\end{equation}

Consider the following:
\begin{equation}
x ^{2 } 
\end{equation}
It can be easily proved, that the derivative of this is equal to
\begin{equation}
2 \cdot x ^{2 - 1 } \cdot 1 
\end{equation}

We will allow ourselves to divert the reader's attentions to this gem of mathematical wonder:
\begin{equation}
1 
\end{equation}
Trivially, the derivative of this is equal to
\begin{equation}
0 
\end{equation}

The following is worth a closer look:
\begin{equation}
1 
\end{equation}
As you can see, the derivative of this is equal to
\begin{equation}
0 
\end{equation}

We are going to study the following:
\begin{equation}
x - 2 
\end{equation}
Clearly, the derivative of this is equal to
\begin{equation}
1 - 0 
\end{equation}

We will allow ourselves to divert the reader's attentions to this gem of mathematical wonder:
\begin{equation}
1 
\end{equation}
As you can see, the derivative of this is equal to
\begin{equation}
0 
\end{equation}

We are going to study the following:
\begin{equation}
x ^{2 } 
\end{equation}
Any self-respecting mathematician would find it obvious, that the derivative of this is equal to
\begin{equation}
2 \cdot x ^{2 - 1 } \cdot 1 
\end{equation}

The following is worth a closer look:
\begin{equation}
2 \cdot x 
\end{equation}
It can be easily proved, that the derivative of this is equal to
\begin{equation}
0 \cdot x + 2 \cdot 1 
\end{equation}

One shall regard the object in question with utmost interest:
\begin{equation}
1 
\end{equation}
Any self-respecting mathematician would find it obvious, that the derivative of this is equal to
\begin{equation}
0 
\end{equation}

Let us take a look at this:
\begin{equation}
x ^{2 } 
\end{equation}
Obviously, the derivative of this is equal to
\begin{equation}
2 \cdot x ^{2 - 1 } \cdot 1 
\end{equation}

The following is worth a closer look:
\begin{equation}
2 
\end{equation}
It can be easily proved, that the derivative of this is equal to
\begin{equation}
0 
\end{equation}

The following is worth a closer look:
\begin{equation}
1 
\end{equation}
It is now obvious, that the derivative of this is equal to
\begin{equation}
0 
\end{equation}

The object of our ultimate interest is the following:
\begin{equation}
1 
\end{equation}
Any self-respecting mathematician would find it obvious, that the derivative of this is equal to
\begin{equation}
0 
\end{equation}

We will take a closer look at this:
\begin{equation}
x ^{2 } 
\end{equation}
Unsurprisingly, the derivative of this is equal to
\begin{equation}
2 \cdot x ^{2 - 1 } \cdot 1 
\end{equation}

Consider the following:
\begin{equation}
1 
\end{equation}
It is now obvious, that the derivative of this is equal to
\begin{equation}
0 
\end{equation}

Consider the following:
\begin{equation}
1 
\end{equation}
As you can see, the derivative of this is equal to
\begin{equation}
0 
\end{equation}

One shall regard the object in question with utmost interest:
\begin{equation}
2 \cdot x 
\end{equation}
Trivially, the derivative of this is equal to
\begin{equation}
0 \cdot x + 2 \cdot 1 
\end{equation}

The object of our ultimate interest is the following:
\begin{equation}
1 
\end{equation}
Trivially, the derivative of this is equal to
\begin{equation}
0 
\end{equation}

We will allow ourselves to divert the reader's attentions to this gem of mathematical wonder:
\begin{equation}
x ^{2 } 
\end{equation}
Trivially, the derivative of this is equal to
\begin{equation}
2 \cdot x ^{2 - 1 } \cdot 1 
\end{equation}

We are going to study the following:
\begin{equation}
2 
\end{equation}
It is now obvious, that the derivative of this is equal to
\begin{equation}
0 
\end{equation}

The object of our ultimate interest is the following:
\begin{equation}
1 
\end{equation}
Obviously, the derivative of this is equal to
\begin{equation}
0 
\end{equation}

We are going to study the following:
\begin{equation}
1 
\end{equation}
It can be easily proved, that the derivative of this is equal to
\begin{equation}
0 
\end{equation}

We shall ponder the following:
\begin{equation}
x ^{2 } 
\end{equation}
Clearly, the derivative of this is equal to
\begin{equation}
2 \cdot x ^{2 - 1 } \cdot 1 
\end{equation}

We are going to study the following:
\begin{equation}
1 
\end{equation}
Any self-respecting mathematician would find it obvious, that the derivative of this is equal to
\begin{equation}
0 
\end{equation}

One shall regard the object in question with utmost interest:
\begin{equation}
1 
\end{equation}
Obviously, the derivative of this is equal to
\begin{equation}
0 
\end{equation}

We will take a closer look at this:
\begin{equation}
1 
\end{equation}
Trivially, the derivative of this is equal to
\begin{equation}
0 
\end{equation}

The object of our ultimate interest is the following:
\begin{equation}
x ^{2 } 
\end{equation}
It can be easily proved, that the derivative of this is equal to
\begin{equation}
2 \cdot x ^{2 - 1 } \cdot 1 
\end{equation}

We will take a closer look at this:
\begin{equation}
1 
\end{equation}
Any self-respecting mathematician would find it obvious, that the derivative of this is equal to
\begin{equation}
0 
\end{equation}

We are going to study the following:
\begin{equation}
1 
\end{equation}
It can be easily proved, that the derivative of this is equal to
\begin{equation}
0 
\end{equation}

We are going to study the following:
\begin{equation}
x ^{2 } 
\end{equation}
Unsurprisingly, the derivative of this is equal to
\begin{equation}
2 \cdot x ^{2 - 1 } \cdot 1 
\end{equation}

We will allow ourselves to divert the reader's attentions to this gem of mathematical wonder:
\begin{equation}
x - 2 
\end{equation}
Unsurprisingly, the derivative of this is equal to
\begin{equation}
1 - 0 
\end{equation}

We shall ponder the following:
\begin{equation}
1 
\end{equation}
It is now obvious, that the derivative of this is equal to
\begin{equation}
0 
\end{equation}

We will take a closer look at this:
\begin{equation}
x ^{2 } 
\end{equation}
Any self-respecting mathematician would find it obvious, that the derivative of this is equal to
\begin{equation}
2 \cdot x ^{2 - 1 } \cdot 1 
\end{equation}

We are going to study the following:
\begin{equation}
2 \cdot x 
\end{equation}
Any self-respecting mathematician would find it obvious, that the derivative of this is equal to
\begin{equation}
0 \cdot x + 2 \cdot 1 
\end{equation}

We will take a closer look at this:
\begin{equation}
1 
\end{equation}
Obviously, the derivative of this is equal to
\begin{equation}
0 
\end{equation}

The object of our ultimate interest is the following:
\begin{equation}
x ^{2 } 
\end{equation}
Clearly, the derivative of this is equal to
\begin{equation}
2 \cdot x ^{2 - 1 } \cdot 1 
\end{equation}

Let us take a look at this:
\begin{equation}
2 
\end{equation}
Any self-respecting mathematician would find it obvious, that the derivative of this is equal to
\begin{equation}
0 
\end{equation}

The following is worth a closer look:
\begin{equation}
1 
\end{equation}
Obviously, the derivative of this is equal to
\begin{equation}
0 
\end{equation}

We shall ponder the following:
\begin{equation}
1 
\end{equation}
As you can see, the derivative of this is equal to
\begin{equation}
0 
\end{equation}

We are going to study the following:
\begin{equation}
x ^{2 } 
\end{equation}
Any self-respecting mathematician would find it obvious, that the derivative of this is equal to
\begin{equation}
2 \cdot x ^{2 - 1 } \cdot 1 
\end{equation}

Consider the following:
\begin{equation}
1 
\end{equation}
It is now obvious, that the derivative of this is equal to
\begin{equation}
0 
\end{equation}

We are going to study the following:
\begin{equation}
1 
\end{equation}
Clearly, the derivative of this is equal to
\begin{equation}
0 
\end{equation}

One shall regard the object in question with utmost interest:
\begin{equation}
x - 2 
\end{equation}
Trivially, the derivative of this is equal to
\begin{equation}
1 - 0 
\end{equation}

The object of our ultimate interest is the following:
\begin{equation}
1 
\end{equation}
Unsurprisingly, the derivative of this is equal to
\begin{equation}
0 
\end{equation}

One shall regard the object in question with utmost interest:
\begin{equation}
x ^{2 } 
\end{equation}
Obviously, the derivative of this is equal to
\begin{equation}
2 \cdot x ^{2 - 1 } \cdot 1 
\end{equation}

We shall ponder the following:
\begin{equation}
1 
\end{equation}
As you can see, the derivative of this is equal to
\begin{equation}
0 
\end{equation}

One shall regard the object in question with utmost interest:
\begin{equation}
x ^{2 } 
\end{equation}
Obviously, the derivative of this is equal to
\begin{equation}
2 \cdot x ^{2 - 1 } \cdot 1 
\end{equation}

Let us take a look at this:
\begin{equation}
2 \cdot x 
\end{equation}
It is now obvious, that the derivative of this is equal to
\begin{equation}
0 \cdot x + 2 \cdot 1 
\end{equation}

The following is worth a closer look:
\begin{equation}
1 
\end{equation}
It can be easily proved, that the derivative of this is equal to
\begin{equation}
0 
\end{equation}

We shall ponder the following:
\begin{equation}
x ^{2 } 
\end{equation}
Obviously, the derivative of this is equal to
\begin{equation}
2 \cdot x ^{2 - 1 } \cdot 1 
\end{equation}

The object of our ultimate interest is the following:
\begin{equation}
2 
\end{equation}
It can be easily proved, that the derivative of this is equal to
\begin{equation}
0 
\end{equation}

The following is worth a closer look:
\begin{equation}
1 
\end{equation}
As you can see, the derivative of this is equal to
\begin{equation}
0 
\end{equation}

We are going to study the following:
\begin{equation}
1 
\end{equation}
Unsurprisingly, the derivative of this is equal to
\begin{equation}
0 
\end{equation}

Let us take a look at this:
\begin{equation}
x ^{2 } 
\end{equation}
Unsurprisingly, the derivative of this is equal to
\begin{equation}
2 \cdot x ^{2 - 1 } \cdot 1 
\end{equation}

We shall ponder the following:
\begin{equation}
1 
\end{equation}
It is now obvious, that the derivative of this is equal to
\begin{equation}
0 
\end{equation}

The object of our ultimate interest is the following:
\begin{equation}
1 
\end{equation}
As you can see, the derivative of this is equal to
\begin{equation}
0 
\end{equation}

We are going to study the following:
\begin{equation}
x - 2 
\end{equation}
Trivially, the derivative of this is equal to
\begin{equation}
1 - 0 
\end{equation}

We are going to study the following:
\begin{equation}
1 
\end{equation}
Clearly, the derivative of this is equal to
\begin{equation}
0 
\end{equation}

One shall regard the object in question with utmost interest:
\begin{equation}
x ^{2 } 
\end{equation}
It can be easily proved, that the derivative of this is equal to
\begin{equation}
2 \cdot x ^{2 - 1 } \cdot 1 
\end{equation}

The following is worth a closer look:
\begin{equation}
1 
\end{equation}
Any self-respecting mathematician would find it obvious, that the derivative of this is equal to
\begin{equation}
0 
\end{equation}

The following is worth a closer look:
\begin{equation}
x ^{2 } 
\end{equation}
Clearly, the derivative of this is equal to
\begin{equation}
2 \cdot x ^{2 - 1 } \cdot 1 
\end{equation}

The following is worth a closer look:
\begin{equation}
x - 2 
\end{equation}
It is now obvious, that the derivative of this is equal to
\begin{equation}
1 - 0 
\end{equation}

We shall ponder the following:
\begin{equation}
x + 1 
\end{equation}
Obviously, the derivative of this is equal to
\begin{equation}
1 + 0 
\end{equation}

Consider the following:
\begin{equation}
2 
\end{equation}
Trivially, the derivative of this is equal to
\begin{equation}
0 
\end{equation}

The following is worth a closer look:
\begin{equation}
\sin x 
\end{equation}
It is now obvious, that the derivative of this is equal to
\begin{equation}
\cos x \cdot 1 
\end{equation}

The object of our ultimate interest is the following:
\begin{equation}
1 
\end{equation}
It can be easily proved, that the derivative of this is equal to
\begin{equation}
0 
\end{equation}

The object of our ultimate interest is the following:
\begin{equation}
x ^{2 } 
\end{equation}
Clearly, the derivative of this is equal to
\begin{equation}
2 \cdot x ^{2 - 1 } \cdot 1 
\end{equation}

We will allow ourselves to divert the reader's attentions to this gem of mathematical wonder:
\begin{equation}
2 \cdot x 
\end{equation}
Unsurprisingly, the derivative of this is equal to
\begin{equation}
0 \cdot x + 2 \cdot 1 
\end{equation}

Consider the following:
\begin{equation}
1 
\end{equation}
Unsurprisingly, the derivative of this is equal to
\begin{equation}
0 
\end{equation}

Consider the following:
\begin{equation}
x ^{2 } 
\end{equation}
Clearly, the derivative of this is equal to
\begin{equation}
2 \cdot x ^{2 - 1 } \cdot 1 
\end{equation}

We will allow ourselves to divert the reader's attentions to this gem of mathematical wonder:
\begin{equation}
2 
\end{equation}
Clearly, the derivative of this is equal to
\begin{equation}
0 
\end{equation}

Consider the following:
\begin{equation}
1 
\end{equation}
Unsurprisingly, the derivative of this is equal to
\begin{equation}
0 
\end{equation}

The object of our ultimate interest is the following:
\begin{equation}
1 
\end{equation}
As you can see, the derivative of this is equal to
\begin{equation}
0 
\end{equation}

One shall regard the object in question with utmost interest:
\begin{equation}
x ^{2 } 
\end{equation}
It can be easily proved, that the derivative of this is equal to
\begin{equation}
2 \cdot x ^{2 - 1 } \cdot 1 
\end{equation}

We will allow ourselves to divert the reader's attentions to this gem of mathematical wonder:
\begin{equation}
1 
\end{equation}
Trivially, the derivative of this is equal to
\begin{equation}
0 
\end{equation}

The object of our ultimate interest is the following:
\begin{equation}
1 
\end{equation}
As you can see, the derivative of this is equal to
\begin{equation}
0 
\end{equation}

We shall ponder the following:
\begin{equation}
2 \cdot x 
\end{equation}
As you can see, the derivative of this is equal to
\begin{equation}
0 \cdot x + 2 \cdot 1 
\end{equation}

The following is worth a closer look:
\begin{equation}
1 
\end{equation}
Clearly, the derivative of this is equal to
\begin{equation}
0 
\end{equation}

The following is worth a closer look:
\begin{equation}
x ^{2 } 
\end{equation}
Obviously, the derivative of this is equal to
\begin{equation}
2 \cdot x ^{2 - 1 } \cdot 1 
\end{equation}

We shall ponder the following:
\begin{equation}
2 
\end{equation}
Any self-respecting mathematician would find it obvious, that the derivative of this is equal to
\begin{equation}
0 
\end{equation}

The following is worth a closer look:
\begin{equation}
1 
\end{equation}
Clearly, the derivative of this is equal to
\begin{equation}
0 
\end{equation}

We will allow ourselves to divert the reader's attentions to this gem of mathematical wonder:
\begin{equation}
1 
\end{equation}
It is now obvious, that the derivative of this is equal to
\begin{equation}
0 
\end{equation}

We will take a closer look at this:
\begin{equation}
x ^{2 } 
\end{equation}
It is now obvious, that the derivative of this is equal to
\begin{equation}
2 \cdot x ^{2 - 1 } \cdot 1 
\end{equation}

Let us take a look at this:
\begin{equation}
1 
\end{equation}
Clearly, the derivative of this is equal to
\begin{equation}
0 
\end{equation}

Let us take a look at this:
\begin{equation}
1 
\end{equation}
It is now obvious, that the derivative of this is equal to
\begin{equation}
0 
\end{equation}

We will take a closer look at this:
\begin{equation}
1 
\end{equation}
It is now obvious, that the derivative of this is equal to
\begin{equation}
0 
\end{equation}

We are going to study the following:
\begin{equation}
x ^{2 } 
\end{equation}
Clearly, the derivative of this is equal to
\begin{equation}
2 \cdot x ^{2 - 1 } \cdot 1 
\end{equation}

Consider the following:
\begin{equation}
2 \cdot x 
\end{equation}
As you can see, the derivative of this is equal to
\begin{equation}
0 \cdot x + 2 \cdot 1 
\end{equation}

We will take a closer look at this:
\begin{equation}
1 
\end{equation}
Trivially, the derivative of this is equal to
\begin{equation}
0 
\end{equation}

Let us take a look at this:
\begin{equation}
x ^{2 } 
\end{equation}
Obviously, the derivative of this is equal to
\begin{equation}
2 \cdot x ^{2 - 1 } \cdot 1 
\end{equation}

The following is worth a closer look:
\begin{equation}
2 
\end{equation}
As you can see, the derivative of this is equal to
\begin{equation}
0 
\end{equation}

The following is worth a closer look:
\begin{equation}
1 
\end{equation}
As you can see, the derivative of this is equal to
\begin{equation}
0 
\end{equation}

We shall ponder the following:
\begin{equation}
1 
\end{equation}
It can be easily proved, that the derivative of this is equal to
\begin{equation}
0 
\end{equation}

We will allow ourselves to divert the reader's attentions to this gem of mathematical wonder:
\begin{equation}
x ^{2 } 
\end{equation}
As you can see, the derivative of this is equal to
\begin{equation}
2 \cdot x ^{2 - 1 } \cdot 1 
\end{equation}

We will take a closer look at this:
\begin{equation}
1 
\end{equation}
As you can see, the derivative of this is equal to
\begin{equation}
0 
\end{equation}

The object of our ultimate interest is the following:
\begin{equation}
2 \cdot x 
\end{equation}
Any self-respecting mathematician would find it obvious, that the derivative of this is equal to
\begin{equation}
0 \cdot x + 2 \cdot 1 
\end{equation}

Let us take a look at this:
\begin{equation}
1 
\end{equation}
It can be easily proved, that the derivative of this is equal to
\begin{equation}
0 
\end{equation}

We will take a closer look at this:
\begin{equation}
x ^{2 } 
\end{equation}
Trivially, the derivative of this is equal to
\begin{equation}
2 \cdot x ^{2 - 1 } \cdot 1 
\end{equation}

We will take a closer look at this:
\begin{equation}
2 
\end{equation}
As you can see, the derivative of this is equal to
\begin{equation}
0 
\end{equation}

We are going to study the following:
\begin{equation}
1 
\end{equation}
Any self-respecting mathematician would find it obvious, that the derivative of this is equal to
\begin{equation}
0 
\end{equation}

One shall regard the object in question with utmost interest:
\begin{equation}
1 
\end{equation}
It is now obvious, that the derivative of this is equal to
\begin{equation}
0 
\end{equation}

The following is worth a closer look:
\begin{equation}
x ^{2 } 
\end{equation}
It can be easily proved, that the derivative of this is equal to
\begin{equation}
2 \cdot x ^{2 - 1 } \cdot 1 
\end{equation}

One shall regard the object in question with utmost interest:
\begin{equation}
2 
\end{equation}
Unsurprisingly, the derivative of this is equal to
\begin{equation}
0 
\end{equation}

We will allow ourselves to divert the reader's attentions to this gem of mathematical wonder:
\begin{equation}
2 \cdot x 
\end{equation}
As you can see, the derivative of this is equal to
\begin{equation}
0 \cdot x + 2 \cdot 1 
\end{equation}

Let us take a look at this:
\begin{equation}
1 
\end{equation}
As you can see, the derivative of this is equal to
\begin{equation}
0 
\end{equation}

We are going to study the following:
\begin{equation}
x ^{2 } 
\end{equation}
Any self-respecting mathematician would find it obvious, that the derivative of this is equal to
\begin{equation}
2 \cdot x ^{2 - 1 } \cdot 1 
\end{equation}

The object of our ultimate interest is the following:
\begin{equation}
2 
\end{equation}
Trivially, the derivative of this is equal to
\begin{equation}
0 
\end{equation}

Consider the following:
\begin{equation}
2 \cdot x 
\end{equation}
Unsurprisingly, the derivative of this is equal to
\begin{equation}
0 \cdot x + 2 \cdot 1 
\end{equation}

The following is worth a closer look:
\begin{equation}
1 
\end{equation}
Trivially, the derivative of this is equal to
\begin{equation}
0 
\end{equation}

Let us take a look at this:
\begin{equation}
x ^{2 } 
\end{equation}
As you can see, the derivative of this is equal to
\begin{equation}
2 \cdot x ^{2 - 1 } \cdot 1 
\end{equation}

We are going to study the following:
\begin{equation}
2 
\end{equation}
Unsurprisingly, the derivative of this is equal to
\begin{equation}
0 
\end{equation}

The following is worth a closer look:
\begin{equation}
1 
\end{equation}
Unsurprisingly, the derivative of this is equal to
\begin{equation}
0 
\end{equation}

We will allow ourselves to divert the reader's attentions to this gem of mathematical wonder:
\begin{equation}
2 
\end{equation}
Obviously, the derivative of this is equal to
\begin{equation}
0 
\end{equation}

One shall regard the object in question with utmost interest:
\begin{equation}
2 
\end{equation}
It is now obvious, that the derivative of this is equal to
\begin{equation}
0 
\end{equation}

We shall ponder the following:
\begin{equation}
1 
\end{equation}
It can be easily proved, that the derivative of this is equal to
\begin{equation}
0 
\end{equation}

Let us take a look at this:
\begin{equation}
x ^{2 } 
\end{equation}
Obviously, the derivative of this is equal to
\begin{equation}
2 \cdot x ^{2 - 1 } \cdot 1 
\end{equation}

Let us take a look at this:
\begin{equation}
2 
\end{equation}
Any self-respecting mathematician would find it obvious, that the derivative of this is equal to
\begin{equation}
0 
\end{equation}

Let us take a look at this:
\begin{equation}
1 
\end{equation}
Obviously, the derivative of this is equal to
\begin{equation}
0 
\end{equation}

One shall regard the object in question with utmost interest:
\begin{equation}
1 
\end{equation}
It can be easily proved, that the derivative of this is equal to
\begin{equation}
0 
\end{equation}

We will allow ourselves to divert the reader's attentions to this gem of mathematical wonder:
\begin{equation}
x ^{2 } 
\end{equation}
It is now obvious, that the derivative of this is equal to
\begin{equation}
2 \cdot x ^{2 - 1 } \cdot 1 
\end{equation}

We are going to study the following:
\begin{equation}
1 
\end{equation}
Unsurprisingly, the derivative of this is equal to
\begin{equation}
0 
\end{equation}

We will take a closer look at this:
\begin{equation}
1 
\end{equation}
Clearly, the derivative of this is equal to
\begin{equation}
0 
\end{equation}

Consider the following:
\begin{equation}
x - 2 
\end{equation}
As you can see, the derivative of this is equal to
\begin{equation}
1 - 0 
\end{equation}

One shall regard the object in question with utmost interest:
\begin{equation}
1 
\end{equation}
Unsurprisingly, the derivative of this is equal to
\begin{equation}
0 
\end{equation}

Let us take a look at this:
\begin{equation}
x ^{2 } 
\end{equation}
It is now obvious, that the derivative of this is equal to
\begin{equation}
2 \cdot x ^{2 - 1 } \cdot 1 
\end{equation}

We shall ponder the following:
\begin{equation}
2 \cdot x 
\end{equation}
Trivially, the derivative of this is equal to
\begin{equation}
0 \cdot x + 2 \cdot 1 
\end{equation}

We shall ponder the following:
\begin{equation}
1 
\end{equation}
As you can see, the derivative of this is equal to
\begin{equation}
0 
\end{equation}

The object of our ultimate interest is the following:
\begin{equation}
x ^{2 } 
\end{equation}
Clearly, the derivative of this is equal to
\begin{equation}
2 \cdot x ^{2 - 1 } \cdot 1 
\end{equation}

One shall regard the object in question with utmost interest:
\begin{equation}
2 
\end{equation}
As you can see, the derivative of this is equal to
\begin{equation}
0 
\end{equation}

The object of our ultimate interest is the following:
\begin{equation}
1 
\end{equation}
It is now obvious, that the derivative of this is equal to
\begin{equation}
0 
\end{equation}

We are going to study the following:
\begin{equation}
1 
\end{equation}
Clearly, the derivative of this is equal to
\begin{equation}
0 
\end{equation}

We shall ponder the following:
\begin{equation}
x ^{2 } 
\end{equation}
Any self-respecting mathematician would find it obvious, that the derivative of this is equal to
\begin{equation}
2 \cdot x ^{2 - 1 } \cdot 1 
\end{equation}

The following is worth a closer look:
\begin{equation}
1 
\end{equation}
Obviously, the derivative of this is equal to
\begin{equation}
0 
\end{equation}

We shall ponder the following:
\begin{equation}
1 
\end{equation}
Obviously, the derivative of this is equal to
\begin{equation}
0 
\end{equation}

We will allow ourselves to divert the reader's attentions to this gem of mathematical wonder:
\begin{equation}
2 \cdot x 
\end{equation}
Unsurprisingly, the derivative of this is equal to
\begin{equation}
0 \cdot x + 2 \cdot 1 
\end{equation}

The following is worth a closer look:
\begin{equation}
1 
\end{equation}
Unsurprisingly, the derivative of this is equal to
\begin{equation}
0 
\end{equation}

Let us take a look at this:
\begin{equation}
x ^{2 } 
\end{equation}
Obviously, the derivative of this is equal to
\begin{equation}
2 \cdot x ^{2 - 1 } \cdot 1 
\end{equation}

The following is worth a closer look:
\begin{equation}
2 
\end{equation}
It can be easily proved, that the derivative of this is equal to
\begin{equation}
0 
\end{equation}

The following is worth a closer look:
\begin{equation}
1 
\end{equation}
Any self-respecting mathematician would find it obvious, that the derivative of this is equal to
\begin{equation}
0 
\end{equation}

We shall ponder the following:
\begin{equation}
1 
\end{equation}
Trivially, the derivative of this is equal to
\begin{equation}
0 
\end{equation}

One shall regard the object in question with utmost interest:
\begin{equation}
x ^{2 } 
\end{equation}
Unsurprisingly, the derivative of this is equal to
\begin{equation}
2 \cdot x ^{2 - 1 } \cdot 1 
\end{equation}

Consider the following:
\begin{equation}
1 
\end{equation}
Any self-respecting mathematician would find it obvious, that the derivative of this is equal to
\begin{equation}
0 
\end{equation}

We will allow ourselves to divert the reader's attentions to this gem of mathematical wonder:
\begin{equation}
1 
\end{equation}
It can be easily proved, that the derivative of this is equal to
\begin{equation}
0 
\end{equation}

The object of our ultimate interest is the following:
\begin{equation}
1 
\end{equation}
Clearly, the derivative of this is equal to
\begin{equation}
0 
\end{equation}

We will allow ourselves to divert the reader's attentions to this gem of mathematical wonder:
\begin{equation}
x ^{2 } 
\end{equation}
It is now obvious, that the derivative of this is equal to
\begin{equation}
2 \cdot x ^{2 - 1 } \cdot 1 
\end{equation}

We will take a closer look at this:
\begin{equation}
1 
\end{equation}
Clearly, the derivative of this is equal to
\begin{equation}
0 
\end{equation}

Let us take a look at this:
\begin{equation}
1 
\end{equation}
Clearly, the derivative of this is equal to
\begin{equation}
0 
\end{equation}

The following is worth a closer look:
\begin{equation}
x ^{2 } 
\end{equation}
As you can see, the derivative of this is equal to
\begin{equation}
2 \cdot x ^{2 - 1 } \cdot 1 
\end{equation}

We will allow ourselves to divert the reader's attentions to this gem of mathematical wonder:
\begin{equation}
x - 2 
\end{equation}
Any self-respecting mathematician would find it obvious, that the derivative of this is equal to
\begin{equation}
1 - 0 
\end{equation}

We will take a closer look at this:
\begin{equation}
1 
\end{equation}
It is now obvious, that the derivative of this is equal to
\begin{equation}
0 
\end{equation}

The following is worth a closer look:
\begin{equation}
x ^{2 } 
\end{equation}
It is now obvious, that the derivative of this is equal to
\begin{equation}
2 \cdot x ^{2 - 1 } \cdot 1 
\end{equation}

We are going to study the following:
\begin{equation}
2 \cdot x 
\end{equation}
Trivially, the derivative of this is equal to
\begin{equation}
0 \cdot x + 2 \cdot 1 
\end{equation}

We are going to study the following:
\begin{equation}
1 
\end{equation}
It can be easily proved, that the derivative of this is equal to
\begin{equation}
0 
\end{equation}

We shall ponder the following:
\begin{equation}
x ^{2 } 
\end{equation}
It is now obvious, that the derivative of this is equal to
\begin{equation}
2 \cdot x ^{2 - 1 } \cdot 1 
\end{equation}

Consider the following:
\begin{equation}
2 
\end{equation}
Any self-respecting mathematician would find it obvious, that the derivative of this is equal to
\begin{equation}
0 
\end{equation}

We shall ponder the following:
\begin{equation}
1 
\end{equation}
It can be easily proved, that the derivative of this is equal to
\begin{equation}
0 
\end{equation}

Let us take a look at this:
\begin{equation}
1 
\end{equation}
Any self-respecting mathematician would find it obvious, that the derivative of this is equal to
\begin{equation}
0 
\end{equation}

The object of our ultimate interest is the following:
\begin{equation}
x ^{2 } 
\end{equation}
It can be easily proved, that the derivative of this is equal to
\begin{equation}
2 \cdot x ^{2 - 1 } \cdot 1 
\end{equation}

We will take a closer look at this:
\begin{equation}
1 
\end{equation}
Any self-respecting mathematician would find it obvious, that the derivative of this is equal to
\begin{equation}
0 
\end{equation}

The object of our ultimate interest is the following:
\begin{equation}
1 
\end{equation}
Unsurprisingly, the derivative of this is equal to
\begin{equation}
0 
\end{equation}

The object of our ultimate interest is the following:
\begin{equation}
x - 2 
\end{equation}
Any self-respecting mathematician would find it obvious, that the derivative of this is equal to
\begin{equation}
1 - 0 
\end{equation}

We will allow ourselves to divert the reader's attentions to this gem of mathematical wonder:
\begin{equation}
1 
\end{equation}
Trivially, the derivative of this is equal to
\begin{equation}
0 
\end{equation}

The object of our ultimate interest is the following:
\begin{equation}
x ^{2 } 
\end{equation}
It is now obvious, that the derivative of this is equal to
\begin{equation}
2 \cdot x ^{2 - 1 } \cdot 1 
\end{equation}

One shall regard the object in question with utmost interest:
\begin{equation}
1 
\end{equation}
Unsurprisingly, the derivative of this is equal to
\begin{equation}
0 
\end{equation}

Consider the following:
\begin{equation}
x ^{2 } 
\end{equation}
It can be easily proved, that the derivative of this is equal to
\begin{equation}
2 \cdot x ^{2 - 1 } \cdot 1 
\end{equation}

We are going to study the following:
\begin{equation}
2 \cdot x 
\end{equation}
Unsurprisingly, the derivative of this is equal to
\begin{equation}
0 \cdot x + 2 \cdot 1 
\end{equation}

One shall regard the object in question with utmost interest:
\begin{equation}
1 
\end{equation}
Clearly, the derivative of this is equal to
\begin{equation}
0 
\end{equation}

Consider the following:
\begin{equation}
x ^{2 } 
\end{equation}
It can be easily proved, that the derivative of this is equal to
\begin{equation}
2 \cdot x ^{2 - 1 } \cdot 1 
\end{equation}

Consider the following:
\begin{equation}
2 
\end{equation}
It can be easily proved, that the derivative of this is equal to
\begin{equation}
0 
\end{equation}

The following is worth a closer look:
\begin{equation}
1 
\end{equation}
Unsurprisingly, the derivative of this is equal to
\begin{equation}
0 
\end{equation}

We will take a closer look at this:
\begin{equation}
1 
\end{equation}
Clearly, the derivative of this is equal to
\begin{equation}
0 
\end{equation}

One shall regard the object in question with utmost interest:
\begin{equation}
x ^{2 } 
\end{equation}
As you can see, the derivative of this is equal to
\begin{equation}
2 \cdot x ^{2 - 1 } \cdot 1 
\end{equation}

We will allow ourselves to divert the reader's attentions to this gem of mathematical wonder:
\begin{equation}
1 
\end{equation}
Any self-respecting mathematician would find it obvious, that the derivative of this is equal to
\begin{equation}
0 
\end{equation}

The object of our ultimate interest is the following:
\begin{equation}
1 
\end{equation}
It can be easily proved, that the derivative of this is equal to
\begin{equation}
0 
\end{equation}

The object of our ultimate interest is the following:
\begin{equation}
x - 2 
\end{equation}
It can be easily proved, that the derivative of this is equal to
\begin{equation}
1 - 0 
\end{equation}

We shall ponder the following:
\begin{equation}
1 
\end{equation}
Trivially, the derivative of this is equal to
\begin{equation}
0 
\end{equation}

Let us take a look at this:
\begin{equation}
x ^{2 } 
\end{equation}
It can be easily proved, that the derivative of this is equal to
\begin{equation}
2 \cdot x ^{2 - 1 } \cdot 1 
\end{equation}

We shall ponder the following:
\begin{equation}
1 
\end{equation}
It can be easily proved, that the derivative of this is equal to
\begin{equation}
0 
\end{equation}

Consider the following:
\begin{equation}
x ^{2 } 
\end{equation}
It can be easily proved, that the derivative of this is equal to
\begin{equation}
2 \cdot x ^{2 - 1 } \cdot 1 
\end{equation}

Let us take a look at this:
\begin{equation}
x - 2 
\end{equation}
Clearly, the derivative of this is equal to
\begin{equation}
1 - 0 
\end{equation}

The following is worth a closer look:
\begin{equation}
x + 1 
\end{equation}
It is now obvious, that the derivative of this is equal to
\begin{equation}
1 + 0 
\end{equation}

The object of our ultimate interest is the following:
\begin{equation}
1 
\end{equation}
It can be easily proved, that the derivative of this is equal to
\begin{equation}
0 
\end{equation}

Let us take a look at this:
\begin{equation}
2 
\end{equation}
It is now obvious, that the derivative of this is equal to
\begin{equation}
0 
\end{equation}

We are going to study the following:
\begin{equation}
\sin x 
\end{equation}
Obviously, the derivative of this is equal to
\begin{equation}
\cos x \cdot 1 
\end{equation}

We will allow ourselves to divert the reader's attentions to this gem of mathematical wonder:
\begin{equation}
x + 1 
\end{equation}
Unsurprisingly, the derivative of this is equal to
\begin{equation}
1 + 0 
\end{equation}

The object of our ultimate interest is the following:
\begin{equation}
4 
\end{equation}
Obviously, the derivative of this is equal to
\begin{equation}
0 
\end{equation}

We shall ponder the following:
\begin{equation}
2 
\end{equation}
It is now obvious, that the derivative of this is equal to
\begin{equation}
0 
\end{equation}

We shall ponder the following:
\begin{equation}
\cos x 
\end{equation}
Any self-respecting mathematician would find it obvious, that the derivative of this is equal to
\begin{equation}
-\sin x \cdot 1 
\end{equation}

Let us take a look at this:
\begin{equation}
x + 1 
\end{equation}
It can be easily proved, that the derivative of this is equal to
\begin{equation}
1 + 0 
\end{equation}

We will allow ourselves to divert the reader's attentions to this gem of mathematical wonder:
\begin{equation}
2 
\end{equation}
Trivially, the derivative of this is equal to
\begin{equation}
0 
\end{equation}

We will allow ourselves to divert the reader's attentions to this gem of mathematical wonder:
\begin{equation}
\sin x 
\end{equation}
Trivially, the derivative of this is equal to
\begin{equation}
\cos x \cdot 1 
\end{equation}

The object of our ultimate interest is the following:
\begin{equation}
1 
\end{equation}
Unsurprisingly, the derivative of this is equal to
\begin{equation}
0 
\end{equation}

Consider the following:
\begin{equation}
x ^{2 } 
\end{equation}
Clearly, the derivative of this is equal to
\begin{equation}
2 \cdot x ^{2 - 1 } \cdot 1 
\end{equation}

We will allow ourselves to divert the reader's attentions to this gem of mathematical wonder:
\begin{equation}
2 \cdot x 
\end{equation}
Clearly, the derivative of this is equal to
\begin{equation}
0 \cdot x + 2 \cdot 1 
\end{equation}

We are going to study the following:
\begin{equation}
1 
\end{equation}
It can be easily proved, that the derivative of this is equal to
\begin{equation}
0 
\end{equation}

We will allow ourselves to divert the reader's attentions to this gem of mathematical wonder:
\begin{equation}
x ^{2 } 
\end{equation}
Obviously, the derivative of this is equal to
\begin{equation}
2 \cdot x ^{2 - 1 } \cdot 1 
\end{equation}

We will take a closer look at this:
\begin{equation}
2 
\end{equation}
It can be easily proved, that the derivative of this is equal to
\begin{equation}
0 
\end{equation}

We shall ponder the following:
\begin{equation}
1 
\end{equation}
Clearly, the derivative of this is equal to
\begin{equation}
0 
\end{equation}

We will allow ourselves to divert the reader's attentions to this gem of mathematical wonder:
\begin{equation}
1 
\end{equation}
Obviously, the derivative of this is equal to
\begin{equation}
0 
\end{equation}

One shall regard the object in question with utmost interest:
\begin{equation}
x ^{2 } 
\end{equation}
Any self-respecting mathematician would find it obvious, that the derivative of this is equal to
\begin{equation}
2 \cdot x ^{2 - 1 } \cdot 1 
\end{equation}

We will take a closer look at this:
\begin{equation}
1 
\end{equation}
Clearly, the derivative of this is equal to
\begin{equation}
0 
\end{equation}

Consider the following:
\begin{equation}
1 
\end{equation}
It is now obvious, that the derivative of this is equal to
\begin{equation}
0 
\end{equation}

Let us take a look at this:
\begin{equation}
2 \cdot x 
\end{equation}
Obviously, the derivative of this is equal to
\begin{equation}
0 \cdot x + 2 \cdot 1 
\end{equation}

We will allow ourselves to divert the reader's attentions to this gem of mathematical wonder:
\begin{equation}
1 
\end{equation}
As you can see, the derivative of this is equal to
\begin{equation}
0 
\end{equation}

The following is worth a closer look:
\begin{equation}
x ^{2 } 
\end{equation}
It can be easily proved, that the derivative of this is equal to
\begin{equation}
2 \cdot x ^{2 - 1 } \cdot 1 
\end{equation}

Consider the following:
\begin{equation}
2 
\end{equation}
Unsurprisingly, the derivative of this is equal to
\begin{equation}
0 
\end{equation}

One shall regard the object in question with utmost interest:
\begin{equation}
1 
\end{equation}
Clearly, the derivative of this is equal to
\begin{equation}
0 
\end{equation}

We will allow ourselves to divert the reader's attentions to this gem of mathematical wonder:
\begin{equation}
1 
\end{equation}
Trivially, the derivative of this is equal to
\begin{equation}
0 
\end{equation}

We shall ponder the following:
\begin{equation}
x ^{2 } 
\end{equation}
Trivially, the derivative of this is equal to
\begin{equation}
2 \cdot x ^{2 - 1 } \cdot 1 
\end{equation}

The object of our ultimate interest is the following:
\begin{equation}
1 
\end{equation}
Obviously, the derivative of this is equal to
\begin{equation}
0 
\end{equation}

Consider the following:
\begin{equation}
1 
\end{equation}
As you can see, the derivative of this is equal to
\begin{equation}
0 
\end{equation}

We will take a closer look at this:
\begin{equation}
1 
\end{equation}
It is now obvious, that the derivative of this is equal to
\begin{equation}
0 
\end{equation}

Consider the following:
\begin{equation}
x ^{2 } 
\end{equation}
As you can see, the derivative of this is equal to
\begin{equation}
2 \cdot x ^{2 - 1 } \cdot 1 
\end{equation}

We are going to study the following:
\begin{equation}
2 \cdot x 
\end{equation}
Clearly, the derivative of this is equal to
\begin{equation}
0 \cdot x + 2 \cdot 1 
\end{equation}

We shall ponder the following:
\begin{equation}
1 
\end{equation}
Obviously, the derivative of this is equal to
\begin{equation}
0 
\end{equation}

Let us take a look at this:
\begin{equation}
x ^{2 } 
\end{equation}
Clearly, the derivative of this is equal to
\begin{equation}
2 \cdot x ^{2 - 1 } \cdot 1 
\end{equation}

We are going to study the following:
\begin{equation}
2 
\end{equation}
It is now obvious, that the derivative of this is equal to
\begin{equation}
0 
\end{equation}

One shall regard the object in question with utmost interest:
\begin{equation}
1 
\end{equation}
It is now obvious, that the derivative of this is equal to
\begin{equation}
0 
\end{equation}

The object of our ultimate interest is the following:
\begin{equation}
1 
\end{equation}
Unsurprisingly, the derivative of this is equal to
\begin{equation}
0 
\end{equation}

We will allow ourselves to divert the reader's attentions to this gem of mathematical wonder:
\begin{equation}
x ^{2 } 
\end{equation}
Unsurprisingly, the derivative of this is equal to
\begin{equation}
2 \cdot x ^{2 - 1 } \cdot 1 
\end{equation}

We are going to study the following:
\begin{equation}
1 
\end{equation}
Obviously, the derivative of this is equal to
\begin{equation}
0 
\end{equation}

We will allow ourselves to divert the reader's attentions to this gem of mathematical wonder:
\begin{equation}
2 \cdot x 
\end{equation}
Unsurprisingly, the derivative of this is equal to
\begin{equation}
0 \cdot x + 2 \cdot 1 
\end{equation}

Let us take a look at this:
\begin{equation}
1 
\end{equation}
Clearly, the derivative of this is equal to
\begin{equation}
0 
\end{equation}

We will allow ourselves to divert the reader's attentions to this gem of mathematical wonder:
\begin{equation}
x ^{2 } 
\end{equation}
Any self-respecting mathematician would find it obvious, that the derivative of this is equal to
\begin{equation}
2 \cdot x ^{2 - 1 } \cdot 1 
\end{equation}

Consider the following:
\begin{equation}
2 
\end{equation}
Trivially, the derivative of this is equal to
\begin{equation}
0 
\end{equation}

Consider the following:
\begin{equation}
1 
\end{equation}
Obviously, the derivative of this is equal to
\begin{equation}
0 
\end{equation}

One shall regard the object in question with utmost interest:
\begin{equation}
1 
\end{equation}
Obviously, the derivative of this is equal to
\begin{equation}
0 
\end{equation}

The following is worth a closer look:
\begin{equation}
x ^{2 } 
\end{equation}
As you can see, the derivative of this is equal to
\begin{equation}
2 \cdot x ^{2 - 1 } \cdot 1 
\end{equation}

We are going to study the following:
\begin{equation}
2 
\end{equation}
Trivially, the derivative of this is equal to
\begin{equation}
0 
\end{equation}

We will take a closer look at this:
\begin{equation}
2 \cdot x 
\end{equation}
It can be easily proved, that the derivative of this is equal to
\begin{equation}
0 \cdot x + 2 \cdot 1 
\end{equation}

One shall regard the object in question with utmost interest:
\begin{equation}
1 
\end{equation}
It can be easily proved, that the derivative of this is equal to
\begin{equation}
0 
\end{equation}

We will allow ourselves to divert the reader's attentions to this gem of mathematical wonder:
\begin{equation}
x ^{2 } 
\end{equation}
It is now obvious, that the derivative of this is equal to
\begin{equation}
2 \cdot x ^{2 - 1 } \cdot 1 
\end{equation}

We will take a closer look at this:
\begin{equation}
2 
\end{equation}
Clearly, the derivative of this is equal to
\begin{equation}
0 
\end{equation}

We will allow ourselves to divert the reader's attentions to this gem of mathematical wonder:
\begin{equation}
2 \cdot x 
\end{equation}
Obviously, the derivative of this is equal to
\begin{equation}
0 \cdot x + 2 \cdot 1 
\end{equation}

We shall ponder the following:
\begin{equation}
1 
\end{equation}
It can be easily proved, that the derivative of this is equal to
\begin{equation}
0 
\end{equation}

We shall ponder the following:
\begin{equation}
x ^{2 } 
\end{equation}
Any self-respecting mathematician would find it obvious, that the derivative of this is equal to
\begin{equation}
2 \cdot x ^{2 - 1 } \cdot 1 
\end{equation}

The following is worth a closer look:
\begin{equation}
2 
\end{equation}
Any self-respecting mathematician would find it obvious, that the derivative of this is equal to
\begin{equation}
0 
\end{equation}

We are going to study the following:
\begin{equation}
1 
\end{equation}
Unsurprisingly, the derivative of this is equal to
\begin{equation}
0 
\end{equation}

We will allow ourselves to divert the reader's attentions to this gem of mathematical wonder:
\begin{equation}
2 
\end{equation}
It can be easily proved, that the derivative of this is equal to
\begin{equation}
0 
\end{equation}

We are going to study the following:
\begin{equation}
2 
\end{equation}
Unsurprisingly, the derivative of this is equal to
\begin{equation}
0 
\end{equation}

One shall regard the object in question with utmost interest:
\begin{equation}
1 
\end{equation}
Unsurprisingly, the derivative of this is equal to
\begin{equation}
0 
\end{equation}

We will allow ourselves to divert the reader's attentions to this gem of mathematical wonder:
\begin{equation}
x ^{2 } 
\end{equation}
It is now obvious, that the derivative of this is equal to
\begin{equation}
2 \cdot x ^{2 - 1 } \cdot 1 
\end{equation}

We shall ponder the following:
\begin{equation}
2 
\end{equation}
Trivially, the derivative of this is equal to
\begin{equation}
0 
\end{equation}

The following is worth a closer look:
\begin{equation}
1 
\end{equation}
Obviously, the derivative of this is equal to
\begin{equation}
0 
\end{equation}

We will allow ourselves to divert the reader's attentions to this gem of mathematical wonder:
\begin{equation}
1 
\end{equation}
Trivially, the derivative of this is equal to
\begin{equation}
0 
\end{equation}

We shall ponder the following:
\begin{equation}
x ^{2 } 
\end{equation}
Any self-respecting mathematician would find it obvious, that the derivative of this is equal to
\begin{equation}
2 \cdot x ^{2 - 1 } \cdot 1 
\end{equation}

Consider the following:
\begin{equation}
1 
\end{equation}
Any self-respecting mathematician would find it obvious, that the derivative of this is equal to
\begin{equation}
0 
\end{equation}

Consider the following:
\begin{equation}
1 
\end{equation}
Unsurprisingly, the derivative of this is equal to
\begin{equation}
0 
\end{equation}

Consider the following:
\begin{equation}
x - 2 
\end{equation}
As you can see, the derivative of this is equal to
\begin{equation}
1 - 0 
\end{equation}

The object of our ultimate interest is the following:
\begin{equation}
1 
\end{equation}
It is now obvious, that the derivative of this is equal to
\begin{equation}
0 
\end{equation}

The object of our ultimate interest is the following:
\begin{equation}
x ^{2 } 
\end{equation}
It can be easily proved, that the derivative of this is equal to
\begin{equation}
2 \cdot x ^{2 - 1 } \cdot 1 
\end{equation}

One shall regard the object in question with utmost interest:
\begin{equation}
2 \cdot x 
\end{equation}
Any self-respecting mathematician would find it obvious, that the derivative of this is equal to
\begin{equation}
0 \cdot x + 2 \cdot 1 
\end{equation}

We are going to study the following:
\begin{equation}
1 
\end{equation}
Trivially, the derivative of this is equal to
\begin{equation}
0 
\end{equation}

Consider the following:
\begin{equation}
x ^{2 } 
\end{equation}
Clearly, the derivative of this is equal to
\begin{equation}
2 \cdot x ^{2 - 1 } \cdot 1 
\end{equation}

Let us take a look at this:
\begin{equation}
2 
\end{equation}
Trivially, the derivative of this is equal to
\begin{equation}
0 
\end{equation}

We are going to study the following:
\begin{equation}
1 
\end{equation}
It can be easily proved, that the derivative of this is equal to
\begin{equation}
0 
\end{equation}

We will allow ourselves to divert the reader's attentions to this gem of mathematical wonder:
\begin{equation}
1 
\end{equation}
Unsurprisingly, the derivative of this is equal to
\begin{equation}
0 
\end{equation}

We shall ponder the following:
\begin{equation}
x ^{2 } 
\end{equation}
It can be easily proved, that the derivative of this is equal to
\begin{equation}
2 \cdot x ^{2 - 1 } \cdot 1 
\end{equation}

We are going to study the following:
\begin{equation}
1 
\end{equation}
Unsurprisingly, the derivative of this is equal to
\begin{equation}
0 
\end{equation}

We are going to study the following:
\begin{equation}
1 
\end{equation}
Obviously, the derivative of this is equal to
\begin{equation}
0 
\end{equation}

We will allow ourselves to divert the reader's attentions to this gem of mathematical wonder:
\begin{equation}
2 \cdot x 
\end{equation}
Trivially, the derivative of this is equal to
\begin{equation}
0 \cdot x + 2 \cdot 1 
\end{equation}

We will take a closer look at this:
\begin{equation}
1 
\end{equation}
Obviously, the derivative of this is equal to
\begin{equation}
0 
\end{equation}

We will allow ourselves to divert the reader's attentions to this gem of mathematical wonder:
\begin{equation}
x ^{2 } 
\end{equation}
Clearly, the derivative of this is equal to
\begin{equation}
2 \cdot x ^{2 - 1 } \cdot 1 
\end{equation}

The object of our ultimate interest is the following:
\begin{equation}
2 
\end{equation}
Obviously, the derivative of this is equal to
\begin{equation}
0 
\end{equation}

The object of our ultimate interest is the following:
\begin{equation}
1 
\end{equation}
Any self-respecting mathematician would find it obvious, that the derivative of this is equal to
\begin{equation}
0 
\end{equation}

We will take a closer look at this:
\begin{equation}
1 
\end{equation}
Trivially, the derivative of this is equal to
\begin{equation}
0 
\end{equation}

We are going to study the following:
\begin{equation}
x ^{2 } 
\end{equation}
Clearly, the derivative of this is equal to
\begin{equation}
2 \cdot x ^{2 - 1 } \cdot 1 
\end{equation}

Let us take a look at this:
\begin{equation}
1 
\end{equation}
It can be easily proved, that the derivative of this is equal to
\begin{equation}
0 
\end{equation}

We will allow ourselves to divert the reader's attentions to this gem of mathematical wonder:
\begin{equation}
1 
\end{equation}
It is now obvious, that the derivative of this is equal to
\begin{equation}
0 
\end{equation}

We will allow ourselves to divert the reader's attentions to this gem of mathematical wonder:
\begin{equation}
1 
\end{equation}
Obviously, the derivative of this is equal to
\begin{equation}
0 
\end{equation}

One shall regard the object in question with utmost interest:
\begin{equation}
x ^{2 } 
\end{equation}
Any self-respecting mathematician would find it obvious, that the derivative of this is equal to
\begin{equation}
2 \cdot x ^{2 - 1 } \cdot 1 
\end{equation}

Let us take a look at this:
\begin{equation}
1 
\end{equation}
Clearly, the derivative of this is equal to
\begin{equation}
0 
\end{equation}

Let us take a look at this:
\begin{equation}
1 
\end{equation}
Any self-respecting mathematician would find it obvious, that the derivative of this is equal to
\begin{equation}
0 
\end{equation}

We will allow ourselves to divert the reader's attentions to this gem of mathematical wonder:
\begin{equation}
x ^{2 } 
\end{equation}
Clearly, the derivative of this is equal to
\begin{equation}
2 \cdot x ^{2 - 1 } \cdot 1 
\end{equation}

The following is worth a closer look:
\begin{equation}
x - 2 
\end{equation}
Unsurprisingly, the derivative of this is equal to
\begin{equation}
1 - 0 
\end{equation}

We are going to study the following:
\begin{equation}
1 
\end{equation}
Any self-respecting mathematician would find it obvious, that the derivative of this is equal to
\begin{equation}
0 
\end{equation}

The following is worth a closer look:
\begin{equation}
x ^{2 } 
\end{equation}
Obviously, the derivative of this is equal to
\begin{equation}
2 \cdot x ^{2 - 1 } \cdot 1 
\end{equation}

The following is worth a closer look:
\begin{equation}
2 \cdot x 
\end{equation}
Obviously, the derivative of this is equal to
\begin{equation}
0 \cdot x + 2 \cdot 1 
\end{equation}

The object of our ultimate interest is the following:
\begin{equation}
1 
\end{equation}
Unsurprisingly, the derivative of this is equal to
\begin{equation}
0 
\end{equation}

One shall regard the object in question with utmost interest:
\begin{equation}
x ^{2 } 
\end{equation}
Unsurprisingly, the derivative of this is equal to
\begin{equation}
2 \cdot x ^{2 - 1 } \cdot 1 
\end{equation}

The following is worth a closer look:
\begin{equation}
2 
\end{equation}
Unsurprisingly, the derivative of this is equal to
\begin{equation}
0 
\end{equation}

We will allow ourselves to divert the reader's attentions to this gem of mathematical wonder:
\begin{equation}
1 
\end{equation}
Obviously, the derivative of this is equal to
\begin{equation}
0 
\end{equation}

We will allow ourselves to divert the reader's attentions to this gem of mathematical wonder:
\begin{equation}
1 
\end{equation}
As you can see, the derivative of this is equal to
\begin{equation}
0 
\end{equation}

One shall regard the object in question with utmost interest:
\begin{equation}
x ^{2 } 
\end{equation}
It is now obvious, that the derivative of this is equal to
\begin{equation}
2 \cdot x ^{2 - 1 } \cdot 1 
\end{equation}

We will take a closer look at this:
\begin{equation}
1 
\end{equation}
Unsurprisingly, the derivative of this is equal to
\begin{equation}
0 
\end{equation}

Consider the following:
\begin{equation}
1 
\end{equation}
As you can see, the derivative of this is equal to
\begin{equation}
0 
\end{equation}

Let us take a look at this:
\begin{equation}
x - 2 
\end{equation}
It can be easily proved, that the derivative of this is equal to
\begin{equation}
1 - 0 
\end{equation}

Consider the following:
\begin{equation}
1 
\end{equation}
As you can see, the derivative of this is equal to
\begin{equation}
0 
\end{equation}

We are going to study the following:
\begin{equation}
x ^{2 } 
\end{equation}
Any self-respecting mathematician would find it obvious, that the derivative of this is equal to
\begin{equation}
2 \cdot x ^{2 - 1 } \cdot 1 
\end{equation}

We shall ponder the following:
\begin{equation}
1 
\end{equation}
Obviously, the derivative of this is equal to
\begin{equation}
0 
\end{equation}

We shall ponder the following:
\begin{equation}
x ^{2 } 
\end{equation}
It can be easily proved, that the derivative of this is equal to
\begin{equation}
2 \cdot x ^{2 - 1 } \cdot 1 
\end{equation}

The following is worth a closer look:
\begin{equation}
2 \cdot x 
\end{equation}
Trivially, the derivative of this is equal to
\begin{equation}
0 \cdot x + 2 \cdot 1 
\end{equation}

The object of our ultimate interest is the following:
\begin{equation}
1 
\end{equation}
Any self-respecting mathematician would find it obvious, that the derivative of this is equal to
\begin{equation}
0 
\end{equation}

We will allow ourselves to divert the reader's attentions to this gem of mathematical wonder:
\begin{equation}
x ^{2 } 
\end{equation}
Obviously, the derivative of this is equal to
\begin{equation}
2 \cdot x ^{2 - 1 } \cdot 1 
\end{equation}

We are going to study the following:
\begin{equation}
2 
\end{equation}
It can be easily proved, that the derivative of this is equal to
\begin{equation}
0 
\end{equation}

The object of our ultimate interest is the following:
\begin{equation}
1 
\end{equation}
Clearly, the derivative of this is equal to
\begin{equation}
0 
\end{equation}

We are going to study the following:
\begin{equation}
1 
\end{equation}
Obviously, the derivative of this is equal to
\begin{equation}
0 
\end{equation}

One shall regard the object in question with utmost interest:
\begin{equation}
x ^{2 } 
\end{equation}
Trivially, the derivative of this is equal to
\begin{equation}
2 \cdot x ^{2 - 1 } \cdot 1 
\end{equation}

We will take a closer look at this:
\begin{equation}
1 
\end{equation}
Clearly, the derivative of this is equal to
\begin{equation}
0 
\end{equation}

We will take a closer look at this:
\begin{equation}
1 
\end{equation}
Any self-respecting mathematician would find it obvious, that the derivative of this is equal to
\begin{equation}
0 
\end{equation}

The object of our ultimate interest is the following:
\begin{equation}
x - 2 
\end{equation}
It is now obvious, that the derivative of this is equal to
\begin{equation}
1 - 0 
\end{equation}

One shall regard the object in question with utmost interest:
\begin{equation}
1 
\end{equation}
It is now obvious, that the derivative of this is equal to
\begin{equation}
0 
\end{equation}

The object of our ultimate interest is the following:
\begin{equation}
x ^{2 } 
\end{equation}
It can be easily proved, that the derivative of this is equal to
\begin{equation}
2 \cdot x ^{2 - 1 } \cdot 1 
\end{equation}

The following is worth a closer look:
\begin{equation}
1 
\end{equation}
It can be easily proved, that the derivative of this is equal to
\begin{equation}
0 
\end{equation}

One shall regard the object in question with utmost interest:
\begin{equation}
x ^{2 } 
\end{equation}
As you can see, the derivative of this is equal to
\begin{equation}
2 \cdot x ^{2 - 1 } \cdot 1 
\end{equation}

We shall ponder the following:
\begin{equation}
x - 2 
\end{equation}
It can be easily proved, that the derivative of this is equal to
\begin{equation}
1 - 0 
\end{equation}

We are going to study the following:
\begin{equation}
x + 1 
\end{equation}
Clearly, the derivative of this is equal to
\begin{equation}
1 + 0 
\end{equation}

One shall regard the object in question with utmost interest:
\begin{equation}
1 
\end{equation}
It can be easily proved, that the derivative of this is equal to
\begin{equation}
0 
\end{equation}

The object of our ultimate interest is the following:
\begin{equation}
2 
\end{equation}
As you can see, the derivative of this is equal to
\begin{equation}
0 
\end{equation}

One shall regard the object in question with utmost interest:
\begin{equation}
\sin x 
\end{equation}
Unsurprisingly, the derivative of this is equal to
\begin{equation}
\cos x \cdot 1 
\end{equation}

We shall ponder the following:
\begin{equation}
x + 1 
\end{equation}
Obviously, the derivative of this is equal to
\begin{equation}
1 + 0 
\end{equation}

The following is worth a closer look:
\begin{equation}
4 
\end{equation}
It is now obvious, that the derivative of this is equal to
\begin{equation}
0 
\end{equation}

We shall ponder the following:
\begin{equation}
2 
\end{equation}
Trivially, the derivative of this is equal to
\begin{equation}
0 
\end{equation}

We will allow ourselves to divert the reader's attentions to this gem of mathematical wonder:
\begin{equation}
\cos x 
\end{equation}
Clearly, the derivative of this is equal to
\begin{equation}
-\sin x \cdot 1 
\end{equation}

Consider the following:
\begin{equation}
x + 1 
\end{equation}
Obviously, the derivative of this is equal to
\begin{equation}
1 + 0 
\end{equation}

Consider the following:
\begin{equation}
2 
\end{equation}
It can be easily proved, that the derivative of this is equal to
\begin{equation}
0 
\end{equation}

Let us take a look at this:
\begin{equation}
\sin x 
\end{equation}
Obviously, the derivative of this is equal to
\begin{equation}
\cos x \cdot 1 
\end{equation}

One shall regard the object in question with utmost interest:
\begin{equation}
1 
\end{equation}
Clearly, the derivative of this is equal to
\begin{equation}
0 
\end{equation}

We will allow ourselves to divert the reader's attentions to this gem of mathematical wonder:
\begin{equation}
x ^{2 } 
\end{equation}
Trivially, the derivative of this is equal to
\begin{equation}
2 \cdot x ^{2 - 1 } \cdot 1 
\end{equation}

We will take a closer look at this:
\begin{equation}
2 \cdot x 
\end{equation}
Clearly, the derivative of this is equal to
\begin{equation}
0 \cdot x + 2 \cdot 1 
\end{equation}

We are going to study the following:
\begin{equation}
1 
\end{equation}
It can be easily proved, that the derivative of this is equal to
\begin{equation}
0 
\end{equation}

We will take a closer look at this:
\begin{equation}
x ^{2 } 
\end{equation}
Any self-respecting mathematician would find it obvious, that the derivative of this is equal to
\begin{equation}
2 \cdot x ^{2 - 1 } \cdot 1 
\end{equation}

The following is worth a closer look:
\begin{equation}
2 
\end{equation}
Clearly, the derivative of this is equal to
\begin{equation}
0 
\end{equation}

Let us take a look at this:
\begin{equation}
1 
\end{equation}
Unsurprisingly, the derivative of this is equal to
\begin{equation}
0 
\end{equation}

One shall regard the object in question with utmost interest:
\begin{equation}
1 
\end{equation}
Unsurprisingly, the derivative of this is equal to
\begin{equation}
0 
\end{equation}

We shall ponder the following:
\begin{equation}
x ^{2 } 
\end{equation}
It can be easily proved, that the derivative of this is equal to
\begin{equation}
2 \cdot x ^{2 - 1 } \cdot 1 
\end{equation}

We will allow ourselves to divert the reader's attentions to this gem of mathematical wonder:
\begin{equation}
1 
\end{equation}
It can be easily proved, that the derivative of this is equal to
\begin{equation}
0 
\end{equation}

We will take a closer look at this:
\begin{equation}
1 
\end{equation}
Any self-respecting mathematician would find it obvious, that the derivative of this is equal to
\begin{equation}
0 
\end{equation}

One shall regard the object in question with utmost interest:
\begin{equation}
x - 2 
\end{equation}
It is now obvious, that the derivative of this is equal to
\begin{equation}
1 - 0 
\end{equation}

We will allow ourselves to divert the reader's attentions to this gem of mathematical wonder:
\begin{equation}
1 
\end{equation}
It is now obvious, that the derivative of this is equal to
\begin{equation}
0 
\end{equation}

The following is worth a closer look:
\begin{equation}
x ^{2 } 
\end{equation}
Any self-respecting mathematician would find it obvious, that the derivative of this is equal to
\begin{equation}
2 \cdot x ^{2 - 1 } \cdot 1 
\end{equation}

Consider the following:
\begin{equation}
1 
\end{equation}
As you can see, the derivative of this is equal to
\begin{equation}
0 
\end{equation}

Let us take a look at this:
\begin{equation}
x ^{2 } 
\end{equation}
Trivially, the derivative of this is equal to
\begin{equation}
2 \cdot x ^{2 - 1 } \cdot 1 
\end{equation}

The object of our ultimate interest is the following:
\begin{equation}
x - 2 
\end{equation}
Trivially, the derivative of this is equal to
\begin{equation}
1 - 0 
\end{equation}

Consider the following:
\begin{equation}
x + 1 
\end{equation}
It is now obvious, that the derivative of this is equal to
\begin{equation}
1 + 0 
\end{equation}

The object of our ultimate interest is the following:
\begin{equation}
1 
\end{equation}
Trivially, the derivative of this is equal to
\begin{equation}
0 
\end{equation}

We will allow ourselves to divert the reader's attentions to this gem of mathematical wonder:
\begin{equation}
2 
\end{equation}
It can be easily proved, that the derivative of this is equal to
\begin{equation}
0 
\end{equation}

We will allow ourselves to divert the reader's attentions to this gem of mathematical wonder:
\begin{equation}
\sin x 
\end{equation}
Clearly, the derivative of this is equal to
\begin{equation}
\cos x \cdot 1 
\end{equation}

We are going to study the following:
\begin{equation}
x + 1 
\end{equation}
Clearly, the derivative of this is equal to
\begin{equation}
1 + 0 
\end{equation}

Let us take a look at this:
\begin{equation}
1 
\end{equation}
As you can see, the derivative of this is equal to
\begin{equation}
0 
\end{equation}

We shall ponder the following:
\begin{equation}
4 
\end{equation}
It is now obvious, that the derivative of this is equal to
\begin{equation}
0 
\end{equation}

We will take a closer look at this:
\begin{equation}
2 
\end{equation}
As you can see, the derivative of this is equal to
\begin{equation}
0 
\end{equation}

Let us take a look at this:
\begin{equation}
\cos x 
\end{equation}
Obviously, the derivative of this is equal to
\begin{equation}
-\sin x \cdot 1 
\end{equation}

We will allow ourselves to divert the reader's attentions to this gem of mathematical wonder:
\begin{equation}
x + 1 
\end{equation}
Any self-respecting mathematician would find it obvious, that the derivative of this is equal to
\begin{equation}
1 + 0 
\end{equation}

We shall ponder the following:
\begin{equation}
16 
\end{equation}
Unsurprisingly, the derivative of this is equal to
\begin{equation}
0 
\end{equation}

We will take a closer look at this:
\begin{equation}
0 
\end{equation}
Clearly, the derivative of this is equal to
\begin{equation}
0 
\end{equation}

One shall regard the object in question with utmost interest:
\begin{equation}
4 
\end{equation}
It can be easily proved, that the derivative of this is equal to
\begin{equation}
0 
\end{equation}

Let us take a look at this:
\begin{equation}
2 
\end{equation}
Any self-respecting mathematician would find it obvious, that the derivative of this is equal to
\begin{equation}
0 
\end{equation}

We shall ponder the following:
\begin{equation}
1 
\end{equation}
Any self-respecting mathematician would find it obvious, that the derivative of this is equal to
\begin{equation}
0 
\end{equation}

We are going to study the following:
\begin{equation}
\sin x 
\end{equation}
As you can see, the derivative of this is equal to
\begin{equation}
\cos x \cdot 1 
\end{equation}

The object of our ultimate interest is the following:
\begin{equation}
x + 1 
\end{equation}
As you can see, the derivative of this is equal to
\begin{equation}
1 + 0 
\end{equation}

The object of our ultimate interest is the following:
\begin{equation}
1 
\end{equation}
As you can see, the derivative of this is equal to
\begin{equation}
0 
\end{equation}

We are going to study the following:
\begin{equation}
4 
\end{equation}
Obviously, the derivative of this is equal to
\begin{equation}
0 
\end{equation}

The following is worth a closer look:
\begin{equation}
2 
\end{equation}
It is now obvious, that the derivative of this is equal to
\begin{equation}
0 
\end{equation}

The following is worth a closer look:
\begin{equation}
\cos x 
\end{equation}
It can be easily proved, that the derivative of this is equal to
\begin{equation}
-\sin x \cdot 1 
\end{equation}

We will take a closer look at this:
\begin{equation}
x + 1 
\end{equation}
It can be easily proved, that the derivative of this is equal to
\begin{equation}
1 + 0 
\end{equation}

One shall regard the object in question with utmost interest:
\begin{equation}
2 
\end{equation}
Clearly, the derivative of this is equal to
\begin{equation}
0 
\end{equation}

We will take a closer look at this:
\begin{equation}
\sin x 
\end{equation}
It is now obvious, that the derivative of this is equal to
\begin{equation}
\cos x \cdot 1 
\end{equation}

Consider the following:
\begin{equation}
x + 1 
\end{equation}
It can be easily proved, that the derivative of this is equal to
\begin{equation}
1 + 0 
\end{equation}

We will take a closer look at this:
\begin{equation}
1 
\end{equation}
Unsurprisingly, the derivative of this is equal to
\begin{equation}
0 
\end{equation}

Let us take a look at this:
\begin{equation}
2 
\end{equation}
Clearly, the derivative of this is equal to
\begin{equation}
0 
\end{equation}

We are going to study the following:
\begin{equation}
\sin x 
\end{equation}
Any self-respecting mathematician would find it obvious, that the derivative of this is equal to
\begin{equation}
\cos x \cdot 1 
\end{equation}

Let us take a look at this:
\begin{equation}
x + 1 
\end{equation}
It is now obvious, that the derivative of this is equal to
\begin{equation}
1 + 0 
\end{equation}

We are going to study the following:
\begin{equation}
4 
\end{equation}
Unsurprisingly, the derivative of this is equal to
\begin{equation}
0 
\end{equation}

One shall regard the object in question with utmost interest:
\begin{equation}
2 
\end{equation}
It can be easily proved, that the derivative of this is equal to
\begin{equation}
0 
\end{equation}

We shall ponder the following:
\begin{equation}
\cos x 
\end{equation}
Obviously, the derivative of this is equal to
\begin{equation}
-\sin x \cdot 1 
\end{equation}

We will allow ourselves to divert the reader's attentions to this gem of mathematical wonder:
\begin{equation}
x + 1 
\end{equation}
Clearly, the derivative of this is equal to
\begin{equation}
1 + 0 
\end{equation}

One shall regard the object in question with utmost interest:
\begin{equation}
1 
\end{equation}
As you can see, the derivative of this is equal to
\begin{equation}
0 
\end{equation}

We shall ponder the following:
\begin{equation}
2 
\end{equation}
Trivially, the derivative of this is equal to
\begin{equation}
0 
\end{equation}

We are going to study the following:
\begin{equation}
\sin x 
\end{equation}
It can be easily proved, that the derivative of this is equal to
\begin{equation}
\cos x \cdot 1 
\end{equation}

We shall ponder the following:
\begin{equation}
x + 1 
\end{equation}
Unsurprisingly, the derivative of this is equal to
\begin{equation}
1 + 0 
\end{equation}

We are going to study the following:
\begin{equation}
4 
\end{equation}
Unsurprisingly, the derivative of this is equal to
\begin{equation}
0 
\end{equation}

The object of our ultimate interest is the following:
\begin{equation}
2 
\end{equation}
Unsurprisingly, the derivative of this is equal to
\begin{equation}
0 
\end{equation}

One shall regard the object in question with utmost interest:
\begin{equation}
\cos x 
\end{equation}
Clearly, the derivative of this is equal to
\begin{equation}
-\sin x \cdot 1 
\end{equation}

We will take a closer look at this:
\begin{equation}
x + 1 
\end{equation}
Clearly, the derivative of this is equal to
\begin{equation}
1 + 0 
\end{equation}

We are going to study the following:
\begin{equation}
2 
\end{equation}
Any self-respecting mathematician would find it obvious, that the derivative of this is equal to
\begin{equation}
0 
\end{equation}

One shall regard the object in question with utmost interest:
\begin{equation}
\sin x 
\end{equation}
Unsurprisingly, the derivative of this is equal to
\begin{equation}
\cos x \cdot 1 
\end{equation}

One shall regard the object in question with utmost interest:
\begin{equation}
1 
\end{equation}
As you can see, the derivative of this is equal to
\begin{equation}
0 
\end{equation}

We are going to study the following:
\begin{equation}
x ^{2 } 
\end{equation}
It can be easily proved, that the derivative of this is equal to
\begin{equation}
2 \cdot x ^{2 - 1 } \cdot 1 
\end{equation}

We are going to study the following:
\begin{equation}
2 \cdot x 
\end{equation}
It is now obvious, that the derivative of this is equal to
\begin{equation}
0 \cdot x + 2 \cdot 1 
\end{equation}

The object of our ultimate interest is the following:
\begin{equation}
1 
\end{equation}
Trivially, the derivative of this is equal to
\begin{equation}
0 
\end{equation}

We will take a closer look at this:
\begin{equation}
x ^{2 } 
\end{equation}
Unsurprisingly, the derivative of this is equal to
\begin{equation}
2 \cdot x ^{2 - 1 } \cdot 1 
\end{equation}

Consider the following:
\begin{equation}
2 
\end{equation}
Trivially, the derivative of this is equal to
\begin{equation}
0 
\end{equation}

We will take a closer look at this:
\begin{equation}
1 
\end{equation}
It can be easily proved, that the derivative of this is equal to
\begin{equation}
0 
\end{equation}

Let us take a look at this:
\begin{equation}
1 
\end{equation}
As you can see, the derivative of this is equal to
\begin{equation}
0 
\end{equation}

One shall regard the object in question with utmost interest:
\begin{equation}
x ^{2 } 
\end{equation}
It is now obvious, that the derivative of this is equal to
\begin{equation}
2 \cdot x ^{2 - 1 } \cdot 1 
\end{equation}

Consider the following:
\begin{equation}
1 
\end{equation}
As you can see, the derivative of this is equal to
\begin{equation}
0 
\end{equation}

The object of our ultimate interest is the following:
\begin{equation}
1 
\end{equation}
It can be easily proved, that the derivative of this is equal to
\begin{equation}
0 
\end{equation}

Consider the following:
\begin{equation}
2 \cdot x 
\end{equation}
Obviously, the derivative of this is equal to
\begin{equation}
0 \cdot x + 2 \cdot 1 
\end{equation}

We will allow ourselves to divert the reader's attentions to this gem of mathematical wonder:
\begin{equation}
1 
\end{equation}
It can be easily proved, that the derivative of this is equal to
\begin{equation}
0 
\end{equation}

We are going to study the following:
\begin{equation}
x ^{2 } 
\end{equation}
Clearly, the derivative of this is equal to
\begin{equation}
2 \cdot x ^{2 - 1 } \cdot 1 
\end{equation}

One shall regard the object in question with utmost interest:
\begin{equation}
2 
\end{equation}
As you can see, the derivative of this is equal to
\begin{equation}
0 
\end{equation}

We will take a closer look at this:
\begin{equation}
1 
\end{equation}
As you can see, the derivative of this is equal to
\begin{equation}
0 
\end{equation}

The object of our ultimate interest is the following:
\begin{equation}
1 
\end{equation}
Obviously, the derivative of this is equal to
\begin{equation}
0 
\end{equation}

One shall regard the object in question with utmost interest:
\begin{equation}
x ^{2 } 
\end{equation}
Any self-respecting mathematician would find it obvious, that the derivative of this is equal to
\begin{equation}
2 \cdot x ^{2 - 1 } \cdot 1 
\end{equation}

The object of our ultimate interest is the following:
\begin{equation}
1 
\end{equation}
Unsurprisingly, the derivative of this is equal to
\begin{equation}
0 
\end{equation}

One shall regard the object in question with utmost interest:
\begin{equation}
1 
\end{equation}
It can be easily proved, that the derivative of this is equal to
\begin{equation}
0 
\end{equation}

The following is worth a closer look:
\begin{equation}
1 
\end{equation}
Trivially, the derivative of this is equal to
\begin{equation}
0 
\end{equation}

We shall ponder the following:
\begin{equation}
x ^{2 } 
\end{equation}
Trivially, the derivative of this is equal to
\begin{equation}
2 \cdot x ^{2 - 1 } \cdot 1 
\end{equation}

We are going to study the following:
\begin{equation}
2 \cdot x 
\end{equation}
Any self-respecting mathematician would find it obvious, that the derivative of this is equal to
\begin{equation}
0 \cdot x + 2 \cdot 1 
\end{equation}

The object of our ultimate interest is the following:
\begin{equation}
1 
\end{equation}
It is now obvious, that the derivative of this is equal to
\begin{equation}
0 
\end{equation}

We will take a closer look at this:
\begin{equation}
x ^{2 } 
\end{equation}
It is now obvious, that the derivative of this is equal to
\begin{equation}
2 \cdot x ^{2 - 1 } \cdot 1 
\end{equation}

We will allow ourselves to divert the reader's attentions to this gem of mathematical wonder:
\begin{equation}
2 
\end{equation}
Clearly, the derivative of this is equal to
\begin{equation}
0 
\end{equation}

We shall ponder the following:
\begin{equation}
1 
\end{equation}
Any self-respecting mathematician would find it obvious, that the derivative of this is equal to
\begin{equation}
0 
\end{equation}

The object of our ultimate interest is the following:
\begin{equation}
1 
\end{equation}
Any self-respecting mathematician would find it obvious, that the derivative of this is equal to
\begin{equation}
0 
\end{equation}

The following is worth a closer look:
\begin{equation}
x ^{2 } 
\end{equation}
Any self-respecting mathematician would find it obvious, that the derivative of this is equal to
\begin{equation}
2 \cdot x ^{2 - 1 } \cdot 1 
\end{equation}

One shall regard the object in question with utmost interest:
\begin{equation}
1 
\end{equation}
It is now obvious, that the derivative of this is equal to
\begin{equation}
0 
\end{equation}

One shall regard the object in question with utmost interest:
\begin{equation}
2 \cdot x 
\end{equation}
Clearly, the derivative of this is equal to
\begin{equation}
0 \cdot x + 2 \cdot 1 
\end{equation}

We are going to study the following:
\begin{equation}
1 
\end{equation}
Any self-respecting mathematician would find it obvious, that the derivative of this is equal to
\begin{equation}
0 
\end{equation}

We are going to study the following:
\begin{equation}
x ^{2 } 
\end{equation}
Obviously, the derivative of this is equal to
\begin{equation}
2 \cdot x ^{2 - 1 } \cdot 1 
\end{equation}

Let us take a look at this:
\begin{equation}
2 
\end{equation}
Unsurprisingly, the derivative of this is equal to
\begin{equation}
0 
\end{equation}

Let us take a look at this:
\begin{equation}
1 
\end{equation}
Any self-respecting mathematician would find it obvious, that the derivative of this is equal to
\begin{equation}
0 
\end{equation}

The following is worth a closer look:
\begin{equation}
1 
\end{equation}
Trivially, the derivative of this is equal to
\begin{equation}
0 
\end{equation}

We will allow ourselves to divert the reader's attentions to this gem of mathematical wonder:
\begin{equation}
x ^{2 } 
\end{equation}
Obviously, the derivative of this is equal to
\begin{equation}
2 \cdot x ^{2 - 1 } \cdot 1 
\end{equation}

The following is worth a closer look:
\begin{equation}
2 
\end{equation}
Trivially, the derivative of this is equal to
\begin{equation}
0 
\end{equation}

One shall regard the object in question with utmost interest:
\begin{equation}
2 \cdot x 
\end{equation}
It can be easily proved, that the derivative of this is equal to
\begin{equation}
0 \cdot x + 2 \cdot 1 
\end{equation}

We will take a closer look at this:
\begin{equation}
1 
\end{equation}
As you can see, the derivative of this is equal to
\begin{equation}
0 
\end{equation}

Let us take a look at this:
\begin{equation}
x ^{2 } 
\end{equation}
As you can see, the derivative of this is equal to
\begin{equation}
2 \cdot x ^{2 - 1 } \cdot 1 
\end{equation}

The object of our ultimate interest is the following:
\begin{equation}
2 
\end{equation}
It is now obvious, that the derivative of this is equal to
\begin{equation}
0 
\end{equation}

We will allow ourselves to divert the reader's attentions to this gem of mathematical wonder:
\begin{equation}
2 \cdot x 
\end{equation}
It is now obvious, that the derivative of this is equal to
\begin{equation}
0 \cdot x + 2 \cdot 1 
\end{equation}

The following is worth a closer look:
\begin{equation}
1 
\end{equation}
It is now obvious, that the derivative of this is equal to
\begin{equation}
0 
\end{equation}

We are going to study the following:
\begin{equation}
x ^{2 } 
\end{equation}
Any self-respecting mathematician would find it obvious, that the derivative of this is equal to
\begin{equation}
2 \cdot x ^{2 - 1 } \cdot 1 
\end{equation}

We are going to study the following:
\begin{equation}
2 
\end{equation}
It can be easily proved, that the derivative of this is equal to
\begin{equation}
0 
\end{equation}

We will take a closer look at this:
\begin{equation}
1 
\end{equation}
Any self-respecting mathematician would find it obvious, that the derivative of this is equal to
\begin{equation}
0 
\end{equation}

We shall ponder the following:
\begin{equation}
2 
\end{equation}
As you can see, the derivative of this is equal to
\begin{equation}
0 
\end{equation}

One shall regard the object in question with utmost interest:
\begin{equation}
2 
\end{equation}
It is now obvious, that the derivative of this is equal to
\begin{equation}
0 
\end{equation}

The following is worth a closer look:
\begin{equation}
1 
\end{equation}
Obviously, the derivative of this is equal to
\begin{equation}
0 
\end{equation}

We will allow ourselves to divert the reader's attentions to this gem of mathematical wonder:
\begin{equation}
x ^{2 } 
\end{equation}
Obviously, the derivative of this is equal to
\begin{equation}
2 \cdot x ^{2 - 1 } \cdot 1 
\end{equation}

The following is worth a closer look:
\begin{equation}
2 
\end{equation}
As you can see, the derivative of this is equal to
\begin{equation}
0 
\end{equation}

We will allow ourselves to divert the reader's attentions to this gem of mathematical wonder:
\begin{equation}
1 
\end{equation}
It is now obvious, that the derivative of this is equal to
\begin{equation}
0 
\end{equation}

The object of our ultimate interest is the following:
\begin{equation}
1 
\end{equation}
Obviously, the derivative of this is equal to
\begin{equation}
0 
\end{equation}

We shall ponder the following:
\begin{equation}
x ^{2 } 
\end{equation}
Any self-respecting mathematician would find it obvious, that the derivative of this is equal to
\begin{equation}
2 \cdot x ^{2 - 1 } \cdot 1 
\end{equation}

We shall ponder the following:
\begin{equation}
1 
\end{equation}
As you can see, the derivative of this is equal to
\begin{equation}
0 
\end{equation}

The object of our ultimate interest is the following:
\begin{equation}
1 
\end{equation}
As you can see, the derivative of this is equal to
\begin{equation}
0 
\end{equation}

We will allow ourselves to divert the reader's attentions to this gem of mathematical wonder:
\begin{equation}
x - 2 
\end{equation}
Any self-respecting mathematician would find it obvious, that the derivative of this is equal to
\begin{equation}
1 - 0 
\end{equation}

The object of our ultimate interest is the following:
\begin{equation}
1 
\end{equation}
Trivially, the derivative of this is equal to
\begin{equation}
0 
\end{equation}

We will allow ourselves to divert the reader's attentions to this gem of mathematical wonder:
\begin{equation}
x ^{2 } 
\end{equation}
Trivially, the derivative of this is equal to
\begin{equation}
2 \cdot x ^{2 - 1 } \cdot 1 
\end{equation}

The object of our ultimate interest is the following:
\begin{equation}
2 \cdot x 
\end{equation}
Trivially, the derivative of this is equal to
\begin{equation}
0 \cdot x + 2 \cdot 1 
\end{equation}

Consider the following:
\begin{equation}
1 
\end{equation}
Obviously, the derivative of this is equal to
\begin{equation}
0 
\end{equation}

Let us take a look at this:
\begin{equation}
x ^{2 } 
\end{equation}
Obviously, the derivative of this is equal to
\begin{equation}
2 \cdot x ^{2 - 1 } \cdot 1 
\end{equation}

One shall regard the object in question with utmost interest:
\begin{equation}
2 
\end{equation}
As you can see, the derivative of this is equal to
\begin{equation}
0 
\end{equation}

The following is worth a closer look:
\begin{equation}
1 
\end{equation}
It can be easily proved, that the derivative of this is equal to
\begin{equation}
0 
\end{equation}

We will take a closer look at this:
\begin{equation}
1 
\end{equation}
Any self-respecting mathematician would find it obvious, that the derivative of this is equal to
\begin{equation}
0 
\end{equation}

We will allow ourselves to divert the reader's attentions to this gem of mathematical wonder:
\begin{equation}
x ^{2 } 
\end{equation}
Unsurprisingly, the derivative of this is equal to
\begin{equation}
2 \cdot x ^{2 - 1 } \cdot 1 
\end{equation}

We shall ponder the following:
\begin{equation}
1 
\end{equation}
Obviously, the derivative of this is equal to
\begin{equation}
0 
\end{equation}

We will allow ourselves to divert the reader's attentions to this gem of mathematical wonder:
\begin{equation}
1 
\end{equation}
It is now obvious, that the derivative of this is equal to
\begin{equation}
0 
\end{equation}

Let us take a look at this:
\begin{equation}
2 \cdot x 
\end{equation}
Obviously, the derivative of this is equal to
\begin{equation}
0 \cdot x + 2 \cdot 1 
\end{equation}

Consider the following:
\begin{equation}
1 
\end{equation}
Clearly, the derivative of this is equal to
\begin{equation}
0 
\end{equation}

We shall ponder the following:
\begin{equation}
x ^{2 } 
\end{equation}
Clearly, the derivative of this is equal to
\begin{equation}
2 \cdot x ^{2 - 1 } \cdot 1 
\end{equation}

We will allow ourselves to divert the reader's attentions to this gem of mathematical wonder:
\begin{equation}
2 
\end{equation}
It is now obvious, that the derivative of this is equal to
\begin{equation}
0 
\end{equation}

One shall regard the object in question with utmost interest:
\begin{equation}
1 
\end{equation}
As you can see, the derivative of this is equal to
\begin{equation}
0 
\end{equation}

One shall regard the object in question with utmost interest:
\begin{equation}
1 
\end{equation}
Trivially, the derivative of this is equal to
\begin{equation}
0 
\end{equation}

We are going to study the following:
\begin{equation}
x ^{2 } 
\end{equation}
It is now obvious, that the derivative of this is equal to
\begin{equation}
2 \cdot x ^{2 - 1 } \cdot 1 
\end{equation}

We shall ponder the following:
\begin{equation}
1 
\end{equation}
Trivially, the derivative of this is equal to
\begin{equation}
0 
\end{equation}

Let us take a look at this:
\begin{equation}
1 
\end{equation}
As you can see, the derivative of this is equal to
\begin{equation}
0 
\end{equation}

We will allow ourselves to divert the reader's attentions to this gem of mathematical wonder:
\begin{equation}
1 
\end{equation}
Obviously, the derivative of this is equal to
\begin{equation}
0 
\end{equation}

We shall ponder the following:
\begin{equation}
x ^{2 } 
\end{equation}
Trivially, the derivative of this is equal to
\begin{equation}
2 \cdot x ^{2 - 1 } \cdot 1 
\end{equation}

The following is worth a closer look:
\begin{equation}
1 
\end{equation}
Unsurprisingly, the derivative of this is equal to
\begin{equation}
0 
\end{equation}

The object of our ultimate interest is the following:
\begin{equation}
1 
\end{equation}
It is now obvious, that the derivative of this is equal to
\begin{equation}
0 
\end{equation}

We will allow ourselves to divert the reader's attentions to this gem of mathematical wonder:
\begin{equation}
x ^{2 } 
\end{equation}
Unsurprisingly, the derivative of this is equal to
\begin{equation}
2 \cdot x ^{2 - 1 } \cdot 1 
\end{equation}

The following is worth a closer look:
\begin{equation}
x - 2 
\end{equation}
Unsurprisingly, the derivative of this is equal to
\begin{equation}
1 - 0 
\end{equation}

We will take a closer look at this:
\begin{equation}
1 
\end{equation}
It can be easily proved, that the derivative of this is equal to
\begin{equation}
0 
\end{equation}

The following is worth a closer look:
\begin{equation}
x ^{2 } 
\end{equation}
It can be easily proved, that the derivative of this is equal to
\begin{equation}
2 \cdot x ^{2 - 1 } \cdot 1 
\end{equation}

We shall ponder the following:
\begin{equation}
2 \cdot x 
\end{equation}
Unsurprisingly, the derivative of this is equal to
\begin{equation}
0 \cdot x + 2 \cdot 1 
\end{equation}

We will take a closer look at this:
\begin{equation}
1 
\end{equation}
Obviously, the derivative of this is equal to
\begin{equation}
0 
\end{equation}

Consider the following:
\begin{equation}
x ^{2 } 
\end{equation}
Unsurprisingly, the derivative of this is equal to
\begin{equation}
2 \cdot x ^{2 - 1 } \cdot 1 
\end{equation}

Consider the following:
\begin{equation}
2 
\end{equation}
Obviously, the derivative of this is equal to
\begin{equation}
0 
\end{equation}

The following is worth a closer look:
\begin{equation}
1 
\end{equation}
Trivially, the derivative of this is equal to
\begin{equation}
0 
\end{equation}

We will take a closer look at this:
\begin{equation}
1 
\end{equation}
Clearly, the derivative of this is equal to
\begin{equation}
0 
\end{equation}

We are going to study the following:
\begin{equation}
x ^{2 } 
\end{equation}
As you can see, the derivative of this is equal to
\begin{equation}
2 \cdot x ^{2 - 1 } \cdot 1 
\end{equation}

Let us take a look at this:
\begin{equation}
1 
\end{equation}
It can be easily proved, that the derivative of this is equal to
\begin{equation}
0 
\end{equation}

Let us take a look at this:
\begin{equation}
1 
\end{equation}
Unsurprisingly, the derivative of this is equal to
\begin{equation}
0 
\end{equation}

We will allow ourselves to divert the reader's attentions to this gem of mathematical wonder:
\begin{equation}
x - 2 
\end{equation}
Unsurprisingly, the derivative of this is equal to
\begin{equation}
1 - 0 
\end{equation}

Consider the following:
\begin{equation}
1 
\end{equation}
Any self-respecting mathematician would find it obvious, that the derivative of this is equal to
\begin{equation}
0 
\end{equation}

Consider the following:
\begin{equation}
x ^{2 } 
\end{equation}
As you can see, the derivative of this is equal to
\begin{equation}
2 \cdot x ^{2 - 1 } \cdot 1 
\end{equation}

Consider the following:
\begin{equation}
1 
\end{equation}
Obviously, the derivative of this is equal to
\begin{equation}
0 
\end{equation}

We shall ponder the following:
\begin{equation}
x ^{2 } 
\end{equation}
Obviously, the derivative of this is equal to
\begin{equation}
2 \cdot x ^{2 - 1 } \cdot 1 
\end{equation}

One shall regard the object in question with utmost interest:
\begin{equation}
2 \cdot x 
\end{equation}
Obviously, the derivative of this is equal to
\begin{equation}
0 \cdot x + 2 \cdot 1 
\end{equation}

We will take a closer look at this:
\begin{equation}
1 
\end{equation}
Unsurprisingly, the derivative of this is equal to
\begin{equation}
0 
\end{equation}

The following is worth a closer look:
\begin{equation}
x ^{2 } 
\end{equation}
Clearly, the derivative of this is equal to
\begin{equation}
2 \cdot x ^{2 - 1 } \cdot 1 
\end{equation}

We shall ponder the following:
\begin{equation}
2 
\end{equation}
Clearly, the derivative of this is equal to
\begin{equation}
0 
\end{equation}

Let us take a look at this:
\begin{equation}
1 
\end{equation}
Obviously, the derivative of this is equal to
\begin{equation}
0 
\end{equation}

Let us take a look at this:
\begin{equation}
1 
\end{equation}
Unsurprisingly, the derivative of this is equal to
\begin{equation}
0 
\end{equation}

We will take a closer look at this:
\begin{equation}
x ^{2 } 
\end{equation}
Unsurprisingly, the derivative of this is equal to
\begin{equation}
2 \cdot x ^{2 - 1 } \cdot 1 
\end{equation}

Let us take a look at this:
\begin{equation}
1 
\end{equation}
It can be easily proved, that the derivative of this is equal to
\begin{equation}
0 
\end{equation}

We will allow ourselves to divert the reader's attentions to this gem of mathematical wonder:
\begin{equation}
1 
\end{equation}
Any self-respecting mathematician would find it obvious, that the derivative of this is equal to
\begin{equation}
0 
\end{equation}

The object of our ultimate interest is the following:
\begin{equation}
x - 2 
\end{equation}
Trivially, the derivative of this is equal to
\begin{equation}
1 - 0 
\end{equation}

We will allow ourselves to divert the reader's attentions to this gem of mathematical wonder:
\begin{equation}
1 
\end{equation}
Obviously, the derivative of this is equal to
\begin{equation}
0 
\end{equation}

We are going to study the following:
\begin{equation}
x ^{2 } 
\end{equation}
Any self-respecting mathematician would find it obvious, that the derivative of this is equal to
\begin{equation}
2 \cdot x ^{2 - 1 } \cdot 1 
\end{equation}

The following is worth a closer look:
\begin{equation}
1 
\end{equation}
It is now obvious, that the derivative of this is equal to
\begin{equation}
0 
\end{equation}

We shall ponder the following:
\begin{equation}
x ^{2 } 
\end{equation}
Trivially, the derivative of this is equal to
\begin{equation}
2 \cdot x ^{2 - 1 } \cdot 1 
\end{equation}

We will take a closer look at this:
\begin{equation}
x - 2 
\end{equation}
It can be easily proved, that the derivative of this is equal to
\begin{equation}
1 - 0 
\end{equation}

We will allow ourselves to divert the reader's attentions to this gem of mathematical wonder:
\begin{equation}
x + 1 
\end{equation}
Trivially, the derivative of this is equal to
\begin{equation}
1 + 0 
\end{equation}

One shall regard the object in question with utmost interest:
\begin{equation}
1 
\end{equation}
Clearly, the derivative of this is equal to
\begin{equation}
0 
\end{equation}

We shall ponder the following:
\begin{equation}
2 
\end{equation}
As you can see, the derivative of this is equal to
\begin{equation}
0 
\end{equation}

Consider the following:
\begin{equation}
\sin x 
\end{equation}
It can be easily proved, that the derivative of this is equal to
\begin{equation}
\cos x \cdot 1 
\end{equation}

We will take a closer look at this:
\begin{equation}
x + 1 
\end{equation}
Unsurprisingly, the derivative of this is equal to
\begin{equation}
1 + 0 
\end{equation}

The object of our ultimate interest is the following:
\begin{equation}
4 
\end{equation}
It can be easily proved, that the derivative of this is equal to
\begin{equation}
0 
\end{equation}

The following is worth a closer look:
\begin{equation}
2 
\end{equation}
As you can see, the derivative of this is equal to
\begin{equation}
0 
\end{equation}

One shall regard the object in question with utmost interest:
\begin{equation}
\cos x 
\end{equation}
Trivially, the derivative of this is equal to
\begin{equation}
-\sin x \cdot 1 
\end{equation}

One shall regard the object in question with utmost interest:
\begin{equation}
x + 1 
\end{equation}
It can be easily proved, that the derivative of this is equal to
\begin{equation}
1 + 0 
\end{equation}

We shall ponder the following:
\begin{equation}
2 
\end{equation}
Any self-respecting mathematician would find it obvious, that the derivative of this is equal to
\begin{equation}
0 
\end{equation}

The object of our ultimate interest is the following:
\begin{equation}
\sin x 
\end{equation}
Any self-respecting mathematician would find it obvious, that the derivative of this is equal to
\begin{equation}
\cos x \cdot 1 
\end{equation}

We shall ponder the following:
\begin{equation}
1 
\end{equation}
Obviously, the derivative of this is equal to
\begin{equation}
0 
\end{equation}

Consider the following:
\begin{equation}
x ^{2 } 
\end{equation}
Trivially, the derivative of this is equal to
\begin{equation}
2 \cdot x ^{2 - 1 } \cdot 1 
\end{equation}

We are going to study the following:
\begin{equation}
2 \cdot x 
\end{equation}
Any self-respecting mathematician would find it obvious, that the derivative of this is equal to
\begin{equation}
0 \cdot x + 2 \cdot 1 
\end{equation}

Consider the following:
\begin{equation}
1 
\end{equation}
Trivially, the derivative of this is equal to
\begin{equation}
0 
\end{equation}

We will take a closer look at this:
\begin{equation}
x ^{2 } 
\end{equation}
Unsurprisingly, the derivative of this is equal to
\begin{equation}
2 \cdot x ^{2 - 1 } \cdot 1 
\end{equation}

We will allow ourselves to divert the reader's attentions to this gem of mathematical wonder:
\begin{equation}
2 
\end{equation}
Unsurprisingly, the derivative of this is equal to
\begin{equation}
0 
\end{equation}

Let us take a look at this:
\begin{equation}
1 
\end{equation}
Clearly, the derivative of this is equal to
\begin{equation}
0 
\end{equation}

The following is worth a closer look:
\begin{equation}
1 
\end{equation}
Trivially, the derivative of this is equal to
\begin{equation}
0 
\end{equation}

Let us take a look at this:
\begin{equation}
x ^{2 } 
\end{equation}
It is now obvious, that the derivative of this is equal to
\begin{equation}
2 \cdot x ^{2 - 1 } \cdot 1 
\end{equation}

We shall ponder the following:
\begin{equation}
1 
\end{equation}
Clearly, the derivative of this is equal to
\begin{equation}
0 
\end{equation}

We are going to study the following:
\begin{equation}
1 
\end{equation}
Trivially, the derivative of this is equal to
\begin{equation}
0 
\end{equation}

One shall regard the object in question with utmost interest:
\begin{equation}
x - 2 
\end{equation}
Unsurprisingly, the derivative of this is equal to
\begin{equation}
1 - 0 
\end{equation}

We will take a closer look at this:
\begin{equation}
1 
\end{equation}
Obviously, the derivative of this is equal to
\begin{equation}
0 
\end{equation}

The object of our ultimate interest is the following:
\begin{equation}
x ^{2 } 
\end{equation}
Any self-respecting mathematician would find it obvious, that the derivative of this is equal to
\begin{equation}
2 \cdot x ^{2 - 1 } \cdot 1 
\end{equation}

Let us take a look at this:
\begin{equation}
1 
\end{equation}
Clearly, the derivative of this is equal to
\begin{equation}
0 
\end{equation}

We are going to study the following:
\begin{equation}
x ^{2 } 
\end{equation}
As you can see, the derivative of this is equal to
\begin{equation}
2 \cdot x ^{2 - 1 } \cdot 1 
\end{equation}

The object of our ultimate interest is the following:
\begin{equation}
x - 2 
\end{equation}
Clearly, the derivative of this is equal to
\begin{equation}
1 - 0 
\end{equation}

We will take a closer look at this:
\begin{equation}
x + 1 
\end{equation}
Any self-respecting mathematician would find it obvious, that the derivative of this is equal to
\begin{equation}
1 + 0 
\end{equation}

We are going to study the following:
\begin{equation}
1 
\end{equation}
It is now obvious, that the derivative of this is equal to
\begin{equation}
0 
\end{equation}

Consider the following:
\begin{equation}
2 
\end{equation}
Obviously, the derivative of this is equal to
\begin{equation}
0 
\end{equation}

We are going to study the following:
\begin{equation}
\sin x 
\end{equation}
Unsurprisingly, the derivative of this is equal to
\begin{equation}
\cos x \cdot 1 
\end{equation}

We will take a closer look at this:
\begin{equation}
x + 1 
\end{equation}
It can be easily proved, that the derivative of this is equal to
\begin{equation}
1 + 0 
\end{equation}

We are going to study the following:
\begin{equation}
1 
\end{equation}
As you can see, the derivative of this is equal to
\begin{equation}
0 
\end{equation}

One shall regard the object in question with utmost interest:
\begin{equation}
4 
\end{equation}
Clearly, the derivative of this is equal to
\begin{equation}
0 
\end{equation}

Let us take a look at this:
\begin{equation}
2 
\end{equation}
Any self-respecting mathematician would find it obvious, that the derivative of this is equal to
\begin{equation}
0 
\end{equation}

We shall ponder the following:
\begin{equation}
\cos x 
\end{equation}
Trivially, the derivative of this is equal to
\begin{equation}
-\sin x \cdot 1 
\end{equation}

We will take a closer look at this:
\begin{equation}
x + 1 
\end{equation}
Clearly, the derivative of this is equal to
\begin{equation}
1 + 0 
\end{equation}

The following is worth a closer look:
\begin{equation}
16 
\end{equation}
Any self-respecting mathematician would find it obvious, that the derivative of this is equal to
\begin{equation}
0 
\end{equation}

The object of our ultimate interest is the following:
\begin{equation}
0 
\end{equation}
Any self-respecting mathematician would find it obvious, that the derivative of this is equal to
\begin{equation}
0 
\end{equation}

The following is worth a closer look:
\begin{equation}
4 
\end{equation}
It is now obvious, that the derivative of this is equal to
\begin{equation}
0 
\end{equation}

We will take a closer look at this:
\begin{equation}
2 
\end{equation}
Obviously, the derivative of this is equal to
\begin{equation}
0 
\end{equation}

One shall regard the object in question with utmost interest:
\begin{equation}
1 
\end{equation}
Trivially, the derivative of this is equal to
\begin{equation}
0 
\end{equation}

We will take a closer look at this:
\begin{equation}
\sin x 
\end{equation}
Clearly, the derivative of this is equal to
\begin{equation}
\cos x \cdot 1 
\end{equation}

One shall regard the object in question with utmost interest:
\begin{equation}
x + 1 
\end{equation}
It can be easily proved, that the derivative of this is equal to
\begin{equation}
1 + 0 
\end{equation}

We will allow ourselves to divert the reader's attentions to this gem of mathematical wonder:
\begin{equation}
1 
\end{equation}
As you can see, the derivative of this is equal to
\begin{equation}
0 
\end{equation}

One shall regard the object in question with utmost interest:
\begin{equation}
4 
\end{equation}
Any self-respecting mathematician would find it obvious, that the derivative of this is equal to
\begin{equation}
0 
\end{equation}

We are going to study the following:
\begin{equation}
2 
\end{equation}
Unsurprisingly, the derivative of this is equal to
\begin{equation}
0 
\end{equation}

We shall ponder the following:
\begin{equation}
\cos x 
\end{equation}
As you can see, the derivative of this is equal to
\begin{equation}
-\sin x \cdot 1 
\end{equation}

The following is worth a closer look:
\begin{equation}
x + 1 
\end{equation}
Trivially, the derivative of this is equal to
\begin{equation}
1 + 0 
\end{equation}

Consider the following:
\begin{equation}
2 
\end{equation}
Obviously, the derivative of this is equal to
\begin{equation}
0 
\end{equation}

The following is worth a closer look:
\begin{equation}
\sin x 
\end{equation}
Unsurprisingly, the derivative of this is equal to
\begin{equation}
\cos x \cdot 1 
\end{equation}

The object of our ultimate interest is the following:
\begin{equation}
x + 1 
\end{equation}
Trivially, the derivative of this is equal to
\begin{equation}
1 + 0 
\end{equation}

We will take a closer look at this:
\begin{equation}
1 
\end{equation}
Clearly, the derivative of this is equal to
\begin{equation}
0 
\end{equation}

One shall regard the object in question with utmost interest:
\begin{equation}
2 
\end{equation}
Unsurprisingly, the derivative of this is equal to
\begin{equation}
0 
\end{equation}

One shall regard the object in question with utmost interest:
\begin{equation}
\sin x 
\end{equation}
Unsurprisingly, the derivative of this is equal to
\begin{equation}
\cos x \cdot 1 
\end{equation}

Let us take a look at this:
\begin{equation}
x + 1 
\end{equation}
It is now obvious, that the derivative of this is equal to
\begin{equation}
1 + 0 
\end{equation}

We will take a closer look at this:
\begin{equation}
4 
\end{equation}
It can be easily proved, that the derivative of this is equal to
\begin{equation}
0 
\end{equation}

One shall regard the object in question with utmost interest:
\begin{equation}
2 
\end{equation}
Any self-respecting mathematician would find it obvious, that the derivative of this is equal to
\begin{equation}
0 
\end{equation}

We are going to study the following:
\begin{equation}
\cos x 
\end{equation}
As you can see, the derivative of this is equal to
\begin{equation}
-\sin x \cdot 1 
\end{equation}

One shall regard the object in question with utmost interest:
\begin{equation}
x + 1 
\end{equation}
Clearly, the derivative of this is equal to
\begin{equation}
1 + 0 
\end{equation}

We shall ponder the following:
\begin{equation}
1 
\end{equation}
Unsurprisingly, the derivative of this is equal to
\begin{equation}
0 
\end{equation}

One shall regard the object in question with utmost interest:
\begin{equation}
2 
\end{equation}
It can be easily proved, that the derivative of this is equal to
\begin{equation}
0 
\end{equation}

We shall ponder the following:
\begin{equation}
\sin x 
\end{equation}
Trivially, the derivative of this is equal to
\begin{equation}
\cos x \cdot 1 
\end{equation}

The object of our ultimate interest is the following:
\begin{equation}
x + 1 
\end{equation}
Obviously, the derivative of this is equal to
\begin{equation}
1 + 0 
\end{equation}

We shall ponder the following:
\begin{equation}
4 
\end{equation}
It can be easily proved, that the derivative of this is equal to
\begin{equation}
0 
\end{equation}

Consider the following:
\begin{equation}
2 
\end{equation}
As you can see, the derivative of this is equal to
\begin{equation}
0 
\end{equation}

We will allow ourselves to divert the reader's attentions to this gem of mathematical wonder:
\begin{equation}
\cos x 
\end{equation}
As you can see, the derivative of this is equal to
\begin{equation}
-\sin x \cdot 1 
\end{equation}

The following is worth a closer look:
\begin{equation}
x + 1 
\end{equation}
Trivially, the derivative of this is equal to
\begin{equation}
1 + 0 
\end{equation}

We shall ponder the following:
\begin{equation}
2 
\end{equation}
Trivially, the derivative of this is equal to
\begin{equation}
0 
\end{equation}

We will take a closer look at this:
\begin{equation}
\sin x 
\end{equation}
Obviously, the derivative of this is equal to
\begin{equation}
\cos x \cdot 1 
\end{equation}

Let us take a look at this:
\begin{equation}
1 
\end{equation}
Obviously, the derivative of this is equal to
\begin{equation}
0 
\end{equation}

We will take a closer look at this:
\begin{equation}
x ^{2 } 
\end{equation}
Obviously, the derivative of this is equal to
\begin{equation}
2 \cdot x ^{2 - 1 } \cdot 1 
\end{equation}

We are going to study the following:
\begin{equation}
2 \cdot x 
\end{equation}
It is now obvious, that the derivative of this is equal to
\begin{equation}
0 \cdot x + 2 \cdot 1 
\end{equation}

We will allow ourselves to divert the reader's attentions to this gem of mathematical wonder:
\begin{equation}
1 
\end{equation}
It can be easily proved, that the derivative of this is equal to
\begin{equation}
0 
\end{equation}

The following is worth a closer look:
\begin{equation}
x ^{2 } 
\end{equation}
Unsurprisingly, the derivative of this is equal to
\begin{equation}
2 \cdot x ^{2 - 1 } \cdot 1 
\end{equation}

Let us take a look at this:
\begin{equation}
2 
\end{equation}
Unsurprisingly, the derivative of this is equal to
\begin{equation}
0 
\end{equation}

Consider the following:
\begin{equation}
1 
\end{equation}
Any self-respecting mathematician would find it obvious, that the derivative of this is equal to
\begin{equation}
0 
\end{equation}

We shall ponder the following:
\begin{equation}
1 
\end{equation}
It is now obvious, that the derivative of this is equal to
\begin{equation}
0 
\end{equation}

We will allow ourselves to divert the reader's attentions to this gem of mathematical wonder:
\begin{equation}
x ^{2 } 
\end{equation}
Clearly, the derivative of this is equal to
\begin{equation}
2 \cdot x ^{2 - 1 } \cdot 1 
\end{equation}

We shall ponder the following:
\begin{equation}
1 
\end{equation}
It is now obvious, that the derivative of this is equal to
\begin{equation}
0 
\end{equation}

We are going to study the following:
\begin{equation}
1 
\end{equation}
Any self-respecting mathematician would find it obvious, that the derivative of this is equal to
\begin{equation}
0 
\end{equation}

The following is worth a closer look:
\begin{equation}
x - 2 
\end{equation}
As you can see, the derivative of this is equal to
\begin{equation}
1 - 0 
\end{equation}

The following is worth a closer look:
\begin{equation}
1 
\end{equation}
It is now obvious, that the derivative of this is equal to
\begin{equation}
0 
\end{equation}

We will allow ourselves to divert the reader's attentions to this gem of mathematical wonder:
\begin{equation}
x ^{2 } 
\end{equation}
Trivially, the derivative of this is equal to
\begin{equation}
2 \cdot x ^{2 - 1 } \cdot 1 
\end{equation}

We are going to study the following:
\begin{equation}
1 
\end{equation}
Obviously, the derivative of this is equal to
\begin{equation}
0 
\end{equation}

Let us take a look at this:
\begin{equation}
x ^{2 } 
\end{equation}
Clearly, the derivative of this is equal to
\begin{equation}
2 \cdot x ^{2 - 1 } \cdot 1 
\end{equation}

We will allow ourselves to divert the reader's attentions to this gem of mathematical wonder:
\begin{equation}
x - 2 
\end{equation}
It can be easily proved, that the derivative of this is equal to
\begin{equation}
1 - 0 
\end{equation}

The following is worth a closer look:
\begin{equation}
x + 1 
\end{equation}
Any self-respecting mathematician would find it obvious, that the derivative of this is equal to
\begin{equation}
1 + 0 
\end{equation}

Consider the following:
\begin{equation}
1 
\end{equation}
It is now obvious, that the derivative of this is equal to
\begin{equation}
0 
\end{equation}

Consider the following:
\begin{equation}
2 
\end{equation}
It can be easily proved, that the derivative of this is equal to
\begin{equation}
0 
\end{equation}

One shall regard the object in question with utmost interest:
\begin{equation}
\sin x 
\end{equation}
Unsurprisingly, the derivative of this is equal to
\begin{equation}
\cos x \cdot 1 
\end{equation}

The following is worth a closer look:
\begin{equation}
x + 1 
\end{equation}
As you can see, the derivative of this is equal to
\begin{equation}
1 + 0 
\end{equation}

The following is worth a closer look:
\begin{equation}
1 
\end{equation}
Clearly, the derivative of this is equal to
\begin{equation}
0 
\end{equation}

Let us take a look at this:
\begin{equation}
4 
\end{equation}
It can be easily proved, that the derivative of this is equal to
\begin{equation}
0 
\end{equation}

The object of our ultimate interest is the following:
\begin{equation}
2 
\end{equation}
Clearly, the derivative of this is equal to
\begin{equation}
0 
\end{equation}

We will take a closer look at this:
\begin{equation}
\cos x 
\end{equation}
It can be easily proved, that the derivative of this is equal to
\begin{equation}
-\sin x \cdot 1 
\end{equation}

Consider the following:
\begin{equation}
x + 1 
\end{equation}
It is now obvious, that the derivative of this is equal to
\begin{equation}
1 + 0 
\end{equation}

We are going to study the following:
\begin{equation}
16 
\end{equation}
As you can see, the derivative of this is equal to
\begin{equation}
0 
\end{equation}

We are going to study the following:
\begin{equation}
4 
\end{equation}
It can be easily proved, that the derivative of this is equal to
\begin{equation}
0 
\end{equation}

Consider the following:
\begin{equation}
2 
\end{equation}
Any self-respecting mathematician would find it obvious, that the derivative of this is equal to
\begin{equation}
0 
\end{equation}

One shall regard the object in question with utmost interest:
\begin{equation}
\sin x 
\end{equation}
Obviously, the derivative of this is equal to
\begin{equation}
\cos x \cdot 1 
\end{equation}

We are going to study the following:
\begin{equation}
x + 1 
\end{equation}
Any self-respecting mathematician would find it obvious, that the derivative of this is equal to
\begin{equation}
1 + 0 
\end{equation}

One shall regard the object in question with utmost interest:
\begin{equation}
1 
\end{equation}
As you can see, the derivative of this is equal to
\begin{equation}
0 
\end{equation}

We shall ponder the following:
\begin{equation}
4 
\end{equation}
Any self-respecting mathematician would find it obvious, that the derivative of this is equal to
\begin{equation}
0 
\end{equation}

The following is worth a closer look:
\begin{equation}
2 
\end{equation}
As you can see, the derivative of this is equal to
\begin{equation}
0 
\end{equation}

Let us take a look at this:
\begin{equation}
\cos x 
\end{equation}
Obviously, the derivative of this is equal to
\begin{equation}
-\sin x \cdot 1 
\end{equation}

We will allow ourselves to divert the reader's attentions to this gem of mathematical wonder:
\begin{equation}
x + 1 
\end{equation}
As you can see, the derivative of this is equal to
\begin{equation}
1 + 0 
\end{equation}

We shall ponder the following:
\begin{equation}
2 
\end{equation}
Obviously, the derivative of this is equal to
\begin{equation}
0 
\end{equation}

We are going to study the following:
\begin{equation}
\sin x 
\end{equation}
As you can see, the derivative of this is equal to
\begin{equation}
\cos x \cdot 1 
\end{equation}

We shall ponder the following:
\begin{equation}
x + 1 
\end{equation}
It is now obvious, that the derivative of this is equal to
\begin{equation}
1 + 0 
\end{equation}

We will take a closer look at this:
\begin{equation}
1 
\end{equation}
It is now obvious, that the derivative of this is equal to
\begin{equation}
0 
\end{equation}

The object of our ultimate interest is the following:
\begin{equation}
2 
\end{equation}
Any self-respecting mathematician would find it obvious, that the derivative of this is equal to
\begin{equation}
0 
\end{equation}

Consider the following:
\begin{equation}
\sin x 
\end{equation}
It can be easily proved, that the derivative of this is equal to
\begin{equation}
\cos x \cdot 1 
\end{equation}

Consider the following:
\begin{equation}
x + 1 
\end{equation}
It is now obvious, that the derivative of this is equal to
\begin{equation}
1 + 0 
\end{equation}

The object of our ultimate interest is the following:
\begin{equation}
4 
\end{equation}
Obviously, the derivative of this is equal to
\begin{equation}
0 
\end{equation}

The object of our ultimate interest is the following:
\begin{equation}
2 
\end{equation}
It can be easily proved, that the derivative of this is equal to
\begin{equation}
0 
\end{equation}

Let us take a look at this:
\begin{equation}
\cos x 
\end{equation}
As you can see, the derivative of this is equal to
\begin{equation}
-\sin x \cdot 1 
\end{equation}

We will allow ourselves to divert the reader's attentions to this gem of mathematical wonder:
\begin{equation}
x + 1 
\end{equation}
Obviously, the derivative of this is equal to
\begin{equation}
1 + 0 
\end{equation}

Consider the following:
\begin{equation}
1 
\end{equation}
Any self-respecting mathematician would find it obvious, that the derivative of this is equal to
\begin{equation}
0 
\end{equation}

We will take a closer look at this:
\begin{equation}
2 
\end{equation}
Clearly, the derivative of this is equal to
\begin{equation}
0 
\end{equation}

The following is worth a closer look:
\begin{equation}
\sin x 
\end{equation}
It is now obvious, that the derivative of this is equal to
\begin{equation}
\cos x \cdot 1 
\end{equation}

The object of our ultimate interest is the following:
\begin{equation}
x + 1 
\end{equation}
Obviously, the derivative of this is equal to
\begin{equation}
1 + 0 
\end{equation}

One shall regard the object in question with utmost interest:
\begin{equation}
4 
\end{equation}
It is now obvious, that the derivative of this is equal to
\begin{equation}
0 
\end{equation}

We are going to study the following:
\begin{equation}
2 
\end{equation}
Obviously, the derivative of this is equal to
\begin{equation}
0 
\end{equation}

Let us take a look at this:
\begin{equation}
\cos x 
\end{equation}
Unsurprisingly, the derivative of this is equal to
\begin{equation}
-\sin x \cdot 1 
\end{equation}

Consider the following:
\begin{equation}
x + 1 
\end{equation}
It can be easily proved, that the derivative of this is equal to
\begin{equation}
1 + 0 
\end{equation}

The object of our ultimate interest is the following:
\begin{equation}
2 
\end{equation}
It is now obvious, that the derivative of this is equal to
\begin{equation}
0 
\end{equation}

We will allow ourselves to divert the reader's attentions to this gem of mathematical wonder:
\begin{equation}
\sin x 
\end{equation}
It is now obvious, that the derivative of this is equal to
\begin{equation}
\cos x \cdot 1 
\end{equation}

Consider the following:
\begin{equation}
1 
\end{equation}
It is now obvious, that the derivative of this is equal to
\begin{equation}
0 
\end{equation}

One shall regard the object in question with utmost interest:
\begin{equation}
x ^{2 } 
\end{equation}
As you can see, the derivative of this is equal to
\begin{equation}
2 \cdot x ^{2 - 1 } \cdot 1 
\end{equation}

We will take a closer look at this:
\begin{equation}
x - 2 
\end{equation}
Trivially, the derivative of this is equal to
\begin{equation}
1 - 0 
\end{equation}

Consider the following:
\begin{equation}
x + 1 
\end{equation}
Obviously, the derivative of this is equal to
\begin{equation}
1 + 0 
\end{equation}

We are going to study the following:
\begin{equation}
x + 1 
\end{equation}
As you can see, the derivative of this is equal to
\begin{equation}
1 + 0 
\end{equation}

We will take a closer look at this:
\begin{equation}
2 
\end{equation}
It is now obvious, that the derivative of this is equal to
\begin{equation}
0 
\end{equation}

We will take a closer look at this:
\begin{equation}
2 
\end{equation}
Trivially, the derivative of this is equal to
\begin{equation}
0 
\end{equation}

The object of our ultimate interest is the following:
\begin{equation}
\sin x 
\end{equation}
Trivially, the derivative of this is equal to
\begin{equation}
\cos x \cdot 1 
\end{equation}

We are going to study the following:
\begin{equation}
x + 1 
\end{equation}
Clearly, the derivative of this is equal to
\begin{equation}
1 + 0 
\end{equation}

We are going to study the following:
\begin{equation}
1 
\end{equation}
Obviously, the derivative of this is equal to
\begin{equation}
0 
\end{equation}

We will allow ourselves to divert the reader's attentions to this gem of mathematical wonder:
\begin{equation}
4 
\end{equation}
As you can see, the derivative of this is equal to
\begin{equation}
0 
\end{equation}

One shall regard the object in question with utmost interest:
\begin{equation}
2 
\end{equation}
As you can see, the derivative of this is equal to
\begin{equation}
0 
\end{equation}

Let us take a look at this:
\begin{equation}
\cos x 
\end{equation}
It can be easily proved, that the derivative of this is equal to
\begin{equation}
-\sin x \cdot 1 
\end{equation}

We are going to study the following:
\begin{equation}
x + 1 
\end{equation}
Unsurprisingly, the derivative of this is equal to
\begin{equation}
1 + 0 
\end{equation}

We will take a closer look at this:
\begin{equation}
1 
\end{equation}
Unsurprisingly, the derivative of this is equal to
\begin{equation}
0 
\end{equation}

Let us take a look at this:
\begin{equation}
4 
\end{equation}
Trivially, the derivative of this is equal to
\begin{equation}
0 
\end{equation}

We shall ponder the following:
\begin{equation}
2 
\end{equation}
It can be easily proved, that the derivative of this is equal to
\begin{equation}
0 
\end{equation}

Consider the following:
\begin{equation}
\cos x 
\end{equation}
It can be easily proved, that the derivative of this is equal to
\begin{equation}
-\sin x \cdot 1 
\end{equation}

We shall ponder the following:
\begin{equation}
x + 1 
\end{equation}
It is now obvious, that the derivative of this is equal to
\begin{equation}
1 + 0 
\end{equation}

Consider the following:
\begin{equation}
1 
\end{equation}
Clearly, the derivative of this is equal to
\begin{equation}
0 
\end{equation}

Let us take a look at this:
\begin{equation}
16 
\end{equation}
Any self-respecting mathematician would find it obvious, that the derivative of this is equal to
\begin{equation}
0 
\end{equation}

We will allow ourselves to divert the reader's attentions to this gem of mathematical wonder:
\begin{equation}
4 
\end{equation}
As you can see, the derivative of this is equal to
\begin{equation}
0 
\end{equation}

Let us take a look at this:
\begin{equation}
2 
\end{equation}
It can be easily proved, that the derivative of this is equal to
\begin{equation}
0 
\end{equation}

We will allow ourselves to divert the reader's attentions to this gem of mathematical wonder:
\begin{equation}
\sin x 
\end{equation}
It can be easily proved, that the derivative of this is equal to
\begin{equation}
\cos x \cdot 1 
\end{equation}

The following is worth a closer look:
\begin{equation}
x + 1 
\end{equation}
Clearly, the derivative of this is equal to
\begin{equation}
1 + 0 
\end{equation}

The object of our ultimate interest is the following:
\begin{equation}
1 
\end{equation}
As you can see, the derivative of this is equal to
\begin{equation}
0 
\end{equation}

One shall regard the object in question with utmost interest:
\begin{equation}
16 
\end{equation}
Trivially, the derivative of this is equal to
\begin{equation}
0 
\end{equation}

The object of our ultimate interest is the following:
\begin{equation}
4 
\end{equation}
Unsurprisingly, the derivative of this is equal to
\begin{equation}
0 
\end{equation}

We shall ponder the following:
\begin{equation}
2 
\end{equation}
Clearly, the derivative of this is equal to
\begin{equation}
0 
\end{equation}

We will take a closer look at this:
\begin{equation}
\sin x 
\end{equation}
It can be easily proved, that the derivative of this is equal to
\begin{equation}
\cos x \cdot 1 
\end{equation}

The object of our ultimate interest is the following:
\begin{equation}
x + 1 
\end{equation}
It can be easily proved, that the derivative of this is equal to
\begin{equation}
1 + 0 
\end{equation}

We will take a closer look at this:
\begin{equation}
256 
\end{equation}
Obviously, the derivative of this is equal to
\begin{equation}
0 
\end{equation}

Let us take a look at this:
\begin{equation}
16 
\end{equation}
Any self-respecting mathematician would find it obvious, that the derivative of this is equal to
\begin{equation}
0 
\end{equation}

We will take a closer look at this:
\begin{equation}
4 
\end{equation}
Clearly, the derivative of this is equal to
\begin{equation}
0 
\end{equation}

We will allow ourselves to divert the reader's attentions to this gem of mathematical wonder:
\begin{equation}
2 
\end{equation}
Obviously, the derivative of this is equal to
\begin{equation}
0 
\end{equation}

We will take a closer look at this:
\begin{equation}
\cos x 
\end{equation}
Obviously, the derivative of this is equal to
\begin{equation}
-\sin x \cdot 1 
\end{equation}

We will take a closer look at this:
\begin{equation}
x + 1 
\end{equation}
It can be easily proved, that the derivative of this is equal to
\begin{equation}
1 + 0 
\end{equation}

Let us take a look at this:
\begin{equation}
1 
\end{equation}
Trivially, the derivative of this is equal to
\begin{equation}
0 
\end{equation}

We are going to study the following:
\begin{equation}
4 
\end{equation}
As you can see, the derivative of this is equal to
\begin{equation}
0 
\end{equation}

We will allow ourselves to divert the reader's attentions to this gem of mathematical wonder:
\begin{equation}
2 
\end{equation}
Any self-respecting mathematician would find it obvious, that the derivative of this is equal to
\begin{equation}
0 
\end{equation}

The following is worth a closer look:
\begin{equation}
\cos x 
\end{equation}
It is now obvious, that the derivative of this is equal to
\begin{equation}
-\sin x \cdot 1 
\end{equation}

We shall ponder the following:
\begin{equation}
x + 1 
\end{equation}
Any self-respecting mathematician would find it obvious, that the derivative of this is equal to
\begin{equation}
1 + 0 
\end{equation}

We are going to study the following:
\begin{equation}
1 
\end{equation}
It can be easily proved, that the derivative of this is equal to
\begin{equation}
0 
\end{equation}

Let us take a look at this:
\begin{equation}
16 
\end{equation}
Trivially, the derivative of this is equal to
\begin{equation}
0 
\end{equation}

The object of our ultimate interest is the following:
\begin{equation}
4 
\end{equation}
Trivially, the derivative of this is equal to
\begin{equation}
0 
\end{equation}

We shall ponder the following:
\begin{equation}
2 
\end{equation}
It can be easily proved, that the derivative of this is equal to
\begin{equation}
0 
\end{equation}

One shall regard the object in question with utmost interest:
\begin{equation}
\sin x 
\end{equation}
Any self-respecting mathematician would find it obvious, that the derivative of this is equal to
\begin{equation}
\cos x \cdot 1 
\end{equation}

We will take a closer look at this:
\begin{equation}
x + 1 
\end{equation}
Obviously, the derivative of this is equal to
\begin{equation}
1 + 0 
\end{equation}

We shall ponder the following:
\begin{equation}
2 
\end{equation}
Unsurprisingly, the derivative of this is equal to
\begin{equation}
0 
\end{equation}

One shall regard the object in question with utmost interest:
\begin{equation}
\sin x 
\end{equation}
As you can see, the derivative of this is equal to
\begin{equation}
\cos x \cdot 1 
\end{equation}

The object of our ultimate interest is the following:
\begin{equation}
x + 1 
\end{equation}
Clearly, the derivative of this is equal to
\begin{equation}
1 + 0 
\end{equation}

Let us take a look at this:
\begin{equation}
1 
\end{equation}
Trivially, the derivative of this is equal to
\begin{equation}
0 
\end{equation}

The following is worth a closer look:
\begin{equation}
2 
\end{equation}
Any self-respecting mathematician would find it obvious, that the derivative of this is equal to
\begin{equation}
0 
\end{equation}

We shall ponder the following:
\begin{equation}
\sin x 
\end{equation}
It is now obvious, that the derivative of this is equal to
\begin{equation}
\cos x \cdot 1 
\end{equation}

The object of our ultimate interest is the following:
\begin{equation}
x + 1 
\end{equation}
It can be easily proved, that the derivative of this is equal to
\begin{equation}
1 + 0 
\end{equation}

Consider the following:
\begin{equation}
1 
\end{equation}
Any self-respecting mathematician would find it obvious, that the derivative of this is equal to
\begin{equation}
0 
\end{equation}

We are going to study the following:
\begin{equation}
4 
\end{equation}
Unsurprisingly, the derivative of this is equal to
\begin{equation}
0 
\end{equation}

Let us take a look at this:
\begin{equation}
2 
\end{equation}
Obviously, the derivative of this is equal to
\begin{equation}
0 
\end{equation}

One shall regard the object in question with utmost interest:
\begin{equation}
\cos x 
\end{equation}
As you can see, the derivative of this is equal to
\begin{equation}
-\sin x \cdot 1 
\end{equation}

Consider the following:
\begin{equation}
x + 1 
\end{equation}
Unsurprisingly, the derivative of this is equal to
\begin{equation}
1 + 0 
\end{equation}

Consider the following:
\begin{equation}
16 
\end{equation}
As you can see, the derivative of this is equal to
\begin{equation}
0 
\end{equation}

The object of our ultimate interest is the following:
\begin{equation}
4 
\end{equation}
Any self-respecting mathematician would find it obvious, that the derivative of this is equal to
\begin{equation}
0 
\end{equation}

Let us take a look at this:
\begin{equation}
2 
\end{equation}
As you can see, the derivative of this is equal to
\begin{equation}
0 
\end{equation}

The object of our ultimate interest is the following:
\begin{equation}
\sin x 
\end{equation}
It is now obvious, that the derivative of this is equal to
\begin{equation}
\cos x \cdot 1 
\end{equation}

We shall ponder the following:
\begin{equation}
x + 1 
\end{equation}
It is now obvious, that the derivative of this is equal to
\begin{equation}
1 + 0 
\end{equation}

The following is worth a closer look:
\begin{equation}
1 
\end{equation}
Unsurprisingly, the derivative of this is equal to
\begin{equation}
0 
\end{equation}

We will allow ourselves to divert the reader's attentions to this gem of mathematical wonder:
\begin{equation}
4 
\end{equation}
Any self-respecting mathematician would find it obvious, that the derivative of this is equal to
\begin{equation}
0 
\end{equation}

One shall regard the object in question with utmost interest:
\begin{equation}
2 
\end{equation}
Obviously, the derivative of this is equal to
\begin{equation}
0 
\end{equation}

Consider the following:
\begin{equation}
\cos x 
\end{equation}
It can be easily proved, that the derivative of this is equal to
\begin{equation}
-\sin x \cdot 1 
\end{equation}

One shall regard the object in question with utmost interest:
\begin{equation}
x + 1 
\end{equation}
As you can see, the derivative of this is equal to
\begin{equation}
1 + 0 
\end{equation}

Let us take a look at this:
\begin{equation}
1 
\end{equation}
Trivially, the derivative of this is equal to
\begin{equation}
0 
\end{equation}

The object of our ultimate interest is the following:
\begin{equation}
2 
\end{equation}
Unsurprisingly, the derivative of this is equal to
\begin{equation}
0 
\end{equation}

The object of our ultimate interest is the following:
\begin{equation}
\sin x 
\end{equation}
It is now obvious, that the derivative of this is equal to
\begin{equation}
\cos x \cdot 1 
\end{equation}

We will take a closer look at this:
\begin{equation}
x + 1 
\end{equation}
It can be easily proved, that the derivative of this is equal to
\begin{equation}
1 + 0 
\end{equation}

One shall regard the object in question with utmost interest:
\begin{equation}
4 
\end{equation}
It can be easily proved, that the derivative of this is equal to
\begin{equation}
0 
\end{equation}

We are going to study the following:
\begin{equation}
2 
\end{equation}
Any self-respecting mathematician would find it obvious, that the derivative of this is equal to
\begin{equation}
0 
\end{equation}

We will allow ourselves to divert the reader's attentions to this gem of mathematical wonder:
\begin{equation}
\cos x 
\end{equation}
Obviously, the derivative of this is equal to
\begin{equation}
-\sin x \cdot 1 
\end{equation}

One shall regard the object in question with utmost interest:
\begin{equation}
x + 1 
\end{equation}
Obviously, the derivative of this is equal to
\begin{equation}
1 + 0 
\end{equation}

We are going to study the following:
\begin{equation}
2 
\end{equation}
It is now obvious, that the derivative of this is equal to
\begin{equation}
0 
\end{equation}

We will allow ourselves to divert the reader's attentions to this gem of mathematical wonder:
\begin{equation}
\sin x 
\end{equation}
It can be easily proved, that the derivative of this is equal to
\begin{equation}
\cos x \cdot 1 
\end{equation}

We will take a closer look at this:
\begin{equation}
x + 1 
\end{equation}
Trivially, the derivative of this is equal to
\begin{equation}
1 + 0 
\end{equation}

We will allow ourselves to divert the reader's attentions to this gem of mathematical wonder:
\begin{equation}
1 
\end{equation}
As you can see, the derivative of this is equal to
\begin{equation}
0 
\end{equation}

Consider the following:
\begin{equation}
2 
\end{equation}
Trivially, the derivative of this is equal to
\begin{equation}
0 
\end{equation}

Let us take a look at this:
\begin{equation}
\sin x 
\end{equation}
Trivially, the derivative of this is equal to
\begin{equation}
\cos x \cdot 1 
\end{equation}

We shall ponder the following:
\begin{equation}
x + 1 
\end{equation}
Obviously, the derivative of this is equal to
\begin{equation}
1 + 0 
\end{equation}

We shall ponder the following:
\begin{equation}
1 
\end{equation}
Obviously, the derivative of this is equal to
\begin{equation}
0 
\end{equation}

The following is worth a closer look:
\begin{equation}
4 
\end{equation}
Trivially, the derivative of this is equal to
\begin{equation}
0 
\end{equation}

We will allow ourselves to divert the reader's attentions to this gem of mathematical wonder:
\begin{equation}
2 
\end{equation}
It can be easily proved, that the derivative of this is equal to
\begin{equation}
0 
\end{equation}

We will take a closer look at this:
\begin{equation}
\cos x 
\end{equation}
As you can see, the derivative of this is equal to
\begin{equation}
-\sin x \cdot 1 
\end{equation}

The object of our ultimate interest is the following:
\begin{equation}
x + 1 
\end{equation}
Clearly, the derivative of this is equal to
\begin{equation}
1 + 0 
\end{equation}

The following is worth a closer look:
\begin{equation}
16 
\end{equation}
It can be easily proved, that the derivative of this is equal to
\begin{equation}
0 
\end{equation}

We are going to study the following:
\begin{equation}
0 
\end{equation}
Trivially, the derivative of this is equal to
\begin{equation}
0 
\end{equation}

The object of our ultimate interest is the following:
\begin{equation}
4 
\end{equation}
It can be easily proved, that the derivative of this is equal to
\begin{equation}
0 
\end{equation}

We will allow ourselves to divert the reader's attentions to this gem of mathematical wonder:
\begin{equation}
2 
\end{equation}
As you can see, the derivative of this is equal to
\begin{equation}
0 
\end{equation}

Let us take a look at this:
\begin{equation}
1 
\end{equation}
Clearly, the derivative of this is equal to
\begin{equation}
0 
\end{equation}

We are going to study the following:
\begin{equation}
\sin x 
\end{equation}
Trivially, the derivative of this is equal to
\begin{equation}
\cos x \cdot 1 
\end{equation}

Let us take a look at this:
\begin{equation}
x + 1 
\end{equation}
Obviously, the derivative of this is equal to
\begin{equation}
1 + 0 
\end{equation}

The object of our ultimate interest is the following:
\begin{equation}
1 
\end{equation}
It is now obvious, that the derivative of this is equal to
\begin{equation}
0 
\end{equation}

The following is worth a closer look:
\begin{equation}
4 
\end{equation}
It can be easily proved, that the derivative of this is equal to
\begin{equation}
0 
\end{equation}

Consider the following:
\begin{equation}
2 
\end{equation}
Trivially, the derivative of this is equal to
\begin{equation}
0 
\end{equation}

We shall ponder the following:
\begin{equation}
\cos x 
\end{equation}
Unsurprisingly, the derivative of this is equal to
\begin{equation}
-\sin x \cdot 1 
\end{equation}

We are going to study the following:
\begin{equation}
x + 1 
\end{equation}
It is now obvious, that the derivative of this is equal to
\begin{equation}
1 + 0 
\end{equation}

We are going to study the following:
\begin{equation}
1 
\end{equation}
Any self-respecting mathematician would find it obvious, that the derivative of this is equal to
\begin{equation}
0 
\end{equation}

Consider the following:
\begin{equation}
2 
\end{equation}
Obviously, the derivative of this is equal to
\begin{equation}
0 
\end{equation}

Consider the following:
\begin{equation}
\sin x 
\end{equation}
Trivially, the derivative of this is equal to
\begin{equation}
\cos x \cdot 1 
\end{equation}

The following is worth a closer look:
\begin{equation}
x + 1 
\end{equation}
Trivially, the derivative of this is equal to
\begin{equation}
1 + 0 
\end{equation}

We will take a closer look at this:
\begin{equation}
4 
\end{equation}
Unsurprisingly, the derivative of this is equal to
\begin{equation}
0 
\end{equation}

We are going to study the following:
\begin{equation}
2 
\end{equation}
As you can see, the derivative of this is equal to
\begin{equation}
0 
\end{equation}

The object of our ultimate interest is the following:
\begin{equation}
\cos x 
\end{equation}
Trivially, the derivative of this is equal to
\begin{equation}
-\sin x \cdot 1 
\end{equation}

Consider the following:
\begin{equation}
x + 1 
\end{equation}
As you can see, the derivative of this is equal to
\begin{equation}
1 + 0 
\end{equation}

Consider the following:
\begin{equation}
2 
\end{equation}
Trivially, the derivative of this is equal to
\begin{equation}
0 
\end{equation}

We will allow ourselves to divert the reader's attentions to this gem of mathematical wonder:
\begin{equation}
\sin x 
\end{equation}
Clearly, the derivative of this is equal to
\begin{equation}
\cos x \cdot 1 
\end{equation}

We are going to study the following:
\begin{equation}
x + 1 
\end{equation}
Any self-respecting mathematician would find it obvious, that the derivative of this is equal to
\begin{equation}
1 + 0 
\end{equation}

We shall ponder the following:
\begin{equation}
1 
\end{equation}
As you can see, the derivative of this is equal to
\begin{equation}
0 
\end{equation}

One shall regard the object in question with utmost interest:
\begin{equation}
2 
\end{equation}
Obviously, the derivative of this is equal to
\begin{equation}
0 
\end{equation}

One shall regard the object in question with utmost interest:
\begin{equation}
\sin x 
\end{equation}
Unsurprisingly, the derivative of this is equal to
\begin{equation}
\cos x \cdot 1 
\end{equation}

The object of our ultimate interest is the following:
\begin{equation}
x + 1 
\end{equation}
Any self-respecting mathematician would find it obvious, that the derivative of this is equal to
\begin{equation}
1 + 0 
\end{equation}

We shall ponder the following:
\begin{equation}
4 
\end{equation}
It is now obvious, that the derivative of this is equal to
\begin{equation}
0 
\end{equation}

Consider the following:
\begin{equation}
2 
\end{equation}
Trivially, the derivative of this is equal to
\begin{equation}
0 
\end{equation}

Let us take a look at this:
\begin{equation}
\cos x 
\end{equation}
Obviously, the derivative of this is equal to
\begin{equation}
-\sin x \cdot 1 
\end{equation}

Let us take a look at this:
\begin{equation}
x + 1 
\end{equation}
Any self-respecting mathematician would find it obvious, that the derivative of this is equal to
\begin{equation}
1 + 0 
\end{equation}

We are going to study the following:
\begin{equation}
1 
\end{equation}
As you can see, the derivative of this is equal to
\begin{equation}
0 
\end{equation}

We will allow ourselves to divert the reader's attentions to this gem of mathematical wonder:
\begin{equation}
2 
\end{equation}
It can be easily proved, that the derivative of this is equal to
\begin{equation}
0 
\end{equation}

Let us take a look at this:
\begin{equation}
\sin x 
\end{equation}
It is now obvious, that the derivative of this is equal to
\begin{equation}
\cos x \cdot 1 
\end{equation}

Let us take a look at this:
\begin{equation}
x + 1 
\end{equation}
It is now obvious, that the derivative of this is equal to
\begin{equation}
1 + 0 
\end{equation}

One shall regard the object in question with utmost interest:
\begin{equation}
1 
\end{equation}
Unsurprisingly, the derivative of this is equal to
\begin{equation}
0 
\end{equation}

We shall ponder the following:
\begin{equation}
4 
\end{equation}
Unsurprisingly, the derivative of this is equal to
\begin{equation}
0 
\end{equation}

The following is worth a closer look:
\begin{equation}
2 
\end{equation}
It is now obvious, that the derivative of this is equal to
\begin{equation}
0 
\end{equation}

Let us take a look at this:
\begin{equation}
\cos x 
\end{equation}
Obviously, the derivative of this is equal to
\begin{equation}
-\sin x \cdot 1 
\end{equation}

The following is worth a closer look:
\begin{equation}
x + 1 
\end{equation}
Any self-respecting mathematician would find it obvious, that the derivative of this is equal to
\begin{equation}
1 + 0 
\end{equation}

We will take a closer look at this:
\begin{equation}
16 
\end{equation}
Obviously, the derivative of this is equal to
\begin{equation}
0 
\end{equation}

We are going to study the following:
\begin{equation}
0 
\end{equation}
As you can see, the derivative of this is equal to
\begin{equation}
0 
\end{equation}

Consider the following:
\begin{equation}
4 
\end{equation}
Any self-respecting mathematician would find it obvious, that the derivative of this is equal to
\begin{equation}
0 
\end{equation}

Let us take a look at this:
\begin{equation}
2 
\end{equation}
Any self-respecting mathematician would find it obvious, that the derivative of this is equal to
\begin{equation}
0 
\end{equation}

We shall ponder the following:
\begin{equation}
1 
\end{equation}
Unsurprisingly, the derivative of this is equal to
\begin{equation}
0 
\end{equation}

Let us take a look at this:
\begin{equation}
\sin x 
\end{equation}
Unsurprisingly, the derivative of this is equal to
\begin{equation}
\cos x \cdot 1 
\end{equation}

We will take a closer look at this:
\begin{equation}
x + 1 
\end{equation}
It is now obvious, that the derivative of this is equal to
\begin{equation}
1 + 0 
\end{equation}

We shall ponder the following:
\begin{equation}
1 
\end{equation}
Any self-respecting mathematician would find it obvious, that the derivative of this is equal to
\begin{equation}
0 
\end{equation}

The following is worth a closer look:
\begin{equation}
4 
\end{equation}
Clearly, the derivative of this is equal to
\begin{equation}
0 
\end{equation}

Let us take a look at this:
\begin{equation}
2 
\end{equation}
It is now obvious, that the derivative of this is equal to
\begin{equation}
0 
\end{equation}

The object of our ultimate interest is the following:
\begin{equation}
\cos x 
\end{equation}
Obviously, the derivative of this is equal to
\begin{equation}
-\sin x \cdot 1 
\end{equation}

We will allow ourselves to divert the reader's attentions to this gem of mathematical wonder:
\begin{equation}
x + 1 
\end{equation}
Obviously, the derivative of this is equal to
\begin{equation}
1 + 0 
\end{equation}

We shall ponder the following:
\begin{equation}
2 
\end{equation}
Trivially, the derivative of this is equal to
\begin{equation}
0 
\end{equation}

The object of our ultimate interest is the following:
\begin{equation}
\sin x 
\end{equation}
It can be easily proved, that the derivative of this is equal to
\begin{equation}
\cos x \cdot 1 
\end{equation}

We are going to study the following:
\begin{equation}
x + 1 
\end{equation}
Trivially, the derivative of this is equal to
\begin{equation}
1 + 0 
\end{equation}

The object of our ultimate interest is the following:
\begin{equation}
1 
\end{equation}
Any self-respecting mathematician would find it obvious, that the derivative of this is equal to
\begin{equation}
0 
\end{equation}

We will allow ourselves to divert the reader's attentions to this gem of mathematical wonder:
\begin{equation}
2 
\end{equation}
It can be easily proved, that the derivative of this is equal to
\begin{equation}
0 
\end{equation}

Consider the following:
\begin{equation}
\sin x 
\end{equation}
It can be easily proved, that the derivative of this is equal to
\begin{equation}
\cos x \cdot 1 
\end{equation}

The following is worth a closer look:
\begin{equation}
x + 1 
\end{equation}
Clearly, the derivative of this is equal to
\begin{equation}
1 + 0 
\end{equation}

We are going to study the following:
\begin{equation}
4 
\end{equation}
Any self-respecting mathematician would find it obvious, that the derivative of this is equal to
\begin{equation}
0 
\end{equation}

One shall regard the object in question with utmost interest:
\begin{equation}
2 
\end{equation}
It can be easily proved, that the derivative of this is equal to
\begin{equation}
0 
\end{equation}

We will allow ourselves to divert the reader's attentions to this gem of mathematical wonder:
\begin{equation}
\cos x 
\end{equation}
It can be easily proved, that the derivative of this is equal to
\begin{equation}
-\sin x \cdot 1 
\end{equation}

We are going to study the following:
\begin{equation}
x + 1 
\end{equation}
Obviously, the derivative of this is equal to
\begin{equation}
1 + 0 
\end{equation}

One shall regard the object in question with utmost interest:
\begin{equation}
1 
\end{equation}
It can be easily proved, that the derivative of this is equal to
\begin{equation}
0 
\end{equation}

The object of our ultimate interest is the following:
\begin{equation}
2 
\end{equation}
Any self-respecting mathematician would find it obvious, that the derivative of this is equal to
\begin{equation}
0 
\end{equation}

The object of our ultimate interest is the following:
\begin{equation}
\sin x 
\end{equation}
Any self-respecting mathematician would find it obvious, that the derivative of this is equal to
\begin{equation}
\cos x \cdot 1 
\end{equation}

We will take a closer look at this:
\begin{equation}
x + 1 
\end{equation}
Any self-respecting mathematician would find it obvious, that the derivative of this is equal to
\begin{equation}
1 + 0 
\end{equation}

Consider the following:
\begin{equation}
4 
\end{equation}
Any self-respecting mathematician would find it obvious, that the derivative of this is equal to
\begin{equation}
0 
\end{equation}

The object of our ultimate interest is the following:
\begin{equation}
2 
\end{equation}
It is now obvious, that the derivative of this is equal to
\begin{equation}
0 
\end{equation}

We will take a closer look at this:
\begin{equation}
\cos x 
\end{equation}
Any self-respecting mathematician would find it obvious, that the derivative of this is equal to
\begin{equation}
-\sin x \cdot 1 
\end{equation}

The following is worth a closer look:
\begin{equation}
x + 1 
\end{equation}
Clearly, the derivative of this is equal to
\begin{equation}
1 + 0 
\end{equation}

The object of our ultimate interest is the following:
\begin{equation}
2 
\end{equation}
Any self-respecting mathematician would find it obvious, that the derivative of this is equal to
\begin{equation}
0 
\end{equation}

The object of our ultimate interest is the following:
\begin{equation}
\sin x 
\end{equation}
Trivially, the derivative of this is equal to
\begin{equation}
\cos x \cdot 1 
\end{equation}

Now the proof that the Taylor series of this function at $x = 0$ is equal to
\begin{equation}
B + \left( \left( \arctan 1 \right) ^{-2 } + A \right) \cdot \frac{x ^{2 } }{2 } + \left( G + H \right) \cdot \frac{x ^{3 } }{6 } 
\end{equation}
Where:
\begin{itemize}
	\item $A = \left( \arctan 1 \right) ^{-2 } \cdot \left( \ln \arctan 1 - 2 \cdot 0 \right) \cdot \left( \ln \arctan 1 - 2 \cdot 0 \right) - \left( \arctan 1 \right) ^{-2 } \cdot 2 \cdot \frac{0.5 \cdot \arctan 1 }{\left( \arctan 1 \right) ^{2 } } $
	\item $B = 0 + \left( \arctan 1 \right) ^{-2 } \cdot \frac{x ^{0 } }{1 } + \left( \arctan 1 \right) ^{-2 } \cdot \left( \ln \arctan 1 - 2 \cdot 0 \right) \cdot \frac{x ^{1 } }{1 } $
	\item $C = \left( \arctan 1 \right) ^{-2 } \cdot \left( \ln \arctan 1 - 2 \cdot 0 \right) \cdot \left( \ln \arctan 1 - 2 \cdot 0 \right) - \left( \arctan 1 \right) ^{-2 } \cdot 2 \cdot \frac{0.5 \cdot \arctan 1 }{\left( \arctan 1 \right) ^{2 } } $
	\item $D = \left( \arctan 1 \right) ^{-2 } \cdot \left( \ln \arctan 1 - 2 \cdot 0 \right) \cdot 2 \cdot \frac{0.5 \cdot \arctan 1 }{\left( \arctan 1 \right) ^{2 } } $
	\item $E = \frac{1 }{\arctan 1 } \cdot 0.5 + \frac{0.5 \cdot \arctan 1 }{\left( \arctan 1 \right) ^{2 } } + \frac{0.5 \cdot \arctan 1 }{\left( \arctan 1 \right) ^{2 } } - 2 \cdot 0 $
	\item $F = \left( \arctan 1 \right) ^{-2 } \cdot \left( \ln \arctan 1 - 2 \cdot 0 \right) \cdot 2 \cdot \frac{0.5 \cdot \arctan 1 }{\left( \arctan 1 \right) ^{2 } } $
	\item $G = \left( \arctan 1 \right) ^{-2 } \cdot \left( \ln \arctan 1 - 2 \cdot 0 \right) - 1.5 \cdot \left( \arctan 1 \right) ^{-2 } + \left( \arctan 1 \right) ^{-2 } \cdot \left( \ln \arctan 1 - 2 \cdot 0 \right) $
	\item $H = \left( \arctan 1 \right) ^{-2 } \cdot \left( \ln \arctan 1 - 2 \cdot 0 \right) + \left( C \right) \cdot \left( \ln \arctan 1 - 2 \cdot 0 \right) - D + \left( \arctan 1 \right) ^{-2 } \cdot \left( E \right) - F $
\end{itemize}

shall be considered an amusing exercise for the reader.
It can be easily proved, that if we simplify this we wil get
\begin{equation}
G + \left( F + \left( \arctan 1 \right) ^{-2 } \cdot \ln \arctan 1 + D + E \right) \cdot \frac{x ^{3 } }{6 } 
\end{equation}
Where:
\begin{itemize}
	\item $A = \left( \arctan 1 \right) ^{-2 } \cdot \ln \arctan 1 \cdot \ln \arctan 1 - \left( \arctan 1 \right) ^{-2 } \cdot 2 \cdot \frac{0.5 \cdot \arctan 1 }{\left( \arctan 1 \right) ^{2 } } $
	\item $B = \left( \arctan 1 \right) ^{-2 } \cdot \ln \arctan 1 \cdot \ln \arctan 1 - \left( \arctan 1 \right) ^{-2 } \cdot 2 \cdot \frac{0.5 \cdot \arctan 1 }{\left( \arctan 1 \right) ^{2 } } $
	\item $C = \frac{1 }{\arctan 1 } \cdot 0.5 + \frac{0.5 \cdot \arctan 1 }{\left( \arctan 1 \right) ^{2 } } + \frac{0.5 \cdot \arctan 1 }{\left( \arctan 1 \right) ^{2 } } $
	\item $D = \left( B \right) \cdot \ln \arctan 1 - \left( \arctan 1 \right) ^{-2 } \cdot \ln \arctan 1 \cdot 2 \cdot \frac{0.5 \cdot \arctan 1 }{\left( \arctan 1 \right) ^{2 } } $
	\item $E = \left( \arctan 1 \right) ^{-2 } \cdot \left( C \right) - \left( \arctan 1 \right) ^{-2 } \cdot \ln \arctan 1 \cdot 2 \cdot \frac{0.5 \cdot \arctan 1 }{\left( \arctan 1 \right) ^{2 } } $
	\item $F = \left( \arctan 1 \right) ^{-2 } \cdot \ln \arctan 1 - 1.5 \cdot \left( \arctan 1 \right) ^{-2 } + \left( \arctan 1 \right) ^{-2 } \cdot \ln \arctan 1 $
	\item $G = \left( \arctan 1 \right) ^{-2 } + \left( \arctan 1 \right) ^{-2 } \cdot \ln \arctan 1 \cdot x + \left( \left( \arctan 1 \right) ^{-2 } + A \right) \cdot \frac{x ^{2 } }{2 } $
\end{itemize}

\end{document}
